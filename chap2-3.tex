%2.3.
\section{Лекция 3}
\label{2.3}

\textbf{А.С.:} Сейчас мы немного вернемся к~теореме Ферма и~Диофанту с~его тринадцатью томами. Шесть сохранившихся
томов, как я~уже рассказывал, были изданы, после чего попали в~руки Ферма. Ферма читал труд
Диофанта и~оставлял замечания на~полях книг. Так вот, в том месте, где Диофант полностью разбирает
классическую задачу о~прямоугольных треугольниках, рукой Ферма была на~полях сделана заметка:
<<В~то же время никак нельзя разложить куб в~сумму двух кубов>>. На~нашем языке это звучит так:
уравнение вида $x^{3}+y^{3}=z^{3}$ не~имеет решений в целых числах.


Далее у~Ферма стоит запятая, и~он продолжает: <<Никакую четвертую степень~--- на~сумму двух четвертых степеней, и~вообще никакую
фиксированную степень~--- в~сумму двух таких же степеней>>. Далее он пишет восхитительную
фразу, за~которой математики гонялись 357~лет. Он пишет: <<Я нашел тому факту поистине удивительное
доказательство, но~поля этой книги недостаточно широки, чтобы его вместить>>.
 Эта
запись рукой Ферма была в~экземпляре трудов Диофанта. Этот комментарий был единственным случаем
в~истории, когда утверждение Ферма не~удалось доказать за~разумный период времени, спустя 20--30~лет.
Один раз Ферма ошибся, но~он не~утверждал определенно. В~случае с~простыми <<числами Ферма>> он
написал <<по-видимому, они простые>>. В~случае с~рассматриваемой нами теоремой Ферма написал, что
нашел доказательство. Он упомянул об этом в~1637~г., а~в~1994~г. ее доказали. Мы сидели на~семинаре
по~алгебре. Пришел преподаватель и~сказал: <<У~меня для~вас потрясающая новость~--- доказана великая теорема
Ферма>>.
 Все решили, что это розыгрыш, не~может такого быть. Мы учимся на~мехмате, и~при~нас
происходит историческое событие. Если быть точнее, теорему доказал Эндрю Уайлз в~1993~г. Но затем~в~доказательстве им самим
была найдена ошибка, которую Уайлз вместе с Ричардом Тейлором исправляли полгода. Поэтому окончательно теорема была доказана в~1994~году.
Некоторое время были сомнения, и в~1996--1997~гг.  не~все были убеждены в~том, что это свершилось,
так как понять это доказательство могли лишь немногие из математиков.
 Сегодня можно утверждать, что
понимают доказательство этой теоремы человек 500 в~мире, детально~--- около 100~человек. Ежу
понятно, что Ферма подобного доказательства выдумать не~мог. Следовательно, или Ферма один раз
ошибся, или мы до~сих пор не~знаем простого доказательства этой теоремы. Математики предпочитают
соглашаться с~первым утверждением, ибо второе~ позорно для~всего человечества.


Ферма не~оставил доказательства общего случая, но~сохранились записи изящного доказательства
для~частного случая, для $n=4$, гласящего, что уравнение $x^{4}+y^{4}=z^{4}$ не~имеет нетривиальных решений в~целых числах.

Я~не~буду приводить этого доказательства, хотя оно и не~очень сложное. Оно использует приемы делимости, что
возвращает нас к~нашей первой задаче (найти все пифагоровы тройки).


Итак, \textbf{теорема Ферма:} уравнение $x^{n}+y^{n}=z^{n}$ не~имеет решений в~натуральных числах. Давайте посмотрим, каким
может быть число~$n$.

Это число можно разложить на~множители. Есть такая теорема, называется \textit{<<Основная теорема
арифметики>>}, которая утверждает, что любое натуральное число можно единственным образом,
с~точностью до~перестановки множителей, разложить в~произведение простых чисел. Смотрим, делится ли
n на~2. Делим, пока делится, получаем 2 в~какой-то степени и~оставшийся нечетный множитель. Если
оставшийся множитель не простой, то мы раскладываем его дальше, пока не~получим произведение простых чисел.


Например,
$$
n=2^{m}n_{1}n_{2} = 2^{5}\cdot 17\cdot 7 = 3808.
$$
Почему процесс разложения на~множители не~может продолжаться до~бесконечности? Каждый раз, когда мы
раскладываем на~множители, числа становятся всё меньше и~меньше. Нельзя бесконечно долго уменьшать
натуральное число. Это аксиома Архимеда, но~для~человека разумного это~--- очевидное
утверждение.


\pagebreak


Переименуем простые множители в~$p$. Математики любят обозначать простые числа буквой $p$ от~английского
<<prime>>:
$$
n=2^{m}p_{1}p_{2}p_{3}\ldots p_{k}.
$$
Некоторые из~множителей могут встречаться несколько раз. А~может быть, у~$n$ есть какие-то другие
множители, которые здесь не~перечислены, и~в~результате оно одновременно равно какому-то другому произведению:
$$
n=2^{m}p_{1}p_{2}p_{3}\ldots p_{k}=q_{1}q_{2}q_{3}\ldots q_{k}.
$$
Может ли такое быть, чтобы одно и~то же число раскладывалось на~простые множители <<существенно
по-разному>> (несущественное отличие~--- например, $2\cdot 3 \cdot 5$ и $3\cdot 5 \cdot 2$)?
 Интуиция подсказывает, что нет,
и~интуиция права. Но~доказать это аккуратно довольно сложно. Мы в~это просто поверим и~не~будем проходить
этой тернистой дорогой.
 Что же следует из~единственности разложения на~простые множители?

Есть два варианта. Либо у~$n$ есть хотя бы один нечетный простой делитель, то есть в~записи:
$$
n=2^{m}p_{1}p_{2}p_{3}\ldots p_{k}
$$
хотя бы одно $p$~--- нечетное. Второй вариант состоит в~том, что ни одного нечетного числа нет. Поговорим сперва о втором варианте.
Что можно сказать про~$n$ в~этом случае?
 То, что $n$ является степенью двойки. Если $n=2$, мы получаем
задачу про Пифагоровы треугольники, которую скоро решим в~этой лекции. Если $n\ne2$, то оно представимо
в~виде $4\cdot k$. Высшая степень двойки -- это либо 4, либо $8=4\cdot2$, либо $16=4\cdot4$ и~так далее. Получаем следующее
уравнение:
$$
x^{4k}+y^{4k}=z^{4k},
$$
но, как известно,
$$
x^{4k}=(x^{k})^{4},
$$
откуда получаем
$$
(x^{k})^{4}+(y^{k})^{4}=(z^{k})^{4}.
$$
Если бы можно было решить это уравнение, то три натуральных числа $x^{k}$, $y^{k}$ и~$z^{k}$ образовали бы решение
задачи Ферма $x^{4}+y^{4}=z^{4}$.

Но~Ферма доказал, что такое уравнение не~имеет решений в~целых числах, строго больших нуля.

Поэтому случай теоремы Ферма для чисел~$n$, являющихся ка\-кой-то степенью двойки, сводится к~$n=2$. В~других случаях решений
нет.

Вспомним, какой случай мы еще не~рассмотрели: $n$~ содержит нечетный простой делитель $p$, $n \neq 1$
(кстати, 1 тоже является степенью двойки), то есть $n=pk$.
 Тогда:
$$
(x^{k})^{p}+(y^{k})^{p}=(z^{k})^{p}.
$$
Получается, что если у~$n$ есть простой нечетный делитель~$p$, то несуществование решения уравнения Ферма
с~показателем $n$ сводится к~несуществованию решения уравнения степени~$p$.

То есть теорема Ферма сводится к~исследованию уравнения простой нечетной степени. И если мы знаем, что ни
при~каком простом нечетном $n$ уравнение $x^{n}+y^{n}=z^{n}$ не~имеет решения, то оно не~имеет решения и~ни при~каком
другом~$n\ge3$.
А~теперь~--- история вопроса.


Про уравнение второй степени было известно уже древним индусам. Уравнение третьей степени оказалось более
сложным. Почти полное решение, которое потом довели до~конца, было получено Леонардом Эйлером.
В лекции~4 я~расскажу, каким изящнейшим путем доказывается теорема несуществования для некоторого уравнения третьей
степени (не~связанного напрямую с~теоремой Ферма), но~сначала про пятую степень:
$$
x^{5}+y^{5}=z^{5}.
$$
Неразрешимость уравнения пятой степени в~целых числах была доказана в~XIX~веке. Потом стали
увеличивать показатели и~доказывать про седьмую, одиннадцатую, тринадцатую степени. Дошли примерно до~сотни. Особо
отличились женщина-математик Софи Жермен, а также Куммер, потративший на теорему Ферма добрую половину своей весьма долгой жизни (1810--1893).

При~решении уравнения Ферма выделяют два разных случая: регулярный и~специальный (нерегулярный).

Регулярный случай: ни одно из чисел $x$, $y$, $z$ не~делится на~$p$. Специальный случай: одна из~переменных делится
на~$p$, а~две другие~--- нет. (Если две переменные делятся на~$p$, то и~третья переменная обязательно
делится на~$p$. Например, если $x$~и~$y$ делятся на~$p$, то в левой части $p$ выносится за скобку, и~$z$ тоже будет делиться на~$p$.
Тогда можно сократить обе части уравнения на максимальную степень числа ~$p$ и~получить какой-нибудь из~двух описанных случаев.)

Софи Жермен далеко продвинулась в~регулярном случае. Она доказала, что уравнение регулярного типа
$x^{p}+y^{p}=z^{p}$ не~имеет решения для~всех таких простых $p$ (нечетных, то есть всех, кроме ${p=2}$), что $2p+1$~--- тоже простое.


Весь XIX~век длилась борьба за~разные простые показатели, и~методология доказательств была типовая.
Выражения раскладывали на~множители типа
$$
x^{3}+y^{3}=(x+y)(x^{2}-xy+y^{2}).
$$
Если вы не~помните эту формулу из~школы, можете ее проверить, раскрыв скобки. Дальше незадача:
$(x^{2}-xy+y^{2})$ на~множители не~раскладывается~--- по~той же причине, по~которой не~раскладывается
$x^{2}+y^{2}$. А~как было бы хорошо разложить его и~в~одну строчку получить решение! Но~это возможно только
с~комплексными числами, а~с~действительными, привычными нам~--- это невозможно. Все дороги, которые
ведут в~настоящую математику~--- идут через комплексные числа. Это сложно, но~интересно и~красиво. К комплексным числам мы вернемся в конце этой лекции.

Сейчас я~хочу доказать математически, что два вышеупомянутых выражения $x^{2}+y^{2}$ и $x^{2}-xy+y^{2}$ не~могут
быть разложены на~множители. Для~этого я~использую сложный, но~наглядный путь через введение
в~алгебраическую геометрию.

Докажем неразложимость $x^{2}+y^{2}$. Допустим, что его можно разложить на множители
(где $\alpha$, $\beta$, $\gamma$ и $\delta$~--- вещественные числа):
$$
x^{2}+y^{2}=
(\alpha x+\beta y)(\gamma x+\delta y).
$$
Рассмотрим, какие множества на~плоскости задают правая и~левая части уравнения:
$$
x^{2}+y^{2}=0\quad
\text{и}\quad
(\alpha x+\beta y)(\gamma x+\delta y)=0.
$$
После работ Декарта мы знаем, что $x$ и~$y$ можно считать координатами на~плоскости. Уравнение
$x^{2}+y^{2}=0$ задает нам только одну точку $(0,0)$. Почему? Потому, что квадраты не~могут быть
отрицательными. Если одна из~переменных положительна, например $x$, то выражение $x^2$ больше нуля, но~квадрат второй
переменной~--- не~меньше нуля, следовательно, сумма будет больше нуля.
 Не~получается. Если сумма
равна нулю, значит $x$ и~$y$ оба равны нулю.

Рассмотрим второе уравнение:
$$
(\alpha x+\beta y)(\gamma x+\delta y)=0.
$$
Оно задает нам две прямые. Иногда они могут совпадать.

%Рис. 1
\xPICi{g1}{Для~левого уравнения получается всего одна точка, для~правого~--- или две прямых, или одна.}

Получается, что с~одной стороны у~нас две прямые (в~случае их совпадения~--- одна), а с~другой
стороны~--- точка (см. рис.~\ref{f:g1}).

Если бы $x^{2}+y^{2}$ раскладывалось на~множители, то второе уравнение должно было бы определять
то же множество на плоскости, что и первое.
Но~так как эти множества не~совпадают, то сумму квадратов нельзя разложить на~множители.

Вот вам пример методов классической алгебраической геометрии. Если я~захочу изучать уравнение от трех переменных $x$,
$y$ и~$z$, то получится уже трехмерное пространство. А~если у~меня 26~переменных? Нам понадобится
26-мерное пространство. Нужно иметь воображение и~жить в~многомерном пространстве. Представьте, что вы
выходите на~улицу и~переходите дорогу на~красный свет. Вас может сбить машина, но~стоит вам перейти
в~четырехмерное пространство, и~вам станут безразличны все светофоры, так как машины будут
проезжать сквозь вас, и~даже не~будут замечать этого. А~ведь вы сделали только один шаг по~четвертой
оси координат!

Немного сложнее доказать, что не~раскладывается на множители $x^{2}-xy+y^{2}$. Допустим, что
\begin{gather*}
x^{2}-xy+y^{2}=(\alpha x+\beta y)(\gamma x+\delta y).\\
\text{Посмотрим на множество }x^{2}-xy+y^{2}=0.
\end{gather*}
Умножим всё на~4, затем преобразуем:
\begin{gather*}
4x^{2}-4xy+4y^{2}=0,\\
4x^{2}-4xy+y^{2}+3y^{2}=0.
\end{gather*}
Свернем $4x^{2}-4xy+y^{2}=(2x-y)^{2}$ по формуле Бинома Ньютона.

Получим $(2x-y)^{2}+3y^{2}=0.$

Если сумма квадратов равна 0, значит, каждый из~них равен 0. Значит, во-первых, $3y^{2} = 0$, то
есть $y=0$. А во-вторых, $(2x-y^{2})=0$, то есть $2x-y=0$, откуда в силу  $y>0$ имеем $x>0$.  То
есть это уравнение задает точку $(0;0)$. Но~$(\alpha x+\beta y)(\gamma x+\delta y)$ по-прежнему
задает две прямые (в~крайнем случае, одну). Множества опять не~совпадают. Значит, разложить
$x^{2}-xy+y^{2}$ на~множители нельзя.

Зачем мы это делаем? Я~снова сделаю переход от~истории к~математике.

Вернемся к~$x^{2}+y^{2}=z^{2}$. Рассмотрим несколько способов решения этой задачи.

Первый способ решения называют <<формулой индусов>>, т.\,к. полагают, что еще древние индусы знали
это решение.

Давайте посмотрим, какие бывают варианты для~четности или нечетности $x$, $y$ и~$z$? Если число четное,
оно имеет вид $2k$, тогда его квадрат имеет вид $(2k)^{2}=4k^{2}$ и он~делится нацело на~4. (В~некоторых
книгах факт делимости изображается так: $4k^{2}\mathbin{\vdots}4$.)

Если число нечетное, то его можно представить в виде выражения $2k+1$ для некоторого целого $k$, и тогда
$$
(2k+1)^{2}=4k^{2}+4k+1=4(k^{2}+k)+1.
$$
$4(k^{2}+k)+1$~--- не~просто нечетное число. Это число, которое при~делении на~4 имеет остаток~1.

Какие бывают остатки при~делении на~4? 1 и~3 у~нечетных чисел и~0 и~2 у~четных. Так вот, выведенные формулы показывают, что
у~квадратов всегда остатки либо 0, либо 1.
 Например,
$$
0^{2}=0,\quad
1^{2}=1,\quad
2^{2}=4,
$$
то есть ноль при делении на 4, далее~--- 9, 16, 25, 36, 49 (с чередованием остатков 1 и 0 при делении на 4).

Тут есть еще один более глубокий <<фокус-покус>>:
$$
(2k+1)^{2} =
4(k^{2}+k)+1=
8\bfrac{k^{2}+1}{2}+1=
8\bfrac{k(k+1)}{2}+1,
$$
где $\bfrac{k(k+1)}{2}$~--- всегда целое число. В~числителе стоят два подряд идущих числа. Оно из~них
всегда четное, значит, это выражение делится на~2.



Получается замечательная вещь. Квадрат любого нечетного числа дает остаток 1 при~делении на~8.
Это~--- очень важный факт. Но~в~нашем случае важен остаток при делении на~4.

Вернемся к~нашему уравнению
\begin{equation} %%% 2-3-1
\label{2-3-1}
x^{2}+y^{2}=z^{2}
\end{equation}
(так как это~--- формулировка теоремы Пифагора, то такие прямоугольные треугольники со сторонами
$x, y, z$, где $x, y, z$~--- целые числа, называются <<пифагоровыми>>).


Прежде всего сократим все на~2.

Делим на~2 все три числа, пока они синхронно будут делиться. Затем, заодно, разделим все три
числа на все их прочие общие простые множители. Так мы опишем не~все треугольники, а~только
качественно разные. Поясним сказанное, воспользовавшись понятием подобия треугольников.

Если два треугольника подобны, то тройки их сторон пропорциональны друг другу. Интересно
в~каждом семействе подобных друг другу пифагоровых треугольников найти самый маленький
треугольник с целыми сторонами. Потом мы сможем умножить найденное решение $(x,y,z)$
на~любое целое положительное число.
 Треугольник увеличится, но~останется пифагоровым.

У~этого самого маленького треугольника не~будет делимости ни на~одно простое число у~всех трех
сторон одновременно. Но~и~длины двух сторон не~могут делиться, например, на~2, иначе длина третьей
стороны тоже будет обязана делиться на~2, так как выполняется равенство~\eqref{2-3-1}. Если делятся
слагаемые, то делится и~сумма, значит, можно сократить все три числа.


То есть у~минимальных троечек из~этих трех чисел на~2 может делиться только одно.
Аналогично и на любое другое простое число может делиться длина не более одной из трех сторон.

Оказывается, что не~подходит тот вариант, когда $x$, $y$, $z$~--- все нечетные числа.
В самом деле, предположим, что все числа нечетные. $x^{2}$~--- нечетное, $y^{2}$~---
нечетное. Следовательно, $z$~--- четное (так как сумма нечетных чисел всегда четна).
Значит, все-таки одно (и только одно) из~$x$, $y$, $z$ должно делиться на~2.


А~могут $x$ и~$y$ быть нечетными? Нет, потому что у~квадратов при~делении на~4 будет остаток 1, а~их
сумма даст остаток 2, но~$z$~--- четное, поэтому его квадрат при~делении на~4 должен дать в~остатке 0.
Значит, в~любой пифагоровой тройке после ее максимального сокращения число $z$~ будет нечетным. Для~примера
возьмем тройку $(30, 40, 50)$. Она сводится к~тройке $(3, 4, 5)$, где 5~--- нечетное число.

\pagebreak

Значит, одно число из~$x$ и~$y$ должно быть четным, другое~--- нечетным. Можно считать, что $x$~--- четное.

А~теперь начинается ключевой момент доказательства, не~очень сложный, но~крайне важный, так как он
работает при~решении многих диофантовых уравнений.

Раз $x$~--- четное число, то $x=2k$ при целом $k$. В~этом случае уравнение будет иметь вид $4k^{2}+y^{2}=z^{2}$.

Перекинем $y^{2}$ направо: $4k^{2}=z^{2}-y^{2}$, то есть
$$
k^{2}=\bfrac{z-y}{2}\bfrac{z+y}{2}.
$$
Так как~$z$ и~$y$~--- нечетные числа, то их разность и~сумма~--- четные числа.
Поэтому $\bfrac{z-y}{2}$ и~$\bfrac{z+y}{2}$~--- целые числа.

Получилось, что $k^{2}$ равно произведению некоторых двух целых чисел.

А~теперь смотрите, мы договорились, что достаточно искать такие тройки, в~которых ни у какой пары чисел нет общих делителей. Поэтому
$y$ и~$z$ не~имеют общих множителей, $y=p_{1}p_{2}p_{3}\ldots p_{a}$, $z=q_{1}q_{2}q_{3}\ldots q_{b}$, и эти наборы простых чисел разные.
Как говорят математики, в этом случае $y$ и~$z$ взаимно просты. Сами они при~этом совершенно не~обязательно простые. Например, $15=3\cdot5$ и~$22=2\cdot11$,
следовательно, 15 и~22~--- взаимно простые числа, хотя ни одно из них не является простым.


Теперь я~утверждаю, что
$\bfrac{z-y}{2}$ и~$\bfrac{z+y}{2}$ также взаимно простые, то есть не~имеют ни одного общего множителя.
Почему? Предположим, что у~них есть общий делитель. Например, они делятся на~3. Тогда, их сумма и~разность тоже делятся на~3. Но
\begin{gather*}
\bfrac{z-y}{2}+\bfrac{z+y}{2}=z,\\
\bfrac{z-y}{2}-\bfrac{z+y}{2}=-y.
\end{gather*}
Получается $y\mathbin{\vdots}3$, $z\mathbin{\vdots}3$, то есть $y$, $z$ \textit{ оба делятся} на~3.
Мы пришли к~противоречию. Значит,
$\bfrac{z-y}{2}$ и~$\bfrac{z+y}{2}$ тоже не имеют общих делителей.


Вернемся к~нашему выражению
$$
k^{2}=\bfrac{z-y}{2}\bfrac{z+y}{2}.
$$

Числа справа состоят из~разных простых делителей. В~каждое из~чисел простые множители могут
входить хоть поодиночке, хоть в~степенях, но~пересечений между разложениями $\bfrac{z-y}{2}$ и~$\bfrac{z+y}{2}$ нет.


Например,
\begin{gather*}
\bfrac{z-y}{2}=p_{1}^{2}p_{2}^{3}p_{3}\ldots p_{k},\\
\bfrac{z+y}{2}=w_{1}w_{2}w_{3}^{5}\ldots w_{m}^{7}.
\end{gather*}
(Вместо 5 и 7 здесь могут быть любые степени.)

С~другой стороны, $k^{2}=q_{1}^{2}q_{2}^{2}\ldots q_{f}^{2}$, поэтому
$$
q_{1}^{2}q_{2}^{2}\ldots q_{f}^{2}=
p_{1}^{2}p_{2}^{3}p_{3}\ldots p_{k} w_{1}w_{2}w_{3}^{5}\ldots w_{m}^{7}.
$$
Согласно~основной теореме арифметики, существует \textbf{единственное} разложение натурального числа на~простые
множители с~точностью до~порядка сомножителей. Значит, по~обе стороны от~знака равенства стоят
наборы одинаковых простых чисел. В частности, $q_{1}^{2}$ равен произведению двух чисел из~правой части.

Так как пересечений простых множителей в наборах $p_1,\dots,p_k$ и $w_1,\dots,w_m$ нет,
то этот квадрат целиком <<сидит>> в одном из наборов. Но то же самое можно сказать и про все прочие
квадраты! Поэтому все простые числа набора $p_i$ входят в разложение числа $\bfrac{z-y}{2}$ в четных
степенях, и то же самое верно для набора $w_j$. Следовательно, числа $\bfrac{z-y}{2}$ и
$\bfrac{z+y}{2}$ \textbf{являются квадратами}\footnote{Этот переход является психологически
сложным. Подумайте над ним самостоятельно: квадрат~--- это такое число, в разложении которого на
простые множители все простые числа входят в четных степенях.}.

Это очень сильное утверждение (потому что квадратов очень мало среди натуральных чисел). $1, 4, 16, 25, 36, 49 \ldots$~---
они встречаются все реже.

Введем новые обозначения. Так как наши выражения~--- квадраты, то обозначим:
\begin{gather*}
\bfrac{z-y}{2}=n^{2},\\
\bfrac{z+y}{2}=m^{2}.
\end{gather*}
Тогда
\begin{gather*}
k^{2}=\bfrac{z-y}{2}\bfrac{z+y}{2}=n^{2}m^{2},\\
z=\bfrac{z-y}{2}+\bfrac{z+y}{2}=m^{2}+n^{2},\\
y=\bfrac{z+y}{2}-\bfrac{z-y}{2}=m^{2}-n^{2}.
\end{gather*}

Вспомним, чему равен~$x$: $x=2k$.

Видно, что $k=mn$, следовательно, $x=2mn$.

Итак, мы доказали, что если $x$, $y$, $z$ являются целыми сторонами прямоугольного треугольника (минимального в серии подобных пифагоровых треугольников), то
существует пара целых чисел $n$ и~$m$ с~таким свойством, что $x$ равен удвоенному произведению этих
чисел, $y$~--- разности квадратов этих чисел, а~$z$~--- сумме квадратов этих чисел. Это~---
обязательное условие:
\begin{gather*}
x=2mn,\\
y=m^{2}-n^{2},\\
z=m^{2}+n^{2}.
\end{gather*}
Остается вопрос: можно ли брать $m$ и~$n$ произвольным образом?

Во-первых, чтобы <<$y$>> было положительным числом, нужно чтобы выполнялось неравенство $m>n$ (<<$y$>>~--- сторона
треугольника, она не~может быть отрицательным числом). Во-вторых, $m$ и~$n$ \textbf{должны быть взаимно простыми числами разной
четности}, чтобы $x$, $y$, $z$ получились взаимно простыми.

Давайте проверим, останется ли верна наша формула для целых решений уравнения $x^{2}+y^{2}=z^{2}$ при~любых целых $m$, $n$:
\begin{multline*}
x^{2}+y^{2} =
4m^{2}n^{2}+m^{4}-2m^{2}n^{2}+n^{4}=
m^{4}+2m^{2}n^{2}+n^{4}=
\\=
(m^{2}+n^{2})^{2}=
z^{2}.
\end{multline*}

Мы видим, что наша формула всегда дает <<пифагоровы>> тройки, но не обязательно положительные и взаимно простые.

Общая формула содержит два произвольных параметра.
Для наглядности построим сетку (рис.~\ref{f:g2}).

%$-2$\quad
%$-1$\quad
%$0$\quad
%$1$\quad
%$2$\quad
%$3$\quad
%$m$\quad
%$n$\quad

%Рис. 2
\xPICi{g2}{Здесь спрятались все пифагоровы тройки!}

В~сетке~--- выберем точку с~координатами $(0;0)$ и~оси: $m$~--- вправо, $n$~--- вверх.
Будем брать точки с~координатами $(m;n)$ и~подставлять их в~нашу формулу. Например, возьмем точку $(2;1)$.
\begin{gather*}
x=2mn=2\cdot2\cdot1=4,\\
y=m^{2}-n^{2}=2^{2}-1^{2}=3,\\
z=m^{2}+n^{2} =2^{2}+1^{2}=5.
\end{gather*}
Давайте возьмем что-нибудь более сложное. Напомню, что для получения минимальных пифагоровых троек нам подходят только $m>n>0$ с~разной четностью.

Возьмем, например, $(5;2)$. Получим $x=20$, $y=21$, $z=29$.

При~подстановке мы увидим, что у~нас появляются разные виды треугольников. Узкие вытянутые
треугольники, у~которых катет и~гипотенуза отличаются на~единицу: 12, 5, 13. Треугольники, у~которых
катеты почти равны друг другу: 20, 21, 29 (рис.~\ref{f:g3}).

%Рис. 3
\xPICi{g3}{Такие разные пифагоровы треугольники.}

В~каждой целочисленной точке плоскости будет возникать вариант
пифагорова треугольника. Возьмем точку $(10;3)$ и~посмотрим, какой треугольник получится:
$$
x=60,\quad
y=91,\quad
z=109.
$$
Задача решена методом Диофанта. Мы получили описание всех пифагоровых треугольников.

\textbf{Второе решение задачи о~пифагоровых треугольниках.}

Алгебраическая геометрия~--- часть~2.

Есть уравнение, которое нужно решить в~целых числах, понимая, что по абсолютной величине $c$ больше $a$ и~$c$ больше~$b$:
$$
a^{2}+b^{2}=c^{2}.
$$
Разделим это выражение на~$c^{2}$ и введем новые обозначения $x=\bfrac ac$, $y=\bfrac bc$:
$$
\left(\bfrac ac\right)^{2}+\left(\bfrac bc\right)^{2}=1.
$$
Обе скобки~--- числа рациональные, т.\,е. дроби.

Какая фигура на~плоскости описывается уравнением: $x^{2}+y^{2}\brokenrel{=}1$? Окружность единичного радиуса.

А~теперь~--- чудо. Задача, которую мы решаем~--- найти на~этой окружности все рациональные
точки (т.\,е. точки, у которых обе координаты являются дробями). Вот как звучит наша задача при втором подходе к~решению!


Какую точку на~окружности даст нам треугольник 3, 4, 5? Точку $(3/5; 4/5)$. Стороны 20, 21, 29
породят точку $(20/29; 21/29)$. Для~любой точки, которая попадает на~окружность, сумма квадратов
координат должна быть равна единице. Но~не~любая из~этих точек \textit{рациональна}.

Нужно найти все такие точки. Возьмем одну очевидную рациональную точку с~координатами $(0, -1)$.

\textbf{Слушатель:} А почему не $(0;1)$ или какую-то другую?

\textbf{А.С:} В принципе, можно выбрать какую угодно точку окружности. Я выбрал такую точку, при которой формулы будут выглядеть проще всего.

Давайте предположим, что есть еще одна рациональная точка $(x,y)$. Тогда прямая, которая проходит
через эти две точки, имеет уравнение с~рациональными коэффициентами
(см. рис.~\ref{f:g4}). Докажем это.

%$$
%a^2+b^2=c^2
%$$
%$$
%$$\left(\bfrac ac\right)^{2}+ \left(\bfrac bc\right)^{2}= 1$$
%$$
%$$
%x^2+y^2=1
%$$
%
%

%Рис. 4
\xPICi{g4}{Прямая, проходящая через точку $(0,-1)$ и еще одну рациональную точку,
обладает рациональным коэффициентом наклона.}

Давайте посмотрим, как выглядит уравнение прямой, проходящей через точку $(0,-1)$ в общем случае. Вспомним, что
$y=kx+b$~--- уравнение прямой <<с~угловым коэффициентом и~свободным членом>>.


Если она проходит через точку $(0,-1)$, то при подстановке $x=0$, $y=-1$ в~наше уравнение мы
должны получить верное равенство. Подставим: $-1=0k+b$, откуда $b=-1$, то есть наше уравнение имеет вид $y=kx-1$.


Мы получили общий вид прямой, проходящий через точку\linebreak $(0; -1)$. При~разных $k$ мы будем получать прямые
с~разным наклоном (рис.~\ref{f:g5}).

%Рис. 5
\xPICi{g5}{Обратите внимание на~вспомогательный прямоугольный треугольник справа
с~вершиной в~точке $(0, -1)$.}

Если точка $(x,y)$ рациональна, то $k$~--- тоже рациональное число ($k$~--- это тангенс угла
наклона прямой, в нашем случае он равен отношению противолежащего катета к прилежащему в полученном
треугольнике). См. рис.~\ref{f:g5}, катеты вспомогательного треугольника.

Вертикальный катет равен $y+1$. Горизонтальный равен~$x$. Для точки $\left(\bfrac{3}{5}, \bfrac{4}{5}\right)$ эти числа равны 9/5 и~3/5.
Получается отношение катетов $k=9/5:3/5=3$. Наша прямая имеет вид: $y=3x-1$.

Итак, на~этом примере продемонстрировано, что если какая-то точка имеет рациональные координаты, то угол
наклона прямой, проходящей через нее и через точку $(0, -1)$, будет рациональным числом. Это следует из того, что оба
катета выражаются в этом случае рациональными числами,
а~отношение двух рациональных чисел является рациональным числом. Говорят, что рациональные числа
<<образуют поле>>, так как сумма, разность, произведение и~частное дробей являются дробью.

Итак, если точка рациональная, то и наклон прямой, проходящей через нее и через точку $(0; -1)$ будет
рациональным числом. Теперь мы докажем и обратное: если в~формулу $y=kx-1$ вместо $k$ подставить
любое рациональное число, то мы всегда получим в~пересечении с~окружностью две точки: $(0; -1)$
и~какую-то другую рациональную точку.



Как найти точку пересечения прямой $y=kx-1$ с~окружностью $x^{2}+y^{2}=1$?

Нужно решить систему уравнений
$$
\begin{cases}
y=kx-1;\\
x^{2}+y^{2}=1.
\end{cases}
$$

Подставим значение $y$ из~первого уравнения во~второе
$$
x^{2}+(kx-1)^{2}=1
$$
 и раскрываем скобки
$$
x^{2}+k^{2}x^{2}-2kx+1=1.
$$
Упрощаем:
$$
x^{2}+k^{2}x^{2}=2kx.
$$
Можно сократить на~$x$, так как случай $x=0$ нам не~интересен~--- он даст уже знакомую точку $(0, -1)$:
$$
x+k^{2}x=2k.
$$
Выразим теперь $x$ и~$y$ через $k$:
$$
x(1+k^{2})=2k;\quad
x=2k/(1+k^{2}),
$$
$$
y=\bfrac{2k^{2}}{1+k^{2}}-1=\bfrac{k^{2}-1}{k^{2}+1}.
$$
Из~этих формул видно, что если $k$~--- рациональное число, то $y$ и~$x$~--- тоже рациональные.
Рациональные числа~--- это числа, с~которыми можно производить действия арифметической природы~--- плюс, минус, разделить, умножить.
Рациональные числа от~этого остаются рациональными (то есть эти действия не~выводят нас за~пределы множества рациональных чисел).

Что значит <<$k$~--- рациональное число>>? Это значит, что $k=\bfrac mn$. Подставим вместо $k$ дробь $\bfrac mn$,
считая, что $m$, $n$~--- положительны, причем $m>n$, а~дробь $\bfrac mn$ несократима:
$$
x=\frac{2\bfrac mn}{\bfrac{m^{2}}{n^{2}}+1}=
\bfrac{2mn}{m^{2}+n^{2}},\quad
y=\bfrac{\bfrac{m^{2}}{n^{2}}-1}{\bfrac{m^{2}}{n^{2}}+1}=
\bfrac{m^{2}-n^{2}}{m^{2}+n^{2}}.
$$
Осталось вспомнить, что в~исходном уравнении $x=a/c$ и~$y=b/c$. Поэтому можно взять в качестве $a$ числитель первой дроби, в качестве $b$~--- числитель второй дроби и в качестве $c$~--- их общий знаменатель. Получится: $a=2mn$, $b=m^{2}-n^{2}$, $c=m^{2}+n^{2}$. Одно
из решений получается сразу, а прочие ему пропорциональны.
Мы имеем тот же ответ, что и~при~первом способе решения.
 Внешне два метода, которыми мы решали эту задачу,
совершенно не~связаны друг с~другом. Координаты и~окружность нам показывают, какие множества
высекают на~плоскости те или иные алгебраические уравнения. А~в~первом способе была делимость
и~основная теорема арифметики. Она, являясь исключительно
арифметическим приемом, не~имеет никакого отношения к~геометрии. Стоит сказать, что если бы математики приходили к~разным результатам, решая
одну и~ту же задачу разными методами, то математика не~была бы наукой. На деле же математика~--- это одно
большое знание, связывающее разные методы между собой одним и тем же ответом.

\pagebreak

Как видим, пифагоровы тройки нами разбиты <<в пух и~прах>>, но~есть одна незадача. При~$k=0$
получается прямая, параллельная оси~$x$ (см. рис.~\ref{f:g6}).

%Рис. 6
\xPICi{g6}{Совпадение двух точек пересечения.}

И~вторая точка пересечения оказывается равной первой. Это как раз и отражает эффект касания.
Алгебраические геометры, когда говорят о~касании, всегда имеют в~виду кратный корень, то есть
корень, в~котором совпали вместе несколько бывших некратных решений.

Есть еще один любопытный момент. Есть еще одна рациональная точка, которую мы не~заметили
на~окружности. Точка $(0,1)$. Это решение появится у~нас при~$k=\infty$.

Если мы хотим \textit{параметризовать} окружность с~помощью рациональных чисел, нужно, чтобы каждому
рациональному числу соответствовала одна, и~только одна точка на~окружности. У~нас же получается
так, что на~окружности есть лишняя точка, которая ни одному рациональному $k$ не~соответствует.
В~таком случае математики рассматривают не~обычную прямую, а~\textit{проективную}. Мы уже сталкивались
с~проективной геометрией. В~задаче на~построение с~помощью линейки у~нас точка пересечения пучка прямых уходила в~бесконечность.

Таким образом, методы алгебраической геометрии часто связаны с~проективной геометрией.

А~теперь \textbf{третий метод} решения той же задачи~--- комплексные числа.
Мы разберем его на~следующей лекции, а~сейчас~--- обещанное введение в арифметику комплексных чисел.

Очень хочется разложить на~множители $x^{2}+y^{2}$. Мы умеем раскладывать разность квадратов. Попробуем
представить нашу сумму в~виде разности:
$$
x^{2}+y^{2} = x^{2}-(-y^{2}).
$$
Если бы я~мог извлечь корень из~$-y^{2}$, то смог бы разложить это выражение следующим путем:
$$
x^{2}+y^{2} =
x^{2}-(-y^{2})=
x^{2}-(-1y^{2})=
(x-\sqrt{-1}y)(x+\sqrt{-1}y).
$$
В~обычной жизни корень из~$-1$ не~извлекается, но~с~помощью комплексных чисел это возможно. Пока
мы исходим из~желания получить комплексное число наиболее естественным образом. Мы хотим разложить
сумму квадратов на~множители. Давайте считать, что есть такое число $\sqrt{-1}$, обозначим его за~$i$. $\sqrt{-1}=i$.
Тогда
$$
x^{2}+y^{2} = x^{2}-(-y^{2})=x^{2}-i^{2}y^{2}=(x-yi)(x+yi).
$$
Это критически важно для~многих задач. Например, для~задачи о том, какие простые числа раскладываются
\textit{в~сумму двух квадратов.} Число 41~--- простое. Оно является суммой двух квадратов: $25 + 16$;
$41=5^{2}+4^{2}$. Если мы умеем раскладывать такую сумму на~множители, то у~нас получатся любопытные вещи:
$41=(5+4i)(5-4i)$. Мы попадем в~знакомую ситуацию, связанную с разложением числа 41 на множители, только
теперь эти множители~--- числа новой природы.

%$i$
%
%$0$
%
%$-1$
%
%$1$

Число $i$~--- не~является вещественным (то есть не лежит на обычной числовой прямой и не может
использоваться для измерения физических величин) и, если мы нарисуем вещественную ось, оно будет
находиться где-то вне нашей оси. Мы можем выбрать сами, где его поместить. Удобнее всего поместить
$i$ на~вертикальной оси, выбрав некоторую плоскость, содержащую обычную вещественную ось (см.
рис.~\ref{f:g7}).


%Рис. 7
\xPICi{g7}{Вот где притаилось загадочное число $i$.}

Тогда получится, что любое число $x+yi$ <<живет на~плоскости>> в~точке с~координатами $(x, y)$.
Если мы хотим ввести в~рассмотрение некоторую новую сущность, которая в~квадрате дает минус
единицу, то нам нужно уметь это число умножать на~любые действительные числа. И~такие произведения
$yi = z$ никогда не~могут быть обычными числами, иначе само $i = z/y$ превращалось бы в~обычное
число. А~мы уже убедились в том, что $i$~ имеет <<невещественную>> природу. Кроме того, мы должны
уметь выполнять действия сложения и~вычитания между обычными (вещественными или действительными)
числами и~числом~$i$.


Давайте посмотрим. Беру вещественные числа и~составляю выражения:
$$
(x+yi); (z+ti).
$$
Вопрос: в~каком случае эти два выражения задают одно и то же число?
Попробуем действовать по~привычным правилам.
\begin{gather*}
x+yi=z+ti,\\
x-z=ti-yi,\\
x-z=i(t-y).
\end{gather*}
Если $t=y$, то из последнего равенства имеем $x=z$.

Если $t\ne y$, то получаем выражение
$$
i=\bfrac{x-z}{t-y}.
$$
Этого \textit{не~может быть}, так как $\bfrac{x-z}{t-y}$~--- вещественное число. А~число $i$~--- НЕ вещественное. Противоречие.
\textbf{Значит, $x+yi$ и~$z+ti$ равны тогда и~только тогда, когда $x=z$ и~$t=y$ одновременно.}

Из~этого следует, что каждой точке плоскости соответствует единственное комплексное число.

Продолжение в~следующей лекции (то есть в~лекции~4 части~2).

\endinput
