%1.3.
\section{Лекция 3}
\label{1.3}

\textbf{А.С.:} В~прошлый раз я~успел поговорить про бесконечность. Умение работать
с~бесконечностью, умение через бесконечность перешагивать, умение различать разные
бесконечности~--- это основная работа в~математике. Как могут быть разные бесконечности? Кажется,
что либо что-то конечное, либо бесконечное. Но~нет. На~самом деле бесконечности бывают разные.
На~прошлой лекции мы говорили про сходящиеся и~расходящиеся ряды. То есть рассматривали суммы
с~бесконечным количеством слагаемых.

Например, сумма $1 + \bfrac12 + \bfrac13 + \bfrac14 +\ldots$, как выяснилось, стремится к~бесконечности. То есть становится
больше любого наперёд заданного числа. Скажите ей: <<Будь больше 1000>>. Тогда нужно взять много
слагаемых.

Возьмем $2^{2000}$~членов. Оказывается, тогда их сумма будет больше 1000. Скажете: <<Будь больше 1000000>>.
Тогда нужно взять $2^{2000000}$~членов. Их сумма будет больше миллиона. И~так далее.

А~вот этот ряд: $1+\bfrac12+\bfrac14+\bfrac18+\ldots$ тоже состоит из~бесконечного числа членов,
но~его сумма никогда не~станет больше, чем двойка.

Теперь еще одна интересная задача. Начнем издалека. В~2000 году, где-то зимой, мы были в~лесу
в~районе станции Радищево, праздновали чей-то день рождения.
 Вокруг было очень мало сухих деревьев, все
спилили до~нас. Было только огромное, совершенно сухое дерево. И~это дерево огромного размера
стояло и~очень нас заманивало. У~нас была двуручная пила, и~мы начали пилить. Пилили-пилили,
пилили-пилили и~допилили. Дерево сделало <<тцук\ldots>> и~село на~нашу пилу. Пила осталась внутри,
а~полностью спиленное дерево стоит и~падать не~собирается (рис.~\ref{f:b1}).

%Рис. 1
\xPICi{b1}{Пальцем слегка толкнем это дерево\ldots}

Но~стоит подуть ветру, и~оно упадет. В~какую сторону оно упадет~--- совершенно не~предсказуемо. Что
делать? Надо или вставать и~уходить, написав со~всех сторон <<внимание, внимание, до~ближайшего
ветра сюда не~подходить>>, или пытаться уронить дерево. Мы решили с~ним побороться.\vadjust{\pagebreak} Взяли
вспомогательное дерево и~прислонили его к~спиленному где-то на~высоте десяти метров. Навалились,
и~оно поддалось (рис.~\ref{f:b2}).

%Рис. 2
\xPICi{b2}{Дерево с~контрфорсом.}

Было видно, как дерево начало падать. Но~скорость была чудовищно медленная: несколько сантиметров
в~секунду, едва-едва. Где-то минуту мы ждали, пока оно медленно наклонялось, и~только потом оно
начало ускоряться и~через несколько мгновений рухнуло со~страшным грохотом. Пришел я~домой
и~написал уравнение падения дерева. В~физике траекторию движения системы под действием сил можно
выписать в~виде уравнений. Такие уравнения называются \textit{дифференциальными}. Это означает, что скорость
изменения скорости, то есть то, что называется ускорением, зависит от~сил, которые действуют
на~тело.
 Это~--- один из~основных законов физики, он позволяет свести всё, что есть в~обычной,
не~квантовой, механике, к~системам уравнений. Можно выписать такое уравнение и~для~нашего дерева.
И~к~своему удовольствию, исследовав это уравнение, я~пришел к~выводу, что, если дать дереву толчок
очень маленькой силы, оно начинает падать очень, очень, очень медленно.

Я~начинаю рассуждать, что дерево~--- это просто вертикальная палка, без~толщины. Она стоит
совершенно вертикально, но~обладает массой. Массивная вертикальная палка. Кто-то толкает ее сверху.
Ударит человек~--- палка падает (скажем) 1~минуту. Пролетит голубь, заденет~--- будет падать 10~минут.
Начальная скорость верхней точки будет, скажем, 1~мм/с. И~очень долго скорость почти
не~будет меняться. А~если врежется муха, то палка будет падать час. Уравнение выдает удивительный
результат: на~самом деле нет никакой границы на~время падения дерева, вообще никакой.

Рассмотрим похожую задачу. Есть вагончик, в~котором на~шарнире установлена тонкая железная
вертикальная палка. Чуть-чуть вправо или влево~--- она падает, так же, как и~рассмотренное выше
дерево.

%Рис. 3
\xPICi{b3}{Падающее дерево едет в~вагоне\ldots}

Теперь представьте обратную задачу. Вы берете уже упавшую или под некоторым углом висящую палку.
После чего придаете ей некоторый импульс~--- толкаете ее снизу вверх (рис.~\ref{f:b4}).

%Рис. 4
\xPICi{b4}{Слева~--- слабый удар по~стержню (он упадет обратно).
Справа~--- сильный удар (стержень упадет с~другой стороны от~точки прикрепления).}

Какие возможны варианты? Во-первых, толчок может быть слишком слабый. Что произойдет с~палкой?
Поднялась и~упала обратно. Теперь, допустим, подошел какой-нибудь бугай. Бабах по~этой палке. Она
р-раз~--- и~перелетела на~другую сторону. Подходит кто-то немножко более сильный, чем я, но~слабее,
чем бугай. Толкает палку, а~она всё равно падает.

Вы качались на~качелях-перевертышах? Мое детство отчасти проходило в~городе Мценске. И~в~парке там
была такая закрывающаяся изнутри кабинка с~противовесом наверху, которую раскачиваешь,
раскачиваешь, раскачиваешь, и~она <<переворачивается>>: противовес оказывается внизу, а~кабинка
сверху (но~благодаря свободному подвесу кабинка при~этом вверх ногами не~переворачивается).
Я~замечал, что наверху она долго движется с~более-менее постоянной скоростью. Мы знаем, что
с~постоянной скоростью движутся тела, на~которые не~действуют силы. На~кабинку силы, конечно,
действуют, но~вертикально вниз. В~момент, когда кабинка проезжает верхушку, сила перпендикулярна
линии движения, поэтому скорость почти не~меняется. И~если аккуратно выверить движение, то кабинка
практически остановится наверху.

Вернемся к~нашей палке и~нарисуем график. По~горизонтальной оси --- сила удара, по~вертикальной~--- результат (рис.~\ref{f:b5}).

%Рис. 5
\xPICi{b5}{Построение графика ступенчатого вида (<<отклик на~удар>>).}

Если вы ударили слишком слабо, то результат будет~--- палка упадет обратно. Если очень сильно
ударить, то результат~--- палка перевернется на~другую сторону. И~есть ровно одно
вещественное число, одна сила удара, которую нужно придать палке, чтобы она остановилась вертикально.
 Вопрос.
Сколько нужно времени в~идеальном мире, в~котором нет воздуха, трения и~так далее, чтобы палка
заняла вертикальное положение?

\textbf{Слушатель:} Смотря под каким углом было изначально.

\textbf{А.С.:} Нет. От~того, под каким углом была изначально палка, зависит только то, с~какой
силой нам надо толкнуть палку. А~времени понадобится бесконечный промежуток. Строгая математическая
бесконечность. То есть, если палке придать такую силу, которая в~точности достаточна для~того,
чтобы она достигла положения вертикального равновесия, то время, за~которое палка будет достигать
этого положения, равно плюс бесконечности.
 В~условиях задачи, когда мы говорим об идеальной
математике, мы, естественно, не~учитываем, что вокруг меняются обстоятельства. В~идеальной ситуации
время равно бесконечности. Я~посчитал всё это в~2000~году, потом рассказал физикам, а~они сказали,
что это очевидно, и~всё они это давно знали. Наверное, кому-нибудь не~хочется верить, что
потребуется бесконечное количество времени. Я~дам еще одно подтверждение. Давайте вернемся
к~тележке. Пусть это будет вагон, внутри которого находится наша палка на~шарнире. Вагон едет
по~маршруту Москва~--- Петербург. И~мне сообщили, с~точностью до~100\% (так, как у~математиков
бывает, а~в~жизни нет) информацию о~скорости, с~которой вагон будет двигаться (рис.~\ref{f:b6}).

%Рис. 6
\xPICi{b6}{График изменения скорости поезда Москва~--- Петербург:
скорость, с~которой поезд выезжает, сначала слегка увеличивается, потом достигает максимума и~далее
обращается в~нуль (остановка поезда в~Бологом). После этого скорость снова растет, чтобы не~было
опоздания в~пункт назначения.}

Утверждение. Существует такое положение палки, такой угол, в~котором я~могу ее выпустить из~рук
в~начальный момент времени, что она не~упадет в~течение всей дороги. Существует такой угол альфа,
что, если я~придам железке на~шарнире этот угол, то она всю дорогу будет болтаться туда-сюда,
но~никогда не~упадет. Обоснование этого факта изложено ниже.

Эта задача разобрана в~книге, которую я~всем рекомендую: Р.~Курант, Г.~Роббинс, <<Что такое
математика?>>.
Еще раз отмечу, что мне в этой задаче известен точный график движения поезда.


Ключ к решению этой задачи~--- в использовании идеи \textit{непрерывности}. Мы один раз уже с~ней
столкнулись в~предыдущей задаче: есть такой импульс, получив который, палка не~упадет ни направо,
ни налево. Она встанет вертикально, но~через бесконечное время. Задача про вагон, в~котором
движется железный стержень на~шарнире, напрямую относится к~предыдущей. Давайте посмотрим. Если
палка уже лежит, то она будет всегда лежать. Она никогда никуда не~встанет. А~теперь рассмотрим
для~каждого начального угла поворота этой палки, в~какое положение она в~конечном счете ляжет:
направо или налево. А~если она останется висеть, значит, мы нашли то, что нам нужно.

Если палка ляжет, то она ляжет в~одно из~этих двух положений. Причем уравнения движения таковы, что
если чуть-чуть поменять угол, совсем чуть-чуть, сторона падения не~изменится. Если палка падала
направо, то чуть-чуть изменив угол, вы не~измените результата. \textit{Она всё равно упадет направо.} Тем
самым, если в~одном положении она падала, скажем, направо, значит, и~в~близких начальных положениях она тоже должна падать направо.
 То есть, как говорят математики, множество положений, в~которых она упадет
направо~--- \textit{открытое множество}. <<Открытое>>~--- значит, вместе с~какой-то точкой содержит все
близкие к ней точки.
 Если из~какого-то положения палка падает, то из~всех достаточно близких положений
она упадет в~ту же сторону.

Интуитивно понятно, что мы можем всегда приподнять палку настолько мало, что она непременно упадет
обратно. Давайте медленно изменять
положение палки. В~каких-то положениях она будет падать направо, а~в~каких-то~--- налево. Значит, где-то
есть переход, угол, такой, что всюду справа она падает направо, всюду слева~--- налево\footnote{Можно доказать, что такое граничное положение~--- единственное.}.
 Что же это
за~угол? Единственный факт, который мы можем сообщить про этот угол, это что для~такого угла палка
не~упадет вообще. Ничего другого про него не~известно. Парадоксально, но~это факт! Если вы в~это
поверили (а~я~вас не~обманываю), тогда в~том, что в~близком к~вертикальному положению палка может
находиться сколь угодно долго, вас убедит следующее соображение. На~стоянке в~Бологом поезд может
стоять 10 минут, а~может~--- час. И~в~течение этого часа палка не~упала. Она ведь не~падала всю
дорогу, в~частности, она не~упала и~в~течение стоянки. Что же она делала в~это время?

\textbf{Слушатель:} Двигалась.

\textbf{А.С.:} Она находилась очень близко к~вертикальному положению. Потому что, если бы она чуть-чуть
от~него отклонилась, она рухнула бы. Поэтому во~время стоянки она была очень близко к~вертикальному
положению. А~так как стоянка может быть сколь угодно долгой, из~этого следует, что палка в~районе
вертикального положения может находиться сколь угодно долго. Поэтому-то она будет подниматься
в~него бесконечное время.

Это~--- наше второе знакомство с~бесконечностью. Сейчас будет третье.

\textbf{Слушатель:} Гвоздь программы.

\textbf{А.С.:} Бесконечность~--- это гвоздь программы, безусловно. Потому что бесконечность~--- это
центральное понятие в~математике. Математика~--- это шаг через бесконечность. Освоение
математики~--- это когда вы становитесь с~бесконечностью <<на ты>>. И~чем больше вы <<на ты>>
с~бесконечностью, тем лучше вы понимаете математику. Это~--- наука о~бесконечности. В~этом смысле
математика и~религия дополняют друг друга. Религия~--- это знание о~бесконечности, математика~---
это наука о~бесконечности. Это две ипостаси бытия.

Сейчас мы поговорим о~бесконечности в~некотором другом разрезе, геометрическом.

Помните ли вы, что такое квадратный корень? Корень квадратный из~100~--- это 10. Потому что
$10\times 10=100$. А~вот что такое корень квадратный из~двух~--- это не~так понятно. А~что такое
рациональное число? Если вы не~знаете, не~страшно. Но~что такое целое число, знают все. Целые
числа~--- это ноль, один, два, три, четыре, пять, шесть и~так далее в~положительную сторону, но
также минус один, минус два, минус три, минус четыре и~так далее~--- в~отрицательную. У~древних
была большая проблема с~отрицательными числами. Число, бесконечность, уравнение~--- это всё то,
с~чем математики всё время имеют дело. Что такое число? Для~древних число~--- это то, чем мы считаем предметы.
Более того, до сих пор натуральными числами часто называют числа, используемые для подсчета предметов.
Ноль~--- это для~древних уже было что-то странное. Число или не~число? Натуральное ли оно? Ноль~--- это отсутствие предметов.
 Отсутствие~--- это количество или нет? Сколько крокодилов в~нашей
комнате?

\textbf{Слушатель:} Ноль.

\textbf{А.С.:} Значит, вы считаете, что ноль все-таки натуральное число\footnote{В~России
натуральные числа по традиции начинаются с единицы. То есть ноль является целым числом, но не
является натуральным.}. А~отрицательных чисел у~древних греков не~было. Вот если бы математика
началась в~России, то проблем с~этим не~было бы. Потому что $-1$, $-10$~--- это мороз, снег идет. Всё
понятно~--- на~улице отрицательная температура.

Когда я~учился в~школе, к~нам как-то приехали американцы. И~они сказали, что уровни умственного
развития школьников в~России и~в~Америке различаются как небо и~земля\footnote{Ради справедливости надо заметить, что эта беседа проходила в школе \No~57.}.
 На~что я~заметил, что всё
очень просто. В~Америке редко бывают отрицательные температуры, и~поэтому у~школьников есть
проблема с~постижением отрицательных чисел. Американец сильно задумался (тем более, что у~нас~---
градусы Цельсия, а~у~них~--- Фаренгейта!).

Действительно, у~нас и~трехлетние дети знают, что такое $-5$ и~$-3$. Это когда снег, и~мама на~голову
шапку надевает.

%Рис. 7
\xPICi{b7}{Наружный термометр~--- инструмент для~освоения отрицательных чисел.}

Это вот ваш градусник (рис.~\ref{f:b7}).

Нулевая температура. А~может, 1, 2, 3, $-1$, $-2$, $-3$ \ldots\ градусов. Но~между ними тоже что-то есть.

\textbf{Слушатель:} Да.

\textbf{А.С.:} Я~же могу сказать, что сейчас два с~половиной градуса выше нуля?

\textbf{Слушатель:} Да.

\textbf{А.С.:} Или три и~три четверти градуса.

\textbf{Слушатель:} Да.

\textbf{А.С.:} То есть я~могу назвать доли. Их сейчас даже в~детсаду рассматривают.

Пришло ко мне на~день рождения 15~детей. И, допустим, у~меня есть 23~яблока. Я~взял нож и~аккуратно
разрезал яблоки на~15 равных частей каждое. Каждому ребенку достанется $\bfrac{23}{15}$~яблока, то есть
по~23~дольки.

Это~--- число между единицей и~двойкой:
$$
1<\bfrac{23}{15}< 2.
$$
Такие числа древние отлично знали. Мы их называем \textit{рациональными}, а~они их называли дробями или
просто числами. Рациональное число~--- это число, которое может выражать количество яблок,
разделенное на~количество детей. Пришло вот к~вам <<$n$>>, целое ненулевое число детей, а~у~вас
имеется <<$m$>>~--- целое число яблок. Получаем рациональное число $\bfrac{m}{n}$. Числитель дроби
может быть меньше нуля.

Ну, скажем, у~вас было $-5$~яблок, и~пришло 7~детей. Каждый получил $-\bfrac57$~яблок.

\textbf{Слушатель:} Бедные дети\ldots

\textbf{А.С.:} Или вы позвали 30~гостей на~день рождения и~сказали: <<У меня $-700$~тысяч рублей,
в~смысле, я~должен за~квартиру 700~тысяч рублей. Скиньтесь, пожалуйста, поровну>>. Вот вам и~минус:
$-\bfrac{700}{30}$. Когда вы говорите о~таких вещах, как долги, то сразу вылезают отрицательные числа.
Я~предлагаю вам понять, что все эти числа живут где-то на~числовой прямой. Число $\bfrac57$ живет где-то
между нулем и единицей (рис.~\ref{f:b8}). Давайте начнем шагать по~оси шагами в~одну сотую. Мы, на~наш
взгляд, целиком замостим нашу прямую: $\bfrac{211}{100}$, $-\bfrac{135}{100}$ и~т.\,д.


%Рис. 8
\xPICi{b8}{Шаги-то получились меньше, чем точка от~мелка!}

И~замостить вы можете сколь угодно плотно, можно ведь шагать шагами, равными $\bfrac1{1000}$ или $\bfrac1{1000000}$.
Где бы вы ни сидели на~числовой оси, где-то рядом с~вами, очень близко живет число вида $\bfrac{m}{n}$.

Математики употребляют в~такой ситуации страшный термин \textit{<<всюду плотное множество>>}. Это такое
множество, в~котором, куда бы вы ни сунулись, в~любой близости от~вас будут точки этого множества.
Рациональные числа образуют всюду плотное множество на~числовой оси. Вроде как вся прямая ими
заполнена. Вполне можно было бы ожидать, что никаких чисел больше нет. Это логично, но~это
неправда. Древние обнаружили, что есть числа, заведомо не представимые в~виде $\bfrac mn$, ни при~каких
целых $m$ и~$n$ (врезка~5).


\medskip

\hrulefill

\smallskip

\textbf{Врезка 5. <<Причина смерти~--- корень из~двух!>>}\footnote{За достоверность этой истории автор книги ответственности не несет.}

Говорят, что пифагорейцы (ученики знаменитого философа\linebreak и~математика Пифагора) сначала верили, что
для~вычислений вполне хватает положительных рациональных чисел, и~что в~этом проявляется
божественная гармония окружающего мира. Однако <<не в~меру способный>> ученик Пифагора додумался
до~того, что строго доказал НЕИЗМЕРИМОСТЬ диагонали квадрата (с~единичной стороной) с~помощью
рациональных чисел. Пифагорейцы в~гробовом молчании выслушали его доказательство и~не~смогли его
опровергнуть. Гармония мира оказалась под угрозой! Поэтому было принято решение: никому про это
не~рассказывать, а~нарушителя мировой гармонии наказали\ldots\ утоплением в~реке, на~берегу которой всё
это и~происходило. К~счастью для~математики, истина потом всё равно <<воссияла>>.

\smallskip

\hrulefill

\medskip

\pagebreak

Вот я~и~утверждаю, что корень из~двух~--- именно такое число. Возьмем 4~квадрата со~стороной
единичка. И~составим из~них новый квадрат (рис.~\ref{f:b8s}).

%Рис. 8s
\xPICi{b8s}{Нарушители мировой гармонии~--- за~работой.}

Какой площади один маленький квадратик?

\textbf{Слушатель:} 1.

Какой площади будет получившаяся фигура?

\textbf{Слушатель:} 4.

\textbf{А.С.:} Теперь я~делаю следующее. Я~провожу диагонали (см. рис.~\ref{f:b9}) и~спрашиваю вас, чему равна
площадь получившегося внутри квадрата?

%Рис. 9
\xPICi{b9}{Внутренний квадрат~--- вдвое меньше по~площади.}

\textbf{Слушатель:} 2.

\textbf{А.С.:} Почему? Потому что в~каждом маленьком квадратике ровно половину взяли, а~половину не~взяли.
Итак, совершенно очевидно, что площадь этой фигуры вдвое меньше, чем у большого квадрата. С~другой
стороны, мы знаем, что если у~квадрата сторона $a$, то площадь его равна $a\cdot a=a^{2}$.

%$\sqrt{2}$\quad
%$1$

Нам нужно найти сторону квадрата с~площадью~2. А~это и~есть корень из~двух. Значит, если у~квадрата
сторона 1, то его диагональ имеет длину <<корень из~двух>> (рис.~\ref{f:b10}).

%Рис. 10
\xPICi{b10}{Сторона внутреннего квадрата равна корню из~двух по~двум причинам:
алгебраическая причина~--- теорема Пифагора и определение корня;
геометрическая причина~--- соотношение площадей наружного и~внутреннего квадратов равно двум.}

\textbf{А.С.:} Из~школьного курса вы знаете теорему Пифагора.

\textbf{Слушатели:} Да.

\textbf{А.С.:} Теорема Пифагора говорит, что квадрат гипотенузы равен сумме квадратов катетов\footnote{Гипотенуза~---
сторона прямоугольного треугольника, лежащая напротив прямого угла. Это самая
длинная сторона в прямоугольном треугольнике.
Катет~--- сторона прямоугольного треугольника, прилегающая к прямому углу.
В~прямоугольном треугольнике имеется два катета и одна гипотенуза.}. Давайте
я~покажу доказательство этого без~единой формулы. Теорему Пифагора не~нужно доказывать формулами,
ее нужно просто узреть, увидеть, она видна. Вот смотрите, я~беру вот такое равенство:
$a^{2}+b^{2}=c^{2}$.

\pagebreak

Мне нужно его доказать для~любого прямоугольного треугольника со~сторонами $a$, $b$, $c$ (рис.~\ref{f:b11}).

%Рис. 11
\xPICi{b11}{Для~такого треугольника мы ниже докажем теорему Пифагора методом <<Взгляни на~чертеж~--- из~него всё ясно>>.}

Беру два квадрата со~стороной $a+b$.

Они будут одинаковые, но~я~их по-разному разобью на~части (рис.~\ref{f:b12}).

%Рис. 12
\xPICi{b12}{Слева и~справа виден квадрат со~стороной $a+b$. Он по-разному разбит на~части, но~и~там,
и~тут мы видим 4 одинаковых треугольника (их стороны указаны на~рис.~\ref{f:b11}). Убирая эти треугольники,
слева видим квадрат гипотенузы, а~справа~--- сумму квадратов катетов. Вот и~всё доказательство.}

Площадь внутри левого квадрата равна~$c^{2}$.

Площади квадратов внутри правого квадрата равны $a^{2}$ и~$b^{2}$.

Теперь смотрите, правый квадрат состоит из~4 треугольников и~двух квадратов, а~левый~---
из~четырех таких же треугольников и~одного квадрата. Но~внешние квадраты имеют одинаковые площади.
Из~площади первого квадрата я~вычел 4 одинаковые площади и~из~площади второго квадрата те же 4
площади. Значит, площади оставшегося должны быть одинаковыми. В~одном случае остается $c^{2}$, а в~другом~--- сумма $a^{2}+b^{2}$.
Значит, $$a^{2}+b^{2}=c^{2}.$$ Теорема Пифагора доказана. Но~это было небольшое
отступление. Я~хотел сказать, что диагональ квадрата со~стороной 1 по~теореме Пифагора равна корню
из~двух, согласно тому, что я~нарисовал, она и~в~самом деле ему равна.
 Древние ничего не~могли
с~этим числом поделать. Потому что, если отложить отрезок, равный нашей диагонали, от~нуля, то вы
попадете в~точку, которая заведомо не~равна никакому числу вида $\bfrac mn$. Ни при~каких $m$ и~$n$. Вы
переберете все целые числа, и~в~числителе, и~в~знаменателе, и~никогда не~получите число, которое
в~точности совпадет с~корнем из~двух.

Есть очень много разных доказательств этого факта, и~одно из~них совершенно геометрическое. Мы разберем ниже два разных доказательства.

Мы сейчас придумаем некую процедуру, которую мы применим к~любому рациональному числу, и~она всегда
будет конечной. А~дальше, я~вам покажу, что та же самая процедура для~числа \textbf{<<корень из~двух>>}
никогда не~прекращается, тем самым это число не~может быть рациональным

\textbf{Слушатель:} То есть это несуществующее число?

\textbf{А.С.:} Существующее, но~не~в~этом круге подозреваемых лиц. Это число существует, и~оно
очень нервировало греков, они не~хотели допустить, что оно существует. Однако они отлично знали,
что оно нужно для~вычислений, но~не~выражается в~виде отношения целых чисел. Они не~понимали, что с ним делать.
 Вроде число не~существует, а~оно-таки есть. Оно не~должно существовать, но~оно существует.
Числа, которые не~представляются в~виде $\bfrac mn$, называются \textit{иррациональными}.

Что такое вообще <<иррациональность>>? Нелогичность. Неразумность. Иррациональное поведение,
например. Но~в~математике, в~отличие от~философии, есть совершенно конкретные объекты,
иррациональные числа. Это такие числа которые не~представляются в~виде $\bfrac mn$. Тем не~менее, они
вполне себе логичные и~очень даже разумные.

\textbf{Слушатель:} А~числа $m$ и~$n$, они целые?

\textbf{А.С.:} Целые. Непременно целые числа. Иррациональные числа~--- это числа, которые не~являются
отношением двух целых чисел. Рациональное число~--- это отношение двух целых.

Есть еще одно труднопроизносимое слово, оно тоже в~философском смысле кое-что означает. Слово
\textit{<<трансцендентно>>}. Что же оно означает в~житейском (не~математическом) смысле?

\textbf{Слушатель:} Находится за~пределами.

\textbf{А.С.:} За~пределами чего бы то ни было.

\textbf{Слушатель:} То есть иррациональное поведение~--- это поведение странное, но~всё же в~каких-то
рамках. А~трансцендентное~--- это что-то за~пределами понимания окружающих.

\textbf{А.С.:} В~математике \textit{трансцендентные числа}~--- это тоже определенный термин. Им
противопоставляются \textit{алгебраические числа}. Согласно строгому математическому определению,
\textit{алгебраическое число}~--- это корень многочлена с целыми коэффициентами. \textit{Трансцендентным числом}
называется такое число, что ни один многочлен с целыми коэффициентами не обнуляется при подстановке
вместо переменной $x$ этого числа.


Внутри множества алгебраических чисел живут как все рациональные, так и~корни любой степени
и~много, много чего еще. Очень много разных чисел. И~вот трансцендентные~--- это те числа, которые
не~являются алгебраическими. Выдумать неалгебраическое число достаточно трудно. Сначала думали, что
все числа алгебраические. А~в~XIX~веке произошел взрыв в~математике, было обнаружено огромное
количество неалгебраических чисел~--- но~это было только в~XIX~веке. Примером трансцендентного числа
является знаменитое число <<пи>>~--- длина окружности с~диаметром, равным 1. Доказательство
трансцендентности одного-единственного числа <<пи>> занимает 10~лекций на~\text{4-м} курсе мехмата МГУ.
Очень мало людей на~Земле, которые знают это доказательство. Это~--- труднейшая теорема.
А~вот про иррациональность корня из~двух всё очень просто.


Я~представлю вам два разных доказательства: одно будет длинным, но~геометрическим (и~оно будет
полезно для~изучения других тем), второе~--- короткое стандартное доказательство.

Я~проведу некую процедуру; ну, сделал это не~я, а~не~кто иной, как Евклид 2,5~тысячи лет назад.
Называется эта процедура \textit{алгоритмом Евклида}.
 Разновидность алгоритма Евклида называется \textit{цепной
дробью}. Цепная дробь~--- это очень просто. Любое число можно разложить в~цепную дробь. Ниже
я~покажу, что числа вида $\bfrac mn$ в~цепную дробь раскладываются конечным образом, а~\textbf{корень из~двух}
в~цепную дробь раскладывается только бесконечным образом.

Приведу пример: дробь $\bfrac{21}{13}$.

Давайте посмотрим, как превращать это число в~цепную дробь. Выделим из~этой дроби целую часть.
Между какими двумя целыми числами она расположена?

\textbf{Слушатель:} Между 1 и~2.

\textbf{А.С.:} Правильно. Значит оно равно 1 плюс сколько?

\textbf{Слушатель:} Мне трудно из~21 вычесть 13.

\textbf{А.С.:} Ничего. Я~вычту.

\textbf{Слушатель:} 8, да?

\textbf{А.С.:} Правильно, да.
$$
\bfrac{21}{13}= 1 + \bfrac{8}{13}.
$$
Но~это не~всё, что я~хотел сказать. Потому что я~напишу так. Один плюс один разделить на~тринадцать восьмых:
$$
\bfrac{21}{13}=
1+\bfrac{1}{\bfrac{13}{8}}=
1+\bfrac{1}{1+\bfrac{5}{8}}.
$$
Этот фокус с~переворачиванием дроби <<вверх ногами>> и~выделением целой части из~знаменателя мы
будем повторять до~тех пор, пока будет возможно. А~возможность такая будет нам представляться до~тех пор, пока
на~очередном шагу после выделения целой части дробная часть не окажется равна нулю. Если этого
никогда не~случится, то исходное число окажется разложенным в~бесконечную цепную дробь.


Итак, продолжим разложение числа $\bfrac{21}{13}$ в~цепную дробь:
\begin{multline*}
1+\bfrac{1}{1+\bfrac{5}{8}}=
1+\bfrac{1}{1+\bfrac1{\bfrac85}}=
1+\bfrac{1}{1+\bfrac1{1+\bfrac35}}=\ldots=
1+\bfrac{1}{1+\bfrac1{1+\bfrac1{1+\bfrac1{\bfrac32}}}}=
\\=
1+\bfrac{1}{1+\bfrac1{1+\bfrac1{1+\bfrac1{1+\bfrac12}}}}.
\end{multline*}
Стоп, машина. После выделения целой части из~числа 2 дробная часть \textbf{равна нулю}. Значит,
числу $\bfrac{21}{13}$ <<суждено>> разлагаться в~конечную цепную дробь. Если кто не~верит, можете
упростить эту <<6-этажную>> дробь, сделав из~нее обыкновенную. Конечно, она будет равна
$\bfrac{21}{13}$.

Эту операцию придумал Евклид. Называется она~--- разложение числа в <<цепную дробь>>. Обратите внимание.
На~последнем шаге мы попали в~целое число $2$ при~переворачивании дроби $\bfrac12$.
На~этом всё заканчивается, так как из~целого числа не~удастся выудить дробную часть.


Другой пример:
$$
\bfrac{20}{17} =
1+\bfrac1{\bfrac{17}{3}}=
1+\bfrac1{5+\bfrac{2}{3}}=
1+\bfrac1{5+\bfrac{1}{1+\bfrac12}}.
$$
Опять пришли к~целому числу. Ура. Закончили.

Евклид утверждал, что для~любой дроби за~конечное число шагов мы придем к~целому числу. Попробую
это пояснить <<без формул>>. Вы берете какую-то очень большую дробь, например, $\bfrac{17284}{3415}$.

Что происходит в~процессе, предложенном Евклидом? Мы просто несколько раз делим с~остатком, и~всё.
На~каждом шагу мы получаем <<нечто>> плюс что-то меньшее, чем то, на~что мы делим. Идея в~том, что
на~каждом шагу числа будут уменьшаться. Числитель и~знаменатель~--- целые положительные числа, и~они будут уменьшаться.
Но~целое число, любое положительное целое число, не~может бесконечно долго
уменьшаться, оно в~конце концов <<закончится>>. Оно придет к~нулю за~конечное число шагов.

То есть любое рациональное число непременно порождает конечную цепную дробь. А~теперь я~возьму
и~покажу, что <<корень из~двух>> порождает бесконечную цепную дробь.

Вот этот фокус-покус. Если <<корень из~двух>>~--- рациональное число, то процедура, которую
я~только что проводил, должна закончиться. Берем <<\textbf{корень из~двух}>>. Между какими целыми
числами он расположен? Вспомним, что, согласно теореме Пифагора, <<корень из двух>>~--- это длина
диагонали квадрата с единичной стороной. Поэтому он расположен между 1 и~2 (см. рис.~\ref{f:b13}).

%Рис. 13
\xPICi{b13}{Ловись, дробная часть~--- большая и~маленькая!
Видно, что диагональ больше стороны, но~меньше суммы сторон.}

Значит, $\text{<<\textbf{корень из~двух}>>} = 1 + \text{дробная часть}$ (она примерно равна $1{,}4142-1=0{,}4142$).

Что я~сделал? Прибавил единицу и~отнял единицу. Больше ничего не~делал. То есть я~выделил целую
часть из~<<корня из~двух>> (дробная же часть записана в~скобках; она равна примерно $0{,}414$).

%$1$\quad
%$\sqrt2-1$\quad

%Рис. 14
\xPICi{b14}{Выделенный отрезок и~есть <<корень из~двух минус один>>.}

%\endinput

Для~получения дробной части я~взял окружность радиуса 1, провел ее до~пересечения с~диагональю,
и~всё (см. рис.~\ref{f:b14}). Эта часть, без~сомнения, меньше единицы.
Далее для~краткости обозначим <<корень из~двух>> через~$\text{К}$. А~выражение
$\text{К}-1$ обозначим за $\text{С}$.
 Значит, $\text{С}<1$. Поэтому будем эту часть <<переворачивать>>:
$$
\text{С}=\bfrac1{\bfrac1{\text{С}}} =
\bfrac1{\text{К}+1}.
$$
Поясню, почему $\bfrac1{\text{С}}$ превратилось в~$\text{К}+1$.

Я~возьму числитель и~знаменатель и~домножу на~одно и~то же число (это не~изменит значения дроби).
Я~числитель и~знаменатель умножу вот на~такое число: $\text{К}+1$. Помним, что $\text{K}\cdot\text{K} = 2$.

Начинаем открывать скобки:
$$
\bfrac1{\text{С}} =
\bfrac{(\text{К}+1)\cdot 1}{(\text{К}+1)\cdot(\text{К}-1)} =
\bfrac{\text{К}+1}{\text{К}\cdot\text{К}-1} =
\bfrac{\text{К}+1}{2-1} =
\text{К}+1.
$$
А~теперь, по~общему правилу, выделяем целую часть.

Между каким двумя целыми числами находится $\sqrt2+1$?

\textbf{Слушатель:} Между двойкой и~тройкой.

\textbf{А.С.:} Конечно. Поэтому, если я~по~правилу Евклида выделяю из~него целую часть, то она равна?

\textbf{Слушатель:} 2.

\textbf{А.С.:} Значит, я~должен написать так:
$$
\sqrt2= 1+ \bfrac1{2 + \text{новая дроб.часть}} =
1 + \bfrac1{2 + 0{,}414\ldots}.
$$
Не~правда ли, мы уже сталкивались выше с~такой дробной частью?

\textbf{Слушатель:} И~так до~бесконечности будет повторяться?

\textbf{А.С.:} И~так до~бесконечности. Значит, исходное число~--- не~рациональное.

Мы получим \textit{бесконечную} цепную дробь:
$$
\text{К}=
1+\bfrac1{2+\bfrac1{2+\bfrac1{2+\bfrac1{2+\ldots}}}}.
$$

Только бесконечное число шагов приведет вас к~числу, равному~$\text{К}$\footnote{Точный смысл
этих слов проясняется в математическом анализе, через понятие предела последовательности.}.
Но~не~конечное~--- а~значит, число~$\text{К}$ иррационально, что и~требовалось доказать.
 Теперь вы знаете, что есть такие числа, страшные
числа, которые не~представляются в~виде <<количество яблок поделить на~количество гостей>>.

Мы еще вернемся к цепным дробям, ибо в них прячется истинная бесконечность.

Пока что дадим стандартное книжное доказательство того, что <<корень из~двух>>~--- число
не~рациональное. Проводится оно от противного.


Предположим, что
\begin{equation} %%% 1-3-2-1
\label{1-3-2-1}
\text{К}= \bfrac mn.
\end{equation}
Тогда, если можно, сократим эту дробь. То есть, если $m$ делится на простое число $p$, и~$n$ делится
на то же самое простое число $p$, поделим и~числитель, и~знаменатель на $p$. После того как мы
сократили всё, что только можно, одно из~чисел обязательно будет нечетным. Может быть, оба
нечетные, но~по~крайней мере одно из~них точно будет нечетным. Потому что если оба четные, значит, можно было еще
раз сократить на~2.

Рассмотрим квадрат равенства~\eqref{1-3-2-1}:
$$
2=\bfrac{m^{2}}{n^{2}}.
$$
Получим $m^{2}=2n^{2}$.

Это значит, что если на~сетке нарисован квадратик с~целочисленной стороной, то в~нём количество
единичных квадратиков такое же, как удвоенное количество квадратиков какого-то другого квадрата
с~целочисленной стороной (рис.~\ref{f:b15}).


%Рис. 15
\xPICi{b15}{Что-то не~получается нарисовать два таких квадрата. И~это~--- неспроста!}

Значит, если $\text{К}$~--- рациональное число, то $m^{2}=2n^{2}$~--- верное равенство. Тогда $m$~--- число четное,
потому что оно делится на~2. Но~если $m$ делится на~2, то это значит, что $m=2k$ для некоторого целого числа~$k$. Тогда $m^{2}=4k^{2}$.

Подставим в~$m^{2}=2n^{2}$ значение для~$m^{2}$. Получим $4k^{2}=2n^{2}$.

Сократим на~2, получится $2k^{2}=n^{2}$.

Но~тогда $n$ тоже делится на~2. А~значит, мы в~начале этого процесса недосократили. Но~мы же
договорились досократить всё, что возможно. В~этом и~заключается противоречие~--- с тем фактом, что в
выражении $\sqrt{2}=\bfrac mn$ можно добиться того, что хотя бы одно из чисел $m$, $n$ будет нечетным.

То, что $\sqrt2$ никогда не~представляется в~виде $\bfrac mn$, на~самом деле означает то же самое,
что равенство $m^{2}=2n^{2}$ всегда неверно. Никогда не~получится взять один квадрат с~целыми сторонами,
умножить его площадь на~два и~получить другой квадрат с~целыми сторонами (удвоенной площади). Ни
для~каких целых чисел.

Попробуем копнуть этот вопрос поглубже.

А~может ли быть так, что они почти будут равны, например, $m^{2} =2n^{2}\pm1$?

Вдруг мы сможем взять какие-нибудь огромные числа, возвести их в~квадрат, умножить одно из~них
на~2 и~выяснить, что результаты отличаются на~1. Может ли такое быть или нет? И~если может быть,
то насколько часто такое бывает? И~можно ли полностью описать все пары целых чисел $(m,n)$, которые
удовлетворяют уравнению $m^{2} = 2n^{2}\pm 1$?
 Вопрос, который ставился еще древними~--- он называется
<<решение \textit{Диофантовых уравнений} в~целых числах>>.

Диофант жил в~Александрии в~III~веке нашей эры. Он оставил после себя 13~томов математических
изысканий, 6 из~них худо-бедно, но~дошли до~нас, 7~--- полностью и безвозвратно потеряны.
 6~томов его
изысканий до~сих пор питают умы математиков. Диофант писал всё словесно. Примерно так: <<Может ли
быть такое, что одно число, будучи взятое то же самое число раз (то есть $n\cdot n$) и~еще столько же раз
(то есть $2n\cdot n$), отличалось бы от~другого числа, взятого другое же число раз (то есть $m\cdot m$) всего лишь
на~единицу?>> Так он записывал уравнение
$$
2n^{2} = m^{2} \pm 1.
$$
Можно ли такое уравнение решить в~целых числах или нет? Мы пишем символами, поэтому далеко
продвинулись в~математике. Но~все идеи буквально, буквально все подряд были в~этих шести томах.
Если чего-то в~них не~было, то, видимо, оно было в~пропавших. Но~мы уже не~узнаем этого.

Диофант~--- человек, оставивший фантастическое наследие.\linebreak В~1651~году Пьер Ферма читал книгу
Диофанта по~целочисленной арифметике. Читал и~комментировал ее на~полях. А~сын Ферма издал книгу
с~комментариями своего отца. На~полях был кладезь математических сокровищ. В~частности, в~одном месте было обнаружено следующее.
У~Диофанта решалось в~целых числах уравнение $a^{2}+b^{2}=c^{2}$. То есть он пытался
выяснить, может ли быть так, что все числа целые? Древним было хорошо известно, что такое может
быть. Например, числа (3, 4, 5), и~много-много других примеров.

Первое решение, возможно, даже имело практическое применение 2,5~тысячи лет назад. Берем веревку, делим ее на~12
равных частей, завязываем узелки в~местах деления. После чего связываем веревку в~кольцо и~делаем
из~нее треугольник так, чтобы на~одной стороне было 5 узелков, на~другой 4, а на~третьей~--- 3.


И~вот вы получили прямой угол кустарными средствами. Это очень важно.

Землемеру этого хватит. Всё. У~него веревка с~12~узлами есть, и~отлично. Но~математик всегда хочет
пойти до~конца. Все варианты найти, все целые $a$, $b$, $c$~--- такие, что получается прямоугольный
треугольник. Задача древнейшая. Ответ был известен еще древним индусам. <<Пифагоровы тройки>>~---
вот как называются эти решения. Интересно то, что в~этом месте слева на~полях было написано рукой
Ферма приблизительно следующее: <<Вместе с~тем, невозможно разложить никакой куб в~сумму двух
кубов, никакую четвертую степень в~сумму двух четвертых степеней и~вообще никакую произвольную
степень числа в~сумму двух таких же степеней. Я~нашел этому факту поистине удивительное
доказательство, но~на~полях оно не~поместится>>. Это~--- начало истории величайшей загадки
математики~--- \textbf{великой теоремы Ферма}.

Ферма утверждает, что при~$n$ большем, чем 2, уравнение $$x^{n}+y^{n} = z^{n}$$ не~имеет решения в~целых числах.
То есть, конечно, можно взять $x=y=z=0$. Или, если мы поставим $x=0$, тогда $y$ и~$z$ могут быть любыми
одинаковыми. Но~это всё неинтересно. А~вот если ноль запретить, то если мы ищем среди \textit{положительных}
целых чисел $x$, $y$, $z$ решения этого уравнения, то их нет, вообще нет. Ни одного, ни одной тройки $(x,
y, z)$, ни для~какого $n$, большего чем 2, то есть ни при~$n=3$, ни при~$n=4$, ни при~каком~$n$.

Эта загадка была страшно популярной среди широких масс населения~--- уж больно просто формулируется
эта теорема (да еще какой-то чудак завещал крупную сумму тому, кто справится с~доказательством
теоремы Ферма). Но~и~опытные математики были озадачены. Дело в~том, что все утверждения, которые
Ферма оставил без~доказательства, оказались правильными (их все доказали после его смерти),
а~с~этим творилось черт знает что: начали все сходить с~ума, потому что всё кажется просто,
и~хочется взять ручку и~начать писать. Вот вы мне не~поверите, но~когда мне было 10~лет, я~этим
занимался, честно. Но~всё это безумие продолжалось только до~1994~года.

В~1994~году она была полностью доказана нашим с~вами современником математиком Эндрю Уайлзом.
На~самом деле ему предшествовали 30~разных имен, которые долго в~разных местах подстраивали большое
здание. А~он просто понял, в~каком месте нужно сшить то, что уже известно. В~частности, безусловную
важность здесь сыграла московская школа алгебраической геометрии. Последним был Уайлз,
но~в~принципе это~--- всемирное творение.

Сейчас доказательство великой (или, как еще говорят, последней) теоремы Ферма входит в~книгу
А.\,А.~Панчишкина, Ю.\,И.~Манина <<Введение в~современную
теорию чисел>>.
Толстенная сложнейшая книга по~теории чисел, \text{7-я} глава целиком посвящена тео\-ре\-ме
Ферма.

Ну а~теперь фокус-покус, ладно?
 А~то лекция уже кончается.

Берем нашу цепную дробь для~<<корня из~двух>>:
$$
\sqrt2=1+\bfrac1{2+\bfrac1{2+\bfrac1{2+\bfrac1{2+\ldots}}}}.
$$
Обрубаем, получаем приближенное значение для~корня из~двух:
$$
1+\bfrac1{2+\bfrac1{2+\bfrac1{2+\bfrac1{2}}}}.
$$
Такую дробь можно превратить в~некоторое рациональное число, то есть в некоторое отношение двух целых чисел.
Сейчас превратим. Получается $\bfrac{41}{29}$.

Всё отлично.

А~вот теперь берите калькулятор, пожалуйста. И~возводите в~квадрат 41 и~29.
Не~забудьте, что 29 в~квадрате при~этом надо умножить на~2, <<по просьбе Диофанта>>:
\begin{gather*}
41^{2} =1681,\\
29^{2} =841,\\
841\cdot 2=1682.
\end{gather*}
Ура! Они отличаются на~единицу. Это те самые решения нашего уравнения
\begin{equation} %%% 1-3-2-2
\label{1-3-2-2}
m^{2} =2n^{2}\pm1.
\end{equation}
Мы нашли решение этого уравнения. Причем нетривиальное.

Теорема, которую я~доказывать не~буду (хотя она и не очень сложная), гласит: \textit{Где бы вы ни обрубили данную цепную
дробь, всегда получается решение нашего диофантова уравнения.}


\textbf{Слушатель:} Любое число разложу в~цепную дробь, обрублю и~получу решение какого-то похожего уравнения?

\textbf{А.С.:} Не~для~любого. Для~любого числа, не~являющегося квадратом. И обрубать надо будет аккуратнее, не в любом месте, как в случае с корнем из двух.

Например, уравнение $m^{2} = 9n^{2}\pm 1$ не~получится решить таким способом
(впрочем, несложно показать, что у него всего два тривиальных решения $(1,0)$ и $(-1,0)$).
Но~таких чисел довольно мало.
Что же я~могу подставить вместо 2 в~уравнение~\eqref{1-3-2-2}: 3~--- могу, 4~--- не~могу, так как квадрат;
5, 6, 7, 8~--- могу, 9~--- не~могу, 10, 11, 12, 13, 14, 15~--- могу, 16~--- не~могу, и~так далее.
Уравнение такого вида (см. подробнее об этом в следующей лекции) носит название \textit{уравнение Пелля}. И, как обычно это бывает, Пелль не~имеет
к~нему никакого отношения. В~математике очень много фактов названо именами людей, которые никакого
отношения к~этому факту не~имели. Шутки ради это явление математики тоже назвали <<теоремой>>. Вот,
получилось так, что эту теорему назвали теоремой Арнольда. Она \textit{самоприменимая} (то есть Арнольд
не~является автором этой теоремы). Шутливую <<Теорему Арнольда>> придумал, вроде бы, Николай Николаевич
Константинов и~назвал теоремой Арнольда специально для того, чтобы она была самоприменимой, чтобы она тоже
называлась не~именем человека, который ее выдумал, а~другим.
 Математики мыслят логически, даже
когда они шутят!

Давайте все-таки, чтобы вас убедить, пообрубаем эту дробь в~разных местах. Смотрите. 1~--- это
ведь <<1 разделить на~1>>. Если подставить в~уравнение~\eqref{1-3-2-2} $m=n=1$, то что получится?
$$
1^{2} = 2\cdot1^{2}-1
$$
(то есть \eqref{1-3-2-2} выполняется).

Обрубаем дальше. Будет $\bfrac32$.

Подставляем: $9=2\cdot 4+1$.

Обрубаем еще раз. Получаем $\bfrac75$. Подставляем.
$$
49=2\cdot25-1.
$$
Вы видите, что теорема верна.

Гуманитарию уже не~надо доказывать теорему, он уже <<видит>>, что она верна. Но~математику нужно ее
доказать, нужно установить, что это действительно всегда будет так. Мало того, оказывается, что все
такие обрубания дадут вам решения этого уравнения, и~других решений в~задаче нет. Вообще никаких.

\textbf{Слушатель:} Ну, или мы просто не~нашли?

\textbf{А.С.:} Нет. Доказали, что больше не~существует.

Ну, последний фокус-покус. Но~берегитесь, он страшный. Знаете ли вы, что такое \textit{бином Ньютона}?
Это~--- правило, по~которому раскладываются выражения, в~которых вы много раз умножили одну скобку
на~себя. В~школе проходят $(a+b)(a+b)=a^{2}+2ab+b^{2}$. Еще проходят: $(a+b)(a+b)(a+b)=a^{3}+3a^{2}b+3ab^{2}+b^{3}$.
Но~есть некая формула, которая верна всегда, для~любого количества скобок. Считается, что ее
придумал Ньютон, но на~самом деле ее, скорее всего, знали и~до~него. Просто он ее огласил.
Так вот, бином Ньютона тоже помогает искать решения уравнения $m^{2}-2n^{2}=\pm1$.

Ниже мы снова за $\text{К}$ обозначим корень из двух.

Возьму $(1+\text{К})^{2}=1+2\text{К}+2=3+2\text{К}$. Решением будет пара ($m=3$, $n=2$),
и~мы уже выше встречались с~ним. Но, может, это случайно так совпало?

Возведение в куб вас должно уже убедить. Имеем:
$$
(1+\text{К})^{3}=1+3\text{К}+6+2\text{К}=7+5\text{К}.
$$
Не~правда ли, это следующее решение нашего уравнения? Здесь $m=7$, $n=5$.

Возведем в~четвертую степень. А~это всё равно, что возвести два раза во~вторую, один раз в~нее мы уже возводили.
$$
(1+\text{К})^{4}=(3+2\text{К})^{2}=9+12\text{К}+8=17+12\text{К}.
$$
Проверяем:
\begin{gather*}
17^{2} =289,\\
12^{2}=144,\\
144\cdot 2=288.
\end{gather*}
Получается: $289=288+1$.

Это работает!

До~встречи на~лекции 4.

\endinput
