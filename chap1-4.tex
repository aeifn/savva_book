%1.4.
\section{Лекция 4}
\label{1.4}

\textbf{А.С.:} На~прошлой лекции я~сказал кое-что про решение уравнения вида $x^{2}-2y^{2} = \pm
1$.  Тогда обозначения были другие. Но на то это и математика, что <<хоть горшком назови>>. В этой
лекции переменные, значения которых мы ищем, будут обозначаться <<$x$>> и <<$y$>>. Теперь кое-что уточним.
Можно взять вместо числа 2 любое натуральное число $m$ и записать аналогичное уравнение:
$x^{2}-my^{2}= \pm 1$.


В~принципе, почти ничего не~изменится в~общем ходе решения. Единственный вариант, при~котором будут
различия, это когда $m$ представляется в~виде квадрата натурального числа (4, 9, 16, 25\ldots),~---
тогда такое уравнение по~неким очевидным причинам никаких решений, кроме $x=\pm 1$, а~$y=0$, не~имеет.

В~самом деле, попробуем найти нетривиальные решения уравнения $x^{2}-9y^{2} = \pm 1$, то есть $x\cdot x =
(3y)\cdot (3y) \pm 1$. При~<<$y$>>, не~равном нулю, получается, что квадраты двух целых чисел <<$x$>> и~<<$3y$>>
отличаются на~единицу. Так мало они отличаться НЕ МОГУТ. Даже квадраты соседних целых ненулевых
чисел (скажем, $M$ и~$M+1$) отличаются больше, чем на~1, а~именно: отличие их равно $2M+1$, причем
$M$ не~равно~0.

Для~всех остальных $m$ прием, которым мы пользовались ранее при~решении этой задачи, срабатывает.
А~прием этот был такой: нужно корень из~$m$ разложить в~цепную дробь. То есть выделяем целую часть,
потом <<переворачиваем>> оставшуюся дробную часть, получаем число, большее единицы, в~нём опять
выделяем целую часть, и~так далее:
$$
\sqrt m=
a+\bfrac1{b+\bfrac1{c+\bfrac1{d+\bfrac1{e+\ldots}}}}.
$$

Я~сказал в~лекции~3, что для~получения решения уравнения мы можем обрубить дробь в~любом месте,
привести к~виду <<целое число разделить на~целое>>, и~числа, которые получатся в~числителе
и~знаменателе, будут нашими решениями. И~для~$m=2$ это действительно можно делать на~любом месте.
Но~если это утверждение применить для~других значений $m$, то получится, что я~немного обманул вас.
Есть теорема, доказанная Ж.\,Л.~Лагранжем, которая утверждает, что если мы разложим корень из~числа,
не~являющегося квадратом, в~цепную дробь, то цепная дробь начиная с~некоторого места начнет
повторяться. Появится период.

\medskip

\hrulefill

\smallskip

\textbf{Врезка 6. О~бессилии <<наблюдения>> без~<<доказательства>>}

Понятие \textbf{периода} последовательности не~такое простое, как хотелось бы думать. Более того, это понятие
демонстрирует бессилие прикладной математики для~установления фактов чистой математики. Например,
допустим, что прикладной математик изучает поведение следующей последовательности десятичных цифр:
2223222322232223\ldots\ Что скажет при~этом <<совсем простой наблюдатель>>? То, что имеется период
<<2223>>, состоящий из~4~цифр. Более <<утонченный наблюдатель>> возразит: не~будем спешить,
понаблюдаем дальше за~поведением этих цифр хотя бы до~\text{34-го} места. Сказано-сделано: получили
$$
22232223222322237\quad
22232223222322237\ldots
$$
Что, убедились?! Период-то имеет длину не~четыре,
а~семнадцать! Но~обиженный <<простой наблюдатель>> возразит: погодите радоваться. Понаблюдаем
теперь хотя бы до~сотого места. И~увидели, что на~\text{69-м} месте (после семерки на~\text{68-м} месте) стоит
не~цифра~2 (как они оба ожидали), а~цифра~0. Вот тут-то они призадумались\ldots\ А~есть ли вообще период
у~этой последовательности? И~может ли <<простое наблюдение>> дать обоснованный ответ на~этот
вопрос? КОНЕЧНО, НЕТ!~--- скажет им чистый математик. Если у~нас в~результате наблюдения появилась
гипотеза, что период равен 2223, то надо остановиться, проверить, есть ли научные предпосылки
для~доказательства этого (либо для~опровержения этого), и~продолжать исследование дальше. И~если
возможную длину периода не~удалось определить или ограничить сверху никакими <<наблюдениями>>, это
вовсе не~означает, что последовательность непериодическая! Это означает, что пока что чистому
математику не~удалось решить эту проблему (может, потому, что он плохо ее решал).

Это, конечно, \textbf{не~означает}, будто бы мы не~доказали, что для~разложения <<корня из~двух>> период
начинается сразу, и~длина его равна единице, а~сам период равен~<<2>>. В~данном случае не~просто
повторяются числа 2222\ldots, начиная со~второго места, а~повторяются \textit{условия} для~повторения этого
числа. Ниже мы не~будем углубляться в~эту философскую проблему, а~просто предположим, что уже
<<кем-то>> доказано наличие именно периода такой длины, и~именно из~таких чисел.

\smallskip

\hrulefill

\medskip


Мы раскладывали для~самого простого случая, и~в~нём сразу пошел период: целые части со~второго
места равны 2, 2, 2, и~т.\,д. Если бы я обрубал цепную дробь в любом месте  для~любого $m$,
я~совершил бы ошибку. А~на~самом деле обрубать нужно ровно в~конце периода, то есть в~том месте,
где начинается повторение. Начало периода~--- это как раз~самое большое число. В этом месте и нужно
обрубать, игнорируя весь последний отрезок дроби, начиная с самого большого числа. Например,
в~идущем ниже примере мы доходим до~4 и~обрубаем. В~следующий раз можем обрубить перед второй
четверкой, и~т.\,д.
$$
\bfrac1{2+\bfrac1{1+\bfrac1{4+\bfrac1{2+\bfrac1{1+\bfrac1{4+\bfrac1{\ldots}}}}}}}.
$$
Но это был модельный пример, не относящийся ни к какому~$m$.

Например, бывает, что повторение начнется на~7 или 8 ступеньке дроби, или еще дальше. Число 61,
среди первых 100 чисел, самое неприятное в~нашем смысле. Ибо $\sqrt{61}$ очень долго раскладывается
в~цепную дробь, пока не~\textit{повторятся условия}, обеспечивающие циклическое повторение всех
выделяемых далее целых частей. И~поэтому самые маленькие решения уравнения $x^{2}-61y^{2}=\pm1$
будут больше миллиарда.

В~костромской области каждые полгода проводится школа для сильных школьников. Вот они у~нас где-то
за~часик этот корень из~61 раскладывали. Потом еще минут десять сворачивали дробь, и~на~выходе
получали два числа порядка миллиарда. Которые, если подставить в~наше уравнение, чудесным образом
дают решение уравнения Пелля.

Цепная дробь (или алгоритм Евклида, который ее породил) может быть изложена геометрическим образом.
Полезно знать, какая геометрия за~этим стоит. Ниже я~ее изображу.

Немного уточню теорему Лагранжа, что приблизит нас к~термину \textit{алгебраические числа}. Что такое
рациональное число? Мы договорились, что это <<целое делить на~целое>> (то есть $\bfrac mn$). Можно
написать и~по-другому. Рациональное число~--- это корень (то есть решение) уравнения $m-nx=0$.

Например, $\bfrac{17}5$~--- корень уравнения $17-5x=0$. Подставьте $x=\bfrac{17}5$ и~проверьте это.

К~чему мы приходим? К~более широкому подходу. Рациональные числа~--- это корни вот таких линейных
уравнений, то есть уравнений первой степени с~целыми коэффициентами.


Корнем какого уравнения является число <<корень из~двух>> (обозначим его просто~К)? Нужно написать
выражение с~иксом, у~которого целые коэффициенты, такое, что при~подстановке получится~0. Вот оно:
$x^{2}-2=0$.

Оно \text{2-й} степени. Вот я~и~говорю поэтому: К~--- число не~рациональное. Ведь это уравнение нелинейное,
оно второй степени.

А~если я~напишу: $x^{10}-3=0$?

Что я~получу на~выходе? Корень \text{10-й} степени из~3. Число не~рациональное, удовлетворяющее
уравнению, где слева стоит многочлен с~целыми коэффициентами.

\pagebreak

Напишем произвольное уравнение \text{2-й} степени: $ax^{2}+bx+c=0$ (тут, конечно, <<$a$>> не~равно нулю).

Такое уравнение вы, без~сомнения, изучали в~школе. Но~вы изучали его для~произвольных $a$, $b$, $c$. А~мы
будем рассматривать \textit{только целые}. То есть многочлен, в~котором целое число раз взята единица (либо
минус единица)~--- получилось <<$c$>>, потом целое число раз взят $x$ (с~тем или иным знаком)~--- это
будет <<$b$>>, и~целое число раз взят $x^{2}$ (это~--- <<$a$>>). Решаем квадратное уравнение по~известной
формуле:
$$
x=\bfrac{-b\pm\sqrt{b^{2}-4ac}}{2a}.
$$

Мы получили выражение, использующее при~своем построении операцию извлечения квадратного корня один
раз. Так вот, теорема Лагранжа звучит так: если $x$ является решением уравнения $$ax^{2}+bx+c=0$$
с~целыми коэффициентами, то тогда его цепная дробь будет либо конечной (если вдруг решение окажется
рациональным), либо периодической. Верно также обратное утверждение. Если цепная дробь устроена
так, что у~нее, начиная с~некоторого места, возникает периодическое повторение целых частей, то она
удовлетворяет такому уравнению с~целыми коэффициентами.

А~вот теперь, опираясь на~эту теорему, я~могу вам дать основное определение. Число называется
\textit{алгебраическим}, если оно является корнем хотя бы одного уравнения с~целыми коэффициентами
произвольной длины. Не~обязательно квадратного, как у~нас, а~произвольного (многочлен любой
степени).

А~\textit{трансцендентное число}~--- это число, которое не~является алгебраическим. С~этим связана долгая
история. Стоял вопрос, существуют ли трансцендентные числа вообще. Древним грекам было известно,
что длина диагонали квадрата не~является рациональным числом. Это было очень неудобно древним. Но,
с~другой стороны, она удовлетворяет элементарному квадратному уравнению, то есть является
алгебраическим числом. Возникает вопрос: все ли числа алгебраические? Ответ~--- нет. Математик Ж.~Лиувилль,
живший в~середине XIX~века, просто выписал конкретное число и~доказал, что оно
не~является алгебраическим. С~этого всё и~началось. На~самом деле алгебраических чисел неизмеримо
\textbf{меньше}, чем не~алгебраических, то есть трансцендентных.

Грубо говоря, если вы возьмете вещественную ось и~случайно воткнете в~нее булавочку нулевой
толщины, вы практически наверняка попадете в~неалгебраическое число.

Мы с~вами на~3~лекции какую-то задачу решали с~какой-то железкой (помните?~--- которую надо куда-то
отправить, чтобы она встала в~вертикальное положение). Если вы эту железку наугад взяли где-то,
со~свалки, установили на~шарнир и~стали отправлять ровно с~той силой, чтобы она за~бесконечное
время встала в~вертикальное положение, то сила, наугад взятая, будет трансцендентная. На~самом деле
сама железка тоже будет трансцендентной по~своей длине. Есть самая большая загадка, которая обычно
совершенно не~понятна людям, не~занимающимся математикой. Как это~--- \textit{бесконечности могут
быть разные}? Вот как можно представить, что бесконечности разные? Вроде бесконечность, она и~есть
бесконечность. Они все одинаковые. Это один наивный взгляд. Другой наивный взгляд на~вещи состоит
в~том, что, наоборот, почти все бесконечности разные. Вот, скажем, возьмем множество всех
натуральных чисел и~множество всех целых чисел. Каких больше?

\textbf{Слушатель:} Целых.

\textbf{А.С.:} <<В два раза больше>>, если вы думаете <<в наивном ключе>> о~бесконечности. Или, так
же наивно: <<Чисел, которые делятся на~3, в~3~раза меньше, чем всех натуральных чисел>>. А~и~тех,
и~других~--- бесконечно много.

Оказывается, математика дает безжалостный ответ, совершенно безжалостный: целых чисел \textbf{столько же},
сколько натуральных. Как говорил один шутник: <<На этот вопрос есть два мнения~--- одно из~них Мое,
а~другое~--- Неверное>>.\vadjust{\pagebreak} В~отличие от~обычного шутника, с~которым можно и~поспорить,
с~математическими <<шутками>> не~поспоришь, ибо они обоснованы строгими доказательствами.

\textbf{Слушатель:} Кстати, и~алгебраических чисел столько же.

\textbf{А.С.:} И~алгебраических чисел тоже столько же, сколько натуральных. И~рациональных чисел
столько же, сколько натуральных, потому что единственный правильный способ, единственный
непротиворечивый способ придать значению <<столько же>> какой-то научный смысл~--- это установить
между двумя множествами взаимно однозначное соответствие. То есть каждому целому сопоставить
некоторое натуральное и~наоборот. Сделать это для~множеств целых и~ натуральных чисел
весьма просто (рис.~\ref{f:c1}):

%$$
%\ldots,\  -5,\ -4,\ -3,\ -2\, -1,\ 0,\ 1,\ 2,\ 3,\ 4,\ 5,\ \ldots
%$$
%$$
%1,\ 2,\ 3,\ 4,\ 5,\ 6,\ 7,\ 8,\ 9,\ 10,\ 11,\ \ldots
%$$

%Рис. 1
\xPICi{c1}{Натуральных чисел хватает, чтобы перенумеровать все целые числа.}

То есть вы сможете все целые числа перенумеровать. У~вас каждое целое число в~конце концов получит
один однозначно определенный номер. Да, кажется, что целых чисел вдвое больше, чем натуральных,
но~это неправда, на~самом деле и тех, и других одинаковое количество. Мы их пересчитали (и~ни
одного не~пропустили). Грубо говоря, целые числа можно перечислить. Все целые точки плоскости тоже
можно перечислить (рис.~\ref{f:c2}).


%Рис. 2
\xPICi{c2}{Начнем с~точки $(0,0)$ и~будем обходить ее постепенно расширяющимися оборотами. Каждому
из~узлов при~этом присваивается какой-нибудь не~повторяющийся номер: точке $(0,0)$~--- номер 1,
точке $(1,1)$~--- номер 2, и~так далее.}

Начинаю с~точки $(0,0)$~--- это будет моя \text{1-я} точка, и~дальше по~спирали. И~в~конце концов
каждая целая точка плоскости получит свой один-единственный уникальный номер. Все номера будут
заняты, все целые точки плоскости будут перечислены. Представьте себе, что у~вас есть комната,
в~которой бесконечное количество стульев, и~в~нее заходит бесконечное количество учеников, как
понять, что это одинаковые количества?

\textbf{Слушатель:} Посадить учеников на~стулья.

\pagebreak

\textbf{А.С.:} И если они займут все стулья, каждый ученик сидит на~одном стуле, все стулья
заняты, и~стоящих учеников нет, то вы констатируете тот факт, что стульев и~учеников одинаковое
количество.

\medskip

\hrulefill

\smallskip

\textbf{Врезка 7. Как Кантор размышлял о~<<взаимно-од\-но\-знач\-ных>> процессах}

Выше было доказано, что учеников <<ровно столько, сколько стульев>> (притом и~тех, и~других~---
бесконечное количество). Но~ведь тем же способом я~сейчас докажу, что учеников (то бишь
натуральных чисел) БОЛЬШЕ, чем стульев (то бишь целочисленных точек на~плоскости). В~самом деле:
ученика номер~1 я~вообще отправлю домой <<как лишнего>>, на~первый стул посажу ученика \No~2,
на~второй~--- ученика \No~3, и~так далее. Итак, я~доказал два взаимоисключающих факта: 1)~что учеников
и~стульев одинаковое количество; и~2)~что учеников БОЛЬШЕ, чем стульев. А~захотел бы~--- доказал бы
и~что 3)~стульев БОЛЬШЕ, чем учеников. Так что же, для~бесконечных множеств, что ли, в~принципе
нельзя сказать, какое из~них <<больше>>?!

Другой бы ученый на~том и~успокоился. Но~гениальность Кантора позволила ему найти верную дорогу
в~этом мраке. Он спросил себя: а~как было с~этими теоремами для~\textbf{конечных} множеств? Для~конечных эти
три теоремы несовместимы друг с~другом, и~любая из~них может быть использована для~выяснения, какое
из~множеств больше: учеников или стульев. А~какой же из~трех надо пользоваться для~сравнения
бесконечных множеств? Оказалось, что для~бесконечных множеств надо взять за~основу \textbf{способ~1}: если
хоть каким-то образом удалось установить взаимно-однозначное соответствие между учениками
и~стульями, значит, торжественно объявляем эти два множества <<равномощными>> и~не~поддаемся ни
на~какие провокации типа <<способа~2>> или <<способа~3>>. Только так можно построить
непротиворечивое сравнение множеств по~мощностям. <<В наказание>> за~это Кантору пришлось доказать
несколько труднейших теорем, которые, к~сожалению, не~только нельзя <<по-простому>> пояснить
гуманитариям, но даже и у~будущих математиков
 (студентов \text{2-го} курса мехмата МГУ) с~пониманием их
доказательства возникают большие проблемы. Но~хотя их трудно понять и~воспроизвести, их уже нельзя
<<запретить>> подобно тому, как пифагорейцы хотели <<запретить>> иррациональные числа~--- ведь
Кантор всё обосновал строго математически, а~другие с~этим согласились.

\smallskip

\hrulefill

\medskip

\textbf{Слушатель:} А~если у~нас, допустим, две сферы, маленькая\linebreak и~большая?

\textbf{А.С.:} Как множества точек это одно и~то же (то есть они равномощны). Объясню на~примере окружностей (вместо сфер).
Установим взаимно однозначное соответствие (рис.~\ref{f:c3}).

%Рис. 3
\xPICi{c3}{Проводя лучи из~общего центра двух окружностей, устанавливаем
взаимно-однозначное соответствие между их точками. (Более того, оно же
отвечает важному условию: близким точкам одной окружности соответствуют
близкие точки другой.)}

Школьник маткласса узнаёт всё это, скажем, в~\text{9-м} классе. И~вот тут у~него, как и~у~Кантора,
возникает мысль: а~может, любое бесконечное множество можно пересчитать? Тогда все бесконечные
множества одинаковые. Возьмем отрезок $[0,1]$ и~пересчитаем его. Получат ли все точки отрезка номера?

Нет. И~это можно формально доказать (Кантор сделал это). Пересчитать точки отрезка
\textbf{невозможно}. И~так как внутри отрезка заведомо уживается бесконечное число точек вида $\bfrac 1n$,
параметризуемое натуральными числами~--- например, множество чисел вида, то мы говорим о~том, что
отрезок имеет как бесконечное множество большую мощность, он больше как бесконечное множество, чем
множество натуральных чисел. На~отрезке, на~окружности, на~плоскости больше точек, строго больше,
чем натуральных чисел.
 Где-то в~конце XIX~века Г.~Кантор понял, что \textit{бесконечности бывают разные}.

Сейчас я~докажу, что множество, любое множество (какое бы оно ни было, конечное или бесконечное),
и~множество его подмножеств~--- не~одинаковы. (Второе множество ОБЯЗАТЕЛЬНО будет больше
по~мощности.)

Сначала возьмем конечный случай. Пусть у~нас есть множество. Оно состоит (например) из~трех чисел:
0, 1 и~2. Подмножество~--- это какая-то компания, составленная из~них. Какие могут быть компании?
Во-первых, может быть компания, в~которой нет ни одного числа. Ну, как говорят, \textit{пустое множество}.
<<Никого в~нём нет>> называется компания. Но~математики никак не~могут без~этого обойтись, они
просто не~могут. Без~нуля и~без~пустого множества математика не~живет. Что значит <<На день
рождения пришло пустое множество гостей>>? Это означает, что вы накрыли стол, и~никто не~явился.
Математик скажет:\vadjust{\pagebreak} <<Ко мне на~день рож\-де\-ния пришло пустое множество гостей>>. Потом, возможно,
пришел только господин~0. И~сразу множество перестало быть пустым!

Продолжаем <<придумывать ``компании''>>. Так сказать, \textit{кампания по~нахождению компаний} (шутка).
Возможно, в~гости пришел не~Господин~0, а~Господин~1 или Господин~2. Вот вам уже целых четыре компании: одна пустая и~три
из~одного <<человека>>. Эти последние могут даже побеседовать\ldots\ сами с~собой (<<с умным человеком
и~поговорить приятно>>).
\begin{gather*}
\emptyset\quad
\text{(пустое множество)},\\
\{0\},\quad
\{1\},\quad
\{2\}.
\end{gather*}
Какие еще варианты?
$$
\{0,1\},\quad
\{0,2\},\quad
\{1,2\}.
$$
Все варианты перечислили?

Еще могли прийти все. Итого~--- 8~разных компаний.
$$
\{0,1,2\}.
$$

Или такая задача. Вы начальник группы. И~вы хотите кого-то наградить. Сколькими способами вы можете
решить эту задачу? Вы можете наградить одного, можете не~награждать никого. Можете наградить двух,
можете всех трех. Сколько у~вас способов решить эту задачу? У~вас 8~вариантов, потому что 8~подмножеств.

Так вот, ни для~какого (ни конечного, ни бесконечного) множества нельзя пересчитать подмножества,
используя элементы исходного множества. Подмножеств гораздо больше, чем элементов. У~нас элементов
всего 3, а~подмножеств оказалось 8. Не~хватит. Если элементов было бы 5, то подмножеств будет 32
штуки. Для~конечных понятно~--- не~пересчитаешь. Я~хочу сказать, что такого не~может быть ни
для~каких вообще множеств. Это доказал Г.~Кантор.

Смотрите. Как мы могли бы доказывать теорему о~том, что множество и~множество его подмножеств
не~одинаковы.

Рассмотрим множество натуральных чисел $1,2,3,4\ldots$ И~множество всех подмножеств этого
множества, например, все четные, или все делящиеся на~3, или все кубы чисел, начиная с~тысячи, и~т.\,д.

Используем доказательство от~противного (но~в~несколько необычной обстановке). Предположим, что мы
смогли пересчитать множество подмножеств. Подмножество четных чисел получило, скажем, номер~15.
Подмножество нечетных получило номер~3. Подмножество <<Все четные, начиная с~десятки>> получило
номер 156. Числа, делящиеся на~3, как множество, получили номер 1376, отдельно взятое подмножество
из~чисел, которые между ста и~тысячью лежат, получило номер 1000000 и~т.\,д.

Допустим, мы пересчитали все подмножества. Приведем это допущение к~противоречию.

Рассмотрим все натуральные числа, для~которых <<их>> подмножество (то есть подмножество с~таким
номером) \textbf{не~содержит} этого числа.

Скажем, четные числа получили номер~15, 15~--- нечетное число, то есть, подмножество, ему
соответствующее, его не~содержит.
 Значит, 15~--- это как раз нужное нам число.

А~если, например, подмножество состоит из~чисел $\{101, 102,\allowbreak 103,\ldots, 200\}$ и~получило номер 195, оно нам
не~подходит, так как 195 лежит внутри своего подмножества. Значит, натуральное число 195 нам
не~подходит.

Далее Кантор сделал шаг к~следующему этапу рассуждения. Он рассмотрел все такие числа, собрал их
в~кучу и~обозвал это подмножеством~$B$.
%Почему не~$A$? Да потому, что это просто ЛЮБОЕ подмножество
%натуральных чисел~--- множество~$A$.

Подмножество $B$ вполне конкретное~--- это все числа, которые сами не входят
в~подмножество с~их номером.
%Подмножество же~$B$ не~любое, а~вполне конкретное~--- это все числа, для~которых оно само не~входит
%в~подмножество с~их номером.
То есть 15 вошло в~$B$, 195~--- не~вошло. И~так далее. Этому
подмножеству $B$ тоже должен быть присвоен некий натуральный номер~$b$. Это же подмножество. Но~если
каждому подмножеству присвоен номер (по~нашему предположению), то такому подмножеству тоже присвоен
номер. Вопрос: число~$b$ входит ли в~подмножество~$B$? С~какой вероятностью вы встретите крокодила
на~улице (если вы не~знаете вообще, что такое <<крокодил>>)?

\textbf{Слушатели:} 50 на~50.

\textbf{А.С.:} Да, правильно, девушки дорогие! Вот это вы правильно говорите. Либо встретите, либо
не~встретите. Правда? Значит, номер $b$ либо принадлежит подмножеству $B$, либо не~принадлежит. Сейчас
я~докажу, что \textbf{не~может быть ни того, ни другого}. То есть сейчас я~докажу, что вы не~можете ни
встретить крокодила, ни не~встретить. Ни то, ни другое не~может произойти. И~это будет то самое
противоречие, которое будет устанавливать тот факт, что соответствия между множеством и~множеством
его подмножеств не~бывает. Потому что оно выведено, исходя из~того, что мы смогли устроить такое соответствие.

Поехали. Я~утверждаю, что <<инвентарный>> номер подмножества, которое состоит из~таких натуральных
чисел, что их собственное подмножество их не~содержит, не~может ни содержаться в~$B$, ни
не~содержаться в~$B$. Предположим, что номер $b$ содержится в~$B$. Это значит, что он не~может входить
в~множество тех чисел, которые в~своих множествах не~содержатся.

\textbf{Слушатель:} И~значит, он не~содержится в~множестве~$B$.

\textbf{А.С.:} И~значит, он не~содержится в~$B$. А~теперь представьте себе, что он не~содержится в~$B$.

\textbf{Слушатель:} Но~тогда он должен содержаться, потому что он элемент~$B$ по~определению множества~$B$.

\textbf{А.С.:} Да. Тогда он должен содержаться в~$B$. То есть если он содержится, то он
не~содержится, а~если он не~содержится, то содержится. Теорема доказана методом <<от противного>>,
ибо мы пришли к~чисто логическому противоречию.

Вот она, математическая логика. Добро пожаловать! Каждый \text{6-й} логик, как говорят, сходит
с~ума. Это мне говорил мой учитель, он тоже математический логик, но не~шестой.

Делаем дальнейший вывод~--- это множество подмножеств\linebreak \textbf{больше}, чем само
множество. (Подсказка: во множестве всех подмножеств находятся все одноэлементные подмножества.)
Заманчивой является мысль, что это не~только для~подмножеств натурального ряда чисел справедливо,
но~и~вообще для~подмножеств ЛЮБОГО множества. Но~понятие <<любое множество>> так вдохновило
некоторых математиков, что в~ход пошли совершенно ужасающие множества типа <<множество всех
мыслимых множеств>>. (Или, например, множество плохо совместимых слов <<огород>>, <<бузина>>,
<<Киев>>, <<дядька>> из~известной поговорки.) И~возникли крупные математические проблемы с~такими
множествами. Но~теория Кантора выдержала это нашествие <<безумных множеств>>. Просто пришлось
внести необходимые уточнения в~некоторые исходные понятия.

Вот еще один яркий образчик <<безумных математических объектов>>. Рассмотрим некоторый шар.
Например, футбольный мяч. Есть способ разбить этот шар на~конечное число кусков, из~которых потом
можно будет составить ровно \textbf{два шара} такого же размера. То есть вы берете футбольный мяч, берете
ножницы, разрезаете мяч на~несколько кусков, они совершенно безумно устроенные, но~все-таки куски.
Потом кххх... и~у~вас два футбольных мяча. Всё. Математики~--- такие вот фокусники.

\textbf{Слушатель:} Такого же размера?

\textbf{А.С.:} Абсолютно такого же размера.

\textbf{Другой Слушатель:} Но~это в~теории возможно?

\textbf{Слушатель:} Только в~теории и~возможно. А~на~практике?

\textbf{А.С.:} А~на~практике ножницы должны быть устроены <<неизмеримым образом>>, так сказать. Эти
куски \textbf{не~имеют объема}. Представление обычного человека о~том, что любая объемная фигура имеет
объем, не~соответствует реальности. Далеко не~у~любой пространственной фигуры можно посчитать
объем. Далеко не~у~любой плоской фигуры можно вычислить площадь. Но~вы не~можете себе представить такую фигуру.
Их выдумали математики. Фигуры эти страшные, и~они никогда не~возникнут ни в~какой
реальности. Но~в~теоретических построениях они есть, и~без~них вот никуда. (А~в~процессе обучения
такие <<безумные>> примеры позволяют лучше понять, как устроены окружающие нас РЕАЛЬНЫЕ предметы.)
На~прямой, например, есть объекты, у~которых \textbf{нет длины}. У~каждой физической теории есть ареал,
в~котором она применима. Механика применима для~размеров порядка нас с~вами, но~она неприменима
для~размеров порядка атома, там она не~работает.
 Там совершенно другая физика. В~экономике
макросистемы, которые генерируют такие вещи как инфляция, безработица и~так далее, они абсурдны
для~множества из~двух, трех человек. И~мое философское мнение, может быть, оно совершенно
дилетантское, что у~математики тоже есть ареал, но~он находится в~мозгу человека. Там вселенная
совершенно другая, там нет воздуха, привычной атмосферы нет. Математика тоже ограниченно применима,
она не~универсальна, она совершенно в~другой области живет, поэтому в~другой области нужно искать
ограничения.

\textbf{Слушатель:} Как называется направление в~математике, где можно разрезать шары таким невероятным
образом?

\textbf{А.С.:} Теория меры. То, что мы называем площадью и~объемом, математики называют мерой.
Совершенно страшные объекты получают меру, а~некоторые не~получают. Это очень существенно. Не~любое
множество можно измерить. В~этом кроются границы того, что подсказывает нам наша интуиция. Она говорит про
очень простые вещи, про многоугольники, многогранники, сферы, шары. Про что-то, достаточно просто
устроенное. У~чего всегда можно измерить объем, площадь. Если разрезать футбольный мяч
на~нормальные <<человеческие>> куски нормальным <<человеческим>> ножом, то вы, конечно, никогда
не~получите двух футбольных мячей, просто из~соображений объема. То есть идея в~том, что объема
у~тех кусков, которые участвуют в~теореме, нет. Никакого. Ни нулевого, ни положительного, никакого
нет. Но~когда куски эти сложат вместе, у~полученного мяча может быть вполне определенный объем.
(<<Объединение двух неизмеримых кусков может быть измеримым>>.)

А~теперь я~все-таки расскажу вам про алгоритм Евклида в~геометрических терминах. Давайте возьмем
дробь $\bfrac3{14}$ и~превратим ее в~цепную дробь:
$$
\bfrac3{14} =
0 + \bfrac1{\bfrac{14}{3}} =
0 + \bfrac1{4 + \bfrac1{\bfrac32}} =
0 + \bfrac1{4 + \bfrac1{1 + \bfrac12}}.
$$
А~теперь смотрите, что я~делаю геометрически. Беру прямоугольник $3\times14$ (рис.~\ref{f:c4}).

%Рис. 4
\xPICi{c4}{Исходный прямоугольник размера $3\times14$.}

%Рис. 5
\xPICi{c5}{Режем пирог методом Евклида.}

Отрезаю от~прямоугольника квадраты (рис.~\ref{f:c5}). Остается прямоугольник 2 на~3. Отрезаю
от~него квадрат. Остается 2 клетки. Вот они и~есть наши целые части. 4 больших, 1 поменьше и~2
совсем маленьких.

Это и~есть алгоритм Евклида. Такой красивый, геометрический способ.

Как показать на~картинке, что рациональное число обязательно раскладывается в~конечную цепную
дробь? Рациональное число~--- это как бы прямоугольник на~клетчатой сетке. Потому, что у~него верх и~низ
целые.

Вот, скажем, $\bfrac{105}{13}$~--- это прямоугольник 105 в~ширину и~13 в~высоту.
%%Рис. 6
%\xPICi{c6}{Теперь отрезаем не~куски $3\times3$, а~куски $13\times13$.}
105 и~13~--- это целые числа, то есть у~нас целое количество квадратиков. Теперь мы начинаем наш
геометрический алгоритм Евклида. Отрезаем, пока можем, огромные квадраты $13\times13$, проводим здесь
границу~--- это наша целая часть. Граница идет по~целым клеткам, потому что отрезали целое
количество квадратиков. Оставшаяся фигура~--- целочисленный прямоугольник. Отрезаем квадраты
от~нее. Остается еще меньший прямоугольник. Понятно, что за~конечное время всё будет вырезано.
Каждый раз минимум один квадратик удаляется. Поэтому в~какой-то момент квадратики закончатся.
Теперь вспомним фокус-покус,
 который я~провел выше с~числом <<корень из~двух>> (точнее, с числом $\sqrt2+1$, см. рис.~\ref{f:c7}).

%Рис. 7
\xPICi{c7}{Прямоугольники, да не~те\ldots\ Стороны их несоизмеримы.}

Начинаем делать ровно то же самое. Отрезаем квадратик\linebreak (рис.~\ref{f:c8}). Потому что мы ищем, сколько раз
единица укладывается в~<<корне из~двух плюс один>>. Она укладывается ровно два раза.
Какие оказываются у~этой конфигурации стороны? $1$ и $\sqrt2-1$.
%Дорисуем к~нашему исходному прямоугольнику слева квадрат со~стороной 1.
Тогда прямоугольник
$ABCD$ подобен $BCD'A'$.

%$A$\quad
%$B$\quad
%$C$\quad
%$D$\quad
%$D'$\quad
%$A'$\quad
%$1$\quad
%$\sqrt2+1$\quad
%$\sqrt2-1$\quad

%Рис. 8
\xPICi{c8}{На~горизонтальном прямоугольнике выделены два квадрата,
остался кусочек, подобный исходному прямоугольнику.}

То есть если мы перевернем и~увеличим $BCD'A'$ в~некоторое количество раз, то получим $ABCD$.
Доказательство нужно? Сейчас будет. Что такое подобие? Это~--- <<сильная похожесть>> фигур. Углы у~прямоугольников одинаковые, не~хватает лишь пропорциональности сторон.
Анализируем, есть ли она.
$1:(\text{К}+1)$ равно ли $(\text{К}-1):1$? <<Напоминаем, что $\text{К}$~--- это корень из~двух>>.


В~прошлый раз мы доказывали, что это одно и~то же. Подобие имеет место. Что же произойдет дальше?
Если мы начнем дальше отрезать квадратики, мы опять получим подобие, и~так будет до~бесконечности
(рис.~\ref{f:c9}).

%Рис. 9
\xPICi{c9}{<<У попа была собака, он ее любил.
Она съела кусок мяса, он ее убил~--- в~землю закопал, камень положил. На~камне написал:
``У попа была собака\ldots'' и~так далее до~бесконечности>>.}

Теорема доказана. Из-за подобия мы будем эту операцию бесконечное количество раз проделывать,
а~значит, ни на какой сетке наш прямоугольник размеров $(\sqrt2+1)\times 1$ не может лежать~--- и, следовательно, $\sqrt2+1$ не
будет рациональным числом.

Что-то это мне напоминает\ldots\ В~детстве у~меня была книга. Она называется <<Вот так история!>>. Там
был мальчик. Он ужасно себя вел. Все были воспитанные, а~он был невоспитанный. И~вот этого мальчика
отправили в~невоспитанный город, где у~него сразу старик отнял кровать, выгнал его, стал спать
в~этой кровати. Потом на~его подушке выросло невоспитанное дерево, мальчика разбудило. Его
все стали обижать, на~улице все толкались, и~он попросился обратно. Просто детский триллер, да
и~только. На~обложке этой книги, однако, была изображена нетривиальная картина.


%Рис. 10
\xPICi{c10}{На~обложке книги сидел дед. Рядом сидели внуки.
Дед держал такую же книгу, как исходная, но~поменьше.
А~на~ее обложке сидел дед (поменьше), внуки (поменьше),
дед держал книгу (еще меньше). А~на~обложке дед (еще меньше\ldots), и~так далее.}

Я~на~нее гляжу, гляжу\ldots\ Это был момент, когда мой папа понял, что
я~математик.
%Я~тогда ничего не~понял, я~был совсем маленьким. Я~просто посмотрел на~картину,
%на~которой сидит дедушка, на~коленях у~него сидел внук. И~дедушка читал внуку эту книгу.
Я~сам тогда ничего не~понял, я~был совсем маленьким. Я~просто посмотрел на~картину,
на~которой сидит дедушка и читает внуку эту книгу.
А~теперь
представьте на~секунду, что это означает? Это значит, что на~маленькой книжке на~картинке
изображены дедушка и~внук, и~у~дедушки в~руках та же книга, на~которой изображены дедушка и~внук.
Я~говорю: <<Папа, так это\ldots\ Это же до~бесконечности повторяется! Это же повтор, а~значит, это
должно быть до~бесконечности, да?>> В~прямоугольниках (рис.~\ref{f:c9})~--- то же самое.

С~этим связана еще одна интересная задача. Есть две карты города, разного масштаба. Одна карманная,
другая~--- большая настенная карта. Предположим, что кто-то взял, сорвал со~стены большую карту
и~кинул на~нее маленькую. (Карты подобны по~форме) (см. рис.~\ref{f:c11}).

Доказать, что можно взять иголку и~проткнуть эти две карты в~одной и~той же точке изображаемого ими города.

Это вроде как игра. Я~беру вот эти две карты и~думаю, как бы мне положить верхнюю на~нижнюю, а~вы
приходите и~иглой протыкаете, где захотите; если вы проткнули иглой одну и~ту же точку (дом~37
по~улице Профсоюзной), значит, вы выиграли, а~если нет, то выиграл я. Теорема утверждает, что
в~этой игре проигрывает тот, кто кладет карту. То есть, как бы ни~положить карты, всегда можно
указать нужную точку.

Доказательство~--- в~одну строчку. Методом <<взгляни~--- и~поймешь>>.

\pagebreak

%Рис. 11
\xPICi{c11}{Ничто не~предвещало
появления Ее величества Бесконечности\ldots}

Нарисуем на~маленькой карте ту территорию местности, которую на~большой карте закрыла маленькая карта.

Теперь нарисуем на~нарисованной карте ту территорию, которую она занимает на~маленькой (рис.~\ref{f:c12}).

%Рис. 12
\xPICi{c12}{И~завертелись карты города аж до~бесконечности!}

Дальше они будут вертеться до~бесконечности, но~в~пересечении всех этих карт будет точка. В~нее
и~надо воткнуть иголку.

А~теперь немножко сложнее: я~беру две абсолютно одинаковые карты города. Верхнюю снимаю, сжимаю,
комкаю, складываю, но~не~рву. Теперь кидаю на~оставшуюся лежать карту так, чтобы верхняя не~вылезла
за~пределы нижней (рис.~\ref{f:c13}).

%Рис. 13
\xPICi{c13}{Иллюстрация к теореме Брауэра.}


\textbf{Теорема.} Всё равно можно проткнуть иголкой эти две карты в~одном месте. Всегда, что бы вы
ни делали (единственное только~--- нельзя резать и рвать). Если вы карту порвали, то можно добиться
того, чтобы проткнуть было негде. А~вот если мы не~рвали, то всегда найдется общая точка, иногда их
будет несколько, но~одна найдется обязательно. При~условии, что смятая до~неузнаваемости
(и~сплющенная) верхняя карта целиком лежит на~нижней. Эта теорема очень эффектна. На~самом деле,
она утверждает нечто про произвольное непрерывное отображение объекта в~себя. Эта теорема не~очень
простая, я~ее рассказываю на~курсе <<математика для~экономистов>> и~в~школе анализа данных Яндекса.
Называется она \textit{Теорема Брауэра}.

На~самом деле, пока она не~была доказана, в~нее не~очень верили. Любое непрерывное отображение
(разрешено всё, кроме разрывания) замкнутого выпуклого объекта (в~теореме Брауэра говорится
о~замкнутом шаре) в~себя всегда обладает неподвижной точкой.
 То есть точкой, которая никуда ни сдвинулась. Вы что-то растягиваете, что-то сжимаете, что-то
складываете, но~вы никогда, никогда не~добьетесь того, чтобы \textit{все} точки изменили свое
положение. Этого \textit{нельзя} сделать. Этому есть математическое препятствие, и~оно называется
<<теорема Брауэра о~неподвижной точке>>.

\centerline{* * *}

\pagebreak

Вернемся к~задаче, которой мы закончили предыдущую лекцию.
$$
(1+\sqrt2)^{2},\
(1+\sqrt{2})^{3},\
\ldots\
(1+\sqrt2)^{n}
$$
--- эти вот выражения почему-то тоже помогали нам решать уравнение Пелля.

Сейчас как раз самое время открыть секрет. Заодно получим еще одно оправдание изучению уравнения Пелля:
$$
x^{2}-2y^{2}=1.
$$
Греки мыслили геометрическими образами. Старались увидеть число, увидеть теорему Пифагора. У~них были <<квадратные>> и~<<треугольные>> числа.

Например, 4 или 9~--- это квадратные числа (рис.~\ref{f:c14}).

%Рис. 14
\xPICi{c14}{Из~4 или из~9 кружков можно сложить квадрат.}

Что такое треугольное число? Это когда из~такого количества кружочков можно
треугольник собрать.
3, 6, 10~--- числа треугольные (см. рис.~\ref{f:c15}).

%Рис. 15
\xPICi{c15}{Перед началом партии в~биллиард шары укладывают в~<<треугольник>>.}

Следующее 15, потом 21. Каждый раз прибавляем на~1 больше, чем в~предыдущий раз.

Сам собой возникает вопрос: \textit{бывает ли так, что одно и~то же число и~квадратное, и~треугольное?} То
есть количество фишек таково, что можно собрать из~них квадратик, а~можно перемешать и~собрать
треугольник.

\textbf{Слушатель:} Число 1 и~такое, и~такое.

\textbf{А.С.:} Безусловно. Человек, который говорит <<число 1>>, обладает математическим мышлением.
Не~пропустить даже простейшего случая. Это очень важно.

Однажды я~ехал в~поезде из~Иркутска в~город Тулун. И~со~мной в~плацкарте ехала женщина с~дочкой лет
пяти. Мама явно не~математик, но~при~этом хочет дочку чему-то научить. И~она спрашивает: <<Вот,
смотри. У~тебя пять кукол. Как их можно разложить? 3 и~2>>.~--- <<Ну, да>>.~--- <<А еще можно
как-нибудь?>> Я~с~интересом наблюдаю. Тут дочка и~говорит: <<Можно $5+0$>>.

Я~вскакиваю с~полки, спускаюсь и~говорю: <<Ваша дочь имеет нетривиальные, очень хорошие
математические способности>>.


Мама немножко помолчала, а~потом согласилась. Но~она не~поняла. Ведь назвать $5+0$ может только
человек, у~которого четко развита логика, другой человек не~назовет, это нетривиальный вариант.

Вернемся к~треугольным и~квадратным числам. Какое следующее, после 1? Следующее <<и такое,
и~такое>> число~--- это 36 (см. рис.~\ref{f:c16}).


%Рис. 16
\xPICi{c16}{Число 36~--- <<дважды чемпион>> среди натуральных чисел.}

Давайте найдем общую формулу для~всех чисел такого рода.

\textbf{Слушатель:} 36 на~6, ну, 36 на~36 умножить?

\textbf{А.С.:} Давайте, во-первых, выведем формулу для~треугольных чисел. То есть, грубо говоря,
есть формула для~всех квадратных: $n^{2}$. Подставляете любое число, получается квадрат. А~вот как
написать общую формулу для~чисел 1, 3, 6, 10, 15\ldots\ Вот что нужно сделать с~$m$, чтобы получить
треугольное число?

Что получается? $1+2+3+4+\ldots+m$.

Нужно посчитать такую сумму. Вот оно, треугольное число. Как посчитать такую сумму? Есть знаменитая
история про то, как Гаусс быстро в~уме подсчитал сумму первых подряд идущих ста чисел.
 (Но~это, мне кажется, байка.)
Маленький Гаусс учился в~школе в~\text{3-м} классе. В~школе к~учителю или к~учительнице пришел знакомый.
Учительница решила дать задачу такую, чтобы дети занялись на~весь урок. <<Дети, а~теперь посчитайте
$1+2+3+$ и~так далее до~100>>. И~ушла довольная. Выбегает маленький Гаусс через 5~минут, говорит: <<Я посчитал: 5050>>.

<<А как ты посчитал? А~ты можешь доказать?>>~--- <<Ну конечно, могу. Смотрите. Я~пишу две строки:
\begin{gather*}
1+2+3+\ldots+100,\\
100+\ldots+3+2+1.
\end{gather*}
По-другому просто перенумеровал. Сумма внизу та же самая будет. Пусть она равна~$x$. И~сверху~$x$ и~снизу~$x$.
\begin{gather*}
1+2+3+\ldots+100=x,\\
100+\ldots+3+2+1=x\text{>>}.
\end{gather*}

Давайте теперь сложим строчки по~столбикам: $1+100$, $2+99$, $3+98$, \ldots

\textbf{Слушатель:} Всегда получится 101.

\textbf{А.С.:} Конечно. А~сколько штук?

\textbf{Слушатель:} 100.

\textbf{А.С.:} 100. Значит, удвоенное значение нашего выражения равно 101 умножить на~100. Откуда
после сокращения на~2, естественно, и~получается $x$ равно 50 умножить на~101.
$$
2x = 10100,
$$
отсюда
$$
x = 5050.
$$


Вот Гаусс и~сказал 5050. И~был совершенно прав, ничего не~считая. Математика~--- это искусство
лени. Математик~--- это тот, кто никогда не~будет делать рутинных действий, он всегда придумает
что-то машинное. Вот вы познали какой-то рутинный метод. Всё. Вы теперь на~нём не~зацикливаетесь,
на~нём будет зацикливаться компьютер. Компьютер ничего не~выдумает, а~математик, он свалит
на~компьютер всю рутину и~найдет закономерность. В~Брауншвейге Гауссу стоит памятник: бронзовый
17-угольник, на~котором стоит математик. А~почему он стоит на~17-угольнике? Потому что Гаусс
придумал, как строить правильный 17-угольник циркулем и~линейкой. <<Правильный>>~--- значит
с~равными сторонами и~углами.

До~него эту задачу не~могли решить. Можно построить правильный треугольник, 4-угольник. На~самом
деле, если вы умеете строить многоугольник с~простыми сторонами, то остальное легко. То есть надо
строить: правильный треугольник, 5-угольник, 7-угольник. А~7-угольник никто строить
не~умеет. Все мучились и~думали: <<Что ж~такое? Какие-то мы глупые, наверное. Почему мы не~можем
построить правильный 7-угольник циркулем и~линейкой?>>

А~потом уже после Гаусса пришел Ванцель и~сказал: <<Это невозможно. Математически невозможно>>. И~доказал это. Так же, как нельзя построить правильный
11- и~13-угольник. Помните, я~рассказывал про теорему Галуа? Про то, что для~уравнения выше
\text{4-й} степени нельзя написать общую формулу корней. Здесь та же ситуация, вы можете взять циркуль
и~линейку, вооружиться ими и~хоть всю жизнь строить какие-то дуги, что-то пересекать, но~вы никогда
не~сможете построить правильный 7-угольник. Ванцель доказал это в~1836~году. Но еще значительно
раньше девятнадцатилетний Гаусс сумел построить правильный 17-угольник. Какая следующая фигура строится циркулем и~линейкой
из~правильных многоугольников? После 17-угольника? Оказывается, 257-угольник. Это очень
долго и~сложно, но~можно.

Вернемся к~треугольным числам.

Понятно теперь, как мы будем выводить общую формулу? Мы запишем всё наоборот. Мы здесь запишем
\begin{gather*}
1+2+3+4+\ldots+m=x,\\
m+(m-1)+(m-2)+\ldots+1=x.
\end{gather*}
Теперь сложим и~получим $m$~экземпляров какого числа? Числа $(m+1)$:
$$
(m+1)m=2x.
$$
Поэтому формула вот такая:
$$
x=\bfrac{m(m+1)}{2}.
$$
Теперь можно подставлять вместо $m$ любые натуральные числа и~получать треугольные.

А~вот теперь задается вопрос. Итак, число является треугольным, значит оно имеет такой вид
$$
\bfrac{m(m+1)}{2}.
$$
С~другой стороны, оно квадратное, то есть имеет вид: $n^{2}$.

Получается замечательное уравнение для~решения в~целых числах
$$
m(m+1)/2=n^{2}.
$$
А~теперь смотрите? Может быть, вы помните, что такое \textit{делители} числа? Делитель числа $a$~---
это такое число, на~которое~$a$ делится (без~остатка).

$m$ и~$(m+1)$~--- два соседних числа. Значит, одно из~них точно четное, а~значит, делится на~2.
Значит $m(m+1)/2$~--- произведение двух целых чисел. Четное поделится на~2, а~нечетное не~поделится.
Важно так же, что у~соседних чисел не~может быть общих делителей.

15 делится на~3, 16 нет. На~3 делится каждое третье число.

16 делится на~2 и~на~4, но~на~3 не~делится.

\textbf{Слушатель:} А~единица?

\textbf{А.С.:} Да. Единица. Но~единицу мы за~делитель не~считаем, на~нее всё делится. Еще пример.
28 делится на~7. Следующее число, которое делится на~7~--- 35, а~предыдущее~--- 21. Значит 27 на~7
не~делится. То есть $m$ и~$(m+1)$ точно не~имеют общих делителей.

В другой части нашего уравнения написан квадрат: $n^{2}$.

Его можно разложить на~простые множители. И~каждый такой множитель будет входить в разложение $n^{2}$ в четной степени. Например, если $n$ делится
на~5, то $n^{2}$ делится на~$5^{2}$.

Значит, чтобы выполнялось наше равенство, $m(m+1)/2$ тоже должно делиться на~$5^{2}$. То есть на~25. Но~$m$
и~$(m+1)$ не~имеют общих делителей, значит одно из~них делится сразу на~$5^{2}$.

И~это будет верно для~каждого простого делителя числа $n$. Иными словами наше равенство возможно,
только если каждый из~$m$ и~$(m+1)$ является квадратом\footnote{Этот переход <<скомкан>> и вовсе
неочевиден. Подробнее речь об этом пойдет во второй части книги.}.

Я, между прочим, в~этой лекции обошел одну тонкость, которая называется \textbf{\textit{основная теорема
арифметики}}. В~школе ее тоже обходят. Звучит она так: \textit{любое число однозначным образом раскладывается
на~простые множители}. Школьников обманывают, говорят: <<Ну, это очевидно>>. Действительно очевидно~--- если вы
посидите, как я, и~пораскладываете все числа от~одного до~1000 на~множители, то вы, конечно,
убедитесь в~этом. Но к абсолютному доказательству такая очевидность отношения не имеет.


Если же поверить в~эту теорему, то получается следующее. Если $n$ делится, скажем, на~11, то $n^{2}$
делится на~$11^{2}$, на~121. Значит, и~$m(m+1)$ делится на~$11^{2}$. Но~11 не~может входить и~в~$m$, и~в~$(m+1)$.
Либо $m$ делится на~11 в~квадрате, либо $(m+1)$ делится на~11 в~квадрате.

Эта важная истина говорит о~том, что если $m(m+1)/2=n^{2}$, то либо $m=a^{2}$ и~$(m+1)/2=b^{2}$ в~случае если
$m$~--- нечетное, а~$(m+1)$~--- четное, либо наоборот $m+1=b^{2}$ и~$m/2=a^{2}$ (если наоборот).


Отсюда уже один шаг до~уравнения Пелля $x^{2}-2y^{2}=1$.

В~первом случае получаем:
$$
m=a^{2}\quad
\text{и}\quad
(m+1)/2=b^{2} \Rightarrow m+1=2b^{2},
$$
а~так как $m$ и~$m+1$ соседние числа, то $a^{2}-2b^{2}=-1$.

Во~втором случае получаем:
$$
m+1=b^{2}\quad
\text{и}\quad
m/2=a^{2} \Rightarrow m=2a^{2} \Rightarrow b^{2}-2a^{2}=1.
$$
Оба раза мы пришли к уравнению Пелля $x^{2}-2y^{2}=\pm 1$.


То есть, решая это уравнение, мы будем получать \textbf{квадратно-треугольные} числа.

Ребенок, который играет в~эти кружочки и~хочет составить одновременно квадратное и~треугольное
число, вынужден решать уравнение Пелля.


Давайте посмотрим. Какие у~нас были решения? 41 и~29.
$$
41^{2}-2\cdot29^{2}=-1.
$$
Следовательно
$$
(m+1)/2=29^{2},\quad
m=41^{2},\quad
n^{2}=29^{2}41^{2},\quad
n=1189.
$$
Кто бы мог подумать, что когда-нибудь мальчик выложит такой треугольник. В~нём должна быть 1681~строка.
Представляете, какую площадь займет этот треугольник!

Но~я~всё ухожу от~ответа про $(\sqrt2+1)^{2}$, хотя и~обещал его вам.

Итак. Почему же, независимо от~степени, у~нас всегда получалось решение уравнения
$$
x^{2}-2y^{2}=1?
$$
Возвожу, например, в~\text{4-ю} степень. (На~самом деле, можно возвести в~любую.)
\begin{gather*}
(1+\sqrt2)^{4}=
(1+\sqrt2)(1+\sqrt2)(1+\sqrt2)(1+\sqrt2),
\\
(1-\sqrt2)^{4}=(1-\sqrt2)(1-\sqrt2)(1-\sqrt2)(1-\sqrt2).
\end{gather*}
Это классический бином Ньютона. Чтобы раскрыть все эти скобки, нам нужно каждый раз из~каждой
скобки взять либо $\sqrt2$, либо~1. Представьте себе, какие из~этих операций дадут целое число, а~какие
будут давать \textit{число с~корнем из~двух}. Во-первых, целое получается, если я~отовсюду взял единичку.
С~другой стороны, если я~из~двух скобок взял корень из~двух, а~из~двух единичку, тоже будет целое.
Если я~из~четырёх скобок возьму $\sqrt2$~--- тоже целое. А~вот если из~трёх скобок взять или из~одной, то
получится число с~корнем. И~после того, как я~сложу, у~меня получится выражение вида $m+n{\sqrt2}$.

В~$m$ сидят все способы раскрытия скобки, где я~беру с~корнем четное число скобок, а~все остальные разы
единичку. А~в~$n$~--- все, в~которых я~взял с~корнем нечетное количество скобок, а~из~остальных~--- единичку.


А~теперь посмотрим на~скобку с~минусом. Получится то же самое, за~одним исключением. Когда я~возьму
$\sqrt2$ нечетное число раз, у~меня получится число с~минусом. Таким образом, при перемножении получается знак минус ровно
у тех слагаемых, которые кратны $\sqrt2$.
 Поэтому после того, как мы всё сложим, у~нас
получится $m-n{\sqrt2}$
\begin{gather*}
m+n{\sqrt2}=(1+\sqrt2)(1+\sqrt2)(1+\sqrt2)(1+\sqrt2),\\
m-n{\sqrt2}=(1-\sqrt2)(1-\sqrt2)(1-\sqrt2)(1-\sqrt2).
\end{gather*}
А~теперь давайте перемножим две наши строчки.
$$
(m+n{\sqrt2})(m-n{\sqrt2})=m^{2}-2n^{2}.
$$
Напоминает Уравнение Пелля, не~правда ли?

Наконец, перемножим правые стороны уравнений. Я~не~зря вам тут про Гаусса рассказывал.\vadjust{\pagebreak}
От~перестановки множителей ведь ничего не~меняется. Поэтому я~могу в~моем произведении перемножать
скобки в~любом порядке. Перемножу их по~столбцам:
$$
(1+\sqrt2)(1-\sqrt2)=1-2=-1.
$$

Ну и~тогда у~нас получается после перемножения по~столбцам
\begin{gather*}
(1+\sqrt2)(1+\sqrt2)(1+\sqrt2)(1+\sqrt2),\\
(1-\sqrt2)(1-\sqrt2)(1-\sqrt2)(1-\sqrt2),
\end{gather*}
в ответе $(-1)(-1)(-1)(-1)$.

А~что получится, если $(-1)$ умножается на~себя много раз? 1 или $(-1)$. То есть, когда мы будем
возводить $(1+\sqrt2)$ в~\textit{четную} степень, будет $(+1)$, а~в~\textit{нечетную}~$(-1)$.

Но~с~другой стороны уравнения у~нас стояло $m^{2}-2n^{2}$. Получаем $m^{2}-2n^{2}=\pm1$.

То есть мы доказали, что в~любой степени $(1+\sqrt2)^{n}$ порождает решение нашего уравнения Пелля.

Теперь \textbf{несколько вопросов} до~следующей лекции.

\centerline{* * *}

1. Вы залезли на~вершину Хибинских гор. Высота их примерно 1~км. И~посмотрели вдаль. А~там~---
дома. Вам померещилось, или это Мурманск? Могли ли вы увидеть Мурманск? На~сколько километров вдаль
можно увидеть с~километровой горы? Обратите внимание, земля круглая, поэтому сильно далеко
не~увидишь. На~сколько километров видно с~Эвереста? С~20-этажного дома? Если кто-нибудь,
например, говорит: <<Я тут пролетал из~Тбилиси в~Дели и~Москву на~горизонте видел>>. Он врет или
возможно такое? Это задача, которая возникает, когда вы идете в~горы.

2. Вы стали астрономом. И~наблюдали за~звездами. Наблюдали, живя в~городе Москве, а~потом
по~семейным обстоятельствам перебрались в~Санкт-Петербург. Вы обнаружили какие-нибудь новые звезды,
которые из~Москвы не~видно? Как устроены между собой два множества звезд этих двух городов?
(Множество звезд, которые видны из~Петербурга, и~множество звезд, которые видны из~Москвы.)
Случилось солнечное затмение. Летним днем в~Москве стали видны звезды, которые никогда не~видны
летним днем в~Москве. Это~--- новые звезды, например, <<Южный крест>>, или это те звезды, которые вы
уже видели над Москвой?

3. Вы идете на~лыжах 22~марта. Начинает темнеть, и~вы вспоминаете, что 22~декабря, когда вы шли
на~лыжах, и~солнце зашло за~горизонт, вы успели добежать до~деревни Морозки до~полной темноты. Вы
идете в~том же самом месте. Успеете ли вы добежать до~деревни Морозки или нет? Тот же вопрос про 22~июня и 22~сентября (но уже вряд ли на лыжах).
Когда быстрее всего наступает темнота после
захода солнца, когда медленнее?

4. Вы подошли с~вашей маленькой двухлетней дочкой к~детской площадке. И~обнаружили там некоторое
количество качелей разного вида.

Картинка: качели, рассчитанные на~двоих детей. Одни с~ручкой для~рук (см. рис.~\ref{f:d14-0}), другие с~ручкой для~ног (см. рис.~\ref{f:d16-0}).
А~также совсем простые качели в~виде обычной доски (см. рис.~\ref{f:d17-0}).

%Рис. 14
\xPICi{d14-0}{Качели с ручками для рук.}

%Рис. 16
\xPICi{d16-0}{Качели с ручками для ног.}

%Рис. 17
\xPICi{d17-0}{Качели без ручек~--- в виде обычной доски.}


Какие качели в~порядке (то есть в~горизонтальном состоянии), а~какие качели вы увидели, скорее
всего, в~перекошенном состоянии?
 Как это зависит от~расположения спинки?

Вы опять пришли и~увидели, что качели устроены как абсолютно плоская доска. Просто обтесанное
с~двух сторон бревно, и~больше ничего. Но~одни качели в~положении равновесия висят, вторые всё
время скатываются набок. Почему?


5.~ Алиса летит сквозь Землю. Помните сюжет книги Л.~Кэр\-ро\-ла <<Алиса в~стране чудес>>?

Алиса летит сквозь Землю и~думает, что она к~антиподам прилетит. В~самом центре Земли выбегает
гномик и~дает ей пинок так, что увеличивает скорость ее полета на~1~метр в~секунду. Вопрос:
на~какое расстояние она вылетит из~Земли вверх благодаря этому дополнительному метру в~секунду?

6. В~лиге чемпионов в~группе четыре~команды. Они играют каждая с~каждой. Причем каждая с~каждой
играет и~у~себя, и~в~гостях. То есть в~каждой группе между каждой парой команд произойдет ровно 2~матча.
В~случае ничьей каждой команде дают 1~очко. Тот, кто победил, получает 3~очка.
Проигравший~--- ноль очков. А~теперь~--- внимание! С~каким минимальным количеством очков еще можно
попасть в~следующий раунд? В~следующий раунд попадают две~команды из~четырёх. Вы должны привести
конкретный расклад. Кто с~кем играл, какие баллы получил. И доказать, что с меньшим числом очков <<выйти из группы нельзя>>.

7. Каково максимально возможное количество родных прапрабабушек?

Обсуждение ответов~--- на~лекции~5.

\endinput
