%1.1.
\section{Лекция 2}
\label{1.2}

\textbf{А.С.:} Сегодня мы займемся тем, что называется \textit{топологией}. Многие cчитают ее центральной наукой
в~математике. Математика~--- это центральная наука во~всех науках. Топология получается тогда как бы  <<центром
внутри центра>>, то есть самой главной дисциплиной.
 Она сформировалась в~начале XX~века,
и~постепенно стало ясно, что она лежит в~сердце математики. На~простом языке, топология~--- это
геометрия плюс анализ. А~можно сказать и~по-другому: тот, кто хочет понять самые глубокие и~важные
закономерности и~геометрии, и~математического анализа, должен изучать эти науки с~топологической
точки зрения.

{\tolerance=7000

100~лет назад топология уже достаточно хорошо оформилась, а~началась она, наверное, с~Эйлера (того
самого Эйлера, формулу которого мы сегодня будем с~вами изучать). Были сформулированы определения
важнейших объектов топологии: \textit{линия, поверхность, объём, многомерное пространство. Было осознано,
что у~топологических объектов имеется важное свойство: \textbf{размерность.}} Например, линия~--- это
одномерный объект (его можно при этом поместить в~1-мерное пространство, в~2-мерное, в~3-мерное и~даже
в~так называемое <<4-мерное пространство>>). Поверхность~--- двумерный объект (он может
располагаться в~2-мерном пространстве, в~3-мерном, 4-мерном и~так далее).
 Тело, имеющее
положительный объём~--- это 3-мерный объект; но~оно может располагаться в~3-мерном, 4-мерном,
5-мерном\ldots\ пространствах. Ниже всё это будет рассматриваться в~самых простых случаях, поскольку
свойства топологических объектов, лежащих в~4-мерном, 5-мерном, 6-мерном\ldots\ пространствах недоступны
непосредственному геометрическому восприятию человека. Может быть, это хорошо, что человек не~может
совершить даже небольшую и~короткую по~времени прогулку в~<<подлинное>> 4-мерное пространство.
Вернувшись из~такой прогулки, этот бедняга мог бы с~ужасом обнаружить, что сердце у~него теперь
находится не~с~левой, а~с~правой стороны (и~ему, кроме того, придется примириться с~тем фактом, что
он стал левшой, хотя ранее им не~был).
 Так что с~4-мерным пространством шутки плохи.
Но~и~в~3-мерном пространстве (казалось бы, так хорошо нам знакомом) топология сумела обнаружить ряд
совершенно сногсшибательных фактов. Приступим же к~ее изучению (конечно, на~общеописательном
уровне, не~достигая стопроцентной строгости изложения).

}

Допустим, у~вас есть глобус, или футбольный мяч, или арбуз. Это объекты по~сути разные, а~по~форме
они одинаковые. Как говорится на~житейском языке, это тела, которые имеют форму шара. Однако с~точки зрения
топологии арбуз резко отличается от~глобуса и~от~футбольного мяча: арбуз внутри заполнен веществом,
а~глобус и~мяч внутри пустые. Разумно считать, что толщина картонной поверхности глобуса и~толщина
оболочки мяча имеют нулевой размер. Тогда глобус и~мяч являются двумерными объектами, а~арбуз~---
трехмерным. Но~можно мысленно рассматривать поверхность арбуза~--- получится <<двумерный объект,
ограничивающий исходный трехмерный арбуз>>. Ниже мы будем говорить просто о~поверхности шара (неважно,
какого диаметра).
 Допустим, что мяч имеет диаметр 20~см, поверхность арбуза~--- диаметр 50~см,
а~глобус~--- 200~см. Для~лучшего понимания, что такое топология, рассмотрим также кубик со~стороной
20~см, склеенный из~бумаги, и~таких же размеров кубик, сделанный из~кусочков проволоки, идущих
вдоль ребер куба. Итого у~нас имеется пять объектов. С~общежитейской точки зрения их можно
разделить на~две группы~--- <<круглые>> (3~шт.) и~<<кубообразные>> (2~шт.). С~точки зрения человека,
привыкшего всё измерять сантиметром (например, портного), их надо разделить на~две группы
по~другому принципу: <<предметы с~размерами порядка 20~см>> (3~шт.) и~<<более крупные предметы>> (2~шт.).
А~с~точки зрения математика-тополога, здесь имеются четыре \textbf{абсолютно одинаковых} предмета
и~один особенный (а~именно, проволочный куб). И~тополог даже даст обоснование, почему он так
считает: первые четыре объекта являются двумерными, а~последний объект~--- одномерный. Таким
образом, топология не~только не~видит разницы между поверхностью шара диаметра 20, 50 или 200~см,
\textit{но~и~не~видит разницы между поверхностью куба и~поверхностью шара!} Итак, тополог надевает на~себя
<<волшебные очки>>, которые не~позволяют определить ни размеры, ни форму предметов. Что же он тогда
через них сможет разглядеть?
 \textit{Он сумеет разглядеть самое глубинное отличие представленных ему
предметов друг от~друга, их, так сказать, конструкцию.} Например, добавим к~этим пяти предметам еще
и~бублик с~внешним диаметром 20~см и~будем интересоваться не~самим бубликом, заполненным тестом,
а~только его поверхностью. А~также добавим обыкновенное кольцо из~проволоки (диаметром 1~см). Что
скажет тогда тополог? <<С~точки зрения размерности здесь имеется два типа объектов: двумерные
и~одномерные. Но~поверхность бублика резко, принципиально отличается от~поверхности шара. Точно
так же проволочный кубик резко отличается от~кольца из~проволоки. Итак, здесь представлены \textbf{четыре}
различных топологических типа: поверхность шара (4~предмета), поверхность бублика, окружность,
проволочный кубик>>.

\medskip

\hrulefill

\smallskip

\textbf{Врезка 1.
Упражнение для~слушателей (необязательное; но ответ полезно прочесть)}

Во~времена фашистской Германии в~ней процветали ученые-шарлатаны. Один из~них на~полном серьезе
утверждал, что всё космическое пространство вокруг Земли заполнено\ldots\ льдом. (То есть, что мечтать
о~космических полетах бессмысленно.) Ну, допустим, это так и~есть. Хм. Рассмотрим тогда три
объекта: поверхность Земли, внутренность Земли и~наружная часть Земли, состоящая, хм, изо льда. Как
называются эти объекты на~языке топологии? Одинаковы ли с~точки зрения топологии второй и~третий
объект?

\textit{ОТВЕТ. Первая часть ответа:} первый объект~--- двумерный, типа сферы. Не~имеет граничных точек.

Второй объект: 3-мерный, типа шара. Его граничные точки~--- все точки поверхности Земли.

Третий объект: 3-мерный, типа шарового слоя. Граничные точки~--- все точки поверхности Земли.

\textit{Вторая часть ответа:} второй и~третий тип топологически различны, так как шаровой слой
существенно отличается от~шара.
 Граничные точки у~них тем не~менее \textbf{одинаковы}.

\textit{Третья часть ответа:} не~следует говорить, что третий объект <<бесконечный по~размерам>>, так как
в~топологии неважно, каковы размеры объектов. Например, если взять поверхность сферы и~выкинуть
из~нее одну-единственную точку, то по~житейским представлениям этот объект <<конечный по~размерам>>,
в~то время как плоскость <<бесконечна>>. По~правилам же топологического исследования, сфера
с~<<выколотой>> точкой имеет тот же топологический тип, что и~плоскость.

\smallskip

\hrulefill

\medskip

Возьмем и~\textit{изогнем, изомнем, растянем} поверхность шара, но нигде \textit{не~порвем}, и~\textit{не~склеим} никакие две
точки в~одну. Мы можем из~нее таким образом получить, например, куб (то есть, естественно, не~сам
куб, а~его поверхность). Чтобы понять, как это делается, покажем, как из~круга, изготовленного
из~резины, получить квадрат (размеры квадрата неважны). Для~этого надо в~четырех равноудаленных
местах границы круга потянуть наружу резиновый слой, пока он не~примет форму квадрата. В~частности,
точки границы круга превратились в~точки периметра квадрата.

Можно много чего сделать из~резиновой камеры сдутого футбольного мяча. Но~есть \textit{интуиция},
которая подсказывает, что автомобильную (или велосипедную) камеру из~камеры футбольного мяча
сделать будет затруднительно, даже используя те широкие возможности, которые предоставляет нам
топология. Куб, эллипсоид (то есть сжатая поверхность сферы), яблоко, арбуз~--- пожалуйста, а~вот
бублик из~шара не~сделаешь, не~порвав его, либо не~склеив между собой некоторые точки. Согласно
сказанному выше, надо различать две разные задачи: 1)~\textit{Из~заполненного шара сделать
заполненный бублик} и~2)~\textit{Из~поверхности шара сделать поверхность бублика.} Первая задача <<решена>>
в~подписи к~рис.~\ref{f:a1}.

%Рис. 1
\xPICi{a1}{Слева~--- шаровой кусок теста, справа~--- бублик из~теста.
Пекарь (или лектор?) взял левый кусок теста, раскатал его так, чтобы из~него получился удлиненный цилиндр
(в~топологии заполненный цилиндр неотличим от~заполненного шара), согнул его и~слепил концы этого
цилиндра. Вот и~получился из~шара бублик\ldots\ Стоп-стоп. Слеплять (то есть склеивать точки) нельзя! Тип
объекта изменился.}

И~Эйлер задался вопросом, а~можно ли это утверждение доказать? Вроде бы интуитивно оно совершенно
понятное.
 Но~математика ставит задачу перевести очевидное на~язык строго доказанного. Ведь если мы
откроем цивилизацию, которая, например, живет на~плоскости, для~ее жителей будет не очевиден
рассматриваемый нами факт (см. врезку~2). А~с~помощью математики мы сможем передать им содержание
теоремы. К~чему я~клоню?

\medskip

\hrulefill

\smallskip

\textbf{Врезка 2. Эйнштейн~--- о~топологии}

Однажды А.~Эйнштейна попросили совсем кратко, на~понятном любому языке, пояснить, в~чем состоит
суть сделанных им открытий. Он ответил: все мы, люди, словно маленькие жучки с~завязанными глазами,
ползающие по~поверхности большого мяча и~воображающие, что двигаемся по~плоскости. Я~же первый
понял, что мир, в~котором я~живу, \textit{искривлен}. Но~пока не~совсем понятно, как именно он искривлен.
(То есть, <<по-научному>>, каков топологический тип космоса.)

\smallskip

\hrulefill

\medskip

А~вот к~чему. Несколько лет назад математик Г.~Перельман установил похожий факт, но~только
в~пространстве больших измерений. Факт про фигуры в~многомерном пространстве, которые
локально похожи на искривленное трехмерное пространство.
 Мы живем в~трехмерном пространстве,
мы четвертого измерения не~видим и~не~чувствуем. Мы можем только рассуждать, что четвертое
измерение~--- это время, но~объять его взором не~можем.
 Поэтому мы не~можем говорить так
спокойно и~убежденно, что сделать из~шара тор в~пространстве больших измерений нельзя. (Ведь
в~4-мерном пространстве, как указывалось выше, МОЖНО, не~нарушая правил топологии, превратить
незаметным образом человека с~сердцем, расположенным слева, в~человека с~сердцем, расположенным
справа.)

Нам нужен язык, на~котором это можно доказать. И~вот для~того, чтобы это можно было доказывать,
для~того чтобы через много лет Перельман смог доказать <<гипотезу Пуанкаре>> (после того как ее
доказали, она вместо гипотезы Пуанкаре стала называться теоремой Перельмана или
Пуанкаре~--- Перельмана), Эйлер начал большой путь. Он перевел то, что мы с~вами считаем очевидным,
в~точное, железобетонное математическое рассуждение. Как же он это сделал? Он нарисовал
на~поверхности шара, мяча, арбуза, глобуса, любого круглого объекта некоторую карту. Иными словами,
некий искривленный многогранник (рис.~\ref{f:a2}).

%Рис. 2
\xPICi{a2}{Начало большого пути в~топологию.}

С~точки зрения топологии, любой многогранник~--- это тоже шар. Тетраэдр~--- это шар, куб~--- это
шар, октаэдр, любой параллелепипед~--- это всё шары. Например, потому что если их выполнить из резины и надуть, то получится футбольный мяч, то есть шар.
 Но~до~работ Эйлера еще не~было <<точки зрения
топологии>>, так как не~было и~самой топологии.

Эйлер <<чувствовал>>, что все эти объекты \textit{одинаковые}. В~чём именно? И~как это объяснить остальным
людям? В~особенности его интересовал вопрос: как доказать, что поверхность шара, поверхность
бублика, поверхность кренделя \textit{неодинаковые?}\footnote{Мы не можем знать этого наверняка, но мне кажется, Эйлер просто не мог пройти мимо такого вопроса!}
 В~ответ на~первый вопрос ясность позже внес Анри
Пуанкаре
 (после того, как Огюст Коши внес должную ясность в~вопрос, что такое <<непрерывная
функция>>). Однако Эйлер сразу обратился ко второй задаче (о~доказательстве неодинаковости двух
поверхностей) и~блестяще решил ее.

Эйлер сделал следующее. Он нанес на~поверхность шара многогранник~--- картиночку <<стран>>, причем
страны необязательно треугольные (рис.~\ref{f:a3}). (Если говорить о~<<странах>>, то надо помнить, что рассматривается
<<Земной шар>>, не~содержащий морей и~океанов.) При этом вся поверхность шара должна быть покрыта многоугольниками.


%Рис. 3
\xPICi{a3}{А~вы не~пробовали жить на~<<Земном шаре>> в~форме огромного бублика?
А~заметили бы жители, что это не~шар, если бы небо всегда было закрыто
беспросветными тучами?}

Главное, чтобы каждая страна была простым плоским объектом, без~дырочек,~--- как круг или квадрат.
И~далее он сделал то же самое с~велосипедной камерой. Нанес такой многогранник, который является
как бы <<остовом>> каретного колеса (машинных колес в~то время еще не~было!).
 При~этом вовсе
не~обязательно, чтобы количество и~вид граней, а~также количество вершин и~ребер этого
многогранника для~шара и~для~колеса были одинаковы. Более того, они и~не~могут быть одинаковыми
(как мы увидим ниже).

А~потом стал считать у~этих многогранников эйлерову характеристику: величину $\text{В}-\text{Р}+\text{Г}$.

Число вершин минус число ребер плюс число граней. Как бы мы ни мяли и~ни изгибали шар, наши
грани~--- <<страны>> от~этого не~меняются. (Но, конечно, нельзя так смять страну, чтобы она вся
превратилась в~отрезок. Такого даже во~время наполеоновских войн не~происходило! А~если говорить
серьезно, то отрезок~--- одномерный объект, а~страна~--- двумерный.) То есть вершины остаются
вершинами, ребра~--- ребрами, а~грани~--- какими были (например, изогнутым пятиугольником или
треугольником), такими и~остались.
 А~значит, величина $\text{В}-\text{Р}+\text{Г}$ не~меняется. Теперь считаем эту
величину на~колесе (по~науке поверхность колеса (или бублика) называется словом <<ТОР>>. А~тор,
заполненный внутри, называется \textit{полноторием}. Поверхность же шара называется, как известно,
\textit{сферой}).
 И~если сфера может перейти в~тор, то картинка на~шаре перейдет в~картинку на~колесе. И,
значит, их эйлерова характеристика должна быть одинакова.

Докажем, однако, что у~любой фигуры, нарисованной на~колесе, эйлерова характеристика равна 0,
а~у~любой фигуры на~шаре~--- равна 2.

\textbf{Слушатель:} А~если бы получилась одна и~та же цифра, то что?

\textbf{А.С.:} Мы не~смогли бы сделать из~этого никакого вывода. Мы бы не~смогли сделать вывод, что
они одинаковые, но~не~смогли бы сделать и~вывод, что они разные. Но~ведь есть и~другие подходы,
кроме формулы Эйлера. Для~более сложных случаев.

\textbf{Слушатель:} Понятно.

\textbf{Слушатель:} А~как взаимосвязаны картинки на~торе и~шаре?

\textbf{А.С.:} То есть как именно они друг с~другом соотносятся? Никак. Каждая из~картинок, независимо друг
от~друга, является как бы <<сетью>>, наброшенной на~данную поверхность. Эту сеть при~желании можно
сделать состоящей из~треугольных ячеек. Тогда она называется <<триангуляцией поверхности>>.

\textbf{Слушатель:} А~не~может быть такого, что будет то же самое количество вершин, ребер и~граней,
но~при~этом картинка будет другая?

\textbf{А.С.:} Смотря, что понимать под словом <<другая>>. Она может, безусловно, немного иначе выглядеть:
ребра могут быть длиннее или короче. Но~мне достаточно того, чтобы имелось то же самое количество
вершин, ребер и~граней. А~при~изгибах, растяжениях и~сжатиях поверхности это будет именно так.

\textbf{Слушатель:} А\ldots

\textbf{А.С.:} Итак, если вы поверили, что не~изменится ни количество вершин, ни количество ребер, ни
количество граней, то всё остальное я докажу совершенно строго.
 Я~продемонстрирую, что величина $\text{В}-\text{Р}+\text{Г}$
на~шаре и~на~торе разная: на~автомобильной камере она равна 0, на~сфере~--- равна~2.

\textbf{Слушатель:} А~если предположить, что дырка у~тора имеет площадь ноль. По-прежнему число Эйлера~---
0?

\textbf{А.С.:} А~что значит <<площадь дырки>>? Это значит, что бублик сходится в~одной точке~--- в~серединке?
%(рис.~\ref{f:a4})
%
%%Рис. 4
%\xPICi{a4}{Дырка в~торе уже исчезла; край дырки склеился в~одну точку.
%Ещё усилие~--- и~верхний край отклеится от~нижнего, и~получится что-то
%в~роде <<эритроцита>>. А~его поверхность уже имеет топологический тип сферы.}

\textbf{Слушатель:} Да.

\textbf{А.С.:} Нет, эйлеров индекс $\text{В}-\text{Р}+\text{Г}$ будет другой. Фигура, которая получится, не~устроена как обычная
плоскость в~окрестности любой своей точки,
потому что в~окрестности серединки, где дырка сходится с~разных сторон, она устроена очень сложно.

Чтобы понять это, рассмотрим сечение тора (с~заклеенной дырой) вертикальной плоскостью, проходящей
в~стороне от~точки заклейки, а~также плоскостью, проходящей через точку заклейки. Рассмотрим две
замкнутые кривые, получившиеся в~сечениях (см. рис.~\ref{f:a4n}).

\xPIC{a4n}

Первая кривая устроена как окружность, окрестность любой ее точки~--- просто интервал, а~вторая
кривая устроена иначе (рис.~\ref{f:a5}). Потому что в~любой микроскоп окрестность точки пересечения видится
как крест, а~не~как отрезок. То же самое с~тором~--- с~автомобильной камерой. С~точки зрения
таракана, который по~ней ползает, это просто плоскость (если, конечно, дырка в~торе не~была
заклеена). Но~и~шар с~точки зрения таракана~--- тоже плоскость (ведь он в~каждый момент времени
видит только маленький кусочек <<у~себя под носом>>, а~он почти плоский). То есть смотрите, что
происходит. Таракан, который ползает по~тору и~по~шару, не~может понять, что это разные объекты. Мы
такие же тараканы, мы живем в~трехмерном пространстве, мы~--- трехмерные тараканы. Мы знаем, что
вокруг нас есть окрестность. Окрестность~--- это обычное трехмерное пространство: его определяют 3
взаимно перпендикулярных оси. То есть я~вижу трехмерную окрестность вокруг себя, но~я~не~знаю, как
устроена вся вселенная целиком. Я~не~могу иметь такого представления. Так вот: топология приоткрыла
эту тайну. Гипотеза Пуанкаре как раз про то, как устроено пространство, где мы живем. Мы видим,
что вокруг нас всё трехмерно, но~мы не~знаем внутри какого рода объекта мы живем. То ли мы живем
в~обычном бесконечном трехмерном пространстве, то ли мы живем на~поверхности трехмерной, извините,
сферы, которая ограничивает четырехмерный шар. Не~можем мы этого понять, просто посмотрев вокруг
себя. Ведь радиус такой <<трехмерной сферы>> может равняться, скажем, 100~миллионам световых лет.
А~на~такие расстояния глаз посмотреть не~способен.

%Рис. 5
\xPICi{a5}{Слева~--- простая замкнутая кривая (не~пересекает сама себя). Справа~---
что-то вроде дороги с~перекрестком. Топологический тип этих двух одномерных объектов разный.}

\newpage

\medskip

\hrulefill

\smallskip

\textbf{Врезка 3.
Еще одно \textbf{упражнение} для~слушателей. Ниже описано странное путешествие неких космических Магелланов.
Могло ли такое быть в~космосе?}

\ldots\ Все астрономы Земли в~3333~году нашей эры были в~глубоком недоумении. Один из~них,
направляя свой телескоп в~разные точки небесной сферы, имел привычку фотографировать не~только ее,
но~и~(перейдя в~другое полушарие Земли), фотографировать также диаметрально противоположную ей
точку. Накопив изрядное количество таких пар фотографий, он принялся их изучать. И~вдруг~---
сюрприз: на~одной из~двух фотографий пары он увидел маленькое, но~вполне различимое созвездие
в~виде \textit{правильного пятиугольника}. Велико же было его изумление, когда на~другой фотографии пары он
увидел ТАКОЕ ЖЕ созвездие, той же величины и~той же яркости! Велико было и~удивление всех остальных
астрономов, когда они услышали это сообщение (и~немедленно проверили его). И~скоро об этом узнали
все жители Земли. Было решено одновременно выслать две космических экспедиции (на~предмет проверки,
не~посылают ли на~Землю сигналы внеземные цивилизации): одна экспедиция~--- прямо в~центр первого
пятиугольника, вторая~--- в~центр диаметрально противоположного пятиугольника.

Долго летели космонавты в~ту и~в~другую сторону с~одинаковой <<субсветовой>> скоростью~--- целых
10~лет. И~всё это время за~их ракетами наблюдали чуткие приборы астрономов. Вдруг в~центре первого
5-угольного созвездия была зафиксирована яркая вспышка неправильной формы, и~первая ракета ИСЧЕЗЛА.
Астрономы решили взглянуть, видна ли вторая ракета. К~своему ужасу, они увидели, что ровно в~тот же
момент с~диаметрально противоположной стороны была зафиксирована вспышка ТОЙ ЖЕ ФОРМЫ, и~вторая
ракета тоже исчезла.

Могло ли такое быть?

\textbf{ОТВЕТ.} \textit{Могло. Если бы только космос, в~который погружена Земля, был не~бесконечным трехмерным
пространством, а~очень большой, но~конечной трехмерной сферой.}

Чтобы лучше понять это, представьте себе, что наша Земля сплошь покрыта мировым океаном, на~котором
имеется (на~экваторе) только один небольшой остров вроде Крита. Поверхность этого океана является
двумерной сферой, но~свойства у~нее похожи на~свойства трехмерной сферы. И~выплыли с~этого острова
два одинаковых корабля (в~один и~тот же момент времени): один поплыл ровно на~запад, другой~---
ровно на~восток. Плыли они быстро и~потому очень сильно столкнулись (в~точке, диаметрально
противоположной острову Криту). От~столкновения они могли взорваться. После отплытия прочие люди
следили за~ними, посылая вслед радиоволны (а~они, как известно, могут огибать поверхность Земли).
На~экране радара и~на~западе, и~на~востоке всё время был виден какой-то странный правильный
пятиугольник (оказалось, что это~--- радиомаяк из~пяти источников, построенный кем-то
на~противоположной точке поверхности Земли). Корабли взорвались как раз в~центре этого
пятиугольника. Взрыв был зафиксирован одновременно и~западным, и~восточным радаром.

\smallskip

\hrulefill

\medskip

Сверху из~нашего трехмерного мира мы видим, что тор и~сфера~--- разные объекты. Но~глазами червя,
который ползает по~двумерной поверхности, этого не~видно, всё одинаковое. Вопрос: как же доказать
червю, что поверхности разные?

Допустим, что у~червя есть мышление, он может воспринять математическое рассуждение. Как я~могу
передать ему знание? А~вот как. Я~ему говорю: <<Ты можешь, экспериментально исползав сферу,
проверить, сколько здесь вершин?>> Он говорит: <<Ну, конечно могу. Я~постепенно все их обползаю,
поставлю метку, найду алгоритм, которым я~посчитаю количество вершин>>. Тогда я~спрошу: <<Можешь ли
ты посчитать количество ребер?>>~--- <<Ну, конечно, могу>>,~--- говорит он. <<А граней?>>~--- <<Тоже
могу. Нет проблем никаких. Каждый раз переходя из~грани в~грань, заливаю ее водой. В~следующий раз
я~к~ней приду, а~она уже мокрая, значит, я~ее уже посчитал>>. Понятно, что, находясь на~двумерной
поверхности, не~выходя в~трехмерное пространство, можно посчитать, сколько ребер, вершин и~граней.
Теперь, если я~пересажу червя на~тор, он посчитает вершины, грани и~ребра и~убедится, что индекс
Эйлера имеет другое значение. На~сфере~--- 2, а~на~торе~--- 0. Тут я~ему и~скажу: <<Теперь ты
понимаешь, что поверхности абсолютно разные, они с~нашей человеческой трехмерной точки зрения
абсолютно разные. Они с~твоей точки зрения одинаковые, потому что ты видишь локально, а~с~нашей
трехмерной~--- они разные>>. То же самое происходит с~нашей трехмерной вселенной, с~точки зрения
четырехмерного пространства. Наше пространство может быть устроено по-разному, но~Г.~Перельман
доказал теорему, которая ограничивает класс того, что нам нужно проверять, когда мы выясняем, где
живем.

\textbf{Слушатель:} А~как Эйлер пришел именно к~этой формуле?

\textbf{А.С.:} Честно говоря, я~не~знаю, но~он вообще был гений. Говорят, что у~него никогда не~было
математических ошибок и~неверных утверждений. Даже не~совсем обоснованные рассуждения Эйлера (после
их очевидной коррекции) были впоследствии подтверждены. Видимо, он настолько верно чувствовал
ситуацию, как будто внутри него находился <<барометр правильности>>, с~которым он постоянно
сверялся.

%\pagebreak

Математика~--- это прозрение. Вы идете по~парку, вокруг листья шелестят, бах~--- и~вы всё поняли.
Это не~от~вас, это как бы сверху идет.

Сейчас я~буду доказывать, что на~сфере индекс Эйлера равен 2, а~на~торе он равен 0, и, может быть,
вам будет ясно, как Эйлер к~этому пришел.

%Рис. 6
\xPICi{a6}{Накидываем <<сеть>> из~ребер и~вершин на~верхнюю половину сферы и~на~небольшой кусок
поверхности тора. Нижняя часть сферы может трактоваться как одна гигантская грань (грани
не~обязательно должны быть треугольными). Оставшийся кусок тора НЕ МОЖЕТ считаться <<гранью>>, так
как грань не~может выглядеть как трубка. Надо эту трубку подразбить на~более мелкие части
(на~треугольники, квадратики и~т.\,д.).}

Допустим, я~уже сформировал <<сеть>>, покрывающую сферу, и~<<сеть>> для~тора (рис.~\ref{f:a6}).

Стираю одно ребро на~сфере (потом буду стирать ребра и~на~торе). Что меняется вот в~этом нашем
выражении (то есть $\text{В}-\text{Р}+\text{Г}$)?

\textbf{Слушатель:} Минус одно ребро.

\textbf{Слушатель:} Минус одна грань.

\textbf{А.С.:} Значит, выражение $\text{В}-\text{Р}+\text{Г}$ не~изменилось (рис.~\ref{f:a7}).

%Рис. 7
\xPICi{a7}{Укрупняем <<страны>> за~счет удаления участков границ.
Начинать удаление можно с~любого ребра.}

Какие еще операции я~могу сделать с~этой картинкой? Могу убрать еще одно ребро. Опять
ничего не~изменится.
 Но~в~какой-то момент меня ударят по~рукам. Некоторые вершины могут стать
странными (что-то вроде куска забора в~чистом поле).

Может получиться <<висячая вершина>>~--- она связана с~единственным ребром (может быть
и~несколько таких кусков, см. рис.~\ref{f:a7n}).

%Рис. 7n
\xPIC{a7n}

%\pagebreak

Давайте превратим вот такое ребро во~что-нибудь человеческое (только не~в~человеческое ребро!).
Что для~этого надо сделать?

\textbf{Слушатель:} Выпрямить.

\textbf{А.С.:} Да. Удалить вершину и~выпрямить границу, убрав ненужный <<кусок границы>>. Что изменилось?

\textbf{Слушатель:} Минус вершина.

\textbf{Слушатель:} Минус ребро.

\textbf{А.С.:} Минус ребро, потому что из~двух соседних ребер стало одно. Заметьте, что в~выражении
$\text{В}-\text{Р}+\text{Г}$ опять ничего не~изменилось. Итак, я~буду упрощать картинку дальше (см. рис.~\ref{f:a8}).

%Рис. 8
\xPICi{a8}{Одна из~стран явно стала <<империей>>!}

Что происходит, когда я~сниму еще ребро?

Пусть возникнет еще одна аномалия такого же типа. Возникнет вершина, из~которой торчит ребро,
и~на~другом конце ребра висит пустая вершина. Но~по-прежнему $\text{В}-\text{Р}+\text{Г}$ такое же, как было раньше. Что
я~теперь могу сделать с~этой вершиной и~этим ребром? Стереть их целиком. При~этом количество
и~вершин, и~ребер уменьшится на~1 (рис.~\ref{f:a8}). Значит, выражение опять не~изменилось, а~<<сеть>>
на~поверхности стала проще.

%%Рис. 9
%\xPICi{a9}{А~вот таких <<границ>> не~бывает. Венский Конгресс бы этого не~одобрил!}

Я~значительно увеличил грань, я~убрал всё внутри нее, а~выражение не~менялось. <<Сеть>> свелась
к~двум граням, охватывающим сферу <<сверху и~снизу>>, разделенным замкнутой ломаной; в~ней
количество вершин равно количеству ребер, то есть $\text{В}-\text{Р}+\text{Г}=\text{Г}=2$.

Для~сферы формула Эйлера тем самым доказана.

Вопрос: <<В какой ситуации логика этих рассуждений не~может быть проведена?>> Математик всегда
изучает, в~каком месте его рассуждение не~пройдет. А~не~пройдет оно, например, на~торе. На~торе
берем вершину и~2~ребра (рис.~\ref{f:a10}).

%Рис. 10
\xPICi{a10}{Ребро вокруг бублика и~ребро вокруг дырки от~бублика (и~одна вершина).}

К~такой картинке (рис.~\ref{f:a10}) приводится сниманием ребер любая <<сеть>> (достаточно общего вида ) на~торе. Почему же
нельзя снять еще одно ребро?
 Здесь я~взываю к~интуиции слушателей. Если мы разрежем тор по~этим
ребрам, а~потом развернем, то получим квадрат. Чтобы лучше себе всё это представить, проделаем
данные операции в~обратном порядке: возьмем обычный квадрат из~гибкой резины и~изогнем его так,
чтобы две противоположные стороны квадрата совпали (и~затем склеим по~совпавшим сторонам).

%%Рис. 11
%\xPICi{a11}{Слева: продолжаем стирать ребра на~сфере. Справа: <<сеть>> на~торе.}

\pagebreak

Получилась трубка (две оставшиеся стороны квадрата превратились при~этом в~два колечка). Изогнем
трубку таким образом, чтобы эти колечки тоже совпали (и~склеим их). Вот и~получился из~квадрата
тор. По~местам склеек восстанавливаем, где на~этом торе расположены два ребра и~одна вершина
(из~четырех вершин квадрата получилась ОДНА вершина на~торе).

Осталось пояснить только один важный вопрос: так все-таки можно или нельзя при~изучении топологии
делать склейки, разрывы и~надрезы? Выше говорилось, что при~этом может измениться топологический
тип объекта. Значит, если мы хотим сохранить топологический тип объекта, этого делать нельзя.
Но~можно безболезненно делать многое другое: растяжение, сжатие, перемещение, поворот объекта,
увеличение его в~несколько раз. Эти операции позволяют представить изучаемый объект в~самом простом
для~понимания виде. Например, конус (заполненный внутри) можно превратить в~шар.

Однако, если мы хотим \textit{изменить} топологический тип, то можно (и~даже нужно) делать разрезы
и~склейки. Эти операции так часто применяются в~топологии, что даже носят специальное название:
\textit{<<топологическая хирургия>>}. Более того, практически любой интересный для~изучения объект можно
склеить из~весьма простых кусков. Скажем, торическую поверхность можно получить склейкой нескольких
треугольных кусков. А~когда склейка будет закончена, места склеек будут определять некоторую
<<сеть>> на~торе. <<Сеть>>, составленная из~треугольников (естественно, криволинейных), называется
\textit{<<триангуляцией>>}. Простейшая <<сеть>> на~торе (рис.~\ref{f:a10}) не~является триангуляцией, так как она
получена не~из~треугольников, а~из~квадратов\ldots\ точнее, из~одного-единственного квадрата. Но~этой
беде легко помочь: когда мы выше делали операции в~обратном порядке, надо было на~исходном квадрате
нарисовать диагональ (то есть вместо квадрата далее рассматриваются <<два склеенных треугольника>>).
После двух вышеописанных склеек из~этого квадрата получится триангуляция тора. Она состоит (хотя
в~это и~трудно поверить) из~двух граней, трех ребер и~одной вершины (к~которой подходят все шесть
концов этих трех ребер!).

Можно порекомендовать слушателям купить свежеиспеченный бублик с~маком и, прежде чем его съесть,
внимательно осмотреть и~понять, как именно проходят по~его поверхности ребра данной триангуляции.
Но~специалист-тополог может представить себе эту триангуляцию даже с~закрытыми глазами!

\looseness=-1
Проверьте, возьмите любую ненужную велосипедную камеру, разрежьте и~попытайтесь развернуть.
Сохранится тот факт, что грань выглядит как квадрат или как круг, то есть она, как говорят
математики, \textit{топологически тривиальна}.
 Она выглядит почти как обычная плоская фигура. А~вот если мы
снимем ребро (т.\,е. сотрем его с~поверхности тора) и~потом разрежем по~оставшемуся ребру, у~нас
возникнет \textit{нетривиальная} фигура в~виде кольца. (Кстати, слово <<тривиальный>> восходит к~слову
<<тривиум>>, обозначающему начальный уровень образования в~средневековых университетах.)


Колечко на~плоскости (рис.~\ref{f:a12}) не~является топологически тривиальным, у~него внутри дырка.
Получается, что нам запрещено убирать это ребро, потому что мы изменим тривиальный объект
на~нетривиальный. Математика прошла долгий путь, прежде чем смогла понять, чем формально квадрат
отличается от~кольца.


%Рис. 12
\xPICi{a12}{Ребро, охватывающее <<дырку от~бублика>>, стерли. Вдоль оставшегося ребра
разрезали. Полученную трубку разогнули. Сильно увеличив радиус одного из~концов
трубки и~прижав ее к~плоскости, получили из~нее кольцо. (Можно стереть вместо этого другое ребро:
убедитесь в том, что получится то же самое, даже наглядно проще!)}

\pagebreak

Но~если мы примем к~сведению этот путь, то сможем воспользоваться его результатами. Сможем сказать,
что можно снимать ребро тогда и~только тогда, когда объект, который возникает, будет топологически
тривиален, то есть будет похож на~квадрат по~своей топологической структуре. Именно поэтому
я~не~имею права стирать на~торе ребро.

Итак, чему равно $\text{В}-\text{Р}+\text{Г}$ для~нашей картинки (рис.~\ref{f:a12})? Сколько у~нас вершин?

\textbf{Слушатели:} Одна.

\textbf{А.С.:} Граней?

\textbf{Слушатель:} 4?

\textbf{А.С.:} Нет, одна грань. Эта одна и~та же грань. Посмотрите, из~любой точки грани я~могу
пройти в~любую другую, не~пересекая рёбра. А~это значит, что грань одна.

На~торе сейчас всего одна грань, одна вершинка и~два ребра. Поэтому $\text{В}-\text{Р}+\text{Г}=0$.

И~всегда для~тора будет ноль.

А~к~чему я~приду на~сфере, когда сниму все возможные ребра и~вершины? Какой объект получится? (То
есть мы не~хотим останавливаться на~сети в~виде двух граней, охватывающих сферу сверху и~снизу, как
выше, а~хотим сделать ее еще проще.) Я утверждаю, что в~итоге останется просто голая сфера с~одной вершиной.
 Все ребра будут сняты.

\textbf{Слушатель:} И~как получится два?

\textbf{А.С.:} Вот как. У~вас одна вершина, одна грань и~ноль ребер. $1-0+1=2$ (см. рис.~\ref{f:a12n}).

\xPIC{a12n}

Почему я~не~могу снять и~точку тоже? Потому что, если я~ее сниму, останется сфера, которая
топологически не~похожа на~квадрат. А~вот, если я~сферу проколол\ldots\ Что происходит с~камерой мяча,
который проткнули иголкой? Он сдувается и~превращается (если сильно увеличить место прокола
и~наложить на~плоскость) в~лоскут~--- в~плоскую фигуру. Сфера отличается от~плоского куска только
одной точкой. Очень хорошо это понимают грузины, буряты и тувинцы. Они делают большие пельмени (хинкали, позы и буузы).

%Рис. 13
\xPICi{a13}{Сфера зажата между двумя круглыми гранями (передняя, малая и~задняя, большая).
Их разделяет $n$-угольник (в~нём, как было сказано выше, n ребер и~n вершин). Странно только,
что $n=1$. Как это понимать, обсуждается в~лекции.}

Как их делают? Берут кусок теста, поднимают за~края, слепляют, и~получается сфера. Так что
в~топологии можно сказать, что сфера отличается от~круга всего одной точечкой. Отсюда и~возникает
одна точка и~ноль ребер.

Давайте к~одной вершине добавим одно ребро (рис.~\ref{f:a13}). Что изменилось? Добавилось одно ребро и~одна
грань. То есть у~нас одна вершина, одно ребро и~две грани. Странно смотрится замкнутое ребро
на~рис.~\ref{f:a13}? Давайте тогда поставим еще одну вершину (рис.~\ref{f:a14}).

%Рис. 14
\xPICi{a14}{Случай $n=2$ (назовем это <<два двуугольника>>).}

Итак: 2 вершины, 2 ребра, 2 грани: $2-2+2=2$.

%\endinput

Не~бывает двугранников? Да еще образованных двумя <<двуугольниками>>? Хорошо. Чтоб не~было сомнений,
добавим еще две вершинки. Получится квадрат на~сфере, то есть $n=4$.

\textit{4~вершины, 4~ребра, 2~грани: $4-4+2=2$. Упорно получается значение~<<2>>.}

Можно остановиться в~любой момент, посчитать количество вершин, ребер и~граней. Но~вы должны
понимать, что всегда можно привести к~ситуации, в~которой останется одна вершина. Поэтому у~любой
картинки на~сфере эйлерова характеристика равна двум, ибо эту картинку можно свести к~простейшему
случаю <<одна вершина, одна грань, ноль ребер>>.


Мы получаем противоречие. На~торе всегда ноль, а~на~сфере~--- два. Но 2 не~равно 0. Значит, это разные
топологические фигуры, что, впрочем, каждый из~вас и~так знал. Но~вопрос не~в~том, чтобы доказать
очевидный факт, а~в~том, чтобы наработать язык, который поможет нам этот факт заметить в~других
пространствах. В~частности, в~пространстве большего числа измерений. А~в~большем числе измерений
верно в~точности то же самое, только появляется то, что называется <<трехмерные грани>>.
И~получается следующее выражение:
$$
\text{В}-\text{Р}+\text{Г}-\text{Т}.
$$
Здесь $\text{Т}$~--- количество трехмерных граней. Так выглядит эйлерова характеристика
для~четырехмерного пространства, в~котором лежит трёхмерный объект. В~общем случае у~формулы тот же
вид $\text{В}-\text{Р}+\text{Г}-\text{Т}+\ldots$ и~так далее, в~$n$-мерном пространстве, которое довольно сложно представить. Если
изучить, что происходит при~стирании вершины, ребра, грани, трехмерной грани, будет
обнаруживаться, что значение нашего выражения не~изменится. Вот основываясь на~примерно таких
вещах, но~гораздо более сложных, была установлена справедливость гипотезы Пуанкаре.

В~2002~году, когда доказали гипотезу Пуанкаре, газета <<Известия>> напечатала о~ней статью.
Помнится, в~СССР было 2~основных газеты: <<Правда>> и~<<Известия>>. И~все знали, раз написано
в~газете <<Известия>>, значит факт. Но~в~2002~году <<Известия>> отступили от~этого замечательного
правила, написав математическую формулировку гипотезы Пуанкаре в~таком виде, в~котором она являла
собой полную чушь. Они не~удосужились позвонить ни одному грамотному математику и~очень сильно опозорились (впрочем, мало перед кем).

А~теперь~--- обещанное в~первой лекции доказательство того, что в~футбольном мяче ровно 12~пятиугольных лоскутков.

Рисуем на~сфере картину футбольного мяча. Он должен состоять из~шестиугольных и~пятиугольных
лоскутков. В~любой вершине должны сходиться ровно 3~ребра. В~остальном он может быть совершенно
произвольным.

%%Рис. 15
%\xPICi{a15}{На~круглой деревянной болванке усердные портные сшивают лоскутки.
%И~вот футбольный мяч готов. Тяжеловат, правда, получился! :-)}

Давайте обозначим за~$x$~--- число шестиугольников, за~$y$~--- число пятиугольников.

Сколько тогда граней у~нашего многогранника, нарисованного на~сфере, то есть на~футбольном мяче?

\textbf{Слушатель:} Граней?

\textbf{А.С.:} Да.

\textbf{Слушатель:} $x+y$.

\textbf{А.С.:} Правильно. Ровно столько, сколько в~сумме количеств шести- и~пятиугольников.
$$\text{Г}=x+y$$ ($\text{Г}$~--- количество граней).

Чему равно количество вершин и~чему равно количество ребер?
Посчитаем наивно. Сколько вершин у~шестиугольника?

\textbf{Слушатели:} 6.

\textbf{А.С.:} 6. Всего $x$~шестиугольников. Значит, у~всех шестиугольников вершин\ldots

\textbf{Слушатель:} $6x$.

\textbf{А.С.:} А~у~пятиугольников?

\textbf{Слушатель:} $5y$.

\textbf{А.С.:} Значит, пишем $6x+5y$, но~это не~совсем то, что надо. Обозначим поэтому не~<<В>>, а~<<М>>,
$$
\text{М}=6x+5y.
$$


\textbf{А.С.:} Почему это не~то, что надо?

\textbf{Слушатели:} Потому что вершины совпадают.

\textbf{А.С.:} Если мы разрежем мяч на~лоскутки или, наоборот, не~начнем сшивать, то сколько будет
вершин у~всех лежащих на~столе лоскутков? Именно столько, $6x+5y$. А~когда мы сошьем, некоторые вершины совпадут.
 Что надо
сделать с~этим числом, чтобы получить правильное число вершин?

\textbf{Слушатель:} Разделить на~3.

\textbf{А.С.:} Да. Правильно, потому что ровно~--- не~больше не~меньше, а~ровно~--- 3~разных грани сходятся в~каждой вершине:
$$
\text{В}=\bfrac{\text{M}}{3}=\bfrac{6x+5y}{3}.
$$

%%Рис. 16
%\xPICi{a16}{Свяжем переменные $(x,y)$ веревками уравнений и~прихлопнем сверху формулой Эйлера!}

Сколько ребер? Первый вопрос: сколько ребер до~того, как мы сшивали? Столько же, сколько было до сшивания вершин:
$$
\text{М}=6x+5y.
$$
У~любого многоугольника вершин и~ребер одинаковое количество. А~на~что делить?

\textbf{Слушатели:} На~2:
$$
\text{Р}=\bfrac{6x+5y}{2}.
$$
Каждое ребро мы считали ровно два раза.

%%Рис. 17
%\xPICi{a17}{Пора применить формулу Эйлера.}

Теперь мы воспользуемся формулой Эйлера. Формула Эйлера утверждает, что $\text{В}-\text{Р}+\text{Г}=2$.
Подставим в~нее выражения через <<$x$>> и~<<$y$>>:
$$
\bfrac{6x+5y}{3}-\bfrac{6x+5y}{2}+x+y=2.
$$

Цель этой формулы~--- доказать, что $y=12$. Давайте решать.
\begin{gather*}
6x:3=2x,\\
6x:2=3x,\\
2x-3x+x=0.
\end{gather*}
Иксы ушли. Осталось уравнение относительно <<$y$>>:
$$
\bfrac{5y}{3}-\bfrac{5y}{2}+y=2.
$$
Умножим все уравнение на~6, чтобы избавиться от~знаменателя. Умножим и~правую, и~левую часть.
Справа будет 12.
Слева будет: $10y-15y+6y$. Отсюда
$$
y=12.
$$
Чудеса, да? И~никакого мошенничества!

\textbf{Слушатель:} Что-то тут есть от~фокуса.

\textbf{А.С.:} Курс <<Математика для~гуманитариев>>~--- это курс черной магии плюс ее разоблачение.
В~чем здесь фокус? Природа фокуса в~том, что сократились все шестиугольники. Получается, они ни
на~что не~влияют. Можно любое количество шестиугольников вклеить дополнительно в~любой футбольный
мяч, так как все <<$x$>> сокращаются\footnote{Строго говоря, это утверждение требует доказательства.}.
 А~с~<<$y$>> вы не~можете сделать ничего, потому что сколько бы
пятиугольников ни было у~нас в~запасе, их количество должно удовлетворять уравнению. А~математики
еще 3~тысячи лет назад научились решать линейные уравнения. У~этих уравнений в~нормальной ситуации
всегда одно решение: $y=12$~--- единственное решение нашего уравнения. Поэтому сколько бы вас ни
просили сшить футбольный мяч из~11~пятиугольников~--- не~получится.

\textbf{Слушатель:} А~если пятиугольников будет 24?

\textbf{А.С.:} Вы сошьете два футбольных мяча. Один не~сошьется. Где-то будут торчащие, несшиваемые части.


Давайте теперь посмотрим на~обычную бесконечную во~все стороны плоскость. С~одной стороны, это
более простой объект, чем сфера, но, с~другой стороны, она бесконечна во~все стороны.
\textit{Бесконечность}~--- это такой краеугольный камень математики. И~как с~ней можно быть <<на ты>>~--- это
очень важная тема. Кажется, плоскость, она и~есть плоскость, посмотрел вокруг~--- везде плоскость.
Но~ведь она бесконечная\ldots\ А~как, кстати, можно понять, что земля не~плоская?

В~принципе, как я~понимаю, то что древние люди считали Землю плоской~--- это сказки. Люди всегда
знали, что она не~плоская. Когда по~морю идет корабль, сначала на~горизонте появляются паруса. Как
еще, кроме как искривлением, можно это объяснить?

\textbf{Слушатель:} Может быть, Земля не~ровная именно в~этом месте\ldots

\textbf{А.С.:} От~того, что ты видишь паруса, до~понимания, что Земля может быть устроена как шар,
уже, в~общем, недалеко.

Люди, на~самом деле, в~прошлом совершали и~более великие открытия. Знаете, когда в~первый раз
(по~крайней мере, документально) была высказана идея о~конечности скорости света? В~1676~году
датский астроном Тихо Браге стал наблюдать затмения спутников Юпитера. И~заметил странности в~их
периодичности: то затмения наступали позже прогнозируемого момента, то раньше.
 Тогда он предложил совершенно невероятное
объяснение. Он предположил, что такое могло бы быть, если бы скорость света была конечна. Так как
Земля и~Юпитер то приближаются друг к~другу, то отдаляются, мы видим объект, который ближе, раньше,
чем тот, который находится дальше. За~счет этого и~возникает неполная периодичность в~затмениях.
Но~тогда нужно было признать, что значение этой скорости настолько велико, что оно превосходит
всякое наше воображение. И~Браге оценил его как 225~тысяч километров в~секунду. Он назвал величину,
которая равна 75\% от~верного значения. Но~тогда ученый мир был еще не~готов к~таким смелым идеям,
и~к~этому предположению отнеслись с~большим сомнением.

Или другая история.

У~вас в~сумке, наверное, живет зарядка от~телефона или наушники. В~каком они будут состоянии?
Обычно получается страшный запутанный провод.

Вопрос: можно ли его как-то распутать, если вы еще и~концы провода свяжете, чтобы он стал
замкнутым, как окружность? Чтобы он стал после этого распутывания нормальной, идеальной
окружностью?

\textbf{Слушатель:} Нельзя.

%\pagebreak

\textbf{А.С.:} Иногда можно, иногда нельзя. Это~--- задача из~теории узлов. Какие-то виды узлов можно
распутать, какие-то нельзя. Сейчас я~расскажу историю, которая может оказаться неправдой. Я~слышал
ее на~лекции примерно 13~лет назад. Знаменитая проблема узлов, топологических типов узлов, встала
в~первый раз на~корабле пирата Дрейка в~конце XVI~века. Один из~матросов этого корабля тоже
занимался узлами. Он завязывал много разных морских узлов и~заметил, что некоторые из~них~---
по~сути один и~тот же узел. Надо просто в~одном месте потянуть, в~другом приспустить шнур,
и~из~первого узла получится второй (имеется в~виду, что при~этом концы узла должны оставаться
связанными). Такие узлы называются <<эквивалентными>>. И~пирату в~голову пришла идея
классифицировать все виды узлов. Какие друг в~друга переводятся без~разрезания, а~какие нет.
Ему это не~удалось, в чем, якобы, он честно признался.

Прошло 400~лет. И~только совсем недавно был сделан большой прорыв в~решении задачи об узлах.
Сделали его отечественные математики Максим Концевич, Виктор Васильев и Михаил Гусаров.

Идея решения в том, что берут два узла, пишут для них некоторые математические выражения, и~если они разные, то и узлы тоже разные.

\textbf{Вернемся к~плоскости.} <<Простой>> вопрос: какими многоугольниками можно замостить плоскость?

Что значит <<замостить многоугольниками>>? Я~имею в~виду следующее. Вы заходите в~магазин
и~выбираете себе паркет. Понравившийся вам паркет состоит из~одинаковых дощечек, например, такой
формы, как на рисунке~\ref{f:a18}.

%Рис. 18
\xPICi{a18}{Замысловатая паркетная плитка.}

Кто-то в~страшном сне придумал такую форму. И~таких дощечек у~вас немыслимое количество. Вопрос:
<<Можно ли собрать из~них паркет? Или они при~сборке входят в~противоречие сами с~собой?>>

\textbf{Слушатель:} Ну, скорее всего, центр еще получится, а~вот по краям комнаты будут проблемы.

\textbf{А.С.:} Вы, наверное, уже видите, что не~всякими плитками можно замостить плоскость.

Но~доказать, что какой-то конкретной плиткой нельзя замостить~--- довольно сложная задача. На~самом
деле, до~сих пор не~классифицированы даже все виды пятиугольников, которыми можно замостить
плоскость. Найдено несколько пятиугольников, которыми можно замостить плоскость, но~неизвестно,
есть ли другие. Открытая проблема\footnote{Уже после чтения этих лекций, в 2015~году,
был изобретен новый вид выпуклого пятиугольника, годный для замощения плоскости!}.
 Но тем не~менее методами Леонарда Эйлера можно доказать
следующую теорему.

\smallskip

\textbf{Теорема.} Не~существует ни одного выпуклого 7-угольника, которым можно замостить плоскость.
Более того, восьми-, девяти-, десяти- и~т.\,д. угольника тоже не~существует.

\smallskip

А~что такое <<выпуклый>>? Выпуклая фигура~--- это такая фигура, у~которой, если вы выбрали любые две
ее точки, то весь отрезок между ними лежит внутри этой фигуры, не~выходит за~ее пределы.

%Рис. 19
\xPICi{a19}{Слева~--- невыпуклая фигура, справа~--- выпуклая.}

\pagebreak

Выпуклость~--- одно из~фундаментальных понятий математики. Такое простое определение, а~на~нём
построена огромная сложнейшая теория с~зубодробительными теоремами.

Почему же теорема требует выпуклости? Представьте себе царскую корону (рис.~\ref{f:a20}). Паркетина такой
формы хотя и~является 7-угольником, но~он не~выпуклый. Ниже мы увидим, что такими паркетинами МОЖНО
замостить плоскость. Значит, если не~требовать выпуклости, доказать указанную выше теорему
нельзя~--- она просто неверна. Нельзя огульно утверждать, что паркетов из~7-угольников не~бывает.
Не~бывает только из~выпуклых.

%Рис. 20
\xPICi{a20}{До~царской короны страшно даже пальцем дотронуться!}

Сколько углов? Семь. Однако такой плиткой можно без~проблем замостить плоскость.

Переворачиваем фигурку и~вставляем корону в~корону, а~потом еще раз, два\ldots\ (см. рис.~\ref{f:a21}).

%Рис. 21
\xPICi{a21}{\ldots\ и~получилась страшная зубастая пасть!
Продолжаем ее до~бесконечности вправо и~влево.}

\textbf{Слушатель:} А~в~конце как?

\textbf{А.С.:} До~бесконечности. Мы же говорим о~бесконечной плоскости. Полосу сделать у~нас
получилось\ldots\ (бесконечную в~обе стороны). Ну, а~если можно полосу, то мы ее размножаем неограниченно
вниз и~вверх, и~всё. Мы <<запаркетили>> всю плоскость. А~теперь я~нарисую выпуклый семиугольник
(рис.~\ref{f:a22}).

%Рис. 22
\xPICi{a22}{А~вот этими нельзя замостить плоскость!}

Априори совершенно не~понятно, почему им нельзя замостить плоскость? Почему это так? Почему
никакого семиугольника нельзя предложить в~качестве дощечки для~паркета?
Если Ваша невеста просит Вас: <<Милый, я так хочу выпуклый семиугольный паркет в нашу ванну!>>,~---
то это вариант <<вежливого посыла>>~--- ибо такого быть не может.
Сейчас мы докажем эту теорему. И~в~этом доказательстве
у~нас в~первый раз возникнет бесконечность <<во весь рост>>. Как доказываются теоремы
не~существования чего-то? Какой прием доказательства таких теорем?..

\textbf{Слушатель:} От~противного?

\textbf{А.С.:} Точно. Предположим, что существует выпуклый семиугольник, которым можно замостить
плоскость. Не~знаю какой, но~какой-то есть. Предположим и~приведем это предположение
к~противоречию. Итак, посмотрим на~плоскость, которая замощена этими семиугольниками.\vadjust{\pagebreak} Посмотрим
на~нее в~<<перевернутый бинокль>> и~увидим часть плоскости, как будто очень большую квартиру (см. рис.~\ref{f:a23}).

Я~предупреждаю, такими доказательствами гоняют на~ночь чертей. Приготовьтесь.

Начнем с~того, что попробуем посчитать, сколько в~квартире многоугольников. Давайте исходить
из~того, что наш семиугольник имеет длину 1~метр, а~размер квартиры~--- примерно 1~км.

%Рис. 23
\xPICi{a23}{<<Чертогон>> в~самом разгаре. Для~справок можно почитать рассказ Н.\,С.~Лескова
с~таким же названием.}

%\pagebreak

На~самом деле, не~важно, какого что размера. Важно, чтобы вторая величина была \textit{неизмеримо больше},
чем первая.

В~данном случае <<длина>> семиугольника в~1000~раз меньше <<длины>> квартиры.

\textbf{Слушатель:} Что мы считаем длиной 7-угольника или квартиры?

\textbf{А.С.:} Например, самую большую диагональ. Это не~очень важно. Тут математика немножко
напоминает физику. Нужно несущественные детали не~замечать, а~на~существенные~--- обращать
внимание. Когда у~физика есть ниточка, она обычно имеет толщину ноль. На~самом деле у~нее, конечно,
есть толщина, но~физикам она не~важна. Вот и~нам не~важно. Возьмем какое-то измерение семиугольника
(например, любую из~его сторон или любую диагональ). Ведь все эти измерения НАМНОГО МЕНЬШЕ, чем
<<длина квартиры>>~--- что бы мы ни понимали под этой длиной. На~полу квартиры в~нормальной ситуации
помещается очень много паркетин. Форма пола квартиры тоже неважна, поэтому будем считать его кругом
радиуса~$R$ (где $R$ может быть как угодно велико).

Не~забывайте, что нам приказано замостить не~пол в~квартире, а~всю бесконечную плоскость.

А~теперь давайте посмотрим, сколько примерно семиугольников таится внутри вот этого огромного
круга? С~точностью до~порядка? Если у~нас диаметр круга в~тысячу раз больше, чем диагональ
семиугольника, сколько семиугольников примерно поместится в~круг?

\textbf{Слушатель:} Миллион?

\textbf{А.С.:} Миллион, правильно. \textit{Правильный физический ответ.} Миллион. Не~важно, что это будет
700\,000 или 5~миллионов. В~районе миллиона. Порядок величины такой. Это примерно миллион.

\textbf{Слушатель:} Почему миллион?

\textbf{А.С.:} Потому что у~многоугольника размером 1~метр площадь сопоставима с~1~$\text{м}^{2}$~--- может
быть, чуть меньше, чуть больше.
 У~круга, у~которого диаметр 1~километр, площадь порядка 1000000~$\text{м}^{2}$.
Значит, в~круг влезает примерно миллион семиугольников.

Зададим теперь следующий вопрос. Сколько примерно семиугольников <<живет>> в~районе границы этого
круга (то есть зацепляет за~границу круга)?

\textbf{Слушатель:} 6000.

\textbf{А.С.:} Да, похоже. $2\pi r=6000$. Порядок этого числа~--- не~миллион, а~тысяча. То есть
внутрь входит в~районе миллиона семиугольников, а~на~границе их несколько тысяч. А~теперь~---
внимание! Я~стираю все многоугольники, которые не~лежат в~этом круге. Затем беру плоскость и, как
грузинский хинкали, сжимаю ее в~сферу (рис.~\ref{f:a24}).

%Рис. 24
\xPICi{a24}{Профессор сжал всю плоскость в~сферу, и черти разбежались!}

Делаю я~это, чтобы воспользоваться формулой Эйлера:
$$
\text{В}-\text{Р}+\text{Г}=2.
$$
Грубо говоря, вместо круга есть поверхность огромного шара, у~которого верхняя шапочка (почти
плоская) вся испещрена семиугольниками. Но~для~картинки на~всей большой сфере верна формула Эйлера:
$$
\text{В}-\text{Р}+\text{Г}=2.
$$

%%Рис. 25
%\xPICi{a25}{А~вместо двойки назло всем чертям возьмём нуль, так как оба этих числа весьма малы по~сравнению с~тысячей и~с~миллионом.}

Давайте оценим примерно, сколько у~этой картинки будет вершин, ребер и~граней? Одна огромная грань
снизу, а~наверху порядка миллиона граней в~виде паркетин. Понятно, что одна грань погоды не~делает.
Более того, так как мы сейчас будем иметь дело с~величинами порядка миллиона, то 2 в~формуле
Эйлера, или 0~--- тоже совершенно неважно. Я~могу написать <<примерно равно нулю>>. $\text{В}-\text{Р}+\text{Г}$ примерно равно $0$.
Или $\text{В}+\text{Г}\approx \text{Р}$.
Граней~--- порядка миллиона.
$\text{Г}\approx 1 000 000$.

Сколько вершин? 7000000~--- это вершин у~всех многоугольников; и~в~каждой из~вершин сходится как
минимум 3~многоугольника. Может быть и~больше (например, если у~нашего 7-угольника есть острый угол
в~30 градусов, и~в~вершине сошлись 12~этих острых углов), \textit{но~не~меньше}~--- это точно (ровно два
угла не~могут со~всех сторон окружить вершину, ибо каждый из~них меньше 180~градусов). Поэтому
вершин <<не больше>> (меньше или равно), чем 7000000/3. На~самом деле я~не~учел вершины,
которые являются вершинами большой нижней грани. Сколько их примерно?

\textbf{Слушатель:} 6000.

\textbf{А.С.:} Да. Поэтому надо прибавить еще 6000. Нам не~жалко!
$$
7 000 000/3 + 6000.
$$
Но~шутка матанализа заключается в~том, что 7000000 и~6000~--- не~сопоставимы по~величине, так как
первая величина значительно больше; так что про тысячи можно забыть.
 Получается:
$$
\text{В}\le 7 000 000/3.
$$
Теперь о~ребрах. Ребер будет $7 000 000/2$.
Причем делим \textit{в~точности} на~2, без~всяких меньше или равно, потому что каждое ребро мы посчитали ровно 2~раза:
$$
\text{Р}=7 000 000/2.
$$

\textbf{Слушатель:} А~почему мы каждое ребро посчитали ровно 2~раза?

\textbf{А.С.:} Потому что мы плиточку к~плиточке прикладываем, без~всяких зазоров (мы ведь
предположили, что \textit{можно} уложить без~зазоров), см. рис.~\ref{f:a26}.

%Рис. 26
\xPICi{a26}{Плиточка к~плиточке! Ребро к~ребру! Без~зазоров!}

\textbf{Слушатель:} Почему в~теореме взято 7~сторон и~более?

\textbf{А.С.:} Потому что шестиугольное замощение давно известно, например, его знают наши друзья
пчелы. Пятиугольное может быть таким: поставил домики рядом и~сверху такие же, но~вверх ногами
(см. рис.~\ref{f:a27}). Домики, в~отличие от~царской короны, которую мы в самом начале рисовали, выпуклые.

%Рис. 27
\xPICi{a27}{Замощение плоскости 4-угольниками, а также некоторыми 5-угольниками и~6-угольниками возможно.}

А~уж квадратами, треугольниками замостить~--- это совсем легко. Любым четырехугольником можно
замостить плоскость и~любым треугольником~--- тоже. А~вот какими пятиугольниками можно~--- это
сложная задача. И~про выпуклые шестиугольники тоже далеко не всё известно. Но~какими-то можно. А~вот выпуклыми
семиугольниками уже никак нельзя.


%Рис. 28
\xPICi{a28}{Паркеты из~<<домиков>>, квадратов и~правильных треугольников.
В~первом из~них к~вершинам подходят либо 3, либо 4~ребра. Во~втором~--- только 4~ребра.
В~третьем~--- только 6~ребер.}

Давайте все-таки доведем до~конца доказательство.

У~нас есть равенство
$$
\text{В}+\text{Г}\approx \text{Р}.
$$
Оно говорит нам, что количество ребер должно быть того же самого порядка, что и~количество вершин
плюс количество граней. Подставим наши значения.
$$
\bfrac{7000000}{2}\approx 1000000+\bfrac{7000000}{3}.
$$
Если посчитать, сократив на~миллион и~умножив на~6, равенства не~получается. Очень заметно
не~получается! Потому, что 21 не~равно 20. Так что никакая добавка слагаемого типа 6000 дела
не~спасет, ибо эту добавку тоже придется делить на 1000000, и она станет исчезающе малой.
 А~ведь мы могли взять не~$R=1000$~км, а~$R=20000$~км. Тогда бы процентное влияние добавки
типа <<6000>> стало бы гораздо меньше. То же самое, естественно, будет \text{с~восьми-,} девяти- и~прочими
<<много-много-угольниками>>. А~вот для~шестиугольников как раз получается
\begin{gather*}
6 000 000/2=6 000 000/3 +1 000 000\\
3 000 000=2 000 000+1 000 000\quad
(\text{при~любом значении $R$}).
\end{gather*}
Точное равенство получается потому, что шестиугольное замощение устроено так, что в~каждой вершине
сходится ровно 3~ребра. А~вот уже 5-угольное замощение устроено иначе. Иногда 3~ребра сходится,
а~иногда~--- 4. У~квадрата везде сходятся 4, а~у~правильных треугольников~--- 6~ребер (рис.~\ref{f:a28}).

{\tolerance=7000

То есть выпуклое замощение бывает треугольное, четырехугольное, пятиугольное, шестиугольное. А~никаких
других не~бывает.

}

\textbf{Слушатель:} А~какая практическая польза?

\textbf{А.С.:} Ну, наверное, есть какая-то. Математик никогда не~думает о~практической пользе.
Другие за~него думают. Посмотрит какой-нибудь строитель: <<О, значит не~надо даже думать о~том,
чтобы использовать семиугольные плитки>>. А~для~математика нет такого вопроса. Это же совершенство.
Это всё равно, что спрашивать, какая практическая польза у~молитвы. Так же и~математик, он просто
показывает: нельзя,~--- ура, вот какая интересная теорема. А~польза? Наверняка какая-то польза
есть. У~любого красивого факта есть польза.

\medskip

\hrulefill

\smallskip

\textbf{Врезка~4.} Ни один из~слушателей не~спросил у~меня: <<А где же в~доказательстве теоремы
используется тот факт, что исходный семиугольник был выпуклым?>> И~даже сложилось превратное
впечатление, что для~проведения доказательства выпуклость 7-угольника вообще не~нужна. Но~она
нужна! Ведь иначе получилось бы, что мы заодно доказали, что для~невыпуклого семиугольника тоже
нельзя придумать замощение плоскости таким кусочком. Выше, однако, приведен пример, что 7-угольным
кусочком типа <<царская корона>> вполне можно замостить плоскость.

{\tolerance=7000

На~самом деле выпуклость была незаметным образом использована, когда мы поделили число 7000000
именно на~3. Только для~выпуклого 7-угольника можно опираться на~число~3. На~рис.~\ref{f:a21} паркет
содержит такие вершины, где сходятся только две плитки паркета (и~на~одной из~них имеется угол БОЛЕЕ 180
градусов).
Подобное явление, однако, возможно только для невыпуклых плиток: любой выпуклый многоугольник содержит в себе только углы менее 180 градусов.


}

\smallskip

\hrulefill

\medskip

\textbf{А.С.:} Скажу напоследок вот что. Если кого-то не~убедят тысячи и~миллионы, надо будет сказать следующее.
Если круг в~$n$~раз больше по~размеру, чем плиточка, то количество граней, вершин и~ребер
имеет порядок $n^{2}$, потому что их количество связано с~площадью круга. А~то, что в~районе большой окружности
<<живет>>, имеет порядок $n$, потому что вопрос связан с~длиной окружности. И~если вы исследуете
некоторое выражение порядка $n^{2}$, например, $\bfrac{7n^{2}}{2}>\bfrac{7n^{2}}{3}+n^{2}$, и~при~этом во~все слагаемые
примешивается мелочь порядка $n$: $2n$, $3n$, $6n$ и~так далее, то матанализ разрешает ее стереть, потому что
$n^{2}$ и~$n$ <<разного порядка роста>>.
 И~неравенство будет верным при~любом $n$, начиная с~некоторого
места. (А~именно с~того места, когда $n^{2}$ станет подавляюще большим по~сравнению с~$n$.)

В~матанализе есть основной принцип: если вы про какое-то число показали, что оно меньше сколь
угодно малого положительного числа, то вы доказали, что оно равно нулю (если оно изначально не~было
отрицательным). Вот вы получили какое-то число, вы хотите доказать, что оно равно нулю. Покажу
типичный прием матанализа. Пусть есть число~$a$. Рассмотрим такое число, как $\bfrac1n$, и~покажем, что
наше число меньше, чем $\bfrac1n$. Допустим, это мы доказали для~любого натурального значения $n$. Для~1000,
для~1000000, для~1000000000\ldots\ Если вы умеете доказать такое неравенство для~любого $n$, значит,
вы умеете доказать, что~$a$ равно нулю.

\pagebreak

Вот в~этом, собственно, весь принцип матанализа и заключен. Всё остальное, что есть в~матанализе: интегралы,
производные~--- не~более чем упражнения с~этой логикой (математики говорят в~этом случае:
<<Применим технику работы с~порядками бесконечно малых>>).

И~самый последний пример. Мне рассказал его папа, когда я~еще даже в~школу не~ходил. Папа взял
яблоко, отрезал от~него половинку и~говорит: <<Это сколько от~яблока?>>~--- <<1/2>>,~--- сказал я.~---
<<А если теперь я~к~этой половинке прибавлю половинку оставшейся половинки, то это что здесь надо
написать?>>

%%Рис. 29
%\xPICi{a29}{Вот что надо написать}

\textbf{Слушатель:} $1/2+1/4$.

\textbf{А.С.:} А~если я~проделаю это бесконечное количество раз? Тогда что я~получу?
$$
\bfrac12+\bfrac14+\bfrac18+\bfrac1{16}+\ldots=1.
$$

\textbf{Слушатель:} Ноль.

\textbf{Другой слушатель:} Единицу.

\textbf{А.С.:} Я~получу число один, причем \textit{в~точности} число~1.

Почему в~точности? Потому что каждый раз число получалось не~больше единицы, это очевидно. Значит,
мы не~можем получить число больше единицы. Но~какое бы маленькое число мы не~взяли, в~конце концов
$\bfrac1n$ станет меньше его. На~самом деле у~нас в~знаменателе вместо $n$ стоят степени двойки: 2, 4, 8, 16, 32,
64, 128, 256, 512, 1024, 2048, 4096, 8192\ldots

Они очень быстро растут, поэтому $\bfrac1{2^{n}}$~--- очень быстро уменьшается. И~в~итоге очередное расстояние
до~числа~<<1>> станет меньше любого наперед заданного числа. То есть они уходят в~ноль. Получается,
что наша сумма неограниченно приближается к~единице, и~вот тогда математик говорит:
<<Следовательно, она равна единице>>.
 Всё. Вот он, \textit{предельный переход}. Это то, что учат в~матанализе
на~любом факультете любого вуза. Больше ничего в~нём нет\footnote{В этом месте, на самом деле, заключается (прячется) значительная психологическая трудность. Она разрешается посредством аксиомы Архимеда.}.



\textbf{Слушатель:} А~если здесь просто включить житейскую мудрость и~подумать, что мы отрезали от~одного
целого яблока?

\textbf{А.С.:} Да. В~данном случае можно. Но~житейская мудрость~--- она такая штука, что она иногда
не~работает. Давайте решим такую задачу.

Кузнечик сначала прыгает на~один метр, а~потом на~$\bfrac12$~метра, а~потом~--- на~$\bfrac13$, а~потом~--- на~$\bfrac14$,
а~потом~--- на~$\bfrac15$, и~так далее\ldots\ Вот он прыгает и~прыгает. Есть ли предел того, куда он может
допрыгать?
$$
1+\bfrac12+\bfrac13+\bfrac14+\bfrac15+\bfrac16+\ldots
$$

\textbf{Слушатель:} Да.

\textbf{А.С.:} При~наивном подходе кажется, что есть, потому что <<шажки все меньше и~меньше>>. Но
тем не~менее, друзья мои, вы будете смеяться, или удивляться, или поражаться, или возмущаться, но
$$
1+\bfrac12+\bfrac13+\bfrac14+\bfrac15+\bfrac16+\ldots=+\infty
$$
(т.\,е. эта сумма равна бесконечности).

Нет никакого предела тому, куда может дойти этот кузнечик. Никакого. Он может дойти до~Луны, может
дойти до~Солнца, и~далее, прямо в~Космос!

В~прошлом примере у~нас шажки были всё меньше и~меньше, они стремились к~нулю, но~в~сумме
получилось число, равное единице. А~эти шажки, хотя и~тоже всё меньше и~меньше, но~уйти этими
шажками можно до~бесконечности, вот такая загадка природы. Хотите, покажу, почему?

\textbf{Слушатели:} Да.

\textbf{А.С.:} Вот смотрите, сейчас я~с~кузнечиком сделаю страшную штуку, я~сейчас его заменю
на~кузнечика, который шагает еще медленнее. А~именно: кузнечик этот будет шагать следующим образом.
$$1+\bfrac12+\bfrac14+\bfrac14,$$ то есть вместо одной трети, он шагает на~одну четверть. Не~правда ли, такой кузнечик
будет отставать от~первого?

\textbf{Слушатель:} Да.

\textbf{А.С.:} А~теперь вместо одной пятой я~сразу одну восьмую поставлю.
То есть первый кузнечик на одну пятую шагает, а мой, второй~--- он сразу прямо раз~--- и <<скис>>~--- только на одну восьмую.
И~так 4 раза
по~одной восьмой:
$$
1+\bfrac12+\bfrac14+\bfrac14+\bfrac18+\bfrac18+\bfrac18+\bfrac18.
$$
А~вместо одной девятой я~напишу что?

\textbf{Слушатели:} Одну шестнадцатую?

\textbf{А.С.:} Правильно. Одну шестнадцатую, и~так повторим эту добавку 8~раз. А~дальше я~что напишу? Вместо одной семнадцатой?

\textbf{Слушатель:} Одна тридцать вторая.

\textbf{А.С.:} Одну тридцать вторую. Отлично. И~повторим ее 16~раз!
\begin{multline*}
1+\bfrac12+\bfrac14+\bfrac14+
\bfrac18+\bfrac18+\bfrac18+\bfrac18+
\\+
\bfrac1{16}+\bfrac1{16}+\bfrac1{16}+\bfrac1{16}+\bfrac1{16}+\bfrac1{16}+\bfrac1{16}+\bfrac1{16}+
\bfrac1{32}+\ldots
\end{multline*}
Похоже, что второй кузнечик всё время отстает от~первого. Небось, он совсем отстанет от~него: ведь
первый, как мы утверждаем, ускачет на~бесконечное расстояние.
 Нет, самое страшное здесь вот что. Хоть второй и~отстает,
но~он ТОЖЕ ускачет на~бесконечное расстояние. Чему равна сумма $\bfrac14+\bfrac14$ (двух равных слагаемых)?

\textbf{Слушатель:} $\bfrac12$.

\textbf{А.С.:} Отлично. А~такая: $$\bfrac18+\bfrac18+\bfrac18+\bfrac18?$$

\textbf{Слушатель:} Одна вторая.

\textbf{А.С.:} Тоже одна вторая! А~для~шестнадцатых долей?

\textbf{Слушатель:} Тоже одна вторая.

\textbf{А.С.:} Теперь вы поняли, почему он дойдет до~бесконечности?

\textbf{Слушатель:} Нет.

\pagebreak

\textbf{А.С.:} Потому что мы каждый раз, в каждой очередной группе шагов, будем получать в~сумме $\bfrac12$.
Значит, он всё снова и~снова отходит на~$0{,}5$.
А~таких \textit{<<одних вторых>>}-то бесконечное количество штук. Вот он и~уйдет на~бесконечность.
\begin{gather*}
\bfrac14+\bfrac14=\bfrac12,\\
\bfrac18+\bfrac18+\bfrac18+\bfrac18=\bfrac12,\\
\bfrac1{16}+\bfrac1{16}+\bfrac1{16}+\bfrac1{16}+\bfrac1{16}+\bfrac1{16}+\bfrac1{16}+\bfrac1{16}=\bfrac1{2},
\end{gather*}
и~так далее. Значит, на бесконечность тем более ускачет и первый кузнечик!

Но~самое неожиданное я~приберег на~конец. (\textit{Берёт в~руки мяч и~держит его над полом}.) Уроним этот мяч
и~послушаем, сколько раз он ударится.

\textbf{Слушатель:} Бесконечность.

\looseness=-1
\textbf{А.С.:} Правильно. Бесконечность, но~она будет <<преодолена>> за~конечный промежуток времени. Законы физики
это подтвердят. Единственное, что, к~сожалению, в~атомных размерах законы физики меняются (надо
применять квантовую механику), и~эта идиллия прекращается. Но если бы ньютоновская механика была
верна до~самого конца, то любой мяч, если его отпускают, за~конечное время делал бы бесконечное
число подскоков. То есть он устроен, как задача с~яблоком. Потому что каждый следующий подскок,
по~законам физики, составляет по~высоте некоторый процент от~предыдущего. Но~процент от~любой
положительной величины~--- это положительная величина. Поэтому каждый следующий подскок~--- это
тоже положительная величина, а~значит, их будет бесконечное количество.
 Но~они суммируются по~времени.
Время подскоков суммируется, а~сумма стремится к~некоторому числу. Временные промежутки будут всё
короче и~короче и, грубо говоря, за~2~секунды мяч уже бесконечное число раз подпрыгнет и~ляжет
на~землю тихо. За~конечное время бесконечное количество прыжков\ldots

До~встречи на~лекции~3!

\endinput
