%2.2.
\section{Лекция 2}
\label{2.2}

\textbf{А.С.:} Сейчас мы вернемся к~Евклиду. Он был не~только геометром, но также еще доказал замечательный факт
из~теории чисел. А именно, что \textit{простых чисел~--- бесконечное количество\footnote{Напомним,
что простым называется натуральное число, которо не делится нацело ни на какое другое число, кроме
единицы и самого себя.}.}


Давайте сначала поговорим о~том, как устроено математическое доказательство, и~по~каким канонам его
можно создавать. Сейчас будет проведено классическое рассуждение от~противного. Что такое <<от
противного>>? Давайте представим, что наше утверждение~--- неверное. Что тогда? Тогда количество простых
чисел конечно. Но~если их конечное количество, их можно просто перечислить. Какое первое простое
число?

\textbf{Слушатель:} Ноль.

\textbf{А.С.:} Нет. Ноль не~является простым числом. Ноль вообще исключают при~рассуждениях
о~делимости. На~ноль не~любят делить. Потому что, если вы делите на~ноль что-то отличное от~нуля,
у~вас не~получится ничего. А~если вы делите ноль на~ноль, то у~вас получится <<сразу всё>>.

Что такое <<разделить>> одно число на~другое? Скажем, что такое разделить 6 на~3? Это значит найти
такое число, которое при~умножении на~3 дает 6. Это число 2. А~если вы 5 разделите на~3, получится
дробное число, среди целых чисел его не~найти. Содержательная теория делимости, в~частности понятие
простого числа, относится только к~целым числам. Если вы рассматриваете все дроби, делимость совершенно
бессмысленна. Потому что любую дробь можно разделить на~любую, главное только на~0 не~делить.
Но~если вы формально попробуете разделить, например, 5 на~0, то вы должны найти такое число,
которое при~умножении на~ноль даст 5. Но~вы явно не~преуспеете в~этом, потому что, какое бы вы
число ни~взяли, при~умножении на~0 оно даст 0. Поэтому 5 на~0 разделить нельзя в~принципе. А~можно
ли разделить 0 на~0?\vadjust{\pagebreak} Нужно найти такое число, которое при~умножении на~ноль дает ноль.

\textbf{Слушатель:} Ноль.

\textbf{Другой Слушатель:} Любое.

\textbf{А.С.:} Любое. То есть ноль на~ноль, формально говоря, можно разделить, но~в~результате
получится любое число, это математикам тоже не~нравится, поэтому решили договориться так, что
на~ноль \textit{просто не~делят}. А~ноль можно разделить на~что-нибудь?

\textbf{Слушатель:} На~всё, кроме нуля.

\textbf{А.С.:} Да. И~что получится?

\textbf{Слушатель:} Ноль.

\textbf{А.С.:} Только ноль. Да. Ноль можно разделить на~что угодно, кроме нуля. В~ответе всегда
будет ноль.

Теперь число единица. Единицу тоже не~считают простым числом, потому что на~единицу делится любое
число.


Итак, первое простое число~--- 2. Оно делится только на~себя и~на~единицу. Больше четных простых
чисел нет, потому что все остальные четные числа делятся на~2. Следующее простое число~--- это 3, затем число 5. Вот
как выглядят первые простые числа:
$$
P_{1}=2,\
P_{2}=3,\
P_{3}=5,
$$
а~далее
$$
7,\ 11,\ 13,\ 17,\ 19,\ 23,\ 29,\ 31,\ 37,\ 41,\ 43,\ \ldots,
P_{N}.
$$
Если утверждение Евклида неверно, то есть если простых чисел конечное число, то в~какой-то момент
выпрыгнет последнее простое число. Обозначим его за~$P_{N}$. Получается, что количество простых чисел равно~$N$.


Если Евклид неправ, то значит существует последнее простое число, а~каждое из~следующих чисел делится
на~какое-то из~предыдущих простых чисел. Потому что если число делится на какое-то число, то оно и на~простое
число тоже делится, просто нужно делить, делить~--- пока не~дойдете до~простого. Давайте теперь составим
произведение: $2\cdot5\cdot7\cdot11\cdot13\cdot\ldots\cdot P_{N}$ (точка~--- знак умножения).

Если $N$ имеет порядок, например, нескольких миллиардов, то в~этом произведении стоит несколько
миллиардов множителей, которые нужно друг на~друга умножить. Но~натуральный ряд так устроен, что
к~любому, сколь угодно большому числу можно прибавить~1.

Так что давайте посмотрим на~следующее натуральное число:
$$
2\cdot3\cdot5\cdot7\cdot11\cdot13\cdot\ldots\cdot P_{N}+1.
$$
Оно не~может делиться ни на~какое из~предыдущих простых чисел, потому что наше произведение от~2
до~$P_{N}$ делится на~все предыдущие, а~если еще единичку прибавить, то делимость сразу уйдет. Если
что-то, например, на~3 делится, то следующее число уже на~3 не~делится. Поэтому, число $2\cdot3\cdot5\cdot7\cdot11\cdot13\cdot\ldots\cdot P_{N}+1$
заведомо не~может делиться ни на~2, ни на~3, ни на~5 и~так далее\ldots\ То
есть оно всегда дает остаток 1 при~делении на~все простые числа. Оно нацело ни на~что не~разделится.
Значит, оно простое. Противоречие. Следовательно, простых чисел бесконечное количество.

Некоторое время, правда, недолго, ученые думали, что таким образом можно получить рецепт
изготовления простых чисел.

$2+1$~--- простое число.

$2\cdot3+1$~--- простое,

$2\cdot3\cdot5+1$~--- простое,

$2\cdot3\cdot5\cdot7+1$~--- простое,

$2\cdot3\cdot5\cdot7\cdot11+1$~--- простое,

$2\cdot3\cdot5\cdot7\cdot11\cdot13+1$~--- уже не~простое (30031), так как оно делится на~59.

Рецепт не~работает.

Мы знаем, что простые числа в~натуральном ряду чисел встречаются в~бесконечном количестве. А~теперь
вопрос, как они распределены? Можно ли тут какие-то закономерности установить? Насколько часто или
редко они встречаются?

Рассмотрим такое произведение
$$
1\cdot2\cdot3\cdot4\cdot5\cdot\ldots\cdot100=100!\quad
\text{(называемое <<сто факториал>>)}.
$$
\textit{100~факториал}~--- это произведение подряд идущих натуральных чисел от~1 до~100.

Это~--- огромное число, его невозможно себе даже представить, но~все-таки где-то в~натуральном ряду оно
есть. Следующее за~ним число $100!+1$, потом $100!+2$. Это число будет делиться на~2. Потому что $100!$
делится на~2, и~2 делится на~2. $100!$+3 делится на~3, $100!+4$ делится на~4 и~на~2, $100!+5$ делится
на~5.

И~так до~ста. $100!+100$ будет делиться на~100. Получается, что цепочка от~$100!+2$ до~$100!+100$~--- из~99~натуральных идущих подряд
чисел \textit{простых чисел в~себе не~содержит.}
 Можно ли сконструировать то же самое,
но~для~1000 подряд идущих непростых чисел? Конечно. Берете $1001!$, то есть произведение всех чисел
от~1 до~1001, прибавляете 2, 3, 4 и~так далее до 1001. Вот вам 1000 подряд идущих чисел,
среди которых нет простых. Значит, регулярности в~проявлении простых чисел ожидать нельзя. Промежутки между
соседними простыми числами могут быть сколь угодно большими. Можно ли сказать что-то про то, насколько маленькими
они могут быть?

\textbf{Слушатель:} Единица~--- минимальный промежуток.

\textbf{А.С.:} Единица. Но~она встречается только в~самом начале, между 2 и~3. Потому что из~двух
соседних чисел одно обязательно четное, а~значит~--- не~простое (кроме случая 2 и~3). Получается,
что минимальное расстояние между соседними простыми числами, начиная с числа $3$, равно~2.


Вначале мы очень много видим этих <<двоек>> (то есть простых чисел, идущих через одно),
потом они становятся всё реже и~реже, и~возникает вопрос:
а~кончатся ли эти <<двойки>> когда-нибудь? Будет ли момент натурального ряда, может быть, ужасно далеко
от~нас, когда появится последняя двойка соседних простых чисел, отличающихся на~2 единицы? Такие
числа, кстати, называются близнецами: 29 и~31, 41 и~43, 71 и~73, 101 и~103. Будет ли момент, когда мы
встретим последних близняшек, а~между оставшимися простыми числами расстояние всегда будет
не~меньше трех (на самом деле четырех, потому что они заведомо оба нечетные)?


Это~--- нерешенная математическая проблема.

Я~вам расскажу про одно маленькое <<но>>, которое
позволяет оптимистам утверждать о~некотором прогрессе в~решении этой проблемы. Чтобы вы сразу
почувствовали, однако, насколько анекдотичен этот прогресс, выслушайте такую \text{\textit{притчу.}}

В~одной стране попытались доказать теорему: \textit{У~каждого мужа должно быть не~более трех жен.} Долго
не~удавалось ее никак доказать. Наконец, дело сдвинулось с~мертвой точки. Была доказана близкая
к~ней теорема: \textit{У~каждого мужа должно быть не~более трех миллионов жен.} Ученые продолжают размышлять
над этой проблемой.

Так вот. Специалисты по~теории чисел решили взглянуть на <<проблему близнецов>> под похожим, так сказать, углом.

Может быть, можно сказать, что вот хотя бы на~каком-то другом расстоянии $d$ (превышающем двойку) уже
можно гарантировать, что бесконечно много раз появятся соседние простые? То есть что расстояние между
соседними простыми не~будет уходить
в~бесконечность? До~2013~года это оставалось открытой проблемой даже в~такой формулировке. В~2013~году
математик Итан Чжан (Yitang Zhang)~--- китаец, работающий в~США (как это часто случается
с~китайцами), доказал, что существует бесконечное множество пар простых чисел на~расстоянии,
не~превышающем некоторого числа $d>2$.
 Доказательство проверили: правильно, есть такое число~$d$.
Но~единственное, что про него известно,~--- что оно не~превышает 70000000 (см. вышеуказанную
притчу). Как говорится, хотели рассматривать простые числа через окошко длиной в три единицы
(2-трафарет)~--- не~добились толку. А~потом взяли трафарет побольше (70000000-трафарет),
и~получилось. Таким образом, здесь просматривается новое понятие <<обобщенные простые
числа-близнецы>>. Так что бывают 2-близнецы,
бывают и 6-близнецы, \ldots, 70000000-близнецы. Проблема распалась на~бесконечную серию
проблем, и~с~какого-то места (то есть с~какого-то числа~$d$) она оказалась решенной положительно.

%\pagebreak

Ведь что такое трафарет? Это такая рамочка. Вот я~беру трафарет длиной в~70000000, и~в~окошечко
рассматриваю натуральный ряд, двигаясь вдоль него. Фиксирую моменты, когда внутри этого промежутка встречаются простые
числа (более одного). Так вот, их будет бесконечное количество, этих моментов. Чжан доказал, что
достаточно взять трафарет длиной в~70000000, чтобы поймать бесконечное количество простых
обобщенных близнецов. Конечно, если я~сделаю $d=70 000 001$, тем более поймаю бесконечное количество
этих обобщенных близнецов. А~если возьму чуть меньше, например, 60000000, то уже точно
сказать ничего нельзя.


Это~--- большой прорыв. Потому что в~1896~году Адамаром и~Валле-Пуссеном было доказано, что частота
появления простых чисел уменьшается. Рассказывают, что Адамар появился в~этот день в~кафе и~сиял
как медный грош. Друзья спросили его: <<Что-то у~тебя такое хорошее настроение, как будто ты
доказал асимптотический закон распределения простых чисел>>. А~он и~отвечает: <<Вы знаете, вы
не~поверите, но~я~именно это и~сделал, и~именно с~этим и~связано мое хорошее настроение>>.

<<Да ладно>>,~--- говорят. Потом проверили и~оказалось, что доказательство верно.

Почему это потрясает и~почему это убедительно говорит о~нерегулярности появления простых чисел?
Потому что закон Валле-Пуссена и~Адамара говорит, что между соседними простыми числами
(в районе натурального числа~$n$) в~среднем расстояние равно $\ln n$ (\textit{натуральный логарифм
числа~$n$}). Эта функция очень медленно растет, но тем не~менее стремится к~бесконечности с ростом числа $n$.

Потом долго пытались придумать \textit{формулу для~простого числа.} Ее искали очень долго и~в~какой-то
момент переключились на~доказательство того, что ее в~принципе быть не~может. А~в~1976~году на основе решения
российским математиком \textit{Ю.\,В.~Матиясевичем} десятой проблемы Гильберта был написан \textbf{конкретный}
многочлен \text{25-й} степени с~целыми
коэффицентами от~$26$~переменных. (В формуле использованы все буквы английского алфавита: $a$, $b$, $c$, $d$ и~так
далее до~$x$, $y$, $z$.) Про этот многочлен удалось доказать, что если подставить в~него вместо букв
целые неотрицательные числа и~вычислить полученное значение, то всегда, когда результат будет положительным
числом~--- оно обязательно будет оказываться простым.


Более того, до~ЛЮБОГО простого числа <<можно добраться>> с~помощью какого-то значения именно этого
многочлена. Этот многочлен не~может в~полном смысле слова считаться <<формулой для~$n$-го простого
числа>>, но~является хорошей (и~неожиданной) заменой этой формулы.

Но~на~этом дело не~закончилось. В~2013~году открылась охота на~\textit{константу Чжана}. Это такое
число, про которое уже удалось доказать, что на~трафарете соответствующей величины, если его
двигать по~натуральному ряду, будет встречаться бесконечное количество пар простых. Вы не~поверите,
но~к~концу \text{2013} года 70000000 заменили на~число 300, даже на~246. Так сказать, еще
не~доказали, что должно быть не~более 3~жен, но~доказали, что их не~более 246.



И~вот я~этим 246-трафаретом еду по~натуральному ряду, и~бесконечное число раз считываю простые
числа, которые попали в~него. Более того, при~условии доказательства некоторого факта, в~который
все верят, но~еще не~доказали, трафарет сократят до~12, а~там и~до~исходной проблемы
чисел-близнецов доберутся\ldots\ 2500~лет люди ничего не~знали про минимальное расстояние между
простыми, полтора года назад\footnote{На момент конца 2014~года, когда читались эти лекции.}
прорывной результат, снижение этой границы. И~теперь гипотеза простых
близнецов трещит по~всем швам (впрочем, еще держится).


Еще немного про простые числа. Давайте рассмотрим ряд, который похож на~ряды с~бесконечным числом
слагаемых, которые мы уже рассматривали:
$$
\bfrac12+\bfrac13+\bfrac15+\bfrac17+\bfrac1{11}+\ldots~.
$$
Я~буду суммировать числа, обратные к~простым. Как говорят математики, <<сходится этот ряд или
расходится>>? Конечная или бесконечная сумма получится? Удивительный ответ состоит в~том, что
\textbf{бесконечная}. Это открыли в~начале XIX~века.
$$
\bfrac12+\bfrac13+\bfrac15+\bfrac17+\bfrac1{11}+\ldots=+\infty.
$$
Какое бы число~$K$ вы ни~взяли, можно указать такое натуральное число $n$, что, дойдя до~члена
с~номером $n$, сумма ряда будет превосходить число ~$K$.

А~если суммировать только обратные к простым близнецам, то сумма будет \textbf{конечна}. Это значит, что простых
близнецов заметно меньше, чем всех простых чисел. Потому что если составить ряд из~обратных
простых, то он пойдет в~бесконечность, а~если составить ряд из~обратных простых близнецов, то он
будет конечным. И~никто не~знает пока, происходит ли это потому, что он где-то закончится
и~фактически эта сумма будет конечной, или из-за того, что простых близнецов бесконечное количество,
но~зазоры между ними быстро растут. Все математики, которые этим занимаются, верят, что простые
близнецы никогда не~кончатся. Но~пока это все-таки остается гипотезой.

%Помните ли вы, что такое совершенные числа? Это числа, которые равны сумме всех своих делителей,
%кроме себя. Например, $6=1+2+3$, $28=1+2+4+7+14$.
%
%Для~нечетных совершенных чисел нет общей формулы и~даже не~известн, существуют они (то есть числа,
%а~не~формулы) или нет. А~для~четных есть формула
%$$
%2^{k}(2^{k+1}-1).
%$$
%Правда, число такого вида совершенное тогда и~только тогда, когда $(2^{k+1}-1)$~--- простое число.
%Тем не~менее никто не~знает, беконечно количество четных совершенных чисел или нет. Потому что
%никто не~знает бесконечно число простых чисел вида $(2^{k+1}-1)$ или конечно. Это называется проблема
%Мерсенна.
%
%Проблема Мерсенна заключается в~следующем: какое число вида $(2^{k+1}-1)$ является простым. Бесконечное
%число таких простых чисел или не~бесконечное. Это неизвестно. Зато, про числа вида $(2^{k}+1)$, которые
%у~Гаусса встречались при~построенях, известно довольно много.
Расскажу заодно про еще одну нерешенную математическую проблему. Знаете ли вы,
что такое совершенные числа? Что означает <<мне исполнилось совершенное число лет>>?

\textbf{Слушатель:} Это $18$-летие.

\textbf{А.С.:} Нет, это вовсе не $18$ лет! $18$~--- число несовершенное, а вот $6$ и $28$~--- совершенные
числа! Поэтому математических совершеннолетий в жизни каждого человека бывает ровно два~--- это когда
человеку исполняется $6$~лет, и затем~--- когда исполняется $28$.

По определению, совершенное число~--- это такое число, которое равно сумме всех своих делителей,
кроме себя самого. Например, $6=1+2+3$, $28=1+2+4+7+14$. Следующее совершенное число~---
$496$, но до этого возраста никто (кроме библейских персонажей) еще не доживал.

%\pagebreak

Для~четных совершенных чисел есть общая формула:
$$
2^{k}(2^{k+1}-1)
$$
(где $k$ обозначает произвольное натуральное число).

Правда, число такого вида совершенное тогда и~только тогда, когда $(2^{k+1}-1)$~--- простое
число. При $k=1, 2, 4$ получаются как раз совершенные числа $6, 28, 496$, а при $k=3$~--- число
$120$, совершенным не являющееся (потому что число $2^{k+1}-1=15$ в этом случае не является простым).


Тем не~менее никто не~знает, беконечно количество четных совершенных чисел или нет.
Потому что никто не~знает, бесконечно ли количество простых чисел вида $(2^{k+1}-1)$, или же оно конечно.

Последняя задача называется проблемой Мерсенна, а сами простые числа вида $(2^k-1)$~--- простыми
числами Мерсенна. Если мы поменяем в этом выражении знак минус на плюс, то получим также весьма интересную
и важную, как мы увидим ниже, задачу~--- а именно, какие из чисел вида $(2^k+1)$ являются
простыми?
 К этой задаче мы вернемся ниже, а пока я расскажу кое-что еще про совершенные
числа~--- конкретно, про {\em нечетные} совершенные числа.

Ровно $28$ лет назад (совершенное число!)\footnote{Эта лекция была прочитана в~2014~году.
Тем не менее здесь допущена ошибка~--- в 2014~году это было 27~лет назад!}
я поступил в \text{$57$-ю} школу.
 На уроках специальной математики мы
решали задачки из так называемых {\em листочков}, в которых приводились только определения
математических понятий и объектов, а все свойства объектов уже мы сами должны были доказать.

Так вот, в листочке номер $6$ (совершенное число!), в задаче под номером $6$ (sic!) речь шла
о совершенных числах. Было дано их определение, а затем сформулированы три пункта: пункт~$6(\text{a})$
предлагал доказать, что любое четное число вышеуказанного вида является совершенным,
если число $(2^{k+1}-1)$ простое; пункт~$6(\text{б})$ шел со звездочкой, и в нём требовалось доказать,
что других \textit{четных} совершенных чисел не существует;
 и наконец, пункт~$6(\text{в})$ шел с тремя~(!!)
звёздочками, и в нём предлагалось доказать, что нечетных совершенных чисел не существует.

Никогда до этого ни в одном листочке не было задачек с тремя звездочками. Редкие задачки
с двумя звездочками вызывали нездоровую конкуренцию математических самцов в нашем
классе за то, кто быстрее решит очень сложную задачку и покрасуется перед немногими и
потому особенно драгоценными для нас одноклассницами.

Увидев задачку с тремя звездочками, я бросил всё и два выходных подряд пытался ее решать.

Я исписал две общие тетради (кто постарше~--- помнит, что это такое!!!). В понедельник я шел
в школу с опущенной головой, уже представляя себе Рому Безрукавникова, Сашу Сидорова
или Сашу Стояновского у доски, взахлеб рассказывающими решение этой задачи.

\looseness=-1
Интернета в те годы не было. Поэтому неудивительно, что всё принималось за чистую монету.
Саша Шень (один из моих учителей в школе $57$) стоял у стола, народ потихоньку собирался.
Я подошел к нему, швырнул на стол свои тетрадки и сказал: <<Сдаюсь>>.

<<Ничего удивительного,~--- ответил Саша,~--- это пока что нерешенная математическая проблема.
Мы дали на авось~--- вдруг кто-нибудь из вас изловчится и решит?..>>

С тех пор прошло много лет, а воз и ныне там~--- до сих пор неизвестно, существуют ли нечетные
совершенные числа, или их нет совсем.
 Примера нет, но и доказательства несуществования~---
тоже нет. Конечно, компьютер перебирает уже лет 50--70 одно за другим и проверяет, но к
{\em абсолютному доказательству} такая проверка будет иметь отношение только в том
случае, если вдруг какое-то нечетное число и впрямь окажется совершенным.

Вернемся теперь к проблеме Мерсенна~-- точнее, к ее <<близнецу>>. Про числа вида
$(2^{k}+1)$ (которые, кстати, использовал Гаусс при построениях!) известно довольно много.

\pagebreak

Фокус состоит в~том, что такое число простым может быть только в том случае, если $k$ тоже является степенью двойки!
И~сейчас я~это докажу.

\textbf{Теорема.} \textit{Если число $(2^{k}+1)$~--- простое, то $k=2^{l}$, то есть исходное число, имеет вид $2^{2^{l}}+1$.}


\textbf{Доказательство.} Для~доказательства нам потребуется один\linebreak факт из~школьной программы:
$$x^{3}+1=(x+1)(x^{2}-x+1).$$ Вместо~$x$ можно подставить любое число.
$2^{3}+1$, $3^{3}+1$, $4^{3}+1$ раскладываются на~такие множители. На~самом деле
раскладывается любая нечетная степень плюс один, например, $x^{5}+1$:
$$
x^{5}+1=(x+1)(x^{4}-x^{3}+x^{2}-x+1).
$$
А~четную степень плюс один~ таким способом разложить не~получится.


Предположим, что $k$ не~является степенью двойки. Это означает, что у~него есть нечетный простой
делитель. У~каждого числа есть какие-то простые делители. У~некоторых чисел есть нечетные простые
делители, у~некоторых нет. Если у~кого нет нечетных простых делителей, значит, это число делится
среди простых чисел только на~2. Потому что все остальные простые числа нечетные. Но~если число
делится среди простых только на~2, то оно, очевидно, есть степень двойки. Если же у~него хотя бы один множитель будет
нечетный, то я~сделаю с~ним следующее.

Предположим, что $k$ не~является степенью 2, тогда $k$ равно некоторому нечетному числу $p$ умножить
на~какое-то число~$t$: $k=pt$.

Что же такое теперь $2^{k}+1$?
\begin{multline*}
2^{k}+1=
2^{pt}+1=
(2^{t})^{p}+1=
\\=
(2^{t}+1)((2^{t})^{(p-1)}-(2^{t})^{(p-2)}+(2^{t})^{(p-3)}-\ldots+1).
\end{multline*}
Я~разложил $2^{k}+1$ на~множители, очевидно отличные от 1 и даже положительные, значит, простым оно быть не~может.

Поэтому число $2^{k}+1$ может быть простым только в~том случае, если $k\brokenrel{=}2^{l}$~--- степень двойки.


Пьер Ферма полагал, что все такие числа простые. Это была гипотеза Ферма. Он написал в~свей
тетрадке <<мне кажется, что все эти числа простые>>. Он не~был уверен, ему только казалось. Первое
такое число: $2^{2}+1 = 5$~--- простое.
Следующее: $2^{4}+1=17$~--- простое.
Дальше, $2^{8}+1 = 257$~--- простое,
$2^{16}+1=65537$~--- простое.

Однако уже следующее число такого вида оказалось составным. Ферма ошибся. Но, вообще, Ферма обычно не~ошибался. Есть такой
принцип, <<Ферма ни разу не~обманул>>\footnote{Согласно другим источникам, наоборот, Ферма сформулировал целый ряд неверных <<теорем>>.}.
 Все утверждения, которые он сделал с~пометкой \textit{<<это удалось
строго доказать>>}, впоследствии были доказаны. Он не~оставлял доказательств, предлагая поверить ему
на~слово. Однако, в этом конкретном случае он написал: <<мне кажется>>. И~таки нет: $2^{32}+1$ раскладывается на~множители.
Единственным исключением из~<<правила Ферма>> была великая теорема Ферма, ее никак ни могли доказать. Но~потом
она тоже перестала быть исключением. Ее тоже доказали. Проблема только в~том,
что то доказательство, которое сейчас существует, ни при~каких условиях не~мог выдумать сам Ферма.
Оно содержит настолько сложную математику, которую Ферма не~мог знать. Но~ведь могло быть, что
Уайлз (доказавший Великую Теорему Ферма в~1993--1994~годах) <<ехал из~Москвы в~Питер через Киев>>, а~Ферма ехал
напрямую. Никто этого не знает наверняка!


Давайте вернемся к~числам $n=3, 5, 17, 257, 65537, \ldots$~. Их назвали {\em простыми числами Ферма}.
Они замечательны тем, что такие правильные $n$-угольники строятся циркулем и~линейкой.
Первый нетривиальный из них, 17-угольник, построил ещё Гаусс.  А затем Ванцель
доказал следующую общую теорему: правильный $n$-угольник может быть построен
с помощью циркуля и~линейки в~том и~только том случае, если в разложение числа $n$ на простые множители
(единственное, в силу Основной Теоремы\vadjust{\pagebreak} Арифметики!) входят только степени двойки, а также простые
числа Ферма: $n=2^k p_1 p_2 \ldots p_r$, где все простые числа $p_l$ имеют вид $p_i = (2^{2^l}+1)$.


Этой теоремой Ванцель вплотную подобрался к~современному разделу математики, который называется
<<теория полей>>.
 Математики очень любят такие интересные объекты: поля, группы и~кольца. Каждое
из~этих слов носит строгий математический смысл, совершенно не~тот, который они носят в~разговорном русском
языке.
 Еще есть термин <<идеал>>, и~даже такое понятие, как <<кольцо главных идеалов>>.

А~сейчас будет рассмотрена одна старинная проблема. Она описана Диофантом в~одном из~шести~сохранившихся томов его
произведений: \textbf{найти все прямоугольные треугольники с~целыми сторонами.}

Итак, есть прямоугольный треугольник (см. рис.~\ref{f:f1}). Согласно теореме Пифагора, доказанной
геометрическим путем в первой части книги, отношения между сторонами $a$, $b$, $c$
такого треугольника задаются формулой
$$
a^2+b^2=c^2.
$$
Таким образом, нам нужно найти все целочисленные тройки $a$, $b$, $c$, удовлетворяющие
данному квадратному уравнению. При решении этой задачи мы будем пользоваться одной
школьной {\em формулой сокращенного умножения}, а именно формулой
$$
a^2-b^2=(a-b)(a+b).
$$
Давайте докажем эту формулу геометрически (аналогично тому, как в первой части книги
мы доказали геометрическим путем саму теорему Пифагора).


%Рис. 1
\xPIC{f1}

Прежде чем приступить к решению этой задачи, давайте немного отвлечемся и~вспомним формулу сокращенного умножения
$$
a^{2}-b^{2}=(a-b)(a+b).
$$
Давайте докажем тождество $a^{2}+b^{2}=c^{2}$ геометрически.


Нарисую квадрат со~стороной~$a$. И~вырежу из~него квадратик со~стороной~$b$.

%\newpage
%$a^{2}-b^{2}=(a-b)(a+b)$\quad
%$a$\quad
%$b$\quad
%$a+b$\quad
%$a-b$
%\newpage

%Рис. 2
\xPICi{f2}{Формула сокращенного умножения <<в картинках>>.}

%\endinput

Я~хочу узнать, чему равна остающаяся площадь? Для~этого я~отрезаю прямоугольник и~приставляю его
снизу. Получаю прямоугольник со~стронами $a-b$ и~$a+b$ и~площадью $(a-b)(a+b)$.

Итак, с~одной стороны оставшаяся площадь равна $a^{2}-b^{2}$. С~другой стороны $(a-b)(a+b)$. Значит,
$a^{2}-b^{2}=(a-b)(a+b)$. Тождество доказано.


А что такое прямой угол?

\textbf{Слушатель:} 90 градусов.

\textbf{А.С.:} А если кто-то прилетел с~Марса, как ему объяснить, что значит <<прямой угол>>?

Есть безупречное определение прямого угла. Это такой угол, который, если вырезать его из~бумаги и~приставить к~самому себе, даст развернутый
угол (рис.~\ref{f:f3}).

%Рис. 3
\xPICi{f3}{Слева~--- исходный угол, справа~--- приставлена к~нему копия этого угла,
вырезанного из~бумаги. А~внизу получилась сплошная прямая линия.
Как говорится, ясно даже марсианину\ldots}

Развернутый угол~--- это прямая. Прямой угол~--- это половина развернутого угла. Это выводит нас
на~очень интересный вопрос: что имел в~виду Евклид, когда писал, что все прямые углы равны между
собой? Что такое <<равны>>? Есть одно очень важное понятие~--- \textit{движение}. Движение~---
преобразование, которое сохраняет расстояние между парами точек. Мы всегда можем померить
расстояние между точками на~плоскости. Потом мы можем плоскость поворачивать, отражать, двигать~---
главное, чтобы расстояние между точками не~менялось. Так вот <<равны>>~--- это всегда означает
<<совмещаются движением>>.

В~1872~году знаменитый немецкий математик Феликс Клейн выступил с~так называемой <<Эрлангенской
программой>>. Он сказал, что геометрия~--- это наука о~том, какие свойства фигур не~меняются
при~<<разрешенных преобразованиях>>. В~частности, школьная геометрия~--- это наука о~том, какие свойства фигур
не~меняются при~движениях. Но~преобразования бывают и~более общего рода: растяжение, инверсия. Есть
много разных преобразований. И~высокая геометрия, геометрия Лобачевского, сферическая геометрия~---
это всё примеры того, как мы следуем Эрлангенской программе Клейна. То есть геометрию можно охарактеризовать как науку
о~свойствах фигур, которые не~меняются при~преобразованиях.

Я~хотел охарактеризовать прямоугольные треугольники. Эта задача, несмотря на~то, что ее полностью
решили еще в~античном мире,~--- не~самая простая. Вы быстро найдете несколько примеров целых сторон,
для~которых верно $c^{2}=a^{2}+b^{2}$. Ну, скажем, вот $a=3$, $b=4$, $c=5$:
$$
9+16=25.
$$
Давайте посмотрим, сможем ли мы угадать еще какие-нибудь тройки? Посмотрим на картинку,
которую тоже должны изучать в~школе (в~древнегреческой школе она была!).

%Рис. 4
\xPICi{f4}{Справа~--- столбики различной высоты (по-гречески <<гномоны>>).
Высота измеряется числом клеточек, помещающихся в столбик, и указана внутри них.
Каждый гномон, начиная со~второго, <<заворачиваем>> на~90~градусов (слева).
Его площадь от этого не изменится! Вот и~получилось доказательство замечательной теоремы: <<Сумма нечетных чисел
от~1 и~до~$2n-1$ равна~$n^{2}$>>.}

Картинка на~рис.~\ref{f:f4} помогает понять тот факт, что сумма
$1+3+5+7+9+\ldots$ на~любом шаге вычислений дает квадрат натурального числа.
Одновременно это~--- способ увидеть, чем отличаются друг от друга два соседних
квадрата. А именно, два соседних квадрата всегда отличаются на нечетное число.

Но нечетные числа тоже иногда бывают квадратами, например, 9~--- это~$3^{2}$.

Значит, в~момент, когда между соседними квадратами слой состоял из 9 квадратиков,
у~нас очевидным образом появилось решение соответствующего уравнения Диофанта.

\pagebreak

Подобным же способом можно получить бесконечное множество таких троек. Все эти
тройки будут иметь следующий специальный вид: ($a$, $b$, $a+1$). Здесь $b$~--- нечетное число, обладающее тем свойством, что в изогнутой полоске
между квадратом размера $a$ на $a$ и квадратом размера $(a+1)$ на $(a+1)$ умещается
ровно $b^2$ маленьких квадратиков (см. рис.~\ref{f:f5}).


То есть между $a^{2}$ и~$(a+1)^{2}$, между этими квадратами, иногда разница будет
являться квадратом. И именно тогда, когда нечетное число будет случайно оказываться
квадратом, будет появляться новая тройка решений уравнения Диофанта.

Какое следующее нечетное число будет квадратом? 25. Давайте посмотрим, чему равна
тогда разница между соседними квадратами. На~сколько клеточек отличаются $a^{2}$
и~$(a+1)^{2}$? На~$2a+1$.

%Рис. 5
\xPICi{f5}{Поиск одной серии решений уравнения $a^{2}+b^{2}=c^{2}$.}

Теперь мы хотим, чтобы $2a + 1$ было равно этому нашему числу $5^{2}$, то есть 25,
$$
2a + 1 = 25,\quad
a=12.
$$


%$a^{2}+b^{2}=c^{2}$
%
%$c=a+1$
%
%$b^{2}=2a+1$
%
%$(a+1)^{2}=a^{2}+2a+1$
%
%$a$
%
%$a+1$
%
%$(a,b,a+1)$
%
%\newpage

\noindent
Итак, мы получили новую тройку: $12^{2} + 5^{2} = 13^{2}$. Без~сомнения, ведь $144 + 25= 169$.

Следующий нечетный квадрат~--- 49. Появится решение 24, 7 и~25.

Кто-нибудь из~вас спросит меня: <<Может быть, это всё?>> Мы получили бесконечный ряд решений
уравнения $a^{2}+b^{2}=c^{2}$ в~целых числах. Но~они все устроены одинаково. Гипотенуза отличается
от~большего катета на~1.

Вопрос: а~есть какие-нибудь другие решения? Ответ: да. Очень много других серий. Вот вам одно
из~решений, устроенных иначе: 84, 187, 205.

Общая формула для всех решений~--- это отдельная история.



\endinput
