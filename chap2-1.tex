%Раздел II.
\chapter{Математика для гуманитариев}{Раздел II}{Математика для гуманитариев}{Раздел II}

{\small
<<Знание геометрии артиллеристу и инженеру необходимо, а каждому, кто только
чему-нибудь учиться хочет, нужно; сия наука есть истинное основание всем наукам
в свете, она научает нас здраво разсуждать, верно заключать и неопровергаемо
доказывать; она сохраняет нас от многих заблуждениев, ибо геометристу труднее
какое-нибудь предложение доказать обманчивыми доводами, нежели философу.

Эвклидовы элементы суть основании сей несравненной науки~--- необходимо
учащимся предлагать должно, и стараться, чтоб они их знали совершенно...>>

\smallskip

\textit{Всеподданейший доклад генерал-фельдцейхмейстера графа П.\,И.~Шувалова об учреждении
при артиллерии шляхетного кадетского корпуса с классом военной науки (1757~г.)}


}

\thispagestyle{empty}

\newpage

%2.1.
\section{Лекция 1}
\label{2.1}

\textbf{А.С.:} Сейчас мы рассмотрим несколько сюжетов. Некоторые мы разберем сразу, а~некоторые~--- оставим
и~потом к~ним вернемся.

Первый сюжет называется \textbf{фотосъемка}.

%Давайте представим себе такую ситуацию: на~прямой дороге расположено несколько контрольных пунктов (КП). Над
%этим отрезком дороги непрерывно идет аэрофотосъемка (рис.~\ref{f:e1})
Давайте представим себе такую ситуацию: на~прямой дороге расположено несколько контрольных пунктов (КП). Над
этим отрезком дороги непрерывно идет аэрофотосъемка (рис.~\ref{f:e1}).

%Рис. 1
\xPICi{e1}{Участок усиленного наблюдения.}

И~вот однажды сверху засекли шпиона (рис.~\ref{f:e2}).

%Рис. 2
\xPICi{e2}{<<Возле самой границы овраг. Может, в чаще скрывается враг!>>}

Требуется понять, где конкретно он находится на~дороге. Из~визуальных соображений ясно, между какими двумя КП находится шпион,
но~нам нужна точная координата. Мы видим только фотоснимок.
Мы можем запросить некоторое количество информации, например, координаты некоторых КП.
Вопрос: сколько координат нам
для~этого достаточно запросить. Задача вполне практическая. Фотосъемка~--- достаточно сложное
преобразование, относящееся к \textit{проективным}.

Что это такое? Давайте немного разберемся (см. рис.~\ref{f:e3}).

%Рис. 3
\xPICi{e3}{Схема аэрофотосъемки. Два четырехугольника~--- это область, снимаемая на~фотопленку
(внизу), и~границы кадра фотопленки (вверху). Эти две плоскости, как правило, не~параллельны друг другу. Из-за
этого искажаются соотношения расстояний между точечными объектами. Прямая внизу~---
охраняемая дорога, на~которой расположены три КП (достаточно далеко друг от~друга). Черный кружок указывает на~место обнаружения подозрительного точечного объекта.
Пунктирные линии изображают отраженные лучи света, исходящие от~точечных объектов на~дороге и~фиксируемые на~кадре пленки.}

При фотографировании происходит перенос каждой точки местности вдоль лучей по~направлению к точке съемки. Прямая, конечно,
переходит в~прямую при~таком проецировании. Но~вот соотношения отрезков-рас\-стоя\-ний
становятся другими.

Ясно, что одной координаты для опеределения местоположения недостаточно. Фокус в~том, что двух координат тоже недостаточно. А~вот три
координаты~--- в~самый раз. Потому что у~этого преобразования~--- у~проецирования~--- есть то, что
математики называют \textbf{<<инвариант>>}.


Если вкратце сказать, <<о чём>> математика, то она о~том, чтобы выявлять инвариантность ситуации.
То есть какие-то соотношения, которые остаются неизменными. Вот вы так измерили (расстояния между
КП), так сфотографировали, этак сфотографировали~--- некоторое соотношение координат точек на~всех
снимках будет одно и~тоже. Я~сейчас просто напишу, что остается неизменным. На~самом деле это
можно строго доказать.

\pagebreak

На~всех фотографиях, для~любых фотоаппаратов неизменным остается так называемое \textit{двойное отношение} <<ДвОт>>
четырех точек (три из~них~--- координаты КП, четвертая~--- координата подозреваемого в~шпионаже).
Оно выражается формулой
\begin{equation} %%% 2-1-1
\label{2-1-1}
\text{ДвОт} = \bfrac{(z-x)}{(z-y)}:\bfrac{(t-x)}{(t-y)}\quad
\text{(см. рис.~\ref{f:e5})}.
\end{equation}

%%Рис. 4
%\xPICi{e4}{<<Вот эти 4 пальца~--- это 4 КП. А~указательный палец~--- это угол съёмки>>.}

%Рис. 5
\xPICi{e5}{Рассчитали число <<$z$>>, и~под кустом с~такой координатой
выловили подозрительного гражданина.}

Если не~знаешь, ни за что не~угадаешь! Это число, которое можно взять и~посчитать. Оно будет одинаковым
и~для~местности, и~для~фотографии. Поэтому я~запрошу координаты трех КП, потом вычислю соотношение
на~фотографии (на~которой отражены и~положения КП, и~расположение неизвестного объекта), приравняю
его к~выражению с~реальными координатами и~точно определю реальную координату искомого объекта
(а~именно, число~$z$).

\medskip

\hrulefill

\smallskip

\textbf{Врезка 9. Как агент ДвОт ловит шпионов.}

Обозначим через $x$, $y$, $t$ координаты ближайших КП, в~районе которых был замечен шпион.

Но~тогда надо ответить на~два вопроса: где находится начало отсчета, и~какая будет единица
измерения длины? Ответ: ЭТО НЕ ИГРАЕТ РОЛИ. В~самом деле, двойное отношение координат (ДвОт)
не~изменится, если от~всех четырех координат отнять одно и~то же число; оно не~изменится также,
если все координаты умножить на~одно и~то же число. Поглядите на~формулу ДвОт (формула~\eqref{2-1-1}), и~вы
сразу поймете, почему это происходит. Итак, давайте запросим координаты точек $x$, $y$, $t$, измеренные
в~километрах до~ближайшей погранзаставы. Допустим, они равны 17, 23, 32 соответственно.\vadjust{\pagebreak} А~как же мы найдем ДвОт, если <<$z$>> нам неизвестно? А~вот
так:
$$
\text{ДвОт} = \bfrac{(z-x)}{(z-y)}:\bfrac{(t-x)}{(t-y)} =
\bfrac{(z-17)}{(z-23)}:\bfrac{(32-17)}{(32-23)}.
$$

А~теперь внимательно изучим кадр аэрофотосъемки, где были зарегистрированы три КП и~один~Ш (шпион).
Их координаты будем выражать в~миллиметрах, а~первое КП будем считать началом отсчета
(для~простоты). Обозначим эти четыре координаты за~А, Б, С, Д (где, как мы решили, $\text{А}=0$). Прочие
(ненулевые) координаты мы просто измеряем с~помощью миллиметровой линейки, приложенной
к~фотоснимку. Допустим, мы получили числа (0, 13~мм, 16~мм, 31~мм). Следовательно, мы можем найти
ДвОт уже не~в~виде формулы, а~в~виде числа:
$$
\text{ДвОт} = \bfrac{(16-0)}{(16-13)} : \bfrac{( 31-0)}{(31-13)} = 3{,}097.
$$
Приравнивая буквенное выражение ДвОт к~его числовому выражению, получаем уравнение первой степени для~нахождения <<$z$>>:
$$
\bfrac{z-17}{z-23}\cdot\bfrac{9}{15}=3{,}097.
$$
Отсюда получаем $z\approx 24{,}4$~км.


После чего агент ДвОт сообщает начальнику погранзаставы, что подозрительного человека имеет смысл
поискать на~расстоянии 24~км и~400~м от~заставы. Где он и~был найден спящим под кустом, чтобы,
дождавшись ночи, начать свою деятельность.

\smallskip

\hrulefill

\medskip


Свойства ДвОт станут понятнее, если рассмотреть следующий пример (рис.~\ref{f:e6}).

\xPICi{e6}{Изображен прямоугольный равнобедренный треугольник $KOT$. Длина катета равна 4~единицы. Из~прямого угла опущена высота $OL$ на~гипотенузу. Из~верхней вершины треугольника $K$ проведены
две пунктирные линии $KR$ и $KS$, делящие основание на~отрезочки длиной 1, 1 и~2~ед. Основание треугольника лежит на плоскости, которую фотографирует самолет (рис.~\ref{f:e3}), вершина~$K$~---
местоположение самолета, а~высота $OL$ лежит в плоскости, в~которой находится кадр фотопленки
(вторая плоскость случайно может оказаться параллельной первой; но~гораздо чаще этого не~случается).
Точки пересечения линий $KR$ и $KS$ с высотой $OL$ обозначены за $M$ и $N$ соответственно.
Длины отрезков $OM$, $MN$, $NK$ равны~--- $6a$, $4a$, $5a$ соответственно, где $a=\bfrac{2\sqrt{2}}{15}$ (для полноты картины!).}

Здесь производится <<одномерная>> фотосъемка линии $ORST$ из точки $K$ на <<линию кадра>> $OMNL$.
Конечно, в~реальной ситуации будет не линия, а плоскость кадра, и лежать она будет значительно
ближе к~точке~$K$. Но~суть дальнейшего исследования можно изложить и~на~таком условном рисунке.

Прежде всего отметим, что если бы линия кадра была параллельна фотографируемой линии, то
соотношение расстояний между точками $O,R,S,T$ и точками $O,M,N,L$ было бы одинаковым
(и равным $1:1:2$), и~никакого <<двойного отношения>> нам бы не~понадобилось.

В~случае же, когда параллельности плоскостей нет, произойдет искажение этого соотношения.

Вычислим, насколько сильным оно будет. Уравнения прямых $KR$, $KS$
легко получить по~формуле <<уравнение в~отрезках>>: $\bfrac x1 + \bfrac y4 = 1$~($KR$)
и~$\bfrac x2 + \bfrac y4 = 1$~($KS$).\footnote{Убедиться в том, что уравнения прямых $KR$ и $KS$
имеют такой вид, можно с помощью подстановки в уравнения координат точек $K,R$ и $K$, $S$ соответственно.
Как известно, прямая однозначно определяется по любым двум своим различным точкам.} Уравнение же высоты
и~того проще~--- оно имеет вид <<$y=x$>>.

Поэтому мы легко находим координаты точек $M$, $N$:\linebreak $M(4/5, 4/5)$ и~$N(4/3, 4/3)$, а также
обычное тройное отношение отрезков $OM : MN : NL = 6:4:5$ (а~не~$1:1:2$, как было <<на~местности>>).
Можно теперь ввести координаты на прямой $OL$ таким образом, что точка $M$ получит координату 6,
точка $N$ координату 10, а точка $L$ координату 15. При этом поменяется масштаб, но он на двойное
отношение четырех точек влияния не оказывает.

Теперь мы убедимся, что <<ДвОты>> для точек $O$, $R$, $S$, $T$ и~для точек $O$, $M$, $N$, $L$ будут
СОВПАДАТЬ, несмотря на то, что обычные отношения для~них не~совпали.

В~самом деле, для~точек $O$, $R$, $S$, $T$ $\text{ДвОт} = \bfrac{2-0}{2-1} : \bfrac{4-0}{4-1} = \bfrac 32$.
Для~точек же $O$, $M$, $N$, $L$ $\text{ДвОт} = \bfrac{10-0}{10-6} : \bfrac{15-0}{15-6} = \bfrac 32$.

\centerline{* * *}

Теперь \textbf{второй сюжет}: построения циркулем и~линейкой.

(В~11 классе я~на~экзамене по~геометрии получил такое задание, что даже и~циркулем пользоваться
было нельзя. Третий сюжет, который мы рассмотрим,~--- это построение одной линейкой. Циркуль отменяется.
Есть только линейка. Здесь всё еще веселее. Я расскажу про одну конкретную очень красивую задачу.
Но~об этом~--- ниже.)

Помните ли вы о~построениях циркулем и~линейкой? Построения циркулем и~линейкой выводят на весьма
сложные математические закономерности. Что мы умеем строить циркулем и~линейкой? Можно, например, построить
равнобедренный треугольник, если известны длина основания и~длина боковой стороны.

Ну, скажем, основание 10~см, а~боковые стороны~--- по~13~см. Линейкой проводим любую прямую,
циркулем делаем на~ней в~любом месте засечку. Циркулем же измеряем основание (10~см) и~делаем
на~прямой вторую засечку, предварительно установив иглу циркуля в~первую. Берем раствор циркуля
равным длине боковой стороны (13~см) и~ставим иглу циркуля сначала на~левый край основания, проводя
достаточно длинную дугу в~верхней полуплоскости, а~затем~--- на~правый край (и~проводим дугу
до~пересечения с~первой дугой). Взяв линейку, соединяем точку пересечения дуг сначала с~левой точкой
основания, а~потом~--- с~правой (см. рис.~\ref{f:ee6}).


\xPICi{ee6}{Построение равнобедренного треугольника с помощью циркуля и линейки.}

В~общем, всё это быстрее сделать, чем описывать. Если Вы устали, вот вам \textbf{задачка}. Один
школьник перепутал, что такое 10~см~--- основание или боковая сторона. И~из-за этого построил
не~тот треугольник, что было нужно. Как вы думаете, что сильнее исказилось (в~процентах) из-за
рассеянности: площадь треугольника или его периметр? (Ответ: площадь изменилась больше, чем на 15 процентов, а
периметр менее, чем на 10 процентов.)

Можно построить квадрат. А~вот можно ли построить правильный пятиугольник с~данной стороной? Это
не~очень просто. Но~можно. Пифагорейцы уже умели строить правильный пятиугольник. Правильный
шестиугольник построить совсем просто: строю окружность с~радиусом, равным заданной стороне 6-угольника.
 Теперь делаю подряд 6 засечек
на~окружности окружностью того же радиуса. О~чудо, шестая попадает прямо в~то место, откуда мы
начали. Так уж вышло! Далее прикладываем линейку к~первым двум засечкам и~проводим отрезок, их
соединяющий. Потом то же делаем для~следующих двух засечек, и~т.\,д., пока не~получим все 6~сторон
шестиугольника (см. рис.~\ref{f:ee7}). Итак, правильный шестиугольник построить просто. Треугольник~--- тривиально, квадрат~--- очень
просто, пятиугольник сложно, но~можно. Вопрос. Можно ли построить правильный семиугольник? Древние сломали
миллион копий в~спорах и~потратили кучу часов, пытаясь решить эту задачу, но решена она была только в~XIX~веке!

\xPICi{ee7}{О, чудо! Шестая засечка совпала с первой вершиной!}

На~самом деле, кроме этой задачи, древние оставили нам еще три \textbf{известные проблемы}.

1. Трисекция угла. Дан какой-то угол на~плоскости. Надо разделить его на~три равные части. Знаменитая
проблема древних; примерно в~том же начале XIX~века было доказано, что трисекция угла
с~помощью циркуля и~линейки невозможна.

2. Квадратура круга. Что значит <<квадратура круга>>? Дан круг. Нужно построить квадрат с~такой же
площадью. Пусть радиус круга $r$ равен единице, тогда его площадь $S$ равна $\pi$ (по формуле $S = \pi r^{2}$).
Значит, нужно построить квадрат со стороной, равной ~$\sqrt{\pi}$. Эта задача эквивалентна задаче
построения числа~$\pi$,  которая тоже была решена (в~отрицательном смысле), но~только в~конце XIX~века.
 Почему она была решена позже, чем другие задачи, оставленные нам древними?
Потому что люди очень долго не~понимали структуру числа~$\pi$. В~начале XIX~века было
доказано, что можно построить те и~только те точки плоскости, у~которых обе координаты могут быть
получены из~единицы за~конечное число операций плюс, минус, умножить, разделить, взять квадратный
корень. Берете единицу. Вам разрешается ее складывать с~самой собой. Так получатся
все натуральные числа. Если будете вычитать~--- все отрицательные. Рассмотрим дроби. Дроби
математики называют специальным термином~--- рациональные числа. То есть это числа, которые
представляются в~виде <<целое делить на~целое>>. Например, число 2~--- рациональное, так как может быть
представлено, как $\bfrac 21$ или $\bfrac 42$. То есть все целые числа также являются рациональными. Давайте
посмотрим, где будут жить точки <<целое число пополам>>. Во-первых, будут жить в~целых, потому что
любое целое, это как бы <<удвоенное целое пополам>>. Во-вторых, они будут жить посередине между
соседними целыми (рис.~\ref{f:e7}).

%Рис. 7
\xPICi{e7}{Изображение чисел вида $\bfrac mn$ при~$n=2$. Если $m$ четно, получаются целые числа (обозначены
кружками). Все прочие числа такого вида изображены вертикальной черточкой.}

А~если я~еще раз разделю пополам? Я~могу нарисовать, где живут числа, полученные из~целых делением
на~3, на~5, на~1000 и~т.\,д. Но~удивительным фактом, который был известен уже древним, является то,
что не~все числа можно представить в~виде дроби. Например, я~построю квадрат со~стороной 1 и~возьму
его диагональ. Ни одна обыкновенная дробь не~равна длине диагонали квадрата. Заметьте, что число,
равное диагонали квадрата, строится циркулем и~линейкой (так как сам квадрат циркулем и линейкой мы построить можем).


%%Рис. 8
%\xPICi{e8}{Слева вверху: как строить 6-угольник. Слева внизу: задача о~трисекции угла.
%Посередине вверху: соединяя линейкой противоположные вершины единичного
%квадрата, строим НЕ РАЦИОНАЛЬНОЕ число <<корень из~двух>>.}

Но~оно не~является рациональным числом. Это было помещено в первой части книги. То есть мы умеем теперь строить квадратные корни, потому
что диагональ квадрата выражается квадратным корнем.

Любое рациональное число можно построить циркулем и~линейкой. Давайте, например, построим~$-\bfrac{11}{7}$.
Знак минус просто означает, что число надо откладывать не~вправо от~нуля, а~влево. Чтобы построить
$\bfrac{11}7$, достаточно построить $\bfrac17$ и~отложить этот отрезок 11~раз. А~чтобы построить $\bfrac17$, придется
использовать очень удобную \textit{теорему Фалеса} (которая изучается в школе). Сейчас мы ею воспользуемся.


\textbf{Теорема Фалеса (в~простейшей формулировке).} \textit{Если на~одной из~двух прямых в~плоскости отложить
последовательно несколько равных отрезков и~через их концы провести параллельные прямые,
пересекающие вторую прямую, то они отсекут на~второй прямой равные отрезки (рис.~\ref{f:e9}).}

%Рис. 9
\xPICi{e9}{Две прямые (горизонтальная и~наклонная) выходят из~начала координат и~пересечены
системой параллельных прямых. Если на~одной из~прямых мы отсекли равные отрезки, то
на~другой прямой отрезки тоже будут равны между собой (Теорема Фалеса).}

Мне дан единичный отрезок. Отложу его по~горизонтальной оси от~начала координат и~проведу произвольную
наклонную прямую (тоже из~начала координат) (рис.~\ref{f:e9}).


На~этой прямой отложу от~начала семь равных отрезков (неважно, какой длины) и~конец последнего
отрезка соединю с~концом единичного (горизонтального) отрезка.

После этого провожу прямые, параллельные той, что соединила концы горизонтального (единичного)
отрезка и~наклонного отрезка, и~проходящие через конец предпоследнего из~семи отрезков; затем~---
через конец пред-предпоследнего, и~так далее.

По~теореме Фалеса получается, что все получившиеся на~горизонтальном единичном отрезке кусочки
равны друг другу~--- то есть мы получили~$\bfrac17$.

Корень строится немножко сложнее. Беру произвольное число $a$, которое я~уже построил циркулем
и~линейкой и~из~которого я~хотел бы извлечь квадратный корень. Откладываю справа от~него единичный
отрезок (рис.~\ref{f:e10}; для~примера рассмотрен случай $a=5$).

%Рис. 10
\xPICi{e10}{Отрезок $a=5$ отложен между $x=-3$ и~$x=2$; единичный отрезок~--- между $x=2$ и~$x=3$. Строим
полуокружность радиуса~3. Проводим перпендикуляр из~точки $x=2$ до точки пересечения с окружностью. Это и~есть отрезок длины~$\sqrt5$. Все три
треугольника~--- прямоугольные, и~все они подобны друг другу.}

Теперь я~рассматриваю новый отрезок длины $a+1$ как диаметр окружности. Делю его пополам (это мы
делать умеем) и~строю верхнюю полуокружность.

Из~точки $x=2$ (отделяющей отрезок <<$a$>> от~отрезка~1) восстанавливаю перпендикуляр. Получаю отрезок
с~концами на~окружности и~отрезке.

\textbf{Теорема:} \textit{длина полученного отрезка равна~$\sqrt5$}.

\textbf{Доказательство}. Обозначим за~$A$ и~$B$ концы диаметра, за~$M$ и~$C$~--- концы проведенного нами
перпендикуляра ($C$ ниже, чем~$M$). Треугольник $ABM$ подобен треугольнику $AMC$, так как у~них острые углы совпадают,
а~один из~углов~--- прямой. (Угол AMB прямой, как и любой угол, вписанный в полуокружность.)
 Значит, и~третьи углы равны. По~той же причине и~треугольник $ABM$ подобен треугольнику
$MBC$.
 Значит, можно записать отношение катетов малых треугольников
$$
\bfrac1x=\bfrac xa
$$
(где <<$x$>>~--- длина проведенного нами перпендикуляра); $x^{2}=a$,\linebreak $x=\sqrt{a}$,
что и~требовалось доказать.

%%Рис. 11
%\xPICi{e11}{Самый большой треугольник прямоуголен потому, что любой угол, вписанный
%в~полуокружность, является прямым. Перпендикуляр разбил этот треугольник на~два более маленьких,
%но~подобных исходному.}

Теперь мы умеем строить всякие страшные <<многоэтажные чемоданы>>. Любое выражение, которое является
результатом конечного числа операций плюс, минус, умножить, делить и~взять квадратный корень, можно
построить циркулем и~линейкой. Например,
$$
\sqrt{\bfrac{1+\sqrt{3-\sqrt{3}}}{\sqrt{\sqrt{5}}}}.
$$


\textbf{Основная теорема о построениях циркулем и линейкой} утверждает, что верно и~обратное: то есть если какую-то точку удалось построить
циркулем и~линейкой, то координаты этой точки должны быть получены с~помощью конечного числа
операций плюс, минус, умножить, разделить и~взять корень. Но~есть точки на~прямой, которые таким
образом не~выражаются, а~значит, и~не~строятся при~помощи циркуля и~линейки. Так,
число $\pi$ не~является результатом конечного числа таких операций (ни миллиона, ни миллиарда!), и, следовательно, построить его невозможно.
Доказательство этого факта придумали только в~конце XIX~века.

Более чем полвека назад, в начале XIX~века  придумали доказательство задачи о~невозможности трисекции угла с~помощью циркуля
и~линейки.
 Идея его такая. Если можно разделить любой угол на~три части, то мы могли бы построить
угол в десять градусов (так как угол в 30 градусов мы построить можем).
 Но~тогда, конечно, мы могли бы построить отрезок,
длина которого равна $\sin10^{\circ}$.


%Рис. 12
\xPICi{e12}{$\sin10^{\circ}$ построить нельзя. Это вам не~какой-нибудь $\sin72^{\circ}$!}

\pagebreak

Но~доказано, что это число построить нельзя. (Если выразить $\sin10^{\circ}$ через $\sin30^{\circ}$, то получится
кубическое уравнение, а~для~построения его решения необходимо уметь строить кубический корень.
К~сожалению, с~помощью циркуля и~линейки этого сделать нельзя.) Мы пришли к~противоречию, значит,
задачу о~трисекции угла решить невозможно.

3. Третья великая задача древности~--- удвоение куба. Вам дан кубик. Нужно построить кубик вдвое
большего объема.

Если у~исходного куба сторона равна единице, то какая сторона\linebreak у~удвоенного куба? Объем исходного куба
равен 1, значит, у~удвоенного он равен~2. По~формуле $V=a^{3}$ получаем, что сторона куба должна быть~$\sqrt[3]{2}$.
Поэтому задача, на~самом деле, очень просто формулируется. Построить корень кубический из~двух.

Сделать это циркулем и~линейкой невозможно. По~тем же соображениям, почему нельзя произвести
трисекцию угла. (Как ни странно, число $\sqrt[4]{2}$ циркулем и~линейкой построить можно! Угадайте, как?) Лет
сто назад еще так мало было известно о~числах, что математики не~имели ответа на самые очевидные вопросы:
например, иррационально ли число~$\sqrt{2}^{\sqrt3}$?\footnote{То, что этот вопрос <<очевидный>>, конечно,
является шуткой. Вопрос этот (в слегка усложненном виде) составляет одну из $23$ знаменитых
проблем Гильберта, сформулированных им в 1900 году. Ответ: это число является иррациональным.}
Приходилось чуть ли не~по~отдельности перебирать такие числа и~разбираться с~ними.

\pagebreak

Весьма трудным оказалось и число~$\pi$, потому что до~конца XIX~века не~было понятно, как
оно устроено.

4. У~четвертой великой проблемы, которая была оставлена древними, особенно интересная судьба. \textit{Какие
правильные многоугольники строятся циркулем и~линейкой?} Про нее мы говорили чуть раньше
и~сейчас еще немного поговорим.
 Древние умели строить правильные треугольники, четырехугольники,
пятиугольники, шестиугольники и~их <<производные>>. Например, десятиугольник или двенадцатиугольник.
А~вот семнадцатиугольник не~умели. Его построил в~1796~году 19-летний (обратите внимание!) Карл
Фридрих Гаусс. Процедура достаточно сложная. Не~буду скрывать. Некоторый секрет состоит в~том, что
построение нельзя придумать, не~зная, что такое комплексные числа. Комплексные числа~--- это такая
волшебная палочка. У~шаманов есть бубны, а~у~математиков~--- комплексные числа. Это такая числовая
структура, которая помогает на~ура решать задачи, кажущиеся нерешаемыми. Ну, при~чем здесь
комплексные числа, когда мы говорим о~семнадцатиугольнике? Тем не менее семнадцатиугольник строится
только с~применением комплексных чисел. Впоследствии (в~1836~г.) Пьер-Лоран Ванцель выявил критерий возможности построения
правильного многоугольника.
 Оказывается, строятся только такие правильные
$p$-угольники (где $p$~--- простое число, то есть делится только на~единицу и~на~себя), для~которых
<<$p$>> может быть записано в~виде
$$
2^{2^{k}}+1\quad
(\text{где }
k= 0, 1, 2, 3, \ldots).
$$
Например, \textit{простое} число 17 удовлетворяет этой формуле, если взять $k=2$.

В заключение дам вам простую задачу. Докажите, что если есть некоторое простое число $p$ и~простое число $q$, и~можно
построить $p$-угольник и~$q$-угольник, то можно построить и~$pq$-угольник.

Наконец, вот \textbf{третий сюжет}, который мы рассмотрим: \textit{построение одной линейкой}. Циркуль отменяется. Есть только
линейка. Здесь всё еще веселее. Казалось бы, с~линейкой многого не~достигнешь: она может лишь
соединять две уже данные точки прямой! А~ведь есть небезынтересные задачки. Например:
на~плоскости дана неравнобочная трапеция. С~помощью одной линейки разделить пополам верхнее
и~нижнее основание этой трапеции. Здесь, кажется, совсем не~за~что ухватиться. Ну, проведем две
диагонали в~этой трапеции. Ну, продолжим боковые стороны трапеции до~пересечения. Получили две
новых точки. Ну, соединим их тоже. А~дальше~--- что?

Оказывается, больше ничего. Последняя из~построенных прямых аккуратно делит оба основания пополам.
Да только как это доказать?

Докажем это <<методом Декарта>>. Разместим эту трапецию в~достаточно удобной системе
координат на~плоскости (рис.~\ref{f:e13}).


%Рис. 13
\xPICi{e13}{Для~решения задачи проведены 5 очевидных прямых, последняя из~которых
(пунктирная) как раз и~разделит оба основания пополам.}

Как следует понимать выражение <<удобная система>>? Ну, например, такая: весь объект целиком лежит
в~первой четверти, как можно больше вершин лежат на~оси иксов, а~одна из~них является точкой $(0,0)$.
Сам объект задан при~этом несколькими параметрами, через которые легко выразить различные части
объекта, а~также можно отразить некоторые особенности расположения частей объекта.

В~нашем случае удобно нижним основанием считать то, которое длиннее (а~равными они быть
не~могут~--- подумайте, почему?). Для примера, скос трапеции направим внутрь первой четверти.
Боковые стороны не~могут быть параллельными (почему?). Задать вершины трапеции (то есть 4~точки) можно четырьмя параметрами (хотя всего координат будет~8). Эти параметры обозначим $a$, $b$, $c$, $d$. А~именно: $a$~--- смещение левой верхней вершины вправо;
$b$~--- смещение правой верхней вершины относительно левой;
$c$~--- расстояние от правой нижней вершины до начала координат;
$d$~--- высота трапеции.

Как следует из~рис.~\ref{f:e13}, для~пояснения хода решения взято $a=5$, $b=3$, $c=10$, $d=4$.
Мы докажем корректность нашего построения именно в этом случае. Можно провести рассуждения
и в более общем виде, пользуясь буквенными обозначениями. Наметим план рассуждения.

Обозначая точку $(0, 0)$ буквой~$O$, а~прочие вершины буквами $S$, $N$, $M$ (против часовой
стрелки), назовем пока еще неизвестную точку пересечения диагоналей буквой~$L$, а~будущую
точку пересечения продолжений боковых сторон~--- буквой~$K$ (см.~рис.~\ref{f:e13}).

Тогда можно написать координаты вершин: $O(0, 0)$, $S(10, 0)$, $N(8,4)$, $M(5,4)$.

Легко понять, что уравнение прямой~$OM$ имеет вид $y = \bfrac{4x}{5}$, а~прямой $ON$~---
вид $y = \bfrac{x}{2}$. Уравнения же прямых $SN$, $MS$ можно получить по~формуле
<<уравнение прямой по~двум точкам>>, что при аккуратном исполнении (проверьте!) даст:
\begin{equation} %%% 2-1-2
\label{2-1-2}
SN:
y+2x=20;\quad
MS: 4x + 5y = 40.
\end{equation}

Решая одновременно два уравнения, задающие прямые $ON$ и $MS$, получаем
координаты точки пересечения диагоналей, то есть точки $L \left(\bfrac{80}{13}, \bfrac{40}{13}\right)$,
в то время как одновременное решение уравнений, задающих прямые $OM$ и $NS$, даст
нам координаты точки $K \left(\bfrac{50}{7}, \bfrac{40}{7}\right)$.


Теперь всё готово для нахождения уравнения искомой <<штрихованной прямой>> $KL$.
Получается уравнение $8x-3y=40$, которое при подстановке $y=4$ и $y=0$ даст абсциссы
построенных точек.
 Они окажутся равными $6.5$ и $5$ соответственно и разделят оба
основания нашей трапеции аккурат пополам!

Разобранная выше задача когда-то давалась на~школьных\linebreak олимпиадах примерно для~7~класса. Но~когда
<<широкие массы абитуриентов и~репетиторов>> познакомились с~методом ее решения, кто-то додумался,
как ее <<слегка изменить>> и~дать для~10 класса.

\textit{Задача.} Дана неравнобочная трапеция. С~помощью одной линейки разделить нижнее основание на~41~равную часть.

Решение этой задачи тесно опирается на~решение предыдущей и~вполне может быть найдено школьником 7~класса.
Делим оба основания пополам, затем тем же методом~--- на~4~равные части. А~потом на~8~равных частей, и~т.\,д.,
пока не~получим 64~равных части (и~на~верхнем, и~на~нижнем основании). После
чего делаем замечательный трюк: на~верхнем основании отсчитываем ровно 41~часть из~64 и~проводим
линейкой НОВУЮ БОКОВУЮ ЛИНИЮ. Получилась новая трапеция, у~которой верхнее основание аккуратно
разделено на~41~равную часть. Соединяем точки деления верхнего основания с~точкой пересечения
боковых сторон новой трапеции. Получится 40~прямых линий, продолжения которых аккуратно делят на~41~равную
часть нижнее основание.

Видите, какие <<волчьи ямы>> нам готовят на~олимпиадах. Но мне досталась еще более плохая. В~ней даже не~было
точек, через которые можно провести прямые. Прямые надо было проводить \textit{наобум}, но~ответ при~этом
получить не~<<наобумный>>, а~вполне конкретный.

Расскажу про эту очень красивую задачу. Я~получил ее на~экзамене по~геометрии
в~11 классе школы \No~57. Мой учитель дал мне эту задачу: \textit{нарисована окружность и~дана точка
вне этой окружности} (рис.~\ref{f:e14}).


%Рис. 14
\xPICi{e14}{Веди прямые <<туда, не~знаю куда>> и~построй касательную.}

\textit{Построить с~помощью линейки касательную к~окружности, проходящую через данную точку.} Дальше
случилось следующее: я~нарисовал, как это делать, но, по~некоторой причине, доказать не~смог.
(О~причине я~вам скажу позже, это будет детективная история.) Я~получаю за~экзамен три балла,
а~учитель алгебры говорит: <<Нет, наверное, у~Савватеева было помрачение\ldots\ давайте четыре
поставим>>.

Давайте посмотрим.

Дана линейка, окружность и~точка. Что делать? Можно провести несколько прямых, <<секущих>> окружность. Я~провел три прямые
<<почти наобум>> и получил шесть точек на~окружности.
 Затем их накрест соединил и~получил еще две
точки (рис.~\ref{f:e15}).

%Рис. 15
\xPICi{e15}{И~тут меня озарило!..}

Дальше я~соединил эти точки, и~мне <<внутренний голос>> подсказывает, что точки, которые получились
на~окружности, как раз и~есть точки касания.

%%Рис. 16
%\xPICi{e16}{Озарение~--- хорошо, а~доказательство~--- лучше.}

--- Да,~--- говорит мне экзаменатор,~--- это правильная конструкция. Докажи. Докажи, что это~--- точки
касания.

Что такое касание в~терминах геометрии? Касание прямой и~окружности в~терминах школьной геометрии
означает, что прямая и~(полная) окружность пересекаются в~одной точке. Имеют одну общую точку. Как
же это можно доказать? Сейчас я~вам покажу такое доказательство, что у~вас от~него пойдут по~коже
мурашки. Но, во-первых, на~экзамене я~его не~дал; а~во-вторых, надо сделать предварительные
пояснения, что такое <<проективные преобразования>> (не~входящие в~общеобразовательный курс обычной
средней школы). Ну и, кстати, это же обоснование можно было сделать обычным <<методом Декарта>> (то
есть, рассмотрев некоторую систему прямоугольных координат). Правда, при~этом останется <<за
кормой>> истинная красота решения этой задачи.

\medskip

\hrulefill

\smallskip

\textbf{Врезка 10. Проективная геометрия~--- новый мир математики}

\looseness=1
Что такое <<проекция>>, знают многие. Это~--- тень, которая отбрасывается на~плоскость предметом,
освещаемым точечным источником света~--- либо находящимся недалеко (и~тогда лучи света, исходящие
из~него, расходятся), либо лежащим на~бесконечном расстоянии (тогда лучи света параллельны). Пример
для~первого случая~--- свет небольшой настольной лампы. Для~второго~--- солнечный свет. Однако тень
от~непрозрачного предмета может привести к~заблуждению. Например, можно изготовить такой предмет,
от~которого тень, отбрасываемая вдоль оси~$X$, имеет форму квадрата, вдоль оси~$Y$~--- форму круга,
а~вдоль оси~$Z$~--- равнобедренного треугольника. Чтобы изготовить такой предмет, возьмите деревянный
цилиндр (с~высотой, равной диаметру) и~аккуратно стешите топором с~двух сторон дерево так, чтобы
снизу остался круг, а~сверху он превратился бы в~отрезок, по~длине равный диаметру круга. Поэтому
при~практическом применении проективной геометрии возникает проблема, как рисовать пунктирные
линии, чтобы лучше понять структуру изучаемого предмета и~в~то же время не~сильно загромоздить
чертеж этими линиями.

Когда математики попытались дать себе более ясное представление о том, что же такое есть <<проекция>>, они
ужаснулись. Оказалось, что тенью от~окружности может служить не~окружность, а~эллипс
(<<сплющенная окружность>>) и~даже просто отрезок.
 И~длина этого отрезка может быть больше
диаметра окружности. Но~это были еще только цветочки. Оказалось, что проекцией эллипса может
оказаться\ldots\ бесконечная кривая, называемая <<ветвь гиперболы>>. Таким образом, понятие
проекции неизбежно включало в~себя \textit{бесконечно удаленные точки} (и~они спокойно могли переходить
в~обычные точки, и~наоборот). Но~тогда возник вопрос~--- в~каком же <<мире>> мы изучаем проективную
геометрию: на~прямой, на~плоскости, в~пространстве? Ответ: не~на~прямой, а~на~\textit{проективной прямой}.
Она получается из~обычной прямой добавлением одной новой точки: бесконечно удаленной. Казалось бы,
от~этого добавления прямая оказалась <<еще более бесконечной>> (ведь она была бесконечной
и~без~добавления новой точки). Но, как выяснили топологи, полученный объект по~своим свойствам
совпадает с~обычной окружностью. А~ведь ее мы не~считаем бесконечной!

Второй вопрос: а~как теперь быть с~плоскостью? ОТВЕТ: не с~плоскостью, а~с~\textit{проективной плоскостью}.
Она получается\linebreak из~обычной плоскости <<подклеиванием>> к~ней <<бесконечно удаленной прямой>>,
составленной из~различных бесконечно удаленных точек. Так она, наверное, бесконечная? Нет,
не~бесконечная. Но очень необычная. Гуляя по~проективной плоскости, можно совершенно незаметно
перейти с~одной стороны плоскости на~другую. Точнее говоря, на~ней НЕТ <<одной>> и~<<другой
стороны>>~--- сторона у~нее только одна. Подумайте над этим! Допустим, гуляли по~этой плоскости два
совершенно одинаковых близнеца: Alexey Savvateev и~Алексей Савватеев. Кто-то из~них остался стоять
на~месте, а~другой тем временем быстро пробежался по~проективной плоскости и~оказался с~другой
стороны~--- как раз под первым из~близнецов. Представляете, как они жестоко поспорили~--- кто
из~них стоит нормально, а~кто~--- вниз головой? ПОДСКАЗКА. Раз они живут на~этой плоскости, то их
рост (как и~толщина проективной плоскости) равен 0~см.


Насчет \textit{проективного пространства}. Вы уже догадались, что для~его получения надо к~обычному
пространству <<подклеить>> бесконечно удаленную плоскость, состоящую из~бесконечно уда\-лен\-ных точек.
Во всех случаях при проективном преобразовании вновь добавленные точки могут
спокойно переходить в~старые точки, и~наоборот. Они никак ~не~отличаются друг
от~друга. Вы скажете: <<Такого не~может быть, потому что не~может быть никогда! Как же можно одну перепутать
с~другой? Ведь прежние точки были близким к нам, а~эта связана с~бесконечностью, то есть очень далекая>>.

Так вот. В~проективной геометрии НЕТ таких понятий, как <<близкий>> и~<<далекий>>. Ее выдумали
для~других целей. Так что все точки (в~пределах понятий этой науки) одинаковы. Зато в~ней есть свои
важные понятия: прямая, плоскость, точка, двойное отношение и~многое другое.
 Например, есть такое понятие, как КОНИКА. Что это за~зверь? Это не~один зверь, а~сразу
три~--- в~нашей обычной геометрии они назывались <<эллипс>>, <<гипербола>> и~<<парабола>>. А~в~этой
геометрии они НЕРАЗЛИЧИМЫ.

И~в~заключение полезно сообщить важную информацию и~важную теорему.

\textit{Информация.} При~проективном отображении одной проективной плоскости на~другую проективную
плоскость любые три точки, лежавшие на~одной прямой, превратятся в~три точки, ТОЖЕ лежащие на~одной
прямой.

\textit{Теорема.} Для~любых двух \textbf{четверок точек} $A$, $B$, $C$, $D$ и~$M$, $N$, $P$, $Q$ (в~каждой из~четверок любые
три точки \textbf{не~лежат на~одной прямой}) существует ровно одно проективное преобразование~$f$,
для~которого $f(A)=M$, $f(B)=N$, $f(C)=P$, $f(D)=Q$.

\smallskip

\hrulefill

\medskip

А~теперь как НАДО было решать эту задачу (построить касательную из точки
к окружности одной линейкой, см. рис.~\ref{f:e14}).

Прежде всего, мы выходим из плоскости в~пространство. Выбираем там еще одну плоскость.
Я~проецирую на~нее мою картинку (рис.~\ref{f:e17}) из~некоторой точки~$A$.
Причем точка~$A$ и~новая плоскость расположены так, что прямая~$AO$ и~эта
плоскость параллельны. ($O$~--- это точка, в которой пересекаются три произвольно
проведенные мною прямые,
 а~также касательная, пока еще мною не~построенная.)

%Рис. 17
\xPICi{e17}{Исходная картинка с гипотетическим построением касательной расположена
в <<левой>> плоскости. Точка пересечения всех прямых на ней обозначена за~<<$O$>>.
Точка~$A$ еще левее. Прямая~$AO$ параллельна правой плоскости, и в ней мы видим
ту самую картинку, которая всё и объясняет.}

Моя точка~$O$ уйдет на~бесконечность, так как ей не~найдется места на~новой плоскости
(для~того мы и~брали прямую $AO$ параллельной новой плоскости), окружность превратится в~эллипс, а~все прямые,
проходившие через точку~$O$~--- в~параллельные прямые. Их точка пересечения ушла на~бесконечность,
прямые не~пересекаются, а~значит, в~геометрии Евклида они параллельны. (Мне, конечно, не~могли
дать эту задачу на~школьном экзамене в~рамках \textit{неевклидовой} геометрии.)

%Рис. 18
\xPICi{e18}{На~новой плоскости картина проясняется. Все пять
прямых (здесь я провел обе касательные) теперь пересекаются в~бесконечно удаленной точке (то есть, по-школьному, не~пересекаются).
Окружность превратилась в~эллипс (а~могла бы превратиться и~в~параболу; но~нам этого не~нужно).
Точки касания находятся в~самой верхней и~в~самой нижней точке эллипса. <<Кресты>> на~этом
рисунке стали симметричными, поэтому ясно, что их центральные точки, а~также точки касания лежат
на~одной прямой. Значит, и~прообразы этих точек лежали на~одной прямой. Это и~есть обоснование того
метода решения, который я~использовал в~этой задаче.}

А~теперь смотрите, как все просто. Наша окружность с~прямыми перешла в~следующую конструкцию
(рис.~\ref{f:e18}). На ней пучок параллельных прямых сечет эллипс. Требуется построить прямую того же
пучка, касательную к эллипсу. Очевидно (и подробно обосновано в пояснении к рисунку), что построение,
аналогичное сделанному мной, решает данную задачу.

Но~свойство касания сохраняется при~проектировании: раз прямая с~эллипсом имеет одну точку
пересечения, значит, на~прообразе (то есть, на~исходной плоскости) тоже должна быть одна общая
точка. Следовательно, это точка касания. Теорема до\-ка\-зана.

Надо только выйти в~пространство. Проектируете, превращаете в~параллельный поток, и~всё очевидно.

Почему же я~не~придумал в~школе такое простое решение? Дело в~том, что в~11~классе,
в~котором изучались проективные преобразования (а~это~--- проективное преобразование), наша
учительница литературы Зоя Александровна Блюмина решила набрать гуманитарный класс. Чем отличается
гуманитарный класс от~математического класса? Конечно же, количеством девушек. Понятно, что вся
математика для~11~класса была отменена явочным порядком! Все бегали к~гуманитарному
классу, поболтать. Я~всю проективную геометрию прогулял.

\centerline{* * *}

Давайте немножко разбавим проективную геометрию. В~математике есть некоторое количество неожиданно
прикольных задач! Они просто падают с~неба. Я~помню одну задачу, которую мне дали на~олимпиаде
в~7~классе. Задача про ученика, который сбежал с~урока и~плавает в~круглом бассейне. Учитель
его обнаружил в~бассейне, подходит к~границе бассейна с~розгами в~руках и~говорит: <<Я~тебе сейчас
всыплю. Сейчас ты только выйдешь из~бассейна, и~я~тебе всыплю>>. А~ученик отвечает: <<Нет, Вы же
не~умеете по~земле бегать быстрее меня. По~земле я~от~Вас убегу>>.~--- <<Конечно, ты от~меня убежишь,
но~ты же где-то должен высадиться из~воды на~землю, правильно? Вот там-то я~тебя и~схвачу
за~шиворот и~выпорю розгами>>.~--- <<Ну да, если Вы сможете меня перехватить в~момент, когда я~буду
вылезать из~бассейна, то~--- да. Я~вылезу в~таком месте, где Вас не~будет>>.~--- <<Нет, ничего у~тебя
не~выйдет>>.~--- <<Нет, выйдет>>.

\textit{Условия задачи следующие:} по~земле быстрее бегает ученик; а~пока он в~воде, его скорость $V$
 в~четыре раза меньше скорости бега учителя $W$, т.\,е. $V=\bfrac W4$.


Вопрос. Кто победит? Строгое решение, которое я~придумал на~олимпиаде, состоит в~следующем. Нам
дано, что скорость ученика в~воде в~четыре раза меньше скорости учителя на~берегу. Давайте нарисуем
окружность в~четыре раза меньшего радиуса, чем бассейн. Ученик по~этой окружности плывет ровно
с~такой же угловой скоростью, с~которой учитель бегает по~берегу (рис.~\ref{f:e21}).

%Рис. 21
\xPICi{e21}{Учитель пыхтит, ученик плывет не~торопясь.}

Если я~еще чуть-чуть уменьшу окружность, угловая скорость ученика будет больше угловой скорости учителя.
Тогда через некоторое время можно добиться того, что учитель и~ученик будут на~противоположных
сторонах (рис.~\ref{f:e22}).

%Рис. 22
\xPICi{e22}{Вот так и~спасся ученик. А~если~бы бассейн был квадратный?}

\pagebreak

В~этот-то момент ученик и~рванет к~берегу. Ему надо проплыть $\bfrac34$~радиуса, а~учителю пробежать $\pi$~радиусов (так как это как раз длина полуокружности).
Затраченное на~это время равно $\bfrac{0{,}75}{V}=4\cdot\bfrac{0{,}75}{W}$ для~ученика и~$\bfrac{\pi}{W}$ для~учителя. Но~$4\cdot0{,}75 =3<\pi$.
Значит, ученик спасется.

Как, кстати, можно доказать, что $\pi>3$? Если я~докажу, значит, ученик сможет убежать. Что такое
<<пи>>? Это длина половины окружности единичного радиуса. Как можно эту длину пощупать?

Был такой древний грек Евклид. Он сформулировал несколько принципов работы с~геометрией -- пять
\textit{постулатов} (то есть правил, справедливость которых очевидна). Вот четыре из~них: 1)~между двумя любыми
точками можно провести прямую, 2)~любую прямую можно неограниченно продолжать в~любую сторону,
3)~из~любой точки любым радиусом можно нарисовать окружность, 4)~все прямые углы равны. В~формулировке
употребляется слово <<равенство>>. Несмотря на~то, что оно так просто звучит, это~--- чрезвычайно
сложная математическая концепция.

Первые двадцать восемь теорем книги Евклида <<Начала>> были сформулированы и~доказаны только
с~использованием четырех постулатов. Он чувствовал, что с~пятым постулатом что-то не~так. Сейчас
я~его сформулирую: \textit{через любую точку, не~лежащую на~данной прямой, можно провести прямую,
параллельную данной, и~только одну.} То есть прямую, которая не~будет пересекать данную. У~этого
постулата длиннющая интереснейшая история.

Вернемся к~числу~$\pi$. Окружность~--- это кривая, соединяющая две точки. Так вот, по~одной из~аксиом
Евклида, отрезок прямой, который соединяет две точки, короче любой кривой, их соединяющей.
Рассмотрим вписанный шестиугольник. Каждая сторона правильного шестиугольника равна радиусу, то
есть~1 (рис.~\ref{f:e23}).

%Рис. 23
\xPICi{e23}{Почему $\pi$ больше трех.}

Он дает нижнюю оценку для~числа $\pi$. Число $\pi$ больше суммы трех сторон шестиугольника. А это как раз три. Так что
ученик точно спасется от учителя-преследователя.

Чтобы точнее оценить значение $\pi$, достаточно привести в~пример какой-нибудь правильный вписанный многоугольник,
у~которого сумма половины сторон еще больше, например, правильный 8-угольник. Сейчас мы найдем сумму его
4~сторон. Тогда само число $\pi$ должно быть еще больше полученного значения.

Правильный 8-угольник, вписанный в~круг радиуса~1, может быть разбит на~8~одинаковых равнобедренных
треугольников с~единичными боковыми сторонами и~с~углом $45^{\circ}$ между ними.
Отсюда легко подсчитать (с помощью так называемой <<теоремы косинусов>>)
что сторона этого 8-угольника имеет длину $\sqrt{2-\sqrt2}$.
 Значит, сумма четырех его сторон равна $\approx 3,0615$.
Следовательно, точное значение $\pi$ превосходит $3,0615$.

А для того, чтобы оценивать число $\pi$ всё точнее и точнее, нужно <<увеличивать число сторон до бесконечности>>.

Рассмотрим еще один пример бесконечности. Будем суммировать числа, обратные к квадратам натуральных
чисел:
$$
1 + \bfrac14 + \bfrac19 + \bfrac1{16} +\bfrac1{25} +\ldots~.
$$
Эта сумма конечная по~величине, но~бесконечная по~количеству слагаемых. (В~отличие от~суммы $1 + \bfrac12 +
\bfrac13 + \bfrac14 +\ldots$, которая неограниченно возрастает по~величине по~мере увеличения количества
слагаемых.) В~принципе, оба утверждения совершенно неочевидны, так как мы не~знаем, каким
правилам подчиняются суммы с~неограниченным количеством слагаемых.
 Сейчас я~докажу, что сумма
обратных квадратов натуральных чисел не~может превышать числа~<<2>>.

Рассмотрим квадрат со~стороной~1. Его площадь также равна~1. Теперь рассмотрим квадрат со~стороной~$\bfrac12$,
а площадью $\bfrac14$ (рис.~\ref{f:e24}). Площадь следующего квадрата со стороной $\bfrac13$ равна $\bfrac19$.

%$1$\quad
%$1$\quad
%$\bfrac12$\quad
%$\bfrac14$\quad
%$\bfrac19$\quad

%Рис. 24
\xPICi{e24}{Геометрический смысл исходной суммы.}

Нам нужно слагаемое $\bfrac19$, но~я~возьму с~запасом. Вместо квадрата со~стороной $\bfrac13$ рассмотрю
квадрат со~стороной~$\bfrac12$. И, значит, с~площадью~$\bfrac14$. Понятно, что если новая сумма будет конечной, то
и~исходная тоже должна быть конечной, потому что я~увеличил третье слагаемое.


Что я~сделаю дальше? $\bfrac1{16}$, $\bfrac1{25}$, $\bfrac1{36}$, $\bfrac1{49}$ все заменю
на~$\bfrac1{16}$. От~этого сумма опять увеличится. А~что такое одна шестнадцатая? Это площадь
квадрата со~стороной~$\bfrac14$. Получаем 4~квадратика, так как мы поменяли четыре прежних слагаемых
на~четыре раза по~$\bfrac1{16}$ (рис.~\ref{f:e25}).

\pagebreak

%Рис. 25
\xPICi{e25}{Геометрический смысл увеличенной суммы. Как известно, $\bfrac12+\bfrac14+\bfrac18+\bfrac1{16}+\ldots=1$.}

%\endinput

Следующие восемь слагаемых я~заменю на~$\bfrac1{64}$. И~они займут еще одну полосочку, вдвое меньшей ширины,
чем предыдущая.

%%Рис. 26
%\xPICi{e26}{Как известно, $\bfrac12+\bfrac14+\bfrac18+\bfrac1{16}+\ldots=1$.}

Следующие шестнадцать слагаемых~--- еще одна полосочка\linebreak и~так далее. Итоговая сумма вся, целиком,
будет меньше, чем площадь двух изначальных квадратов. Потому что каждая следующая полосочка
по~ширине вдвое меньше предыдущей. Поэтому вся сумма не~больше двух (рис.~\ref{f:e25}).

Но~хотелось бы узнать, чему она на~самом деле равна. Ее посчитал Леонард Эйлер. Один из~величайших
математиков в~истории человечества. Он выяснил, что сумма обратных квадратов равна $\bfrac{\pi^{2}}{6} \approx 1{,}6449$.
И~доказал это с~помощью дифференциального и~интегрального исчисления. Дифференциальное
и~интегральное исчисления совершают чудеса. Потому что дифференциальное и~интегральное
исчисления~--- это способ простого перешагивания через бесконечность. Как устроено доказательство?
Примерно так: предположим, что сумма меньше $\bfrac{\pi^{2}}{6}$. Тогда она меньше на~некоторую величину. Но~этого быть не~может:
мы можем взять так много членов этой суммы, что разница между ней и числом $\bfrac{\pi^{2}}{6}$ была меньше любого наперед заданного числа,
в~том числе и~этой величины.
 Про бесконечную сумму обычно показывают, что она не~может быть больше
какого-то числа по~одним соображениям, и~меньше~--- по~другим. На самом деле в доказательстве используются намного
более сложные соображения, связанные с~математическим анализом, которые и показывают, что эта сумма в~точности равна $\bfrac{\pi^{2}}{6}$.


Другая потрясающая страница развития математики связана \textbf{с~пятым постулатом Евклида}. Евклид
сформулировал пять постулатов. Но~многие теоремы он доказал, не~используя пятого постулата. Из-за
этого геометры стали думать, что пятый постулат на~самом деле не~постулат, а~теорема, и~он должен
следовать из~предыдущих четырех. С~античных времен до~середины XIX~века ученые пытались
доказать это. Сначала пытались вывести этот постулат из~других, а~потом пошли от~противного:
допустим, \text{5-й} постулат НЕВЕРЕН. Что из~этого будет следовать? Некоторых ученых интересовал вопрос:
может быть, нам только кажется, что через данную точку можно провести только одну прямую,
параллельную данной. А~на~самом деле, если рядом через эту же точку нарисовать еще одну прямую
(почти неотличимую от~первой), она тоже будет параллельна данной (только мы этого не~можем
разглядеть) (рис.~\ref{f:e27}).

%Рис. 27
\xPICi{e27}{Этот пятый постулат дал им всем прикурить!}

Значит, если так, то должна быть какая-то непротиворечивая геометрия, в~которой пятый постулат
будет заменен на~\textit{альтернативный}. Например, что через данную точку можно провести как минимум две
прямые, не~пересекающие исходную. И~вот ученые начинают пытаться привести к~логическому
противоречию такое предположение. И~каждый из~них, обнаружив какие-то необычные следствия, говорит:
ну а~это уже полный абсурд. Например, пятый постулат эквивалентен утверждению, что сумма углов
треугольника равна $180^{\circ}$. И, если пятый постулат неверен, то в~такой геометрии не~существует ни
одного треугольника с~суммой углов, равной $180^{\circ}$. Ну, это же абсурд! Значит,~--- говорили ученые,~---
всё доказано. Никто не~замечал, что абсурд наступает исключительно из-за непривычности этих
следствий, а~не~как логическое противоречие. Возьмите сферу и~посмотрите, какая будет сумма углов
у~треугольника на~сфере. Она всегда БОЛЬШЕ 180~градусов (нарисуйте треугольник на глобусе и убедитесь в этом!
Только учтите, что <<прямой>> на глобусе является кратчайший путь, то есть <<дуга большого круга>>). Этот пример показывает, что логических
противоречий в~таком факте нет.

{\tolerance=7000

Следующий шаг работы математиков привел к~эквивалентности пятого постулата и~утверждения, что
существует \textit{хотя бы два подобных, но~не~равных друг другу треугольника.} Опять же, на~сфере нет
такого понятия, как <<подобие фигур>>. Абсолютно нетривиальное утверждение про поверхность нашей
планеты. Нарисуем два треугольника на поверхности земного шара, или на глобусе, например, треугольник Москва~--- Лондон~--- Иркутск
и~треугольник (Нью-Йорк)~--- (Лос-Анджелес)~--- (Рио-де-Жанейро).
 Если мы измерим углы у этих двух
треугольников и, например, обнаружим, что они попарно равны друг другу, то и сами треугольники окажутся
равными, то есть совмещающимися движением сферы~--- поворотом, отражением или их композицией\footnote{На
самом деле можно показать, что {\em любое} движение сферы является либо поворотом, либо отражением.}.
А тогда и соответствующие стороны у них будут равными, то есть попарные расстояния между городами
окажутся одинаковыми! (Разумеется в приведённоми выше примере это не так.)

}

%%Рис. 28
%\xPICi{e28}{Тут уж и~астрономы призадумались~--- а~у~всех ли <<космических>> треугольников сумма
%углов равна в~точности 180~градусов?}

Таким образом, на~сфере (в частности, на поверхности Земли) справедлив \textbf{Признак равенства треугольников по~трем углам}.
Это можно доказать совершенно строго. (Почему же люди,~--- даже такие, как
Евклид,~--- этого не~замечали? Потому, что их жизненный геометрический опыт ограничивался
наблюдением малых участков поверхности Земли~--- и~они казались плоскими.) На~сфере вообще много
чудес. Например, на сфере, радиус которой равен 1, площадь треугольника равна сумме углов (в~радианах) минус~$\pi$. Вот такая
вот теорема. Это всё очень красивые результаты сферической геометрии. То есть в~<<абы-как заданной
геометрии>> все привычные нам утверждения не~обязательно верны. Была разработана целая наука,
объединенными усилиями многих ученых было выведено \textit{двадцать} утверждений, эквивалентных пятому
постулату. Но~никакого прямого (логического) противоречия в~его отрицании не~было. В~начале
XIX~века Гауссу написал письмо венгерский математик Янош Бойяи. Он писал, что разработал
геометрию без~пятого постулата, не~видит в~ней никаких противоречий и~спрашивал, что ему делать.
А~Гаусс ответил, что знает, что там нет противоречий, но~сказать этого вслух нельзя, потому что <<мы
разворошим осиный улей, и~нас искусают осы>>. Однако великий русский математик Николай Иванович
Лобачевский, ректор Казанского университета, в~1829 году написал: <<Геометрия, разработанная мною,
не~только не~противоречива, а~на~самом деле всё именно так и~происходит во~Вселенной>>.
 Когда Гаусс
узнал, что Лобачевский не~побоялся и~опубликовал свои результаты, он сразу предложил выбрать
Николая Ивановича в~иностранные члены германской академии наук и~перестал скрывать свои разработки
в~этой области. Лобачевский построил геометрию с~огромным количеством теорем. В~частности, одна
из~теорем гласила, что сумма углов в~ЛЮБОМ треугольнике меньше \text{180} градусов! Он даже пытался
мерить углы между звездами, чтобы доказать, что сумма углов треугольника хоть чуть-чуть, да
меньше ста восьмидесяти градусов. В~одной из~современных космологий всё именно так и~устроено
(но~для~проверки этого надо делать замеры не~в~масштабах Земли, а~в~гораздо больших масштабах).
Итак, у~любого треугольника в~геометрии Лобачевского сумма углов меньше $180^{\circ}$. И~площадь
треугольника равна сто восемьдесят градусов минус сумма углов. То есть сферическая геометрия как
бы <<выпуклая>>, а геометрия ~Лобачевского~--- вогнутая.
 Современная топология многим обязана Лобачевскому, потому
что он открыл этот <<ящик Пандоры>>.
\textit{Подведем итог <<поумнения>> человечества в~результате исследований 5-го постулата Евклида.}

\pagebreak

Возможны три типа <<геометрий>>: 1)~геометрия Евклида (сумма углов любого треугольника равна $180^{\circ}$);
2)~геометрия того типа, который исследовал Лобачевский (в~ней через точку, взятую вне прямой,
проходит МНОГО прямых, не~пересекающих данную; в~любом треугольнике сумма углов меньше $180^{\circ})$;
3)~геометрия того типа, который исследовал Риман (через точку, взятую вне прямой, не~проходит НИ ОДНОЙ
прямой, не~пересекающей данную; в~любом треугольнике сумма углов больше~$180^{\circ}$).

И все эти геометрии логически непротиворечивы!
\endinput
