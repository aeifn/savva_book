%Раздел I.
\chapter{Математика для гуманитариев}{Раздел I}{Математика для гуманитариев}{Раздел I}

{\small
<<Математику уже затем учить надо, что она ум в порядок приводит>>.

%\smallskip

\begin{flushright}
\textit{М.\,В.~Ломоносов}
\end{flushright}


}

\thispagestyle{empty}

\newpage



%1.1.
\section{Лекция 1}
\label{1.1}

\textbf{Алексей Савватеев:} Здравствуйте!

\textbf{Аудитория:} Здравствуйте!

\textbf{А.С.:} Для~начала я~расскажу кое-что о~себе.

В~\text{1990} году я~закончил \text{57-ю} школу г.~Москвы. Может быть, кто-то из~вас ее знает.
Тогда это была математическая школа. Но~в~\text{1989} году по~инициативе нашей учительницы литературы Зои
Александровны Блюминой было принято решение набрать гуманитарный класс, чтобы уравновесить огромный
перекос учащихся в~мужскую сторону в~нашей школе. Я~был в~\text{11-м} классе, а~набрали \text{9-й}.

Зачем я~об этом говорю? У~меня в~гуманитарном классе была подруга, которой я~объяснял математику.
Однажды надо было решить уравнение типа $\sin x=\bfrac12$. Я~\textit{объяснял, объяснял, объяснял,} а~потом
перешел непосредственно к~решению уравнения $\sin y=\bfrac12$. Она говорит: <<Стой... Там же был $\sin x$>>.

Я~начинаю объяснять: <<Это закрытый терм, замкнутая переменная, она уничтожится>>.
Ничего
не~понимает, в~глазах ужас. Я~продолжаю: <<Мы должны решить уравнение относительно $y$, да?>>

--- Нужно же было относительно $x$ решать, ну зачем ты <<$y$>> написал?

У~меня случился \textit{затык}, не~могу объяснить и~всё. И~тогда я~понял, что из~меня получается
очень плохой учитель. Я~привык говорить со~школьниками, которые приходили на~маткружок. А~они
не~задавали таких вопросов.

Спустя годы я~закончил мехмат МГУ и~Российскую Экономическую Школу (РЭШ), после чего ездил
по~регионам России, вел курсы повышения квалификации для~преподавателей экономики. В~Москве
считалось, что экономика~--- наука совершенно математизированная и~точная. По~крайней мере, и~в~РЭШ,
и~в~Высшей Школе Экономики до~сих пор всячески насаждается, что математика~--- это главное
в~экономике. В~регионах мы столкнулись, однако, с~преподавателями, которым было трудно перешагнуть через
вещи, для~математиков очевидные. Но~через пять дней курса многие слушатели оказывались очень
способными к математике.
Просто в~некотором месте у~них стоял заслон. Его полезно преодолеть всем, ибо
это~--- часть интеллектуальной культуры. Даже если вы никогда не~занимались математикой, некоторые
вещи знать надо... Так же как я~должен знать что-то про историю или химию.

Давайте теперь поговорим о~словосочетании <<абсолютное доказательство>>. Если вы в~общем и~целом
поймете, что это такое, то значит, мы не~зря сегодня позанимались с~вами.

Что такое абсолютное доказательство, я~объясню на~примерах. Начнем с~игры в~<<пятнадцать>>.

\textbf{Слушатель:} Пятнашки?

\textbf{Слушатель:} Шестнашки?

\textbf{А.С.:} Чтобы мы говорили об одном и~том же, я~объясню правила этой игры.

В~квадрате $4\times4$ имеется пятнадцать одинаковых квадратных фишек, пронумерованных от~1 до~15. Их
нельзя вынимать, можно только передвигать на~свободное место. Стандартная исходная позиция: 1, 2,
3, 4, 5, 6, 7, 8, 9, 10, 11, 12, 13, 14, 15 и~пустое место, которое используется для~передвижения
фишек. (См. рис.~\ref{f:1}; может быть задана и~нестандартная исходная позиция.)

%Рис. 1
\xPICi{1}{Стандартная исходная позиция игры в <<пятнадцать>>.}

%\pagebreak

Пустое место можно гнать по~всей игровой зоне, т.\,е. \textit{разрешенное действие} при~игре~--- передвижение
на~пустое место одной из~соседних с~ним фишек.

Игру придумал где-то 130~лет назад американский математик-популяризатор Сэм Лойд.\vadjust{\pagebreak} А~чуть позже он
пообещал большой приз (\$1000) тому, кто переведет комбинацию с~картинки рис.~\ref{f:2} в~исходную позицию
на~рис.~\ref{f:1}.

%Рис. 2
\xPICi{2}{Позиция, которую получили, поменяв местами фишки 14 и~15.}

Такая вот детская игра. Делайте, что хотите (в~рамках указанного правила). Передвигайте фишки как
вам угодно. Только приведите игру в~исходную позицию. Начался настоящий пятнашечный бум. Примечательно, что
на~этот момент наука алгебра в~другой части света находилась в~очень продвинутом состоянии.
Математики сказали свое веское слово, предоставив абсолютное доказательство того, что выиграть
в~такую игру невозможно. Тем не~менее ажиотаж с~игрой в~пятнашки продолжался еще много лет~--- так
много было желающих посрамить математику и~<<срубить>> тысячу долларов.


Что же означает в~этой игре <<абсолютное доказательство>>? Это значит: какие бы действия вы ни совершали,
сколько бы времени и~каким количеством способов бы
ни передвигали фишки, вы \textit{никогда, ни при~каких условиях} не~вернетесь из позиции на рис.~\ref{f:2} в~исходную позицию на рис.~\ref{f:1}.
В~частности, если кто-то предъявил такое решение, значит он~--- лгун. Он, видимо, взял, выдрал фишки
из~коробки и~расставил их в~правильном порядке. Абсолютное доказательство~--- это точное, настолько
точное утверждение, насколько вообще что-то может быть точным. Математика~--- наука точных
утверждений. Не~<<примерно>>, не~<<может быть>>, не~<<скорее всего, не~приведете>>, а~\textit{никогда, ни
при~каких условиях не~приведете, какие бы способности к~этой игре у~вас ни были.}

Я~постараюсь доказать эту теорему. Но~что значит <<постараюсь доказать>>? Что вообще означает
<<доказать>>? Что значит <<я ее докажу>>? Как вы это понимаете?

\pagebreak

\textbf{Слушатель:} Мы будем убеждены.

\textbf{А.С.:} Вот именно. Я~найду способ вас убедить. Но~с~другой стороны, это не~совсем то, что нам нужно.

Расскажу такую историю. Один рыцарь объяснял другому рыцарю математику. Первый рыцарь был очень умный,
а~второй~--- очень глупый. Второй рыцарь никак не~мог понять доказательство. И~тогда умный рыцарь
говорит: <<Честное благородное слово, это так>>. И~второй сразу поверил: <<Ну, тогда о~чем разговор.
Мы же с~Вами люди безупречной чести, и я, конечно, Вам верю. Я~полностью убежден>>.

У~нас разговор пойдет не~о~таком способе убеждения. Идея математического, абсолютного
доказательства не~в~том, что я~дам честное слово, а~в~том, что я, апеллируя к~вашему \textit{разумению},
передам вам какое-то знание, которое вы потом столь же спокойно передадите дальше. Вы придёте
и~скажете: <<Мы знаем, почему в~``пятнашки'' бессмысленно играть. Мы это знаем совершенно точно, нам это
доказал Алексей. И~не~просто доказал при~помощи какого-то там шаманства, пошаманил-пошаманил
и~сказал, что нет решения у этой задачи. Мы получили такое знание, которое сможем воспроизвести
и~доказать, что выиграть в~игру ``пятнашки'' невозможно>>.

Насчет пошаманить есть очень поучительный эпизод из~жизни математиков. В~начале XX~века жил в~Индии
математик Сринив\'аса Рамануджан. На~момент начала нашей истории ему было 26~лет. Он заваливал
письмами лондонское математическое общество, в~которых были формулы, содержащие числа <<$\pi$>> и <<$e$>>
(мы с~ними позже познакомимся) и~страшные бесконечные суммы, которым эти выражения равны. В~Лондоне
проверяют~--- всё верно. А~Рамануджан присылает всё новые и~новые письма. Профессор математики Г.~Харди приглашает его приехать
в~Англию и~рассказать, как он выводит эти формулы. Рамануджан
отвечает, что формулы сообщает ему во~сне богиня
Маха-Лакшми\footnote{По другим сведениям, это была богиня Намаккам.}. Харди, конечно, посмеялся, решив, что
индус не~хочет делиться секретом.

Английский математик пишет новое письмо, в котором пытается заверить Рамануджана, что никто
не будет претендовать на его открытие. Такое предположение оскорбляет индуса. Он отвечает,
что совершенно не~дорожит такими вещами, как авторство.

В~конце концов Рамануджан все-таки приехал в~Лондон, где стал профессором университета. Многие
присланные им формулы оказались верны. Но~далеко не~все из~предложенных им формул на сегодняшний день доказаны. Некоторые из них остаются
откровениями, которые были сообщены богиней Рамануджану.
 <<Абсолютное>> их доказательство пока неизвестно.

А~теперь отдохнем, посмотрим на~этот футбольный мяч (рис.~\ref{f:3}).

%Рис. 3
\xPICi{3}{Неужто и~здесь прячется абсолютное доказательство?}

Из~чего состоит мяч? Он сшит из~лоскутков. Вы когда-нибудь задумывались над тем, как именно сделан
футбольный мяч и~почему именно так? Это~--- чисто математический вопрос. Вы пока подумайте, где же
тут математика. А~я~приступаю к~математическому доказательству невозможности выиграть в~игру~<<15>>.

Начнем с~гораздо более простой ситуации. Возьмем доску $8\times8$ (рис.~\ref{f:4}) и~достаточно большой запас (заведомо
больший, чем нам может понадобиться) костей домино (одна доминошка покрывает две клеточки
на~доске).

%%Рис. 4
%\xPICi{4}{Смотрите на~картинку.}
%Рис. 4
\xPICi{4}{Доска $8\times8$.}

%Рис. 5
\xPICi{5}{<<Урезанная>> доска $8\times8$.}

\pagebreak

Теперь я~аккуратненько отрезаю у~квадрата $8\times8$ два противоположных угла (рис.~\ref{f:5}). Получилась фигура,
которая состоит из~62~квадратиков. Число, делящееся на~2. Поэтому почему бы не~попробовать
замостить ее доминошками. Но~если вы начнете пытаться сделать это один, два, три, четыре раза,
у~вас ничего не будет получаться.
 30~доминошек влезет, а~\text{31-я}~--- нет. Физик, когда увидит эту ситуацию,
поэкспериментирует 1000~раз и~скажет: <<Экспериментально установлен закон~--- нарисованная фигура
не~замощается доминошками $1\times2$>>. Физик\footnote{Это обидная для физиков аналогия.} также может наблюдать за~игрой в~футбол много-много раз
и~сказать: <<Экспериментально установлено, что мяч падает вниз, а~также, знаете, все остальные тела,
похоже, тоже падают вниз>>. Все знают, что все тела падают вниз. Это экспериментальный факт.
Но~доказать этот факт, в~принципе, невозможно. Никто на~свете не~гарантирует, что завтра этот закон
не~прекратит действовать. Придумают какую-нибудь гравицапу, и~всё полетит не~вниз, а~вверх. Это~---
физический закон, он не~может быть доказан. Он может быть только проверен очень много раз. Еще хуже
с~социальными и~экономическими законами, например, с~законом <<спрос рождает предложение>>.
У~экономистов много таких заклинаний. И~они очень часто не~работают. Наступает кризис, наступает
новая фаза развития социума~--- и~всё.
 Перестают быть верными старые законы. Социальная
реальность постоянно ломает стереотипы, которые связаны с~ее поведением, развитием, эволюцией.
Физическая реальность так не~делает, но тем не~менее доказательств в~ней тоже нет.

В~нашем случае с~доской мы, в~принципе, можем попробовать перебрать все варианты и~сделать
вывод~--- не~получилось. Но~сколько времени нам нужно будет потратить? Давайте примерно оценим.
Сколькими способами можно положить первую доминошку?

\textbf{Слушатель:} Тремя.

\textbf{А.С.} (показывая на~доске $8\times8$ различные положения кости домино):
Раз, два, три, четыре, пять, шесть, семь, восемь, девять...

\textbf{Слушатель:} Двумя.

\textbf{Слушатель:} Тридцатью.

\textbf{А.С.:} Ну, тридцатью~--- хорошо. А~почему двумя?

\textbf{Слушатель:} Вертикально и~горизонтально.

\textbf{А.С.:} Это два способа ее расположения. А~сколько положений на~самой доске она может занять?

\textbf{Слушатель:} Кучу.

\textbf{А.С.:} Очень-очень много. 30~--- это довольно хороший ответ. На~самом деле около 50.
Давайте исходить из~50. На~самом деле не~важно, что 30, что 50, даже 10. Потому что после того
как мы положили первую такую фишечку, сколько способов остается для~второй?

\textbf{Слушатель:} $(n-1)$.

\textbf{А.С.:} Грубо говоря, 49. Еще, на~самом деле, надо учесть порядок, в~котором мы положили
доминошки. Нужно поделить на~два. То есть 50 умножим на~49 и~поделим на~два.

Дальше кладем третью, четвертую и~так далее. И~каждый раз домножаем и~домножаем~--- количество
вариантов очень быстро растет.
 (Показывает на~доске всё новые варианты.)

{\it

Есть миф, будто математика состоит из~формул. В~\text{1993} году, когда я~учился на~\text{3-м} курсе мехмата,
я~ехал на~Урал к~тете. Со~мной в~купе ехала мама с~маленькой 4-летней дочкой. И~дочка говорит:
<<Мама, а~можно почитать дядину книгу?>> Книга моя называлась <<Алгебра>>. Мама сказала: <<Ты в~ней
ничего не~поймешь, там одни формулы>>. Я~передаю книгу и~говорю: <<Найдите первую формулу. На~какой
она странице?>> Формул в~книге по~алгебре не~очень много, и~самое страшное не~в~их количестве,
а~в~том, что они ужасающие, в~них одни буквы, даже цифр почти нет. Это, скорее, похоже на~какой-то
древний язык. Совершенно не~то в~книге, что должно быть с~точки зрения людей. Идея, что математика
состоит из~формул, столь же чудовищна, как мысль, что в~храм люди заходят, чтобы просто совершить
обряд, поставить свечку. Моя дочь говорит, например: <<Пойдем, поставим огоньки>>. Для~нее это
нормально, она маленькая. Математика~--- это вселенная, в~которой есть язык формул. Но~суть
не~в~нём, а~в~том, какие глубинные законы есть в~математике. И~вот эти законы, эту внутреннюю
красоту я~постараюсь вскрыть.

}

После такого философского отступления вернемся к~нашей \text{доске.}

Произведение, которое получится уже через 20~умножений, имеет порядок количества атомов во~вселенной (как
любят говорить в~научно-популярных книгах). Вот с~чем сравнимо количество способов, которые нужно
перебрать, чтобы заявить: <<Мы перебрали все варианты, задачу решить нельзя>>.\vadjust{\pagebreak} Надо придумать что-то
другое. И~то, что мы сейчас придумаем~--- это \textit{абсолютное доказательство}.

Может быть, у~кого-то есть идеи?

\textbf{Слушатель:} Взять площадь каждой фишки и~разделить на~нее общую площадь поля.

\textbf{А.С.:} Ничего не~выйдет. Общая площадь 62, у~каждой фишки~--- 2, значит нужна 31~доминошка,
это мы понимаем. Но~30 умещается, а~31~--- нет (рис.~\ref{f:6}).

%Рис. 6
\xPICi{6}{Внимание! На~доске уже не~квадратики, а~<<кирпичи>>!}

Последняя доминошка распадается на~два квадратика в~разных местах. И~что бы вы ни делали, последняя
будет, как заколдованная, распадаться на~два квадратика.

Теперь я~доказываю, что замостить доску доминошками невозможно.

Ведь перед нами, по~сути, шахматная доска. Давайте вернем ей ее шахматный вид. Клетки на~ней будут
то черные, то белые (рис.~\ref{f:7}).

%Рис. 7
\xPICi{7}{Теперь все <<кирпичи>> сделаем двуцветными.}

После вырезания двух угловых квадратиков, сколько черных и~сколько белых клеточек останется?

%Рис. 8
\xPICi{8}{Сеанс черной магии заканчивается полным разоблачением!}

\textbf{Слушатель:} Одних будет больше, других~--- меньше.

%\pagebreak

\textbf{Слушатель:} Одна доминошка должна покрывать и~белую, и~черную, да?

\textbf{А.С.:} Кто-то уже всё понимает (см. рис.~\ref{f:8}). Любая доминошка, уложенная на~эту
доску, покрывает и~белую, и~черную клетку. Поэтому, если бы фигуру, которую я~сейчас нарисовал,
можно было бы заложить доминошками, черных и~белых клеток было бы одинаковое количество. Но~мы
вырезали две белых. Осталось 30~белых и~32~черные клетки. Противоречие. Количества черных и белых клеток не равны друг другу.
Значит, нашу фигуру нельзя замостить доминошками. Абсолютное доказательство закончено. Не~надо
ничего перебирать.

Повторю еще раз.

\pagebreak

Я~взял урезанную с~двух сторон шахматную доску. Исходная шахматная доска имела 32~черные и~32~белые клетки.
А~в~урезанной шахматной доске пропали две белые угловые клетки. Поэтому стало 30~белых
и~32~черных. Теперь предположим на~секундочку, что мы решили задачу, и~все клетки заполнены
доминошками. Следует заметить, что каждая доминошка обязана лежать одной своей половиной на~черной,
а~другой своей половиной на~белой клеточке, как ты ее ни клади. Следовательно, если бы мы смогли
замостить эту фигуру доминошками в~количестве 31~штуки, то была бы 31~черная и~31~белая клетка.
У~нас же 32~черные и~30~белых клеток. А~значит, замостить обрезанную доску нельзя.
В~этом и~состоит \textit{препятствие}, как говорят математики, препятствие к~решению задачи. Заметьте, что мы
проводили \textit{доказательство от~противного}. Это очень важный прием. Я~предположил, что мы задачу
решили, и~привел ситуацию к~явному противоречию.

Переходим к~более сложному сюжету~--- <<разоблачению игры в~пятнадцать>>.

Сейчас вы узнаете тайну, которую почти никто не~знает: почему в~пятнашки нельзя <<выиграть>>,
то есть перевести игру из позиции на рис.~\ref{f:2} в~исходную позицию на~рис.~\ref{f:1}. Посмотрим на измененную позицию:

%Рис. 9
\xPICi{9}{На~доске выписываем <<извивающуюся змею>>.}


Глядя на~рис.~\ref{f:9}, выпишу числа от~1 до~15 в~линеечку, но~не~подряд, а~хитрым способом. Зачем
я~это сделаю, будет ясно потом. Вот они:
$$
1, \ 2, \ 3, \ 4, \ 8, \ 7, \ 6, \ 5, \ 9, \ 10, \ 11, \ 12, \ 14, \ 15, \ 13.
$$
Такой порядок движения древние греки называли <<бустрофедон>>, что в~переводе значит <<так, как
пашет бык>> (рис.~\ref{f:9n}).

%Рис. 9
\xPIC{9n}

С~помощью такого движения я~закодировал информацию об игровом поле в~виде одной строки. Обратно
раскодировать так же просто, как и~закодировать (\textit{с~точностью до~нахождения пустого места}).

Если, например, сдвинуть 14 в~угол, то при~кодировании я~получу такую же строчку (см. рис.~\ref{f:10}). Вообще, легко
понять, что правила игры~<<15>> позволяют быстро и~уверенно перегнать пустое место на~игровом поле
на~любую клетку из~шестнадцати, двигаясь бустрофедоном.

\smallskip

\textbf{Примечание.} Кодированием называется процедура изображения элементов одного множества
с~помощью элементов другого (обычно более простого) множества, желательно таким образом, чтобы
не~потерялась никакая существенная часть информации о~первом множестве.

\smallskip

%Рис. 10
\xPICi{10}{Пустое место может перемещаться вдоль <<змеи>>.}

\pagebreak

При~этом если пустое место находилось где-то в~другом месте, в~середине, например, то всегда можно
передвинуть фишки так, чтобы оно оказалось в~конце.

Теперь мы, начиная с положения
рис.~\ref{f:9}, должны каким-то образом менять это положение, гонять пустое место, чтобы прийти
к~последовательности, соответствующей рис.~\ref{f:1}:
$$
1, \ 2, \ 3, \ 4, \ 8, \ 7, \ 6, \ 5, \ 9, \ 10, \ 11, \ 12, \ 15, \ 14, \ 13.
$$
Каждый раз, когда я~переставляю пустое место, наша строка меняется. Я~хочу показать, что как бы она
ни~менялась, кое-что сохраняется. В~математике это называется словом \textit{инвариант}.

Инвариант~--- что-то, что не~меняется.

Понятие инварианта~--- одно из~ключевых математических понятий.

Итак, есть \textit{что-то}, что связано с~нашей последовательностью, что при~выполнении разрешенных действий
не~будет меняться. Что это, угадать так просто нельзя, иначе миллионы людей в~Америке и~в~Европе
не~занимались бы ерундой.

В~процессе перестановок строка будет сильно меняться, вплоть до~очень серьезного перемешивания.
Но~что-то меняться не~будет никогда. Давайте напряжемся и~поймем, что это такое.

Рассмотрим все пары чисел (чисел всего 15). Сколько всего можно составить пар?

\textbf{Слушатель:} 7.

\textbf{А.С.:} Нет. Всех различных пар. Пусть у~нас есть 15~кружочков разного цвета. Сколько
существует способов вынуть два кружка?

%Рис. 11
\xPICi{11}{Нарисуем 15 разных кружочков с~номерами.}

Пока будем считать, что порядок, в~котором мы вынимаем кружочки, нам важен. Сколькими способами я~могу взять первый кружок?

\textbf{Слушатель:} Пятнадцатью.

\textbf{А.С.:} Пятнадцатью. Правильно.

\textbf{Слушатель:} 15 факториал?

\textbf{А.С.:} Нет. 15 факториал~--- это множество всех возможных строк.

%%Рис. 11s
%\xPIC{11s}

Но~ведь гуманитарии не~обязаны знать слово <<факториал>>, о~котором мы с~вами говорим. Термин <<факториал>>
нам понадобится в~других темах, поэтому я~его определю.

Есть некоторое число~$n$. Простите, я~употреблю буковку. Умножим $n$ на~$n-1$, потом на~$n-2$ и~так далее, и, наконец, умножим на~2 и~на~1.
То, что получилось, обозначается $n!$ (это и~есть факториал числа~$n$):
$$
n!=n\cdot (n-1)\cdot (n-2)\cdot \ldots\cdot 2\cdot 1.
$$
Например, <<15 факториал>>:
$$
15!=15\cdot 14\cdot 13\cdot 12\cdot 11\cdot \ldots\cdot 2\cdot 1.
$$

\textbf{Слушатель:} Мы сейчас что-то прояснили?

\textbf{А.С.:} Нет. Было произнесено слово <<факториал>>. И~теперь я~его объясняю.

Факториал~--- это произведение подряд идущих, убывающих до~единицы натуральных чисел. В~нашей
задаче он не~понадобится (понадобится в~другой лекции).

Первый кружок вы можете выбрать \text{15} разными способами. Сколько остается способов для~выбора
второго кружка?

\textbf{Слушатели:} 14.

\textbf{А.С.:} 14. Итого? Есть такое знаменитое правило произведения. Число способов выбрать
пару~--- это произведение количества способов выбрать первый ее элемент на~количество способов выбрать второй.
Почему? Мы выбрали первый. Посмотрим, сколько пар мы с~ним можем получить. Второй выбирается \text{14}
способами, значит пар 14. А~теперь мы выбрали другой первый, с~ним тоже можно составить 14~пар.
И~так далее. Получается $14 + 14 + 14\ldots$ и~так 15~раз.

Отсюда и~берется правило произведения: $15\cdot 14$~способов.

Но~есть одна хитрость. Я~хочу посчитать пары независимо от~порядка кружочков. Чтобы вот такие пары (см. рис.~\ref{f:12}) не~различались.
Что надо сделать с~количеством способов?

%Рис. 12
\xPICi{12}{Эти пары для~нас не~разные, а~одинаковые.}

\textbf{Слушатель:} Разделить на~два.

\textbf{А.С.:} Да. Мы любую такую пару посчитали два раза. Один раз, когда мы сначала взяли синий
круг, а~потом белый. В~другой раз мы первым взяли белый круг, а~вторым~--- синий. То есть мы каждую пару посчитали два раза.
Поэтому ответ $(15\cdot 14):2 = 105$.

%%Рис. 12s
%\xPIC{12s}

Мы посчитали число имеющихся пар из~15 элементов~--- <<Цэ из~\text{15} по~2>>, как говорят математики.
<<Цэ>> означает первую букву слова combination (комбинация). См. формулу~\eqref{19n}.
\begin{equation} %19n
\label{19n}
C_{15}^{2}=\bfrac{15\cdot 14}{2}=105.
\end{equation}
%%Рис. 13
%\xPICi{13}{Формула <<цэ из~15 по~2>>}

Математики любят символы. Но~зачем они? Затем, что иначе придется очень много писать. Символы
и~язык математики нужны, чтобы сокращать запись. Почему древние греки и~римляне не~дошли
до~современных высот математики? Потому, что они тратили очень много времени на~лингвистическую
работу перевода математики в~слова (и~обратно: слов в~математику).
А~вот когда математика перешла
на~символы, начался прорыв, о~котором я~еще расскажу.

Вернемся к~нашим змейкам (формула~\eqref{19n2})\footnote{А~нет ли тут пар, в~которых первое число больше второго? Это непорядок!}.
Первая из них соответствует измененной позиции, а вторая~--- исходной:
\begin{equation} %19n2
\label{19n2}
\begin{gathered}
(1,\ 2,\ 3,\ 4,\ 8,\ 7,\ 6,\ 5,\ 9,\ 10,\ 11,\ 12,\ 14,\ 15,\ 13)\\
(1,\ 2,\ 3,\ 4,\ 8,\ 7,\ 6,\ 5,\ 9,\ 10,\ 11,\ 12,\ 15,\ 14,\ 13)
\end{gathered}
\end{equation}
%
%%Рис. 14
%\xPICi{14}{А~нет ли тут пар, в~которых первое число больше второго? Это непорядок!}

Для~каждой пары чисел в~каждой строке (а~пар всего 105) мы спрашиваем, в~правильном ли порядке написаны числа.

\textbf{Слушатель:} Частично да, частично нет.

\textbf{А.С.:} Верно. Например, 1 и~2~--- в~правильном порядке.

\textbf{Слушатель:} И~последующая пара $(2,3)$~--- тоже.

\textbf{А.С.:} Да, и~следующая, и~следующая за~ней. То есть $(4,8)$.

\textbf{Слушатель:} В~смысле <<в~правильном порядке>>?

\textbf{А.С.:} <<В~правильном>> не~значит, что числа в~паре соседние: и в~$(2, 3)$, и в $(2, 7)$~---
числа в~паре расположены в~правильном порядке.


\textbf{Слушатель:} По~возрастанию.

\textbf{А.С.:} Да, по~возрастанию. Большее следует за~меньшим. Но, например, пара $(15, 13)$
<<нарушает порядок>>, потому что вначале идет большее число, потом меньшее.

Посчитаем количество пар, которые стоят в~неправильном порядке. То есть по~убыванию.

\textbf{Слушатель:} Простите, но~ведь мы сами выбрали такую запись в~виде извивающейся змеи. Мы разве
не~могли записать как-то иначе?

\textbf{А.С.:} Могли. Могли записать иначе, но~тогда мы бы не~преуспели в~доказательстве того факта, который нам нужен.

\smallskip

{\it

Математика дает полную свободу исследователю. Когда он провел рассуждение и~сказал: <<Теперь всё
доказано>>,~--- он оправдывает всё, что построил. Математик скажет: <<Рассмотрим то-то и~то-то>>. Зачем?
Ужас, зачем это рассматривать? А~потом раз~--- и~всё получилось (невзирая на~<<ужас>>). Математика~---
самый свободный род занятий. Никакой моды, нет, ничего. Если вы доказали недоказанную гипотезу, то
чем бы вы ни пользовались, всё прощается. Победителей не~судят (но~иногда их слегка журят
за~сложноватое доказательство).

Итак, зачем я~считаю пары и~почему так выписал змейку, пока не~будет понятно. Мы договорились
о~некотором правиле. Мы именно так выписываем числа. Вам придется принять это как есть. А~дальше
я~считаю количество пар, которые стоят в~неправильном порядке. Раз, два, три, четыре, пять, шесть...
(см. рис.~\ref{f:15}).
%$$
%(1,\ 2,\ 3,\ 4,\ 8,\ 7,\ 6,\ 5,\ 9,\ 10,\ 11,\ 12,\ 14,\ 15,\ 13)
%$$
%Рис. 15
\xPICi{15}{Вылавливание неправильных пар в~самом разгаре!}

}

\smallskip

Условно разобьем наш ряд из 15 чисел на 4 группы в соответствии с номером строки.
Рассмотрим для начала пару, элементы которой принадлежат разным группам.
Ясно, что такая пара обязательно будет <<правильной>>, так как любой элемент из~группы слева меньше
любого элемента из~группы, стоящей правее: у~нас группы от~1 до~4, от~5 до~8, от~9 до~12 и~от~13 до~15.
Значит, <<неправильные>> пары следует искать внутри групп.
В~первой и~третьей группе всё хорошо, поэтому считать надо только оставшиеся две группы.
Во~второй группе 6 неправильных пар ($8,7$; $8,6$; $8,5$; $7,6$; $7,5$; $6,5$).
В~четвертой группе чисел (для змейки, соответствующей измененной позиции) неправильных пар~--- 2.
Итого~--- 8. А сколько неправильных пар в~исходной позиции?
(См. нижнюю строку на~рис.~\ref{f:16} или в формуле \eqref{19n2} выше.)
%
%Сколько неправильных пар в~исходной позиции? (См. нижнюю строку на~рис.~\ref{f:16}.)
%\begin{gather*}
%(1,\ 2,\ 3,\ 4,\ 8,\ 7,\ 6,\ 5,\ 9,\ 10,\ 11,\ 12,\ 14,\ 15,\ 13)\\
%(1,\ 2,\ 3,\ 4,\ 8,\ 7,\ 6,\ 5,\ 9,\ 10,\ 11,\ 12,\ 15,\ 14,\ 13)
%\end{gather*}

%Рис. 16
\xPICi{16}{В верхней строке~--- восемь <<беспорядков>>, а в~нижней строке~--- девять.}

\textbf{Слушатель:} 9.

%\endinput

\textbf{А.С.:} Да, 9. Мы находимся на~подступах к~пониманию. Сейчас я~покажу, что никакие изменения
пустого места не~меняют \textit{четности} количества неправильных пар. Само количество, конечно, меняется.
У~нас оно пока равно 8, однако, если перемешать все фишки, согласно правилам игры~<<15>>, то количество
неправильно стоящих пар изменится. Но~удивительный факт состоит в~том, что вы никогда не~измените
\textit{четности} этого количества. Само количество будет прыгать в~сторону увеличения или уменьшения, но~только на~2,
на~4, на~6, словом, на~ЧЕТНОЕ число единиц.


Начнем доказывать это утверждение. Где-то есть пустое место в~коробке $4\times4$ (пусть конфигурация
чисел, окружающих его, такая, как на~рис.~\ref{f:17}).

%Рис. 17
\xPICi{17}{И~вот нашли пустое поле. Есть разгуляться где на~воле!}

%Рис. 18
\xPICi{18}{Витязь на~распутье. По~какой же дороге пойти?..}

Пустое место может сдвинуться в~\text{4} направлениях (рис.~\ref{f:18}).

Давайте рассмотрим все 4~варианта и~посмотрим, что произойдет со~змейкой.

Что происходит с~выписанной змейкой чисел, если я~передвигаю клетку с~числом 11 налево?

\textbf{Слушатели:} Ничего.

\textbf{А.С.:} Правда. А что происходит со~змейкой, если я~передвигаю клеточку с~числом 9~направо?

\textbf{Слушатели:} Ничего.

\textbf{А.С.:} Ответ верный. Два других варианта немного более сложные, но~совершенно однотипные.

Что происходит, когда клетка движется сверху вниз или снизу вверх?

\textbf{Слушатель:} У~нас появляются неправильные пары.

\textbf{А.С.:} Да, у нас либо появляются, либо пропадают неправильные пары.
Вопрос, сколько таких пар появляется и сколько пропадает? Ответ на~этот вопрос зависит от~того, где стояло пустое место.
И~вот здесь придется
рассмотреть уже 4~варианта, но~не~для~исходной стандартной змейки, а~для~\textit{любой}. От~самых простых
в~сторону самых сложных. Например, пусть в~третьей строке получилось <<9, 10, 11, пусто>> (а~номер 12
оказался в~четвертой строке за~счет каких-то предыдущих перемещений) (см. рис.~\ref{f:19}).

%Рис. 19
\xPICi{19}{Следите только за~второй и~третьей строкой.}

Записываю фрагмент змейки:
$$
\ldots 8,\ 7,\ 6,\ 5,\ 9,\ 10,\ 11,\ \text{пусто}\ \ldots
$$
Нас интересует только этот фрагмент, потому что при~движении, которое будет совершено, слева
и~справа в~змейке ничего не~изменится. Будет меняться только этот набор цифр. Расположение остальных
пар не~меняется. Внимание: <<8>> пошло вниз, пустышка~--- наверх (рис.~\ref{f:20}).

%Рис. 20
\xPICi{20}{<<Восьмерка>> и~<<пустышка>> поменялись местами.}

Как теперь будет выглядеть середина змейки? Вот так:
$$
\ldots\ \text{пусто},\ 7, \ 6, \ 5, \ 9, \ 10, \ 11, \ 8 \ \ldots
$$
Что произошло? Восьмерка из~начала группы скакнула в~конец. Какие пары свое значение поменяли?
Группа из~шести чисел $(7, 6, 5, 9, 10, 11)$ целиком сохранилась. Она просто поменялась местами
с~восьмеркой. Значит, какие пары поменяли, как говорят математики, <<свой тип монотонности>>, то есть
возрастание сменилось убыванием (или, наоборот, убывание~--- возрастанием)?

\textbf{Слушатель:} $(8, 7)$.

\textbf{А.С.:} $(8, 7)$. Здесь теперь $(7, 8)$; а~еще?

\textbf{Слушатель:} $(8, 6)$, $(8, 5)$ \ldots

\textbf{А.С.:} При~том движении, которое я~произвел, поменяют взаимное расположение чисел
только те пары, в~которых участвовало число~8. Поэтому 6~пар изменили тип монотонности.
Если были возрастающими~--- стали убывающими, и~наоборот.

Рассмотрим каждую пару в~отдельности.

Было $(8, 5)$ (числа в~порядке убывания), стало $(5, 8)$~--- возрастание. Количество неправильных
пар изменилось на~единицу вниз. Было $(8, 10)$, стало $(10, 8)$, количество неправильных пар изменилось
на~единицу вверх. С~остальными парами~--- то же самое. Каждый раз мы добавляем или вычитаем
единицу. Не~может быть, чтобы где-то (вместо плюс/минус единицы) получился нуль, так как среди
указанных шести чисел нет восьмерок (ведь каждое число, написанное на~фишке, единственно).

Вне зависимости от~знаков, количество изменивших тип монотонности пар всегда четно. Имеется 64
способа расставить знаки, но~в~результате всегда в~качестве суммы получится четное число. Соседние
плюс/минус единички либо добавят к~сумме 2, либо добавят $(-2)$, либо взаимно уничтожатся, давая
ноль:
$$
\pm 1\pm 1\pm 1\pm 1\pm 1\pm 1
$$
В~каждой паре соседних плюс/минус единичек получится или 0, или 2 или~$-2$. То есть общее изменение
количества неправильных пар может произойти на~6, 4, 0, $-2$, $-4$, $-6$.

Изменения происходят на~четную величину, поэтому исходное количество <<беспорядков>> (оно было равно~8)
могло стать числом~14, если все единички оказались бы с~плюсом, могло остаться~8 (если бы было
$+1$, $+1$, $+1$, $-1$, $-1$, $-1$). Могло стать 6, могло 4 или 2. Но~никак не~могло стать ни 5, ни~7.

В~принципе, на~этом месте я~мог бы сказать <<остальное проверьте сами>>, потому что в~других случаях
передвижения пустой фишки происходит ровно тот же самый эффект. Но давайте для~аккуратности
проверим что-нибудь еще. Например, вверх могло пойти число~14 (вместо того, чтобы опустить вниз
число~8) (см. рис.~\ref{f:21}).

%Рис. 21
\xPICi{21}{Еще один способ <<освоения>> пустого поля.}

\pagebreak

Что произойдет, где начались изменения? Только в~нижних двух строках. Было 1, 2, 3, 4, 8, 7, 6, 5,
а~потом вместо 9, 10, 11, 14, 12, 15, 13 мы увидели 9, 10, 11, 14, 12, 15, 13. Ничего вообще
не~изменилось.

Давайте теперь представим себе внутреннюю пустую фишку. Скажем, если в позиции на рис.~\ref{f:19}
клеточку 11 сдвинули к краю, а~7 сдвинули вниз (рис.~\ref{f:22}):

%Рис. 22
\xPICi{22}{Семерка <<переселилась>> с~3-го этажа на~2-й.}


Выпишем змейку до~того, как подвинули 7:
$$
1, \ 2, \ 3, \ 4, \ (8, \ 7, \ 6, \ 5, \ 9, \ 10, \ 11), \ 14, \ 12, \ 15, \ 13.
$$

Теперь я~двигаю 7 вниз и получаю вот такой фрагмент змейки:
$$
1, \ 2, \ 3, \ 4, \ 8, \ 6, \ 5, \ 9, \ 10, \ 7, \ 11\ \ldots
$$
Выделяю в~змейке группу, которая менялась.
\begin{gather*}
\text{Было:}\ 1, \ 2, \ 3, \ 4, \ 8, \ (7, \ 6, \ 5, \ 9, \ 10), \ 11, \ 14, \ 12, \ 15, \ 13.\\
\text{Стало:}\ 1, \ 2, \ 3, \ 4, \ 8, \ (6, \ 5, \ 9, \ 10, \ 7), \ 11, \ 14, \ 12, \ 15, \ 13.
\end{gather*}
6, 5, 9, 10 переехали на шаг левее, а~7 через них перепрыгнула. Сколько будет изменений? Ровно 4. Пары
опять поменялись. Правильные стали неправильными, и~наоборот. Опять каждый раз мы прибавляем или
отнимаем единицу. И~так 4 раза. А~4 ведь~--- четное число, вот незадача. Опять результат меняется
на~четное число.

Что мы можем еще сделать? Мы могли вместо 7 подвинуть 12 (рис.~\ref{f:22n}). Тогда 12 прыгнет за~пару $(11, 14)$. Изменятся ровно две пары.

%Рис. 22n
\xPIC{22n}

\textbf{Слушатель:} То есть нечетное число поменяться не~может.

\textbf{А.С.:} Ни при~каких условиях. Мы уже знаем, что движение по~горизонтали~--- бессмысленно.
Получится та же самая змейка. Если мы движемся сверху вниз, то количество неправильных пар меняется
либо на~2, либо на~4, либо на~6, либо ничего не~меняется. Можно честно перебрать все возможные
переходы снизу вверх. Можно просто понять, что никаких других вариантов, кроме четных, нет. То есть
в~пятнашку выиграть нельзя, потому что в~стандартной исходной позиции количество неправильных пар
8, и~изменить его можно только на~четное число. А~в~требуемой позиции имеется 9~неправильных пар.

\textbf{Слушатель:} Из~любой ли позиции выиграть невозможно?

\textbf{А.С.:} Почему? На~самом деле из~половины всех исходных позиций. Из~половины невозможно, из~половины
возможно. Потому что в~<<высокой>> математике учат, что половина последовательностей имеет четное
число неправильных пар, а~половина~--- нечетное\footnote{Это утверждение требует доказательства
и является далеко не очевидным. Тут как раз поможет <<искусство игры в пятнашку>>.}. Поэтому половина вариантов будет собираться
в~стандартную исходную позицию. Если пятнашки как угодно перемешать, вывалив из~коробки и~затем
вставив обратно как придется, то перестановкой фишек всегда можно прийти либо к~случаю <<13, 14,
15>>, либо к~случаю <<13, 15, 14>>.

Чтобы понять, можно ли привести фишки в~исходную позицию, нужно посчитать количество неправильных
пар в змейке, соответствующей изучаемой исходной позиции. Если оно нечетное~--- привести к исходной позиции можно. Если четное~--- то нельзя.

\textbf{Слушатель:} Какие числа можно поменять местами?

\textbf{Другой слушатель:} Например, 1 и~3 можно поменять?

\textbf{А.С.:} Если я~меняю 1 и~3 местами (было 1, 2, 3,~--- стало 3, 2, 1), то как изменилась четность?
Было отсутствие беспорядков (то есть 0), стало три беспорядка. Четность, стало быть, \textit{изменилась}.
Так что поменять в игре <<пятнадцать>> 1 и 3 местами, сохраняя остальные фишки на своих местах, тоже невозможно.
Ваши вопросы относятся к~\textit{теории групп}, основе современной алгебры. Что и~как можно поменять, чтобы
четность менялась~--- этот вопрос напрямую к~теории групп\footnote{Его можно задавать не только для игры в <<пятнадцать>>.}.
Почему ровно половина позиций имеет
четное количество беспорядков? Это тоже связано с~некоторым фактом из~теории групп. Сейчас
я~продолжу развивать эту тему. Рассмотрим <<кубик Рубика>>. Венгерский инженер Рубик достойно
продолжил дело, начатое Сэмом Лойдом.

Давайте разберем этот кубик и~соберем его обратно.


\textbf{Слушатель:} По-моему, есть даже какие-то соревнования на этот счет.

\textbf{А.С.:} На~соревнованиях надо собрать тот, который теоретически возможно собрать. Под словом
<<разобрать>> я~понимаю более радикальную операцию: <<разодрать>>.

Как только мне купили кубик Рубика, я~сразу его разодрал. Потому что мне было интересно, любую ли
позицию можно привести к~исходной. Мне было это настолько долго интересно, что на~мехмате МГУ
я~решил соответствующую задачку в~качестве зачета. Возможно (если мне не изменяет память) 12~разных расположений,
не~переводящихся друг в~друга.
 В~пятнашках~--- 2, а~для~кубика Рубика~--- 12~ситуаций. Это тоже
следует из~теории групп (по~которой я~и~сдавал зачет).

Если перевернуть угловой кубик в~кубике Рубика путем принудительного <<раздирания>>
и~восстановления его формы~--- его нельзя будет собрать. Если перевернуть центральный кубичек
в~ребре~--- тоже нельзя. Если поменять местами два кубика малой <<диагонали>>  любой грани~---
опять не~получится.
 Эти изменения и~все их сочетания задают набор различных позиций кубика Рубика,
которые нельзя собрать. Однако это~--- трудная задача.

А~теперь поговорим про мяч (рис.~\ref{f:3}). То есть, как ни странно, снова про математику.

\smallskip

{\it

Математика состоит из~двух важных составляющих: что такое число, и~что такое доказательство. Моя
старшая дочка не~могла в свое время решить задачу: есть 3 апельсина и~2 яблока, сколько всего фруктов? Она
совершенно не~понимала, как можно сложить яблоки с~апельсинами. Это же совершенно про разное. Мне
кажется, что это типичное гуманитарное мышление. Человек фокусируется на~содержании объекта
и~не~может от~него уйти. А~вот старший сын решал эту задачу, когда ему было два с~половиной года. Я~ему
говорил: <<У тебя было 3~грузовика и~2~легковушки\ldots>>~--- <<Ой, пап, давай просто $3+2$,~--- зачем
всё это\ldots\ ерунда\ldots\ Говори три и~два, и~будем складывать>>. Ведь что такое число? Число~--- это умение
абстрагироваться от~объекта.
 Говорят, в~каких-то таежных культурах, где-то далеко на~востоке
Сибири, имеются до~сих пор разные числительные для~обозначения, например, количества белых медведей
и~количества деревьев. У~них формализация числа 5 как выражающего общность пяти медведей и пяти сосен
еще не~произошла. На~осознание того, что у~5 медведей и~5 сосен есть общее, человечество потратило много тысячелетий.
И~в~тот момент, когда это осознание настало, началась математика. А~на~память об
этом процессе в~русском языке до~сих пор говорят <<сорок>> вместо <<четырьдесят>>, хотя раньше можно
было сказать <<сорок собольих шкурок>>, но~не~<<сорок деревьев>>.

}

\smallskip

А~теперь рассмотрим поближе футбольный мяч. Он состоит из~шестиугольников и~пятиугольников: двадцати шестиугольников и двенадцати пятиугольников.

Зачем? Почему так сложно? Вот вы, допустим, шьете футбольные мячи, чем вам не~угодили просто
шестиугольники? Взяли, сшили их по~краям. Плоскость, например, отлично замощается шестиугольниками.

\textbf{Слушатель:} Но~они, может быть, в~мяч не~сложатся.

\pagebreak

\textbf{А.С.:} Давайте попробуем сложить огромный мяч. Возьмите 200, 300 шестиугольников.
Плоскость-то элементарно замощается? Вот так, как я~нарисовал. Пчелиные соты (рис.~\ref{f:23}).

%Рис. 23
\xPICi{23}{Пчелы~--- они тоже математики (сами того не~зная\ldots).}

\textbf{Слушатель:} Они на~стыках не~будут совпадать.

\textbf{А.С.:} Ну тут-то, на~плоскости, вроде всё совпадает. А~потом взял, свернул очень большой
кусок плоскости и~получил мяч.

\textbf{Слушатель:} Не~остается места для~того, чтобы правильно согнуть.

\textbf{А.С.:} Я~даже не~знаю, как выразить простым языком Ваше правильное интуитивное замечание.
Но~математическая теория этого вопроса неумолима. Из~шестиугольников \textit{нельзя} собрать поверхность
шара. Вообще, никак, никаким способом~--- даже если их нарисовать на~поверхности шара в~слегка
искривленном виде\footnote{Арсению Акопяну это удалось; см. http://pub.ist.ac.at/\allowbreak\textasciitilde edels/\allowbreak hexasphere/. Обратите внимание на дату публикации :-)))).}.

\textbf{Слушатель:} А~из~пятиугольников?

\textbf{А.С.:} Сейчас мы проясним ситуацию, связанную с~пятиугольниками. Во-первых, давайте договоримся
о~том, что сшивать надо так, чтобы в~каждой вершине сходилось три образующих поверхность
мяча многоугольника. Будем называть такую сшивку \textit{регулярной}.
Сразу скажу, что никакой,
регулярной ли, не~регулярной, никакой сшивкой из~шестиугольников нельзя сшить футбольный мяч.
Но~давайте сейчас рассмотрим подробно регулярные сшивки. Возьмем всевозможные футбольные мячи,
любого размера, которые составлены из~пятиугольников и~шестиугольников.

\smallskip

\textbf{Неожиданная теорема:}\\
Если поверхность шара <<сшита>> регулярным образом из~некоторого количества~$x$ шестиугольников и~некоторого количества $y$ пятиугольников,
то $y$ обязательно равно~12.

\smallskip

\textbf{Слушатель:} В~любом случае?

\textbf{А.С.:} В~любом. Как ни экспериментируй, что ни делай, чему бы $x$ ни равнялось. $x=200$, $x=300$, \ldots\
Но~$y=12$. Ровно 12, не~12\,000, не~120. От~размера мяча не~зависит, от~размера лоскутков
не~зависит, от~того, как сшивать, не~зависит.
 Это~--- математическая теорема.

\textbf{Слушатель:} Невероятно\ldots

\textbf{А.С.:} Есть абсолютное доказательство этой теоремы. Если вы хотите сшить футбольный мяч
из~пятиугольников и~шестиугольников, пятиугольников обязательно будет ровно 12.

\textbf{Слушатель:} Какой диаметр?

\textbf{А.С.:} Не~важно: ни диаметр, ни размер лоскутков, ни то, как сшивать. Вы никогда не~сошьете
ничего другого. Какие бы приказы ни издавала\ldots\ ну, скажем, фабрика <<Спортинвентарь>>. Скажем, придет
к~власти новая футбольная партия и~скажет: <<Отныне сшивать мячи так, чтобы в~них было поровну
шестиугольников и~пятиугольников>>. Тогда их обязательно будет 12 к~12.

\textbf{Слушатель:} То есть такое тоже может быть? Прямо 12 к~12?

\textbf{А.С.:} Да. А~знаете, как еще может быть? Ноль шестиугольников и~12 пятиугольников. Ни
одного шестиугольника, одни пятиугольники.

\textbf{Слушатель:} А~зачем тогда шестиугольники?

\textbf{А.С.:} Видимо, для~того, чтобы мяч был гладкий. Ноль шестиугольников~--- 12 пятиугольников.
200 шестиугольников~--- всё равно 12 пятиугольников.

\pagebreak

\textbf{Слушатель:} Скажите, а~вот эта теорема появилась уже после футбольного мяча? Или футбольный мяч
появился раньше?

\textbf{А.С.:} Футбольный мяч появился <<чуть-чуть>> раньше. Если честно, теорему эту полностью
осознали примерно 150~лет назад. Но~этот результат, как и~очень многие другие, должен быть отнесен
к~Эйлеру. Леонард Эйлер жил больше половины жизни в~Петербурге и~похоронен там же на Смоленском лютеранском кладбище.
Он ввел в~математику понятие \textit{инварианта}. Эйлер показал, что есть в~математике такие вещи,
которые не~меняются, что бы ты ни делал. И~настоящая математика~--- это поиск таких вещей. Эйлер
доказал потрясающую по~красоте формулу, сейчас я~ее нарисую, а~может быть, даже докажу. Кстати,
есть такой архитектурный объект <<Монреальская Биосфера>> или геодезический купол, созданный
Ричардом Фуллером Бакминстером. Гигантский сегмент шара, чуть больше, чем полушар, составленный
из~маленьких шестиугольников. Я, когда его увидел, сказал: <<Нет. Нет. Нет\ldots\ Вы не~правы, там
не~могут быть все шестиугольники, либо он сильно искривлен, либо там где-то живут
пятиугольники. \text{Ищите}>>.


Мне говорят: <<Алексей, как вы это угадали? Мы нашли \text{5-уголь}\-ни\-ки>>. Эта конструкция не~полный шар,
поэтому в~ней не~12, а~примерно 7 пятиугольников.
Как же я~узнал? Теорема, математика. Она же
универсальная для~всего. Что абсолютно одинаково в~России, в~Канаде и~в~Америке? Только математика.

\textbf{Слушатель:} Положение этих пятиугольников, оно тоже определено?

\textbf{А.С.:} Нет. Можно их все сцепить в~одном месте. Только получится сильно искривленная форма.
Лучше пятиугольники разнести. Пятиугольники отвечают за~искривление. А что такое искривление?
Беру Земной шар и~рисую на~нём треугольник (рис.~\ref{f:24}).

%Рис. 24
\xPICi{24}{Сейчас мы заколдуем этот треугольник на~сфере, и~у~него все три угла будут прямыми.}

На~Земном шаре есть где развернуться. Одну из~вершин возьмем на~Северном полюсе, две другие~---
на~экваторе. А~сторонами треугольника, как и~положено в~геометрии, будем считать отрезки двух
меридианов и~отрезок экватора (ведь по~ним измеряется кратчайшее расстояние между точками на~земной
поверхности!). Вот и~получился у~нас равнобедренный треугольник, у~которого оба угла при~основании
\textit{прямые}. А~угол при~Северном полюсе~--- любой. Так давайте возьмем его тоже прямым!!!

У нарисованного нами треугольника все углы прямые. Такого не~бывает на плоскости. Это геометрия
шара, поверхности шара, и~вот с~этой геометрией связан рассматриваемый нами факт. Он открывает
очень глубокую теорию~--- дифференциальную гео\-мет\-рию, а~также теорию римановых многообразий.
Вернемся к футбольному мячу, состоящему из $x$ шестиугольников и $y$ пятиугольников, и к нашей <<неожиданной теореме>>.


\textbf{Слушатель:} Кратен ли $x$ чему-нибудь?

\textbf{А.С.:} <<$x$>> может быть равен чему угодно. А~вот <<$y$>> обязательно равен~12.

\textbf{Слушатель:} То есть четное, нечетное~--- не~важно.

\textbf{А.С.:} Абсолютно.

\textbf{Слушатель:} То есть мы можем сделать шар из~130 шестиугольников и~12 пятиугольников, или из~131 и~12?

\textbf{А.С.:} Да, надо подумать и~аккуратненько вклеить эти наши 12 пятиугольников.

\textbf{Слушатель:} А~связано ли это с~количеством сторон в~пятиугольнике и~в~шестиугольнике?


\pagebreak

\textbf{А.С.:} Безусловно. Терпение, доказывать этот факт мы будем позже. Пока что нам нужна подготовительная работа, проделанная математиком Эйлером.  Леонард Эйлер обнаружил следующий факт. Что
такое \textit{многогранник}, каждый понимает. Любой многогранник~--- это как бы изломанная поверхность шара.
Эйлер нарисовал многогранник на~шаре: спроецировал ребра и~вершины многогранника, лежащего внутри
шара, на~поверхность шара. (Слово <<спроецировал>> означает следующую
процедуру: расположил внутри стеклянного шара макет многогранника, сделанный из~проволочек, и~зажег
в~центре шара маленькую лампочку. На~поверхности шара будут видны тени от~ребер~--- это и~есть
проекции ребер.)

%Рис. 25
\xPICi{25}{Повторяя путь Эйлера, нарисуем на~шаре остов многогранника.}

И~с~помощью этого приема доказал замечательную теорему с~совершенно удивительной формулировкой.
Называется теорема <<Формула Эйлера для~многогранника>>.

Пусть у~многогранника будет:
$\text{В}$~--- количество вершин, $\text{Р}$~--- количество ребер, $\text{Г}$~--- количество граней.
Эти количества можно непосредственно подсчитать, глядя на~модель многогранника.
Тогда обязательно будет
$$
\text{В}-\text{Р}+\text{Г}=2.
$$
Независимо от~того, какой мы взяли многогранник. Теорема верна и~для~куба, и~для~тетраэдра
(рис.~\ref{f:26}), и~для~любого другого многогранника, имеющего границей <<изломанную поверхность шара>>.
Всегда это выражение будет равно~2.


%Рис. 26
\xPICi{26}{Слева~--- куб (невидимые линии не~изображены), справа~--- тетраэдр из~проволоки.}

Тетраэдр~--- это любая треугольная пирамида. Раньше в~такой форме делали молочные пакеты. Давайте
посчитаем у~молочного пакета количество вершин, ребер и~граней.
Сколько вершин у~молочного пакета?

\textbf{Слушатель:} 4.

\textbf{А.С.:} $\text{В}=4$. Сколько ребер у~нашего тетраэдра?

\textbf{Слушатели:} 6.

\textbf{А.С.:} Без~сомнения. А~граней?

\textbf{Слушатели:} 4.

\textbf{А.С.:} Верна формула? $4-6+4=2$. Верна.

А~теперь рассмотрю другую пирамиду~--- четырехугольную (рис.~\ref{f:27}).

%Рис. 27
\xPICi{27}{Схема 4-угольной пирамиды.}

У~нее 5 вершин, 8 ребер и~5 граней. Формула верна: $5-8+5=2$.

\textbf{Слушатель:} А~количество вершин и~граней всегда совпадает?

\textbf{А.С.:} Нет, ни в~коем случае не всегда.
Давайте посмотрим на~куб (рис.~\ref{f:26}, слева).

У~обычного куба~--- 8 вершин, 12 ребер и~6 граней. (Бывают еще и~необычные кубы\ldots\ например, 4-мерные.)

Снова получаем два: $8-12+6=2$.

Никуда от~этой формулы не денешься. Думаю, что до~Эйлера эту закономерность тоже кто-то замечал,
но~важно не~первым заметить, а~громко об этом заявить. Так сказать, довести до~сведения широких
масс.

Не~буду сегодня ничего больше доказывать. Вместо этого я расскажу о некоторых великих математических загадках прошлого.

Давайте вспомним формулу для~решения квадратного уравнения с~коэффициентами $a$, $b$, $c$:
$$
x=\bfrac{-b\pm \sqrt{b^{2}-4ac}}{2a}.
$$
На~самом деле не~очень важно, как конкретно она выглядит. Важно то, что это~--- универсальный метод
решения квадратного уравнения. Какие бы они ни были, эти $a$, $b$ и~$c$, если действие произвести, вы
получите какое-то число.

Тут есть две точки зрения на~эту ситуацию. Если
написана некоторая формула, то она может случайно оказаться верной для~каких-то чисел $a$, $b$,
$c$, то есть для~какого-то квадратного трехчлена. Для~одного случайно оказалась верной, для~другого
оказалась верной.
Сколько раз нужно проверять, чтобы точно сказать, что она всегда верна?
Бесконечное количество раз. Но~можно сделать иначе. Можно взять эту формулу, подставить в~исходное
уравнение
$$
ax^{2}+bx+c=0
$$
и~\textit{убедиться} в~том, что всё сократится, и~вместо символов $a$, $b$, $c$ слева
возникнет ноль. Это и~будет означать, если мы верим в~язык символов, что формула верна. У~нас всё
сократилось, в~любом случае, какие бы $a$, $b$, $c$ мы ни~взяли.


\textbf{Слушатель:} Простите, а~для~чего нужна эта формула?

\textbf{А.С.:} Для~чего она нужна? Ну, я~бы сказал так. Лично для~меня ответ такой: для~красоты.
Для~того, чтобы быть уверенным, что математика может дать какие-то универсальные рецепты вычислений. Сейчас, конечно,
компьютеры решают задачи посложнее этого уравнения, но~раньше она была нужна для~быстрого
вычисления.

Вы распределяете земельные участки, измеряете какие-то прямоугольные куски, у~вас получается
квадратное уравнение. Можно медленно прикидывать, как это сделать, а~можно быстро получить ответ.

\textbf{Слушатель:} То есть практическое применение какое-то было?

\textbf{А.С.:} Ну, раньше~--- да. Дальше эта идея развивалась так. А~что, если я~напишу уравнение:
$$
ax^{3}+bx^{2}+cx+d=0?
$$
Могу я~написать универсальную формулу, с~помощью которой можно вычислить~$x$? При~этом разрешается
складывать, вычитать, умножать, делить и~даже извлекать корни, причем любой степени. Но~больше
ничего не~разрешается.

\textbf{Слушатель:} От~куба и~дальше такого сделать нельзя.

\textbf{А.С.:} Можно; но~эту формулу не~изучают в~школе. Формула для~кубического случая была
придумана в~первой половине XVI~века. Несколько математиков работали над этой проблемой
одновременно. Сейчас формула носит имя Джироламо Кардано, но~он не~придумал ее, а~опубликовал метод
другого математика (т.\,е. <<громко об этом заявил>>).

Чтобы выписать эту формулу, мне понадобится целая доска, поэтому я~не~буду этого делать. Как только
поняли механизм решения кубического уравнения, сразу придумали формулу для~решения уравнения
четвертой степени. Она была еще страшнее. Вывел ее ученик Кардано, по~фамилии Феррари. Всё это
происходило в~XVI~веке, когда математики уже свободно обращались с~буквами, поэтому был
сформулирован самый общий вопрос. Можно ли написать формулу для~решения уравнения произвольной
степени:
$$
a_{n}x^{n}+a_{n-1}x^{n-1}+\ldots+a_{0}=0
$$
($a_{n}, a_{n-1},\ldots$~--- известные числа. Так обозначают для~удобства. А то вдруг не~хватит букв алфавита для~их обозначения?)?

Пусть она займет 10~досок, пусть она займет 100~досок. Погоня за~этой формулой продолжалась
до~конца XVIII~века. А~в~самом начале XIX~века прозрение спустилось на~несколько человек сразу,
из~которых самым главным я~считаю французского математика Эвариста Галуа (хотя первым ситуацию в общих чертах осознал Жозеф Луи Лагранж).
 Было доказано, что
никакая конечная формула не~может быть решением уравнения произвольной степени. Такой формулы
не~существует. Не~потому, что люди еще глупые или не~все формулы перебрали или, может быть, они
не~так ставили корни. Никакое выражение, содержащее плюс, минус, умножить, разделить и~извлечь
корень любой степени не~может при~подстановке в~уравнение $a_{n}x^{n}+a_{n-1}x^{n-1}+\ldots+a_{0}=0$ полностью
сократиться. Это~--- математически строгий результат начала XIX~века\footnote{Более того, никакая формула не может дать универсального ответа, даже если не требовать <<буквенного сокращения>>!}.

Еще очень известна теорема Ферма. Доказательство теоремы Ферма~--- это примерно 120~страниц
трудного текста для~очень посвященного человека.

Про нее мы поговорим потом, а~сейчас просто запишем ее формулировку. Она очень простая.

Ни для~каких целых чисел $x$, $y$, $z$, отличных от~нуля, и~никакого натурального $n$, большего 2,
не~может выполняться равенство:
$$
x^{n}+y^{n}=z^{n}.
$$

Эту теорему доказывали с~1637 по~1994~год. Впоследствии были решены еще две или три величайшие
математические проблемы прошлых веков. Сейчас математика пожинает плоды всего своего существования.

\textbf{Слушатель:} Это сделано с~помощью компьютеров?

\textbf{А.С.:} Нет. Единственное, что сделали с~помощью компьютера~--- это так называемая
<<проблема четырех красок>>. XX~век~--- прорыв в~авиации, в~космосе. Но~самый большой прорыв в~это
время был в~математике. В~ней перевернули всё вверх дном: сняли кучу гипотез, превратили их
в~теоремы. На~моей памяти сняли несколько проблем, которые стояли веками, если не тысячелетиями.

\textbf{Слушатель:} А~это правда, что у~теоремы Ферма нет практического применения?

\textbf{А.С.:} А~кто его знает? Она (точнее, метод ее доказательства) может иметь некоторое отношение к~физической модели мира. На~самом
деле, последнее, что интересно математику, это то, какое у~теоремы практическое применение.
Математика в~каком-то смысле сродни настоящей религии. Это вещь в~себе. Если она кому-то
помогает, математиков это особо не~интересует.
 Люди, которые занимаются прикладной математикой, имеют
совершенно другое настроение. Это~--- другие люди. Как, например, разнятся между собой учителя
и~чиновники. То же самое с~математиками. Человек, который формулу ищет, и~человек, который хочет
с~помощью нее что-то сделать,~--- это два разных человека.

На~этом мы закончим первую лекцию. На~следующем занятии мы будем доказывать теорему про футбольный
мяч и~формулу Эйлера.

\endinput