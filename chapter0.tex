{\small %\baselineskip=11.5pt

\section{Предисловие}

\begin{flushright}
\it
%Посвящается моей жене Марине, деточкам Мише, \\
%Гале, Свете, Юре, моим Маме и Папе, \\
%а также всем родственникам и друзьям.
Посвящается моей жене Марине, деточкам Мише, \\
Гале, Свете и Юре, любимым Маме и Папе, \\
а также всем моим родственникам и друзьям!
\end{flushright}

История написания этой книги такова. Долгие годы я ездил по всей стране с лекциями
<<Математика для экономистов>> в разных ее вариантах (теория игр, микроэкономика или
теория принятия решений~--- все эти на первый взгляд разные по содержанию дисциплины
очень быстро превращались в простой математический ликбез для преподавателей экономики).

Постепенно я понял, чем отличается гуманитарное мышление от мышления представителей
точных наук~--- понял на таком уровне, когда <<понимание>> переходит в качество преподавания.

Я стал преподавать математику (под видом всех этих <<псевдоэкономик>>) неспециалистам
таким образом, чтобы как можно скорее <<зацепить их за живое>> и заставить напряженно
думать над сложными вещами. Пришлось придумать (или накопать в разных учебниках)
целый ряд якобы жизненных ситуаций, которые при формализации давали нетривиальное
игровое взаимодействие; при поиске же равновесия в соответствующей игре слушатели
уже были готовы решать уравнения, изучать построения и графики, зачастую вспоминать
производные и интегралы и~т.\,п. Чувство, что всё это проходили когда-то в университете
только ради муштры и зазубривания, постепенно уходило.

Потом я понял, что ограничение только околоэкономическими сюжетами (и аудитория
преподавателей экономических дис\-цип\-лин) мне начинает, так сказать, жать. Душа просила
большего~--- а именно: бесед с~широким кругом нематематиков, и прежде всего гуманитариев
(технарей я всегда сторонился~--- они всюду ищут конкретику и прикладное значение).

И вот я сказал Михаилу Викторовичу Поваляеву, ктитору\footnote{<<Ктитор>>~--- по-гречески
<<строитель>>, то есть человек, не просто дающий деньги на какое-то дело, но и активно в
этом деле участвующий.} Университета Дмитрия Пожарского и одновременно директору
Филипповской Школы в Москве, что хочу где-нибудь испробовать свои идеи относительно
преподавания математики гуманитариям. Он предложил Вечерние Курсы Университета
Дмитрия По\-жар\-ского.

И тут началось.

Уже на первый запуск записалось около пары сотни слушателей, из~которых 30--40 регулярно
ходили на занятия. Иногда приходилось перемещаться в актовый зал~--- все слушатели просто
не помещались в~обычных классных комнатах Филипповской школы.

Но и это еще не всё. Уж не знаю, что тут сказалось~--- вековой голод гуманитариев по
человеческому преподаванию <<великой и ужасной>> Царицы всех наук, <<сарафанное
радио>> или моя фанатичная манера чтения лекций, но через пару лет записанные и
выложенные в интернет видеозаписи посетило несколько десятков тысяч человек.

Мне реально казалось, что я ухватил <<гуманитарного бога>> за бороду. Разумеется,
тут не могло обойтись без какой-то особенной удачи, <<звезд особого схождения>>.

И вот спустя год мне звонит Катя Богданович, учитель математики в Филипповской
Школе, и говорит: <<Лёша, я всё перенесла на бумагу>>. Я опешил~--- это же какая работа:
вырезать все мои слова-паразиты, перенести мою рваную речь с видеозаписи на текст!

Несомненно, без этих 70 часов работы (как мне призналась Екатерина!) книга бы никогда
не смогла увидеть свет~--- у меня ни в каком обозримом будущем не появилось бы столько
свободного времени. Бегло проглядев рукопись и снова посоветовавшись с Михаилом
Викторовичем Поваляевым, я понял: нужно <<пропустить>> книгу сквозь взгляд истинного
гуманитария, раз уж она предназначена именно для этой аудитории.

И я попросил известного историка Сергея Владимировича Волкова прочитать рукопись
и пометить все места, в которых мне что-то казалось очевидным, но таковым не было, или
где используются термины, которые могут гуманитария сбить с толку (<<трансцендентность>>
и <<иррациональность>>, например, как оказалось, в жизни тоже имеют какие-то значения,
о чем я сам до общения с Сергеем Владимировичем и не подозревал!).

После этого я передал рукопись своему папе, Владимиру Васильевичу Савватееву, тоже
математику по образованию и преподавателю многих математических дисциплин.
Текст был им доработан с включением так называемых <<врезок>> (мой папа их так назвал) и
иногда отдельных абзацев, выделенных \textit{курсивом}.
Папа обработал замечания
Сергея Владимировича, а также нашел и исправил множество неточностей и ошибок.

Кроме того, огромную работу проделал верстальщик Евгений Иванов. Отдельные несуразности выловил
мой ученик Саша Новиков, который прочел рукопись на одной из последних стадий готовности, а также
мой друг и коллега, ведущий специалист по комбинаторной геометрии в~нашей стране Андрей Михайлович
Райгородский. Наконец, целиком от~начала и до конца одну из финальных версий книги прочла Дарья
Орешенкова, секретарь Университета Дмитрия Пожарского, и <<выловила>> несколько дополнительных
спорных мест.

Всем, кто мне помогал в процессе работы над книгой на разных стадиях, я хочу
здесь выразить свою глубочайшую признательность и благодарность~--- ваши
поддержка и помощь были бесценны, книга без вас бы не вышла никогда!


Наконец, самое большое спасибо хочу сказать моей жене Марине за~крайне внимательное неоднократное
вычитывание всей книги, а также за моральную поддержку в течение всего этого времени!

В итоге, несмотря на окончательное наше решение оставить в качестве автора одного
меня, фактически книга является совместной работой многих людей! Разумеется, все
оставшиеся ошибки и недочеты при этом я целиком и полностью беру на себя.

Надеюсь, что всем, кто эту книгу прочтет и проработает, математика уже никогда не
будет казаться чем-то скучным, нудным или отвлеченным! Приятного вам чтения, друзья!

В процессе работы над курсом лекций, легших в основу этой книги, автор
получал поддержку от Русского фонда содействия образованию и~науке,
а также от Министерства образования и науки Российской Федерации в
рамках гранта Правительства РФ \No~14.U04.31.002 от 26~июня 2013~года
<<Изучение разнообразия и социальных взаимодействий с фокусом на
экономике и обществе России>> (администрируемого Российской
Экономической Школой).



}
