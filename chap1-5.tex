%1.5.
\section{Лекция 5}
\label{1.5}

\textbf{А.С.:} Сначала обсудим задачи, заданные на~дом в~конце лекции~4. Они не~являются обязательными и~ничего
не~проверяют (в~отличие от~ЕГЭ). Просто они показывают особенности математического подхода
в~различных жизненных ситуациях. Всего задач было семь: 1)~Обзор окрестностей с~вершины горы;
2)~Наблюдение московских звезд во~время затмения; 3)~Скорость наступления темноты в~различные дни
года; 4)~Странное поведение детских качелей; 5)~Полет девочки Алисы к~антиподам; 6)~Математика
футбольного чемпионата; 7)~А~сколько у~человека прапрабабушек?

\textbf{Все разобрались, сколько у~<<среднего человека>> пра\-пра\-ба\-бу\-шек?} (Несредним
человеком будем называть такого, у~которого две прапрабабушки (или два прадедушки) совпадают в~одном лице. Всякое бывает.
 Например, изучение росписи русских дворянских родов приводит к~выводу, что
у~А.\,С.~Пушкина\linebreak и~Н.\,А.~Романова (то бишь Николая~I) были общие предки.)

%Рис. 1
\xPICi{d1}{<<Дерево>> бабушек и~прабабушек.}

С~каждым коленом удваивается количество как дедушек, так и~бабушек (см. рис.~\ref{f:d1}). Папа один, мама одна, дедушек
два, бабушек две, прадедушек уже четыре, и~прабабушек тоже четыре, ну, а~прапрабабушек,
соответственно, получается восемь.

\textbf{Теперь разберем задачу про футбол} (рис.~\ref{f:d2}).

\pagebreak

C каким минимальным количеством баллов можно выйти из группы в~следующий раунд? Ниже приведен
пример, когда это количество равно четырем (рис.~\ref{f:d2}, справа).

%Рис. 2
\xPICi{d2}{Слева: Матчи команд самих с~собой не~имеют смысла.
Справа: финальные счета в матчах, где A всех победил, в то время как прочие игры сыграны вничью.}

Итак. <<A>> выигрывает все игры. Все остальные матчи сыграны вничью. (Напомним, что за победу дается 3 очка, за ничью~--- 1 очко, за поражение~--- 0.)

Получается, что выходит из~группы в~следующий раунд лучшая команда <<А>> и~еще одна из~команд
с~4~баллами. Я~не~знаю, было ли такое в~лиге чемпионов, но с~6~баллами точно выходили.

Встает другой вопрос. Как доказать, что с~3~баллами выйти \textit{нельзя}? Конечно, можно было бы
перебрать все $3^{12}$~вариантов розыгрышей матчей (ибо имеется 12~незаполненных мест, и~на~каждом месте
может быть 3~различных исхода). Но~хотелось бы этого избежать. Давайте попробуем формально
рассуждать. Как мы будем это делать? От~противного. Предположим, что какая-то команда вышла с~3~баллами.
Какие в~этом случае количества баллов могут быть у~всех команд? Вышеуказанный вариант
для~команд A, B, C, D был 18, 4, 4, 4. Если какая-то команда вышла в~следующий раунд с~3~баллами,
значит, как минимум у~двух команд должно быть тоже не~больше 3~баллов. Потому что иначе наша
команда не~вышла бы из~группы. Раз она вышла, следовательно, у~двух других команд баллов меньше или
равно~3.

Вопрос: сколько баллов у~команды, которая больше всех на\-брала?

\pagebreak

В~каждом матче разыгрывается суммарно или 2, или 3~балла (ничья дает $1+1$, победа дает $3+0$). Поэтому
за~все матчи все команды могут набрать минимум $2\cdot12=24$~балла, если все сыграли вничью, максимум~---
$3\cdot12=36$, если каждый матч был выигран.

В~нашей ситуации три команды набрали не~более, чем по~3~балла, в~сумме 9, значит у~четвертой
команды не~меньше $24-9\brokenrel{=}15$~баллов.

Значит, она выиграла почти все матчи.

Давайте уточним, как команда могла набрать 3~балла. Она либо три раза сыграла вничью, либо один раз
выиграла. Больше способов нет. Одна победа и~5~проигрышей, либо 3~ничьих и~3~проигрыша. Обозначим это
так:

\textbf{Вариант~1.} $+$ $-$ $-$ $-$ $-$ $-$

либо

\textbf{Вариант~2.} $0$ $0$ $0$ $-$ $-$ $-$

\textbf{Слушатель:} Это значит, что 18~очков в~розыгрыше.

\textbf{А.С.:} Рассмотрим вариант~1. В~6~матчах было разыграно 18~очков, значит, в~оставшихся
6~матчах будет не~менее 12~баллов, так как в~каждом матче разыгрывается не~менее 2~баллов.
 Значит,
в~сумме получается не~меньше 30~баллов. Значит, у~четвертой команды не~менее $30-9=21$, чего быть
не~может, так как максимальный результат любой команды равен 18, то есть все выигранные матчи.

Итак, вариант с~одним выигрышем отпадает. Рассмотрим другой вариант: три ничьи.

Вариант~2. Четвертая команда набрала не~меньше 15~баллов (минимальное количество 24, три команды
набрали в~сумме 9, получаем $24-9=15$).
 Значит, она одержала минимум 5~побед. (Меньше не~может быть, так как
всего 6~матчей. Даже если 4~победы и~2~ничьи, получится $4\cdot3+2<15$).


Получается минимум, который команды могли набрать вместе: не~24, а~29 (ибо пять матчей с победами
принесли 15 очков, а остальные 7 матчей~--- не меньше 14 очков).\vadjust{\pagebreak} Значит у~четвертой команды минимум
$29-9=20$~баллов. $20>18$, где 18~--- максимально возможное количество баллов. Противоречие.

Другой вопрос, так сказать, <<обратный>> к~первому. С~каким максимальным количеством очков можно
\textit{не~выйти} из~группы в~следующий раунд?

\textbf{Слушатель:} 12.

\textbf{А.С.:} Да. И~как должна быть устроена таблица? Одна команда (скажем, команда~D) всем
проиграла~--- 0~очков. Остальные 3~команды выигрывали по~кругу: A у~B, B у~C, C у~A. Тогда у~трех
команд будет по~12~очков. И~одна из~них должна будет покинуть чемпионат.

Почему \textit{нельзя не~выйти} в~следующий этап с~13~очками? Предположим, что вы набрали 13 или больше
очков, почему вы точно знаете, что вы вышли в~следующий этап?

Подсказка. Если бы кто-то с~13~баллами не~вышел, то две команды, которые вышли, имели бы
не~меньше, чем 13.

\textbf{Теперь поговорим о~том, с~какой горы на~сколько километров видно.}

Я~залез на~Хибинские горы, могу ли я~видеть Мурманск, который находится на~расстоянии 100~км
от~гор? Ответ: на самом ~горизонте~--- смогу (если сопки около Мурманска не помешают).

Сейчас мы получим точную формулу для~максимальной видимости.

Уже первые шаги вверх от~земли сразу дают очень большую видимость.

С~поверхности ничего не~видно. Ноль, он и~есть ноль. Горизонт стянулся в~точку. Чуть-чуть выше нуля
поднялись, на~10~сантиметров, и~сразу видно примерно на~километр (это если Земля~--- идеальный
шар). Обозначим высоту горы за~$h$. Расстояние до~центра земли обозначим за $R\approx 6400$~км. Эта числовая
величина нам также пригодится, когда мы будем решать задачу про Алису.



%\newpage
%$\bfrac h2$
%\newpage

%Рис. 3
\xPICi{d3}{Земля в~разрезе. Расстояние от~центра до~вершины
горы $R+h$, расстояние от~центра до~точки касания~$R$.
Через $L$ обозначено расстояние от~вершины до~точки касания.}

Посмотрим на рис.~\ref{f:d3}.
Я~хочу знать, чему равно $L$, то есть на~сколько километров видно? Здесь есть одна тонкость.

\pagebreak

Вдали будет горизонт. А~вот если с~другой стороны (за~горизонтом) имеется такая же гора высоты $h$,
то ее видно вдвое дальше.
 А~если там гора высоты $\bfrac h2$, то всё равно гораздо дальше, чем просто
до~горизонта (рис.~\ref{f:d4}).

%Рис. 4
\xPICi{d4}{<<Взгляд за~горизонт>>.}

Один раз при~мне на~Байкале видный ученый совершил детскую ошибку. Он сказал: <<Мы никак не~можем
видеть горы, которые находятся в~районе Улан-Удэ. Не~можем, потому что\ldots>>~--- и~дальше привел
вычисления по~формуле, которую мы сейчас выведем. Я~говорю: <<Ты не~учитываешь, что мы сами сейчас
не~на~Байкале, а~на~сильном возвышении>>.~--- <<О\ldots>>,~--- говорит: <<Конечно, это всё удваивает>>.
На~Байкале очень здорово наблюдать, что Земля круглая. Племена, которые жили на~Байкале, наверняка
издавна знали, что Земля круглая.

Помните теорему Пифагора? У~нас образуется прямоугольный треугольник с~гипотенузой $R+h$ и~катетами $R$ и~$L$. Значит,
$$
(R+h)^{2}=R^{2}+L^{2}.
$$
Теперь раскроем скобки:
$$
R^{2}+2Rh+h^{2}=R^{2}+L^{2}.
$$
Страшная величина (квадрат радиуса Земли) сокращается, и~оста\-ет\-ся:
$$
L^{2}=h^{2}+2Rh.
$$
Здесь нужно быть не~только математиком, но~и~физиком~--- для~того, чтобы сказать, что $h^{2}$,
в~общем-то, равно нулю.

Потому что по~сравнению с~двойным радиусом земли $h^{2}$~--- очень маленькое число. Это~--- вещи
несопоставимые, в~том смысле, что первая подавляюще больше, чем вторая. Поэтому, чтобы без~лишних
усилий \textit{оценить}, на~сколько километров видно, достаточно положить $h^{2}=0$ и~написать:
$$
L=\sqrt{2Rh}.
$$
Что здесь важно понимать? Что $2R$~--- величина постоянная, корень из~нее равен~113. А~если совсем
грубо~--- то просто 100.


Есть такая оценочная формула:
$$
L=100\sqrt{h},
$$
$\sqrt{h}$~--- эта функция, которая сначала очень быстро возрастает, а~после отхода от~нуля
увеличивается медленно (см. рис.~\ref{f:d5}).

%Рис. 5
\xPICi{d5}{<<О! Уже хорошо видно! А~наших лыж что-то не~видно\ldots>>}

Чуть-чуть $h$ отлично от~нуля, прямо самую малость, а~корень уже очень большой. Поэтому и~получается,
что вы чуть-чуть подняли голову от~Земли и: <<О! Уже хорошо видно!>> Давайте немного покрутим эту
формулу.

Вот при~$h=1$~км с~Хибин видно на~100~км, а~более точный результат~--- действительно~--- на~113~километров.
113~километров вполне достаточно, чтобы увидеть Мурманск с~Хибинских гор.

\textbf{Слушатель:} То есть на~один градус, а~точнее, на~1,0128~гра\-дуса.

\textbf{А.С.:} Да, на~один градус. С~километра видно на~1~градус.

\textbf{Другой слушатель:} Вы схалтурили.

\textbf{А.С.:} Где?

\textbf{Слушатель:} Расстояние между Хибинами и~Мурманском. Вы считаете по~прямой. А~на~самом деле надо
считать длину дуги.

\textbf{А.С.:} Да. Разница будет примерно одна сотая процента. Она почти равна нулю.

\textbf{Слушатель:} А~при~Джомолунгме?

\textbf{А.С.:} Будет примерно 10 или 15~метров\footnote{Возможно, уже в районе ста метров, я не прикидывал. Прикиньте сами в качестве упражнения!}.
Давайте еще что-нибудь пощупаем. Джомолунгма, какая высота?

\textbf{Слушатель:} 8848~метров.

\textbf{А.С.:} Да. 8848. Округлим до~9~километров.
$$
h=9,\quad
L=300.
$$
Заметьте, что вершина Джомолунгмы~--- это то, на~какой высоте летит самолет, поэтому с~самолета вы
видите примерно на~300--350~километров. Если лететь из~Петербурга в~Москву, то, чисто теоретически,
пролетая Бологое, можно увидеть и~Москву, и~Санкт-Петербург, правда, для~этого самолет должен
развернуться. Но~если кто-то говорит, что он видел Ташкент, летя из~Москвы в~Питер, он вас
обманывает.

\pagebreak

Когда вы летите на~<<гагаринской высоте>>, на~высоте 100~км, соответственно, вы видите на~1000~километров.
С~40-метровой вышки видно на~20~километров. Что, конечно, довольно много.

А~вышка в~10~метров даст обзор на~10~километров во~все стороны. Вышка уменьшилась в~4~раза, а~обзор
в~2. Если увеличить вышку в~9~раз, обзор увеличится в~3~раза. Обратите внимание, не~линейная
зависимость, а~корневая.

\textbf{Переходим к~Алисе.}

Это~--- пример того, как не~работает наша интуиция. Вот она иногда работает, а~иногда\ldots\
Сейчас я~вам расскажу чисто физическую вещь. Итак, давайте сделаем следующее.

%Рис. 6
\xPICi{d6}{Алиса начинает падать в~шахту. Но~вы не~волнуйтесь, она не~погибнет.
Шахта ровная, проходит всю Землю насквозь, Земля не~раскаленная внутри,
а~саму Алису можно считать материальной точкой массы 40~кг.
Ее уже ждут на~той стороне американские ребята! А~от~ее массы в~этой задаче вообще ничего не~зависит.}

Что происходит? Алиса падает (рис.~\ref{f:d6}). У~нее меняется скорость. С~какой скоростью меняется ее
скорость? То есть какое у~Алисы \textit{ускорение}? Оно в~первые моменты равно ускорению свободного падения
$g$; оно, грубо говоря, такое: $g\approx 10~\text{м/с}^{2}$ (точнее, 9,81).

То есть за~одну секунду полета падающий человек увеличивает скорость на~10~м/с каждую секунду.

\textbf{Слушатель:} А~у~нее юбки парусят, это никак не~влияет?

\textbf{А.С.:} Влияет. Поэтому пусть не~парусят.

\textbf{Слушатель:} В~вакууме\ldots

\textbf{А.С.:} Да, это вакуумная Алиса. Потому что если есть сопротивление воздуха, то
на~некоторой скорости прекращается всякий рост скорости (например, для~парашютиста, падающего
без~парашюта, скорость быстро стабилизируется на~значении примерно 50~м/с). Вот. Поэтому пусть
у~нашей Алисы не~парусят. И~вообще\ldots\ она ведь бросилась в~шахту вниз головой, чтобы удобнее
было вылетать из~нее на~другой стороне Земли.

%Рис. 7
\xPICi{d7}{А~если отважная Алиса летит не~сквозь Землю, а~сквозь Юпитер? У~него радиус в~11,2~раза
больше земного. И~она уже пролетела до~глубины в~2,2~земных радиуса.
 Тогда всю массу Юпитера можно
поделить на~две части: внутренний шар радиуса~9 (земных радиусов) и~наружный шаровой слой толщины
в~2,2 (земных радиуса). Точный математический подсчет показывает, что суммарная сила, с~которой
шаровой слой притягивает к~себе летящую Алису (да вы мне всё равно ведь не~поверите!..) РАВНА НУЛЮ.
А~внутренний шар, через который Алисе осталось пролететь, притягивает ее к~себе так же, как
притягивала бы её к~себе точка, равная массе этого шара и~находящаяся в~центре шара.}

Но~есть одна важная физическая тонкость. Проблема в~том, что когда вы начинаете падать в~такого
рода <<колодец>>, у~вас $g$ начинает меняться. Ускорение свободного падения тем меньше, чем ближе вы
к~центру. В~центре ваша скорость не~меняется вообще, ибо там силы тяготения в~сумме дают нулевой
вектор. Есть такой физический закон, который утверждает, что ускорение свободного падения
пропорционально расстоянию до~центра планеты, если плотность планеты постоянна. (То есть если
осталось пролететь долю <<$x$>> от~поверхности до~центра, то ускорение свободного падения будет равно
$gx$. При~$x=1$ будет просто $g$, при~$x=0{,}1$ будет $g/10$.)

\textbf{Пояснение и~лирическое отступление <<о футболе на~Луне>>.}
Например, радиус Юпитера в 11,2~раза больше
радиуса Земли (рис.~\ref{f:d7}). Ускорение свободного падения на~Юпитере в~2,52~раза больше, чем на~Земле
(оно, кроме радиуса, зависит и~от~массы планеты). Радиус же Луны меньше радиуса Земли в~3,67~раза,
и~на~ней в~6,05~раза меньше ускорение свободного падения. Я~представил себе лунный футбол, это,
должно быть, совершенно замечательно. Огромные ворота высотой 12~метров и~шириной 40~метров,
и~в~них медленно летающий вратарь. Но~бьют-то по~мячу с~той же силой. Должно быть, очень занятно.
Я~не~знаю, может быть, в~будущем когда-нибудь будет лунный футбол.

У~нас ускорение получается такое:
$a=-gx$, и~надо пояснить, что означает знак <<минус>>.

%Ускорение\quad
%$g$\quad
%$x\cdot g$\quad

Сначала попытаемся понять, какое движение совершала бы Алиса в~шахте, просверленной по~диаметру
земного шара, если бы в~центре Земли никакой гномик не~толкал бы ее в~спину для~увеличения
скорости. Она полетела бы <<вниз>> (ощущение у~нее было бы такое же, как у~человека, упавшего
в~колодец). До~момента достижения центра Земли скорость всё время нарастала бы (в~этой задаче
считается, что сопротивление воздуха отсутствует); максимальная скорость будет при~пролете через
центр Земли (в~этой точке Алиса будет лететь по~инерции, так как сила тяготения обратится в~нуль).
Затем начнет проявлять себя сила тяжести, направленная против движения. Она будет постепенно
нарастать, всё сильнее уменьшая скорость полета Алисы. Ее скорость станет нулевой как раз в~тот
момент, когда Алиса пролетит всю Землю насквозь. Теперь сила тяжести направлена в~противоположную
сторону. И~затем всё будет повторяться. Такое движение называется <<колебательным>>. Чтобы оно
могло возникнуть, тело должно испытывать действие так называемой \textit{возвращающей силы}. Эта сила всегда
направлена в~сторону \textit{положения равновесия} (в~этой задаче точкой равновесия является центр Земли).
При~расчетах именно центр Земли удобно выбрать за~начало координат, а~ось иксов направить вдоль
шахты от~начала шахты (место вылета) к~ее концу на~другом краю Земли. Именно при~таких условиях
и~была написана формула $a=-gx$.

В~этой формуле $0<x<1$ (доля пути, оставшаяся до~положения равновесия). Но~обычно в~теории
колебаний этой буквой обозначают \textit{отклонение} от~положения равновесия (то есть от~нуля). В~этом
случае появление знака <<минус>> становится понятным: возвращающая сила противоположна направлению
отклонения, и~она тем больше, чем больше отклонилась точка от~центра. Она похожа по~своему действию
на~заботливого пастуха: чем больше овца отклонилась от~лужайки с~травой, тем сильнее он гонит ее
обратно. Скорость изменения какой-нибудь физической величины <<$x$>> обозначается в~учебниках физики
точкой вверху $\dot x$; а~ускорение (то есть <<скорость изменения скорости>>)~--- двумя точками $\ddot x$.
При~расчете любого движения точки вдоль прямой (в~том числе и~колебательного) математики заимствуют
из~физики \textit{второй закон Ньютона}: <<сила равна массе, умноженной на~ускорение>>. Если <<$x$>> означает
(как в~нашей задаче про Алису) отклонение от~начала координат, то уравнение движения материальной
точки имеет вид $$m\ddot x=f(x),$$ где $m$~--- масса точки, $f(x)$~--- закон изменения силы, управляющей
движением точки, при~изменении ее положения~<<$x$>>. Простейшее колебательное движение
(<<гармоническое колебание>>) получается при $$f(x)=-kx$$ (линейная возвращающая сила). В~этом
случае закон движения $x(t)$ (где $t$~--- время, прошедшее с~момента начала движения) выражается суммой
синуса и~косинуса с~некоторыми коэффициентами (отражающими информацию о~начальном отклонении точки
от~центра и~о~начальной скорости движения точки). То, что физики называют \textit{скоростью}, математики
называют \textit{первой производной}. А~то, что физики называют \textit{ускорением}, математики называют \textit{второй производной}.
 Математики имеют в~своем <<арсенале>> большой запас математических методов
для~решения различных уравнений движения. В~частности, самое простое колебание описывается
с~помощью изменения значений косинуса (или синуса).

Если ускорение точки, движущейся вправо, отрицательное, значит, она тормозит, уменьшая свою
скорость. Может быть такое, что при~неизменном ускорении точка достигнет нулевой скорости и~затем,
остановившись на мгновение, будет двигаться в~отрицательном направлении.
 Аналогичная ситуация может быть
при~движении точки влево и~воздействии на~нее \textit{положительного} ускорения.

\textbf{Слушатель:} А~<<$x$>> в~каких единицах измеряется?

\textbf{А.С.:} Вопрос о~единицах очень важный и~правомерный. Можно измерять $x$ в~метрах, можно
в~километрах. В нашем случае $x$~--- доля радиуса~--- безразмерная величина.
 Решением уравнения, описывающего полет Алисы от~падения в~шахту и~до~достижения
центра Земли, на~самом деле служит обычный косинус.

Однако следует иметь в~виду, что бывают задачи и~с~другими единицами измерения переменной~<<$x$>>.

Например, если <<$x$>> означает запас бензина в~баке автомобиля (он изменяется с~течением времени), то
единицей измерения будет литр,
 а~скорость расхода бензина будет тогда измеряться в~литр/час.
Но~математики всё равно называли бы скорость расхода бензина <<первой производной>>.

%Рис. 8
\xPICi{d8}{Вот так Алиса миновала центр Земли и~долетела до~Аме\-рики.}

Вначале координата будет меняться медленно, потом Алиса будет набирать всё большую и~большую
скорость, ускорение же будет уменьшаться. Алиса сначала долетит до~центра, а~потом и~до~поверхности
Земли с~другой стороны (рис.~\ref{f:d8}), преодолевая нарастающую силу тяжести (<<возвращающую силу>>), потом
опять полетит обратно, и~так до~бесконечности.

Так вот. Спрашивается, какая скорость в~центре Земли? Скорость в~центре Земли очень серьезная. Там
много, много метров в~секунду. (Вычисления с~помощью косинуса показывают, что эта скорость
(независимо от~массы тела Алисы) составляет 7910~м/с.) Что же происходит с~энергией в~полете то
туда, то обратно? Есть такой \textbf{закон сохранения энергии}. Потенциальная энергия переходит
в~кинетическую, и~наоборот. На~поверхности Земли есть только потенциальная, в~центре~--- только
кинетическая. В~полете к~центру Земли потенциальная энергия постепенно уменьшается,
а~кинетическая~--- на~такую же величину увеличивается.

Пока что мы обсуждаем особенности полета Алисы, не~усложняя его появлением гномика, который толкнул
Алису вперед в~центре Земли.

Кинетическая энергия имеет формулу: $E=\bfrac{mv^{2}}{2}$, где $v$~--- скорость в~данный момент.

Гном добавляет 1~м/с, то есть кинетическая энергия Алисы становится: $E=\bfrac{m(v+1)^{2}}{2}$.

\pagebreak

Это должно компенсироваться тем, что Алиса вылетит вверх на~некоторую высоту, чтобы прибавка
потенциальной энергии компенсировала убавку дополнительной кинетической энергии.

Потенциальная энергия записывается формулой $E=mgh$.

При~удалении от~поверхности Земли $g$ будет немного меняться по~сравнению со~значением 9,81,
но~не~очень сильно.

Давайте посмотрим на~разницу этих двух кинетических энергий.
$E=\bfrac{mv^{2}}{2}$~--- до~прибавки в~скорости,
$\bfrac{m(v^{2}+2v+1)}{2}-\bfrac{mv^{2}}{2}$~--- увеличение энергии после прибавки.

Единицу мы, как водится, игнорируем, она очень маленькая, $2v$ остается. В~итоге разница получается
равной $mv$. Этой разнице и~должна быть равна потенциальная энергия в~точке максимального подъема
Алисы на~другой стороне Земли:
$$
mv=mgh.
$$

Масса сокращается, то есть масса Алисы здесь никакого отношения к~делу не~имеет. Совсем, казалось
бы, не~важно, кого пнул гном. Однако следует отметить, что худенькой Алисе гному будет легче
неожиданно придать лишний 1~м/с, чем увесистой Алисе. Впрочем, это проблемы гнома, а~не~наши с Алисой.

Остается формула: $v=gh$.
То есть $h$ равно $\bfrac vg$ (если посчитать, получается 805~метров).
Добавление 1~м/с в~центре земли дает вылет вверх из~дыры примерно на~800~метров.

А~что говорит наша интуиция? Не~нарушил ли наш гном закон сохранения энергии? Нет, не~нарушил.
Просто гному будет не~так-то легко увеличить скорость Алисы на~1~м/с. Ему придется со~страшной
силой размахнуться кулаком, чтобы кулак приобрел скорость Алисы (7910~м/с) плюс еще 1~м/с. Мы бы
с~вами этого не~смогли сделать. А~гном, как существо волшебное, может!

%\textbf{Проверим задачу про звезды.}
%
%\textit{Первое, что надо осознать}: те звезды, которые вы не~видите днем из-за солнца, можно будет увидеть
%через полгода (ночью). См. схему рис.~\ref{f:d9}.
\textbf{Проверим задачу про звезды.}

\textit{Первое, что надо осознать}: те звезды, которые вы не~видите летним днем из-за солнца, можно
будет увидеть через полгода (зимней ночью). См. схему на рис.~\ref{f:d9}.


%Рис. 9
\xPICi{d9}{Земля, не~меняя наклона своей оси к~плоскости эклиптики, вращается вокруг Солнца
(справа в северном полушарии Земли~--- лето, и~Солнце не позволяет
людям разглядеть
звезды, расположенные с той же стороны; слева в северном
полушарии~--- зима, и те же звезды оказываются <<по другую сторону от Солнца>>,
поэтому прекрасно видны по ночам). Важно, что расстояние между указанными двумя
положениями Земли ничтожно мало по~сравнению с~расстоянием до~ближайших (не~считая Солнца) видимых
на~небе звезд.}

Поэтому то, что вы видите во~время затмения, вы уже видели полгода назад и~увидите опять через
полгода вперед (если этому не~помешают тучи и~т.\,п.).

\medskip

\hrulefill

\smallskip

\textbf{Врезка 8. Кое-что о~небесной сфере}

Космос велик и~необъятен. Ему, в~общем-то, нет дела до~ничтожной пылинки, называемой <<планета
Земля>>. Но~на~ней проживает многочисленное племя людей, которые в~свободное от~работы время
не~прочь поразмыслить над вопросами <<что такое Космос>> и~<<влияет ли Космос на~жизнь человека>>.
По~поводу первого вопроса до~сих пор спорят ученые~--- даже не~смогли до~сих пор понять, ограничен
ли Космос по~размерам, или не~ограничен. А~простые люди думают так: если и~ограничен, то всё же он
так велик, что его всё равно облететь нельзя. Так какое нам до~этого дело? На~второй вопрос многие
люди склонны ответить так: наверное, Космос как-то влияет на~жизнь людей~--- но~гораздо меньше, чем
\textit{курс доллара}! Космос, конечно, может преподнести нам подарок в~виде огромного метеорита, который
сотрет жизнь на~Земле~--- да ведь это когда еще будет! Может, в~это время и~долларов уже не~будет.

Тем не~менее было решено, что на~всякий случай за~Космосом надо следить с~помощью специального
воображаемого <<телевизора>> со~сферическим экраном. Это и~есть НЕБЕСНАЯ СФЕРА. Где же находится
центр этой воображаемой сферы и~чему равен ее радиус? Для~ответа на~этот вопрос надо понять, что же
является самым важным для~жизни человеческой цивилизации. Конечно же, <<земная сфера>>, то есть
поверхность Земли. Поэтому и~центр небесной сферы был выбран в~центре Земли, с~учетом <<земного
эгоизма>> людей. (Наверное, если бы земляне и~марсиане жили бы в~виде единой цивилизации, центр
небесной сферы был бы выбран не~в~центре Земли, а~в~центре Солнца.) Что же касается радиуса
небесной сферы, то его надо выбрать побольше (чтобы эта сфера находилась далеко от~Земли),
но~не~слишком большим (чтобы внутри этой сферы не~оказались ближайшие (не~считая Солнца) звезды).
Конкретное же значение радиуса никого особенно не~интересует~--- лишь бы мы на~этой сфере сумели
разглядеть все подробности из~жизни Космоса. Итак, земная и~небесная сфера являются
\textit{концентрическими}. На~земной~--- имеются две важные для~землян точки: северный и~южный полюс. Через
них проходит важная для~землян прямая~--- земная ось. Земля вращается вокруг этой оси с~периодом
24~часа\footnote{Дьявол кроется в деталях. Строго говоря, это утверждение {\em неверное} (попробуйте понять, почему!).}.
А~вот и~каверзный вопрос: почему 24? Ответ (неубедительный): потому, что так решили древние
астрономы. Но~ведь это было пару тысяч лет назад. За~это время старушка-Земля могла притормозить
свое вращение. А~ну как вдруг период ее вращения теперь равен 24,37~часа? Ответ (нелогичный):
часов-то по-прежнему 24, но~сам час стал немного длиннее. Нелогичность его в~том, что нам всё равно
надо знать, происходит ли замедление (или, скажем, ускорение)~--- неважно, как мы это назовем~---
удлинение периода или удлинение часа. К~счастью, сейчас физики могут определять длительность
промежутка времени независимо от~вращения Земли, причем с~высокой точностью. И~никаких признаков
изменения периода вращения земного шара не~обнаружено. Пока не~обнаружено. А~завтра прилетит
какой-нибудь укрупненный метеорит и~врежется в~Землю\ldots

Следующие два важных термина~--- \textit{зенит} и~\textit{надир}. Зенит~--- это точка, лежащая на~небесной сфере
прямо над головой наблюдателя, надир~--- точка, лежащая на~противоположной стороне небесной сферы
(то есть под ногами наблюдателя, так что он и~наблюдать-то ее не~сможет). Вы чувствуете, какой
подвох есть в~этом определении? Земля у~нас одна, а~наблюдателей на~ней может быть очень много~---
и~на~суше, и~на~море. Значит, и~точек зенита будет очень много. И~даже какие-то два наблюдателя
могут сильно поспорить по~поводу одной и~той же точки на~небесной сфере: один скажет, что это
<<зенит>>, другой~--- что это <<надир>>. Надо как-то ограничить <<персональный эгоизм>> наблюдателя.
Поэтому было объявлено, что все эти наблюдатели <<воспомогательные>>, кроме двух <<основных>>. Один
из~двух наблюдает на~северном полюсе, другой~--- на~южном. Кстати, а~какой из~полюсов назвать
северным, а~какой~--- южным? Ведь и~здесь, и~там очень холодно\ldots\ Этот вопрос не~очень важен,
но~всё же решено было, что над северным полюсом находится зенит, а~над южным~--- надир. И~остался
только один <<основной>> наблюдатель~--- тот, у~которого над головой зенит. Через северный и~южный
полюс провели прямую и~продолжили ее до~пересечения с~небесной сферой. И~далее стали именовать эту
прямую не~<<земная ось>>, а~<<ось мира>>. Вот тебе и~раз! Да какое же право имеет маленькая, совсем
незаметная в~масштабах Космоса планета Земля указывать, как должна быть направлена ОСЬ МИРА?
Это~--- типичнейший пример <<земного эгоизма>>. Погодите, дальше еще и~не~такое будет!

Итак, стоит на~Северном полюсе наблюдатель, смотрит в~небо, и~совершает (вместе со~всем земным
шаром) один оборот за~24~часа. А~ему кажется, что и~он, и~Земля стоят на~месте, а~весь огромный
Космос, со~всеми его звездами и~кометами (да и~с~нашим Солнцем тоже), медленно вращается в~другую
сторону. И~чтобы убедиться в~этом, достаточно поглядеть довольно долго на~экран того <<телевизора>>,
через который земляне наблюдают Космос (то есть на~небесную сферу). Осталось совсем немного, чтобы
силами землян достроить для~всего Космоса космическую систему координат (в~которой, смеха ради,
считается, что Земля абсолютно неподвижна, а~всё остальное (в~том числе и~Солнце) вращается вокруг
нее). Рассмотрим плоскость, проведенную через земной экватор. Продолжим ее до~пересечения
с~небесной сферой. Получится <<мировая экваториальная плоскость>> для~всего космического
пространства. Вы, наверное, думаете, что именно в~ней находится Солнце и~вращаются все другие
планеты? Ничего подобного! Солнце и~планеты находятся в~другой плоскости (она называется
<<эклиптикой>>). Обе эти плоскости пересекаются в~центре Земли (так что эту точку называют <<начало
координат мира>>). Конечно, у~этих двух плоскостей есть и~другие точки пересечения (они пересекаются
по~прямой). Плоскость эклиптики пересекает экваториальную плоскость под углом 23,5~градуса. Земная
ось направлена в~зенит (zenit), поэтому ее и~назовем <<ось $Z$>>. Осталось указать в~экваториальной
плоскости, как провести через центр Земли ось~$X$ и~ось~$Y$. Главное~--- это задать направление оси
иксов. Для~этого надо найти на~экваторе (естественно, на~земном, а~не~на~небесном) \textit{нулевой
меридиан}. На~этот счет имеется как минимум два мнения. Англия считает, что надо таковым считать
Гринвичский меридиан, а~Россия~--- что Пулковский меридиан. (А~какая-нибудь цивилизация
из~созвездия Тау Кита, считает, что центр мира вообще не~должен находиться в~центре Земли.) В~целях
унификации общеземной системы космических координат можно провести ось иксов в~направлении,
например, Гринвича\footnote{Так и быть, сделаем этот щедрый подарок англичанам!}.
 Теперь уже можно определить для~каждой точки на~поверхности Земли (а~также
и~для~любой точки Космоса) две координаты: \textit{долготу и~широту}. Нужна еще третья координата~---
расстояние от~центра Земли до~интересующей нас точки. Для~точки на~поверхности Земли (считаемой
<<идеальным шаром>>) эта координата равна усредненному радиусу Земли $R$ (примерно 6371~км). Для~звезды
в~Космосе (как бы далеко она ни~находилась от~небесной сферы)
 в~качестве третьей координаты надо
брать радиус небесной сферы, потому что все эти звезды надо спроектировать из~бездны Космоса
на~экран <<телевизора>> для~разглядывания Космоса, то есть на~небесную сферу. Так как радиус этой
сферы не~уточняется, то в~Космосе используются только две (угловые) координаты: долгота и~широта
луча, идущего из~центра Земли в~данную звезду (или комету, или метеорит\ldots)

Имея систему координат на~небесной сфере, можно уже составлять карту всех созвездий. В~этой системе
Солнце описывает по~небесной сфере замкнутый путь, причем оно при~этом отнюдь не~видно в~виде точки
(подумайте, почему?). Поэтому необходимо говорить не~про <<путь Солнца>>, а~про путь центральной
точки солнечного диска на~небесной сфере. На~этот путь у~солнечного диска уходит ровно один год (то
есть примерно 365,25~суток). Несмотря на~неудобство такой системы координат по~сравнению с~системой
Коперника, в~ней успешно рассчитали (не~пользуясь даже компьютерами!) в~каждой точке Земли восходы
и~заходы Солнца и~их длительность. (См. далее основной текст.)

\smallskip

\hrulefill

\medskip

Приведенная выше врезка для~чтения необязательна, хотя она дает первоначальный обзор трудностей,
связанных с~выбором общекосмической системы координат. Те, кто хотят глубже понять то, что сказано
во~врезке, могут попробовать ответить на~вопрос: верно ли, что наблюдатель на~Северном полюсе,
глядящий вертикально вверх, увидит, что в~точке <<зенита>> находится Полярная звезда? ВАРИАНТЫ
ОТВЕТОВ: 1)~Он увидит там центр созвездия <<Южный крест>>; 2)~В~течение ночи он увидит в~этой точке
разные звезды; 3)~Полярная звезда находится близко к~зениту, но~не~совпадает с~ним; 4)~Так как
радиус небесной сферы не~определен, а~расстояние до~Полярной звезды (в~принципе) определено, то
этот вопрос бессмысленный.

\pagebreak

{\baselineskip=13pt

\looseness=-1
На~самом деле, для~решения этой задачи правильной моделью является одинокая (то есть без~Солнца
и~планет) Земля в~черном страшном Космосе, которая вертится вокруг своей оси. Людям, глядящим ночью
на~небо, кажется, что Земля стоит на~месте, а~всё звездное (видимое им) небо медленно вращается
в~противоположную сторону вокруг Полярной звезды. (Пока, в~<<наше>> время. Через тысячу лет
определять центр вращения будет \textit{другая} звезда.) Вопрос: что и~откуда видно этим наблюдающим людям?
Ответ: Петербург отличается от~Москвы северной координатой широты. Остальные координаты (долгота)
в~данном случае нам не~важны. Чтобы понять, кто видит на~небе больше звезд (москвичи или
петербуржцы), рассмотрим сначала наблюдение из~двух особенных точек: с~экваториальной точки (любой
из~этих точек) и~с~Северного полюса. Что видно с~Северного полюса? Краткий ответ: только то, что
<<сверху>>. Но~где же в~Космосе верх, и~где низ? Сразу для~всего Космоса вряд ли можно разумно
ответить на~этот вопрос. Но~любой конкретный наблюдатель Земли прекрасно ответит на~него: <<То, что
у~меня под ногами~--- это ``низ''. А~остальное~--- это ``верх''. Я~сейчас стою на~плоскости,
отделяющей верх от~низа, и~потому я~вижу звезды только половины небесной сферы>>. Человек, имеющий
чисто математическое образование, не~умеряемое здравым смыслом физика (или астронома), сразу же
запальчиво возразит этому \textit{неучу} (по~его мнению): <<Никакая это не~плоскость, а~сферическая
поверхность, и~на~ней только мысленно можно пририсовать касательную плоскость, именно в~той точке,
где стоит \textit{этот неуч}>>.~--- <<А~вот и~нет!>>~--- очень разумно ответит <<неуч>>: <<Я~вижу только кусок этой
поверхности от~моих ног и~до~горизонта. А~горизонт примерно в~4~километрах от~меня, поэтому видимый
мною кусок сильно похож на~плоскость~--- ведь 4~км очень мало по~сравнению с~6400~км (то есть
с~радиусом Земли). И~эта плоскость сильно мешает мне увидеть звезды на~второй половине небесной
сферы>>. И~в~этом он будет абсолютно прав. Короче говоря, если мы хотим понять, какую часть небесной
сферы (с~мерцающими на~ней звездами) видит тот или иной земной наблюдатель, надо через подошвы
этого наблюдателя провести плоскость, касательную к~земной поверхности. Она разделит небесную сферу
на~две равные части. Он видит ту, которая у~него над головой. Если бы старушка-Земля была
прозрачной, он бы увидел у~себя под ногами и~вторую часть. Если бы этот наблюдатель внезапно вырос
бы до~размеров\ldots\ ну, скажем, Джомолунгмы, он бы увидел, что под ногами у~него не~плоскость,
а~часть земной сферы, и~она, искривляясь, мешает ему увидеть целиком всю небесную сферу. Однако
при~этом он, конечно, видел бы БОЛЬШЕ ПОЛОВИНЫ поверхности небесной сферы, а~меньшая ее часть была
бы ему не~видима. А~если бы он и~еще <<немножко>> подрос~--- чтобы его рост <<хотя бы>> стал равен
расстоянию до~ближайших звезд,~--- тогда бы он смог наблюдать почти все звезды небесной сферы
(маленькая старушка-Земля у~него под правым каблуком почти не~мешала бы ему изучать звёзды). Но~мы,
земные люди, не~могли бы быть такими гигантами~--- нас бы раздавила наша собственная тяжесть.
Поэтому человеку надо быть очень скромным и~считать себя маленькой незаметной точечкой по~сравнению
с~земным шаром. Однако по~поводу сказанного выше можно было бы задать два коварных вопроса. Я~лучше
их сам сразу сформулирую. 1)~Допустим, на~Земле не~было бы гор и~океанов. Тогда люди жили бы
во~всех местах земной сферы. И~по~ночам наблюдали бы небо. Так что же, в~этом случае всю Землю
пришлось бы покрыть касательными плоскостями? ОТВЕТ: представьте себе, именно так и~делают
настоящие математики. И~то, что получается, у~них даже носит специальное название: \textit{<<касательное
расслоение для~сферы>>}. И~потом успешно изучают полученный объект. 2)~Стоп. Только что было
сказано, что для~каждого наблюдателя видна ровно половина звезд небесной сферы. А~вот и~нет.
Давайте возьмем двух диаметрально противоположных наблюдателей Земли. Для~каждого из~них проведем
касательную плоскость. Ведь эти две плоскости будут параллельны? А~две параллельные плоскости делят
пространство, в~котором мы живем, на~\textbf{три части}. Как же может каждый из~двух этих наблюдателей
видеть половину небесной сферы? Каждый из~них должен видеть меньше половины! ОТВЕТ: <<с точки
зрения звезд>> не~только человек является ничтожной точкой, но~и~даже Земля~--- ничтожная точка.
И~с~их точки зрения зазор между двумя параллельными плоскостями (равный диаметру Земли)
пренебрежимо мал. То есть вместо двух плоскостей они <<видят>> как бы одну слившуюся плоскость.
И~если какая-то звезда окажется на~этой плоскости, она погоды не~делает. Умные математики в~таких
случаях любят говорить что-нибудь успокоительное, типа: \textit{множество этих звезд имеет меру нуль}.
Ситуация тут примерно такая, как при~сравнении между собой двух чисел: 1000000,098 и~999999,978. Ну
да, первое число чуть-чуть больше второго, но~с~точки зрения физика можно (и~нужно) пренебречь этой
разницей и~сказать, что эксперимент с~высокой точностью подтвердил их совпадение. А~не~совпали они
полностью потому, что во~время земных измерений кто-то неожиданно чихнул на~Марсе\ldots

}


Итак, каждый человек в~любой момент времени видит звезды только половины небесной сферы. Другая
половина не~видна, ее загораживает Земля (хотя она вовсе не~занимает половины пространства).
Фактически, модель ситуации (при~наблюдении из~космоса) такая. Вы прижимаете к~Земле плоскость
в~любой точке~--- хотите, в~Москве, хотите~--- в~Питере, и~наблюдаете ее полный оборот вместе
с~Землей. В~этой плоскости отметим прямую, касательную к~меридиану в~выбранной точке. В~процессе
поворота Земли вокруг полярной оси эта прямая опишет поверхность некоторого конуса, которого
в~каждый момент касается плоскость (оставаясь при~этом касательной и~к~поверхности Земли~---
потому что Земля оказывается вписанной в~этот конус).
 Вот теперь мы, наконец-то, добрались до~изучения
двух особых наблюдателей: экваториального и~полярного (см. рис.~\ref{f:d10}). А~там, глядишь, и~с~точкой
<<Москва>>, и~с~точкой <<Петербург>> станет ситуация понятной. Они же ведь точки по~сравнению
с~Землей, правда?

%Рис. 10
\xPICi{d10}{Поверхность Земли с~указанием экватора и~Северного полюса (нижняя часть рисунка).
Вертикальные линии~--- образующие бесконечной в~обе стороны цилиндрической поверхности.
Для~удобства восприятия верхняя часть цилиндрической поверхности срезана горизонтальной плоскостью
(верхняя линия~--- край среза). Отдельная точка~--- это Северный полюс. Через него
проведена горизонтальная плоскость, касающаяся Земли. Эта плоскость схематически изображена в~виде
параллелограмма.}

Рис.~\ref{f:d10} показывает, какие звезды увидят наблюдатели, находящиеся на~широте 0~градусов
(экватор) и~на~широте 90~градусов (полюс), если они будут в~течение 24~часов наблюдать звездное
небо (то есть пока Земля совершит полный оборот). Понятно, что в~дневное время свет Солнца помешает
видеть звезды (а~если случится полное солнечное затмение, то всё же увидят). Но~есть такие участки
звездного неба, которые наблюдатель \textbf{в~принципе} не~сможет увидеть. Например, наблюдатель на~экваторе
не~увидит (ни в~какое время суток) тех звезд на~небесной сфере, которые лежат \textit{внутри цилиндрической
поверхности}\footnote{Вышеописанные соображения позволяют заключить, что таковых <<почти нет>>.}.
 (Половина невидимых звезд лежит <<выше северного полюса>>, другая же половина <<ниже
южного>>.) Наблюдатель же, находящийся на~Северном полюсе, не~увидит (ни в~какое время суток) звезд
<<нижней половины небесной сферы>>. Но~зато звезды <<верхней половины>> он увидит не~постепенно
в~течении 24~часов, а~все СРАЗУ. Дело в~том, что горизонтальная касательная плоскость на~рис.~\ref{f:d10}
по~мере вращения Земли хотя и~будет поворачиваться, но~она всё время будет совпадать сама с~собой.
Если мы рассмотрим только наблюдателей из~северного полушария, то можно сформулировать такое
достаточно простое правило:

\textbf{<<Видно всё, кроме внутренности конуса>>.} Чем ближе к~северу находится наблюдатель, тем более
<<плоский>> получается конус, и~тем меньше звезд видно. Чем ближе к~экватору он находится, тем
более <<острый>> получается конус (а~на~экваторе его вершина оказывается удаленной
до~бесконечности; на~полюсе же его вершина совпадает с~самим полюсом). См. рис.~\ref{f:d11}.

%Рис. 11
\xPICi{d11}{Широта Москвы (56~град.) меньше широты Петербурга (60~град.), поэтому коническая
поверхность, внутри которой располагаются невидимые звезды, для~Москвы имеет более значительную
высоту над полюсом, но~угол расхождения левого и~правого луча из~вершины меньше, чем
для~Петербурга. Поэтому из~Москвы видно \textit{больше} звезд (если наблюдение вести 24~часа). (Рекомендуем
мысленно вращать этот рисунок относительно оси симметрии~--- пунктирная линия образует тогда северное
полушарие Земли, а~сплошные~--- <<конусы невидимости>>.)}

Поэтому из~Санкт-Петербурга видно меньше звезд, чем из~Москвы (рис.~\ref{f:d12}). А~на~экваторе не~видно ни
Полярной звезды, ни Южного Креста (да и~соседних с~ними звезд тоже не~видно). Однако
последнее утверждение является схоластическим.
Чтобы понять это,
представьте себе, что цилиндрическая поверхность, изображенная на~рис.~\ref{f:d10}, продолжена вверх
до~пересечения с~небесной сферой. Тогда на~этой сфере получится линия пересечения в~виде
окружности, радиус которой примерно равен 6400~километров (радиусу Земли).
 А~радиус небесной
сферы, как указано выше, примерно равен расстоянию до~ближайшей звезды (не~считая Солнца). Это
расстояние неизмеримо больше, чем 6400~км. Так что даже с~помощью самого мощного современного
телескопа будет проблематичным понять, какие же звезды попали в~<<область невидимости>>
для~экваториального на\-блю\-да\-теля!

%Рис. 12
\xPICi{d12}{Семейство прямых, касательных к~меридианам.}

\textbf{Задача про 22 мая.}

В~формулировке задачи упоминаются такие известные даты, как <<летнее солнцестояние>> (22~июня)
и~<<зимнее солнцестояние>> (22~декабря). Здесь уже без~Солнца никуда! В~этом случае, казалось бы,
надо использовать гелиоцентрическую модель Коперника, но~вполне работает и~модель Птолемея,
в~которой Солнце вращается вокруг Земли. В~этой модели, как объяснено во~врезке, Земля преспокойно
стоит на~месте, несмотря на~возмущение других миров, населенных мыслящими существами (согласно
пословице <<Вся рота шагает не~в~ногу, один поручик~--- в~ногу>>).


%%Рис. 13
%\xPICi{d13}{Круг~--- это наша Земля, мешающая нам увидеть, как сзади крадётся маленькое Солнце,
%чтобы неожиданно выскочить из-за горизонта и~затем снова юркнуть за~край земного шара.
%Я~не~оговорился, сказав <<маленькое Солнце>>~--- на~небесной сфере солнечный диск действительно
%занимает маленькую часть этой сферы. Подумайте, как это совмещается с~тем известным фактом, что
%на~диаметре Солнца можно уложить 109~диаметров Земли?}

Что же происходит в~нашей земной системе координат? В~зависимости от~времени года солнце всходит
и~заходит в~разное время.
%(рис.~\ref{f:d13}).
Но~восходит/заходит оно под разным углом, поэтому меняется
скорость восхода и~заката. Когда наступает темнота? Когда Солнце достаточно низко <<залегает>> под
уровнем горизонта. \textit{(ПОЯСНЕНИЕ. В~этом месте в~астрономию вмешиваются\ldots\  государственные
законы. Надо дать юридическое определение, что такое <<сумерки>>. Ведь с~наступлением сумерек
государство должно потратиться на~освещение объектов цивилизации. Вот и~появились на~свете два вида
сумерек: <<гражданские сумерки>>, начинающиеся с~глубины залегания Солнца в~6~градусов,
и~<<астрономические сумерки>>, когда на~небе можно различить практически все видимые звезды,
начинающиеся с~глубины в~18~градусов.)} В~дни равноденствия Солнце уходит под крутым углом к~линии
горизонта, поэтому темнеет очень быстро. А~в~дни солнцестояния~ угол, под которым заходит
солнце, оказывается гораздо более пологим, поэтому темнеет и~светает медленно.


В~высоких широтах северного полушария (равно как и~в~низких широтах южного) заметен эффект, который
можно назвать <<фальшивыми сумерками>>~--- или, более поэтично, <<белыми ночами>>.

Солнце начинает нисходить за~горизонт под малым углом, но очень быстро выходит на~угол еще меньший,
чтобы <<успеть быстро выскочить обратно>>, ибо широта этого места такова, что еще немного,
и~начнется сплошной полярный день, без~всяких сумерек. В~итоге еще не~кончились вечерние сумерки,
как пора уже готовиться к~рассвету.
%Как это было у Пушкина: <<Заря одна другую спешит сменить\ldots>>
Как это было у Пушкина: <<Одна заря сменить другую // Спешит, дав ночи полчаса\ldots>>
 И~что же получается? Не~только не~достигнута <<глубина
залегания Солнца>> в~18~градусов (астрономические сумерки), но~и~даже не~дотянули до~6~градусов
(гражданские сумерки). Значит, градоначальник не~включит освещение, и~придется нам уподобиться тому же ~Пушкину
(\textit{<<когда я~в~комнате моей пишу, читаю без~лампады\ldots>>}).
 Вот в~Питере уже заметен этот
эффект~--- там просто никогда не~темнеет до~конца. Даже вот это расстояние до~Полярного круга (это
расстояние примерно 7~градусов в~Питере)~--- оно такое, что не~темнеет просто, да и~всё тут! 7~градусов
под горизонтом позволяет получать от~спрятавшегося Солнца вполне достаточную подсветку.
Ведь Питер~--- это \text{60-я} параллель, а~$66^{\circ}$ и~33~минут ($66{,}5622^{\circ}$)~--- это Полярный круг, где уже
светло, где Солнце 22 июня просто сияет круглые сутки.
 Из~этого, в~частности, следует, что в~Мурманске
на~самом деле зимой немного светает, потому что \text{69-я}~параллель отличается от~\text{67-й} всего на~2 с~лишним
градуса, поэтому на~самом деле зимой в~Мурманске в~3 или 4~часа вполне себе светло, просто Солнца
нет. Потому что если бы там было темно, то тем более было бы темно ночью в~Петербурге летом. Но~это
ведь не~так. Летом в~Петербурге достаточно светло, а~уж тем более на~Соловках, \text{64-й} градус.
Соловки и~Мурманск симметричны относительно Полярного круга. Поэтому если 22~июня на~Соловках
можно просто читать газету в~час ночи, то значит, что в~Мурманске 22~декабря в~час дня тоже можно
будет читать газету, это будет то же самое состояние светового дня. Это надо понимать, потому что
когда вам говорят, что там, в~Норильске или Мурманске, никогда не~светает в~течение полугода, это
сказки просто. Светает, каждый день, но~солнце не~восходит. А~вот на~Диксоне, да. В~порту Диксон
на~широте 73,5~градусов состояние <<22~декабря в~час дня>> почти такое же, как в~Москве <<22~июня в~час
ночи>>,
 то есть практически темно, поэтому уже в~Диксоне, например, никакого вам не~будет
<<рассвета>> в~течение пары месяцев подряд. Вот. Учтите это, когда будете планировать ваши
путешествия. Все загадки отгаданы.

\textbf{Слушатель:} Нет. Не~обсудили про качели.

\textbf{Задача про качели.}

Итак. Начнем с~простого. Качели, которые устроены как палка и~сидение (рис.~\ref{f:d14}).
Какие у~них устойчивые положения равновесия и~почему?

%Рис. 14
\xPICi{d14}{Почему эти качели не~поддаются уравновешиванию?}

Почему положение уравновешивания (рис.~\ref{f:d15}) не является\linebreak устойчивым?
Предположим, что качели чуть-чуть покачнулись от~ветерка. Они перешли в~следующее состояние: (рис.~\ref{f:d15}).

%Рис. 15
\xPICi{d15}{Проследим-ка мы, что произошло с~центром тяжести системы <<доска+спинки>>.}

Видно, что одно плечо рычага после поворота уменьшилось, а~другое увеличилось. Поэтому качели
наклонятся в~сторону большего плеча. И~это можно показать математически. Но~не~нужно. Это
давным-давно уже сделано в~механике, причем в~виде общей и~очень простой теоремы: \textbf{чтобы жесткая
система сохраняла равновесие, центр тяжести этой системы должен лежать ниже точки опоры.}

А~теперь рассмотрим ситуацию, когда у~качелей вместо спинок есть держалки для~ног (рис.~\ref{f:d16}).

%Рис. 16
\xPICi{d16}{Центр тяжести ниже точки опоры.}

Пусть опять подул ветерок, и~качели наклонились. Опять изменились плечи, только увеличилось то
плечо, которое сверху. Поэтому качели потянет вниз, обратно к~горизонтальному состоянию равновесия.
А~дело в~том, что теперь центр тяжести НИЖЕ точки опоры (сами поймите, почему).

Но~самое интересное, почему качели без~спинки и~подножки всё равно находятся в~наклоненном виде?
Здесь и~без~теоремы ясно.

Дело в~том, что у~доски есть толщина. И~когда качели наклоняются, одно плечо получается
на~маленький треугольничек больше, чем другое (см. рис.~\ref{f:d17}).

%Рис. 17
\xPICi{d17}{Центр тяжести доски понизился.}

Он-то и~перевешивает качели в~сторону.

\centerline{* * *}

А~теперь~--- \textbf{нерешенные проблемы школьной математики.} Но~прежде я~хочу рассказать о~том, как мыслят
математики, как они решают задачи. Есть такой английский принцип: <<Think out of the box>>, то есть
<<Подумай, не~выглянуть ли за~пределы исходного ящика>>. Давайте посмотрим, как он работает.

Задача:

На~плоскости даны три различные по~радиусам окружности, не~пересекающиеся друг с~другом. К~каждой
паре окружностей проведена пара внешних касательных, и~отмечена точка их пересечения (см.
рис.~\ref{f:d18}). Лежат ли три отмеченных точки на~одной прямой?

%Рис. 18
\xPICi{d18}{Рисунок-шутка. (Из-за нарочито небрежно нарисованных пар касательных
читателю пытаются внушить НЕВЕРНЫЙ вывод о~том, лежат ли точки пересечения
пар касательных на~одной прямой).}

ОТВЕТ: \textit{точки пересечения касательных лежат на~одной прямой.}

А~как же быть с~рис.~\ref{f:d18}? Он что, нас обманывает? Да!!! С~рисунками это часто бывает. Поэтому делать
выводы надо не~после беглого взгляда на~рисунок, а~после строгого математического доказательства
(или строгого опровержения).

Доказательство состоит в~следующем. Рассмотрим три полусферы, которые пересекаются нашей плоскостью
по~своим большим окружностям. Представьте себе 3~сферических купола большого, среднего и~малого
радиуса.

Эти 3~сферы сверху накрываются постепенно опускающейся вниз горизонтальной плоскостью, пока
не~произойдет касание самого большого купола. (Такая плоскость ровно одна.) Теперь (глядя на~рис.~\ref{f:d18}
и~мысленно выходя за~пределы исходной плоскости) будем <<вертеть>> получившуюся плоскость до~тех
пор, пока она, оставаясь касательной к~большому куполу, не~коснется среднего купола; затем вертим
ее дальше (не~теряя точек касания с~большим и~средним куполом), пока она не~коснется малого купола.
Такая <<трижды касательная плоскость>> уже ровно одна (здесь надо предполагать, что центры окружностей не лежат на одной прямой).
Так вот. После очень простых соображений
становится очевидно, что наши отмеченные точки лежат в~этой новой плоскости.

(ПОЯСНЕНИЕ. Считая, что рис.~\ref{f:d18} нарисован не~на~плоскости, а~в~пространстве, содержащем исходную
плоскость, представьте себе, что вместо пары внешних касательных мы нарисовали \textit{конус}, внутри
которого <<спрятались>> касающиеся этого конуса сферы. Таких конусов будет ТРИ. Вершина каждого
из~них находится (как нам подсказывает <<пространственное воображение>>) как раз там, где находятся
отмеченные в~условии задачи точки.)

Но~отмеченные точки также лежат и~в~исходной плоскости. Значит, они лежат \textit{на~прямой пересечения
этих плоскостей}. Теорема доказана.

Я~сейчас пояснил, как думают математики. Это задача никаким простым методом не~решается без~выхода
в~пространство. Выход в~пространство решает ее в~один ход. Так происходит с~математикой.
Идея~--- \textit{выйти за~пределы того, что у~вас дано}. Математика~--- это выход за~пределы. Все великие
открытия, все великие доказательства связаны с~покиданием пределов изученного, пределов данного
и~требуемого в~задаче.

\textbf{Нерешенные математические проблемы}

Самая старая~--- из~известных мне~--- нерешенная математическая проблема. Говорят, был такой
в~Сиракузах царь или вельможа, который занимался странным видом
деятельности\footnote{Согласно последним данным, которые я получил, никакой царь ничем таким не занимался.
Да и проблема гораздо моложе.  См. http://egamath.narod.ru/Nquant/Collatz.htm.}.
 Он брал натуральное
число. Если оно было четное, он делил его на~2, пока оно не~станет нечетным. Например, если это
было число 12, оно превращалось в~3.
$$
12\to 6\to 3.
$$
А~вот когда оно становилось нечетным, он умножал его на~3 и~прибавлял единицу. То есть 3 он
превратил бы в~число~10. А~10? Сначала в~5, потом в~16. 16 в~8, 4, 2, 1 и~в~итоге в~4.
$$
12\to 6\to 3\to 10\to 5\to 16\to 8\to 4\to 2\to 1\to 4.
$$

Как видите, мы сейчас пришли к~циклу:
$$
4\to2\to1\to4\to2\to1\to4\to2\to1\ldots
$$
Давайте возьмем еще какое-нибудь число. Скажем, 13 возьмем.
$$
13\to40\to20\to10\to5\to16\to8\to4\to2\to1\to4.
$$
Опять начинается такой же цикл. Возьмем 17.
\begin{multline*}
17\to52\to26\to13\to40\to20\to10\to5\to16\to8\to4
\to\\\to
2\to1\to4.
\end{multline*}
Заметьте, что от~17 уже довольно далеко до~повторяющейся части. Сиракузский царь (или кто-то еще)
перебрал первую тысячу ~чисел и~обнаружил, что все они приходят к~одинаковому концу, их всех ждет цикл <<$4\to2\to1$>>.
При~этом для~некоторых чисел цепочки получаются очень длинные, и~числа в~процессе преобразований
достигают очень больших значений. Число~27 достигает 9232, приходит к~циклу за~112~шагов. Вопрос:
любое ли число придет к циклу? Ответ был (якобы) неизвестен 2500~лет назад, и~до~сих пор неизвестен. Конечно
же, компьютеры давно запущены, и~числа давно проверены до~величины порядка~$10^{15}$. Компьютер
продолжает работать. Но~все числа перебрать нельзя. И~даже если бы компьютер не~довел какое-то
число до~цикла~--- это не~доказывает, что этого сделать нельзя. Возможно, цепочка просто очень
длинная. Что в~этой проблеме интересно? Вспомним теорему Ферма (для~нее тоже получалась всё более
длинная цепочка степеней <<$n$>>, для~которых она верна). Но~ее доказали для~любого <<$n$>> без~помощи
компьютера (в~принципе, если бы компьютер предложил три числа $x$, $y$, $z$ и~степень <<$n$>>, которые
опровергли бы теорему Ферма, он бы решил проблему). А~в~нашем случае компьютер ничего не~может. Разве сам только он найдет какой-то новый цикл!

В~1994~году ~Уайлз, готовясь к докладу, нашел ошибку в своем ~доказательстве <<великой теоремы Ферма>>. К счастью, ошибка оказалась
несущественной и~была им исправлена. А~1~апреля ему пришло электронное письмо, в~котором
математик, известный Уайлзу, писал, что, пользуясь его методами, он \textbf{опроверг} теорему Ферма.
В~письме приводились числа и~опровержение, содержащее маленькую, незаметную ошибку\ldots\ У~Уайлза был
шок (он забыл про 1~апреля). К~счастью, эта шутка оказалась не~смертельной.

Но~теорему Ферма в~итоге долгих усилий доказали, а~рассматриваемую нами~--- нет. Уже столько лет
требуется человек (апрелеустойчивый), который это докажет.


Следующая по~сложности проблема -- тоже простая (по~формулировке, конечно). Она поставлена
сравнительно недавно. И~я~совершенно уверен, что ее скоро решат.


Давайте рассмотрим прямую линию. Можно ли раскрасить прямую линию в~две краски так, чтобы точки
на~расстоянии единица всегда получались разноцветными? Ясно, что одного цвета недостаточно, а~в~два цвета раскрасить можно. Например, всю прямую можно
разбить на~полуотрезки длины~1 с~отброшенным правым концом. И~эти полуотрезки поочередно
закрашивать то красным, то зеленым цветом.


Поэтому для~прямой минимальное количество цветов, которое требуется, чтобы любые две точки
на~расстоянии~1 были разноцветными, равно двум. Соответствующее число для~плоскости \textit{никому
не~известно}.

\pagebreak

Давайте рассмотрим некоторые начальные соображения.\linebreak На~плоскости есть равносторонний треугольник
со~стороной~1 (рис.~\ref{f:d19}). Закрашивая всю плоскость, мы, конечно, закрасим и~всю площадь этого
треугольника, и~всю его границу~--- в~частности, закрасим и~все вершины этого треугольника.

%Рис. 19
\xPICi{d19}{Выбираем самый трудный для~закрашивания треугольник.}

Сколько нам нужно цветов?

\textbf{Слушатель:} Хотя бы 3\ldots

\textbf{А.С.:} Да, двух уже недостаточно. Иначе из~трех вершин на~расстоянии~1 друг от~друга две
окажутся одноцветными. Ведь этот треугольник специально взят таким, чтобы длины его сторон были
<<запрещенными>>.

Поэтому нужно хотя бы три разных цвета (скажем, К~--- красный, С~--- синий, З~--- зеленый). Представьте себе,
что с~трех сторон к~этому треугольнику пририсованы такие же треугольники, затем еще и~еще
приклеиваем множество таких <<особо трудных>> треугольников, пока вся плоскость не~окажется сплошь
покрытой ими. (Математики в~этом случае говорят так: рассмотрим на~плоскости ТРЕУГОЛЬНЫЙ ПАРКЕТ.)
Раскрасив правильным образом этот паркет цветами К, С, З (если бы нам это удалось), мы бы полностью
решили поставленную задачу для~плоскости. Вы, конечно, догадываетесь, что нам не~удастся этого
сделать (иначе бы эту задачу давно бы уже решили опытные математики). Но~мы всё же попробуем это
сделать~--- возможно, от~этого расширится горизонт наших знаний. Сначала раскрасим правильным
образом только \textit{вершины 3-угольного паркета}. Эти вершины образуют горизонтальные ряды на~плоскости;
в~каждом ряду вершины смещены на~$\bfrac12$ по~отношению к~предыдущему (и~последующему) ряду. Предлагается
такой способ раскраски вершин в~этих рядах (рис.~\ref{f:d19-1}).

\xPIC{d19-1}

%\begin{gather*}
%~\underline{\text{К}}~\text{С}~\text{З}~\text{К}~\text{С}~\text{З}~\text{К}~\text{С}~\text{З}~...\\
%\underline{\text{С}}~\underline{\text{З}}~\text{К}~\text{С}~\text{З}~\text{К}~\text{С}~\text{З}~\text{К}~\text{К}~...\\
%~\underline{\text{К}}~\text{С}~\text{З}~\text{К}~\text{С}~\text{З}~\text{К}~\text{С}~\text{З}~...\\
%\text{С}~\text{З}~\text{К}~\text{С}~\text{З}~\text{К}~\text{С}~\text{З}~\text{К}~\text{К}~...
%\end{gather*}

Как вы видите, каждые три ближайшие вершины закрашены разными цветами, как и~должно быть
по~условию. Ведь расстояние между ними как раз равно 1. Обратите внимание, что третий ряд раскрасок
совпадает с~первым, четвертый~--- со~вторым, и~т.\,д. до~бесконечности. В~данном случае (когда мы
используем только три разных цвета) это не~случайность~--- легко понять, что две вершины <<через
ребро>> будут одноцветными (см. рис.~\ref{f:d19-1}). В самом деле, если разрешенных цветов
только три, то для <<отраженной>> вершины просто нет выбора. Однако при любой попытке продолжить
раскраску на ребра паркета (и далее на собственно плитки) нас постигнет неудача. Эта неудача
не~является случайной (из-за того или иного неверного подхода к~этой задаче), а~является следствием
такой (уже доказанной) теоремы: \textbf{Для~правильного закрашивания плоскости заведомо не~хватит
трех красок.}
 Доказано и~полезное
добавление к~этой теореме: для~успешного закрашивания заведомо хватит СЕМИ разных красок. Таким
образом, в~настоящее время известно следующее: минимальное количество красок для~закрашивания
плоскости равно либо 4, либо 5, либо 6, либо~7.

Эту теорему можно доказать вполне школьными вычислениями, и~мы это сделаем~--- жалко было
бы не~познакомить вас с~таким элегантным доказательством.
 (Его придумали братья Мозеры. Просто
взяли и~предъявили 7~конкретных точек на~плоскости и~доказали, что даже эти 7~точек НЕЛЬЗЯ
закрасить тремя разными цветами. Значит, всю плоскость и~подавно нельзя ими раскрасить. Эти семь
точек образуют замысловатую конструкцию, похожую на~два веретена, нижние концы которых соединены,
а~прочие два конца связаны веревочкой длины~1 (понятно, почему~1?). С~тех пор в~жаргоне математиков
появилось звучное выражение \textbf{<<Веретено братьев Мозеров>>}).

Сейчас я~покажу, что на~самом деле нужно хотя бы 4~цвета.

Предположим, что мы можем раскрасить плоскость в~3~цвета. Тогда любой правильный треугольник будет
разноцветным (в~смысле окраски своих вершин), а~значит, у~ромба из~двух таких треугольников (рис.~\ref{f:d20})
две противоположные вершины окажутся одного цвета. Вот и~получилось первое веретено!

%Рис. 20
\xPICi{d20}{Веретено. Будем вращать его вокруг нижней точки~<<К>>.}

Повернем ромб вокруг одной из~вершин ровно настолько, чтобы расстояние между второй вершиной
и~между новым положением второй вершины стало равным~1 (см. рис.~\ref{f:d21}).

%Рис. 21
\xPICi{d21}{Веретено братьев Мозеров. Длина всех ребер равна~1. Конструкция содержит 7~вершин (точки кажущегося пересечения
на~оси симметрии вершинами не~являются).}

На~рис.~\ref{f:d21}, строго говоря, ребра нам вообще не~нужны, а~нужны только 7~вершин. Ребра нарисованы
только для~лучшего понимания идеи доказательства. Раскрасим вершины правого веретена с~помощью
цветов К, С, З~(мы предположили от~противного, что тремя цветами можно правильно раскрасить вершины
<<двойного веретена>>) (см. рис.~\ref{f:d20}). Если нижняя вершина
окрашена цветом~<<К>>, то и~противоположная вершина левого веретена должна быть
покрашена цветом~<<К>>. Так как горизонтальное верхнее ребро нарочно выбрано так, чтобы длина его была
равна~1, то нарушается основное условие закраски.
 Теорема доказана от~противного.

Человечество научилось красить плоскость в~7~цветов; ни в~6, ни в~5, ни в~4 оно красить плоскость
не~умеет и~не~знает, возможно ли такое.

Андрей Михайлович Райгородский, который очень любит эту проблему, считает, что возможно покрасить
плоскость в~4~цвета. Но~это пока никаким \textit{абсолютным доказательством} не~подтверждено.


%Рис. 22
\xPIC{d22}

\pagebreak

Чтобы покрасить в~7~цветов, делается (рис.~\ref{f:d22}) 6-угольное замощение плоскости (шестиугольный
паркет). Подбирается размер 6-угольника и~предъявляется аккуратная раскраска.

С~этой задачей связана еще одна проблема. Посмотрите на рис.~\ref{f:d21} не~как на~схему соединения вершин
<<двойного веретена>>, а~как на~карту некоего 5-угольного острова, на~котором расположились
9~различных государств (каждый связный кусочек, даже самый маленький, является государством).
 Стало быть, на~географической карте этого острова каждое из~государств надо
было бы, по-хорошему, закрасить своим собственным цветом. Но~государств на~свете имеется ужасно
много, а~количество цветов, различаемое человеком, ограничено. Да и~при~изготовлении карты
полиграфисты хотели бы иметь сильно ограниченный набор цветов (резко отличающихся друг от~друга).
Возникает чисто математический вопрос (<<проблема четырех красок>>):

Можно ли любую карту на~плоскости раскрасить в~4~цвета так, чтобы страны, имеющие общую границу
ненулевой длины, были разных цветов? Или нужно 5~цветов? (То, что 3~цветов мало, довольно быстро
показывается на~примере.)

Вопрос: можно ли карту 5-угольного острова раскрасить 2; 3; 4~цветами? (см. рис.~\ref{f:d21}).

Проблема четырех красок решена в~1976~году. Путем длиннейшего компьютерного
перебора, который увенчал длинное математическое рассуждение, было доказано,
что четырех цветов хватает для любой карты на плоскости. Даже математическая
часть была столь сложна, что всерьез взялись за ее проверку только через 10~лет.
Несколько <<дырок>> нашли, но все они были успешно <<залатаны>>.

Чтобы застраховаться от ошибки в компьютерной части, написали две полностью
независимые программы~--- ни о какой ручной проверке речи быть уже не могло.
Наконец, в~\text{1990-х} годах первая часть тоже была автоматизирована, а в~\text{2000-х} всё
доказательство целиком было записано на формальном языке\vadjust{\pagebreak} и верифицировано
программой Coq (представьте себе, есть такая программа, которая верифицирует
формальные доказательства!).


Следующий набор проблем связан с~\textbf{простыми числами} и~с~делимостью.

Что такое простое число? Простое число~--- это такое целое положительное число, которое делится
только на~два числа: на~себя и~на~единицу. Простые числа: 2, 3, 5, 7, 11, 13, 17, 19, 23, 29, 31,
37, 41, 43, \ldots

Еще Евклид знал, что простых чисел бесконечное количество. Но, тут есть одно <<но>>. Заметили, что
простые числа любят появляться парочками через один. Например, 11 и~13, или 41 и~43. Такие числа
назвали <<близнецами>>. (Числа 2 и~3 <<близнецами>> не~называют, потому что это единственный
случай, когда расстояние между соседними простыми числами равно единице~--- кстати, почему?)
Нерешенная проблема заключается в~том, что никто не~знает, бесконечно ли множество простых
<<близнецов>>.


Если мы перебираем подряд простые числа, то то и дело встречаем пары близнецов. Так вот, никто не~может
доказать, что какая-то конкретная пара <<близнецов>> последняя, или что таких пар бесконечное
количество.

С~удалением от~нуля простые числа встречаются всё реже и~реже. В~конце XIX~века Адамар
и~Валле-Пуссен доказали закон распределения простых чисел. Согласно этому закону, у~произвольного
числа от~1 до~$n$ в районе большого натурального числа $n$  шанс оказаться простым равен $\bfrac1{\ln n}$.


Функция <<логарифм>> постепенно растет, поэтому данная дробь постепенно убывает, стремясь к~0, то
есть вероятность встретить простое число падает вплоть до нуля.


ПРИМЕР. Пусть $n=20$. Тогда шанс встретить простое число среди первых 20 натуральных чисел равен $\bfrac1{\ln 20}$,
что примерно равно $\bfrac{1}{2{,}996} = 0{,}3338$. Значит, ожидается, что среди первых 20 чисел
простых будет $20\cdot0{,}3338 = 6{,}676$. На~самом деле их ровно~8.

\pagebreak

А~вот простые <<близнецы>> встречаются не~регулярно~--- более нерегулярно, чем сами простые числа.
Разрыв между ними то маленький, то большой. Вопрос: стремится ли к~нулю минимальный разрыв? В~2013~году было доказано, что нет.

Следующая проблема. Если вы перебираете четные числа, то их можно разбить на~два слагаемых:
$6=3+3$,
$8=3+5$,
$10=3+7$.
Всегда получается представить четное число в~виде суммы двух простых:
$$
22=11+11,\quad
36=19+17,\quad
66 = \text{<<напишите сами, какие>>},
$$
и~так далее.

Пока все четные числа, которые смог проверить компьютер, удалось разложить в~сумму двух простых.
Гипотеза И.\,М.~Виноградова состоит в~том, что любое \textit{четное} число можно представить в~виде суммы
двух простых чисел. Виноградов доказал, что любое нечетное число можно представить в~виде суммы 3
простых чисел. \textit{А~вот про четные пока не~могут доказать.}

\textbf{Совершенные числа}

Сколько совершеннолетий в~жизни человека? Многие думают, что одно~--- 18-летие. На самом деле совершеннолетий в~жизни человека~--- два!
Это <<6-летие>> и~<<28-летие>>. Потому что числа эти~--- <<совершенные>>.

Что же такое \textit{совершенное число}? Совершенное число~--- это число, которое равно сумме своих
делителей, меньших, чем само это число. Какие делители у~числа 6, считая единицу, но~не~считая его
самого? 1\ldots

\textbf{Подсказка из~аудитории:} 1, 2, 3.

\textbf{А.С.:} Мы видим, что $1+2+3 = 6$.
Какие делители у~числа 28?
1, 2, 4, 7, 14. Всё. И~снова выполняется равенство такого же типа:
$$
1+2+4+7+14 = 28.
$$
В~жизни человека ровно два~совершенных возраста, потому что следующее совершенное число равно 496.

У~математиков есть тост на~совершеннолетие. Они, правда, празднуют 28, а~не~18~лет. Тост всегда
такой: <<Чтоб тебе дожить до~следующего совершеннолетия>>. Но вроде как никому еще не~удавалось.

Так, а~в~чём же загадка? Априори совершенными числами могут быть как четные числа, так и~нечетные.
Более того, все четные уже описаны.

Над этим потрудились \textit{Евклид} и~\textit{Эйлер}. Первый обратил внимание на~следующую изящную
формулу: $2^{p-1}(2^{p}-1)$ (произносится она весьма своеобразно: \textit{<<два в~степени (пэ минус
один) умножить на~[(два в~степени пэ) минус один]>>}).
Буква <<пэ>> означает некоторое простое число.
Первый множитель можно раздробить на~самые мелкие из~возможных множители (равные двум). А~второй
множитель хотелось бы взять таким, чтобы его вообще нельзя было раздробить, то есть в~виде простого
числа. (Я~думаю, Евклид рассуждал именно так. Если когда-нибудь повстречаюсь с~ним, непременно
спрошу его об этом.) Вот и~высказал Евклид такую гипотезу:

Если число $(2^{p}-1)$ простое, то число $2^{p-1}(2^{p}-1)$~--- совершенное.

И~что вы думаете? Так оно и~оказалось! А~потом за~дело взялся Эйлер и~доказал теорему посложнее:
\textbf{любое} четное совершенное число можно записать в~таком виде. Чтобы вас немного <<попугать>>, давайте
проверим формулу Евклида при~$p=13$. Получается четное число
$
33550336.
$
Странные цифры, правда? Кто не~верит, что это число совершенное, проверьте.

А~с~нечетными не~всё так хорошо. Когда я~учился в~матклассе, у~нас были листочки с~задачами. \textit{И~вот
на~одном листочке была задача с~тремя звездочками:} <<Докажите, что нечетных совершенных чисел
не~существует>>.

%\pagebreak

Я~посидел дома денек, другой. Пришел в~школу и~говорю учителю: <<Что-то\ldots\ я~не~могу доказать,
честно\ldots>> \textit{А~он мне в~ответ:} <<А\ldots\ Да, это никто не~может доказать! Я~на~всякий случай дал. Вдруг
кто-нибудь решит\ldots>>

\pagebreak

\looseness=-1
Вот такая проблема! Существуют ли нечетные совершенные числа? Компьютеры пока
перебирают варианты. Если компьютер найдет, то проблему снимут. А~если не~найдет, то надо
доказывать, что их не~существует. В~конце этой темы я~хочу задать задачу-шутку (а~решение~---
не~шутка): бывают ли совершенные числа, которые в~десятичной системе записываются одними семерками?

Напоследок две решенные недавно задачи.

Возьмем \textit{много-много одинаковых} шаров. Начнем приставлять \textit{их друг к~другу} с~разных сторон
(в~пространстве).

Сколько \textit{одинаковых} шаров можно приставить вплотную к~одному шару такого же размера? Она называется задачей Ньютона.
Ньютон очень долго переписывался с~Д.~Грегори. \textit{Ньютон} был уверен, что \textit{можно приставить только 12~шаров},
а~Грегори утверждал, что~13.
В~результате доказали, что \text{13-й} шар \textit{чуть-чуть}
не~влезает. Ну, разумеется, возникает естественный вопрос, а~в~4-мерном пространстве сколько
шаров влезет? Задача решена в~2013~году нашим соотечественником О.~Мусиным. Он еще жив и~вполне
себе в~рабочем настроении. То есть в~4-мерном пространстве она решена, а~в~5-мерном, кажется,
еще нет.

А~теперь, наконец, \textbf{Гипотеза Пуанкаре.}

Что мы знаем о~нашем мире? Во-первых, что он 3-мерный. \textit{Во-вторых, у~него нет края.} Края в~том
смысле, в~котором его воспринимает таракан, подползая к~краю стола. \textit{Мир везде одинаковый.} То есть
таракан ползет по~сфере или по~бесконечной плоскости. А~люди <<ползают>> по~трехмерной сфере или
по~бесконечному пространству (а~где именно~--- надо бы уточнить).

\textit{А~еще наш мир ориентированный.} То есть что бы вы ни делали в~этом 3-мерном мире, ваша правая нога
никогда не~станет левой.

Исследования в области теоретической физики (так называемые уравнения
космологии Фридмана и других ученых) не исключают того, что наш мир конечен.
Можно даже представить себе, что сверхдалекие звезды, которые видны справа
и слева от Земли~--- это одни и те же звезды. И, может быть, мы сможем увидеть
на небе Землю, улетая от нее вертикально вверх, долго-долго летя и возвращаясь
на эту же Землю с другой ее стороны! Это трудно себе представить, но такая
гипотеза не противоречит современным научным данным.

Наше пространство, возможно, является искривленным, то есть служит примером
нетривиального {\em трехмерного многообразия}. Может ли к нему быть применена
гипотеза Пуанкаре, доказанная Перельманом? Вернемся к <<двумерным мирам>>.
Если я беру камеру от колеса (рис.~\ref{f:d23}), продеваю в него нитку и завязываю, то я
никогда не смогу ее снять. А если я завяжу нитку на сфере, я сниму ее без проблем.
Всё, что нам осталось предположить про наш мир, чтобы применить к нему гипотезу
Пуанкаре, это принять на веру, что в нашей вселенной <<трюк с завязыванием петли>>
не пройдет, и любую петлю можно стянуть.  Описанное свойство поверхности~--- сферы, но не камеры!~--- носит название
\textit{односвязности}\footnote{На самом деле ориентируемость поверхности, и вообще любого
топологического пространства, является следствием односвязности, но это уже сложнее.}.

%Рис. 23
\xPICi{d23}{Пусть наш Космос имеет форму <<бублика>>, только не~двумерного, а~трехмерного,
расположенного в~пространстве более высокой размерности. Как бы могли подтвердить этот факт земные
космонавты? По~наличию <<дыры>> в~этом бублике.}

\textit{Так вот, если наш трехмерный мир конечен и односвязен, то мы попадаем в~условия теоремы
Пуанкаре~--- Перельмана.}
 И тогда он обязательно является 3-мерной сферической поверхностью
4-мерного пространства-шара.

Обычная сфера радиуса~1 задается уравнением: $x^{2}+y^{2}+z^{2}=1$.

А~3-мерная того же радиуса вот так: $x^{2}+y^{2}+z^{2}+k^{2}=1$. (Подумайте, почему координат на единицу больше, чем размерность!)

Раньше это была гипотеза Пуанкаре и~относилась она только к~топологии. Теперь это~--- \textbf{теорема}
Пуанкаре~--- Перельмана. И~теперь ее можно пытаться применять в~космологии.

\endinput
