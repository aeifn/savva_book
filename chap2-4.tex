%2.4.
\section{Лекция 4}
\label{2.4}

\textbf{А.С.:}
Сегодня мы будем заниматься комплексными числами. Но~для начала интересная зарисовка из~теории
вероятностей. Если бы нас было человек 30, я~бы поставил 5 мороженых к~1, что у~двоих из здесь
присутствующих совпадут дни рождения. На~самом деле граница проходит на~числе 23. Если
в~аудитории 23~человека, то вероятность совпадения хотя бы двух дней рождения примерно равна 50\%. Правда,
совпадут только месяц и~число рождения, но не обязательно год. Для~людей, которые об этом не~задумывались, это совершенно
удивительный факт. Вроде бы всего 23~человека, как же такое может быть?
 Но~математика открывает
этот секрет.

Еще один интересный сюжет: два человека решили встретиться в~метро на~станции Кропоткинская.
Но~вышло так, что они не~договорились о~времени. Известно лишь, что они свободны между 9 и~10 утра.
Стратегия у~них такая: человек приходит и~ждет 15~минут. Если не~дождался, уходит.
Вопрос: что вероятней, встретиться или разминуться? Чему равна вероятность того, что они встретятся?

<<Математическая>> теория вероятностей на~эту тему говорит следующее. Давайте расположим на~плоскости все возможные исходы в этой <<игре>>.


По~оси~$x$ будем откладывать момент прихода первого, а~по~$y$~--- второго (в~минутах после 9~часов).
Получившийся квадрат называется \textit{фазовым пространством задачи} (рис.~\ref{f:h1}).
А вот если первый может появиться в~любой момент от~9 до, например,
11~часов, то фазовое пространство будет не~квадратом, а~прямоугольником. Так
как и~момент появления первого, и~момент появления второго совершенно непредсказуемы в рамках промежутка с 9 до 10,
следует представлять себе, что и~квадрат (слева), и~прямоугольник (справа) покрыты равномерной сетью
из~большого количества точек.

%Рис. 1
\xPICi{h1}{Два разных фазовых пространства.}

Теория вероятностей постоянно оперирует с~понятием <<зависимости>> и~<<независимости>> нескольких
случайных величин. Здравый смысл подсказывает, что наши события (то есть приход \text{1-го}
и~приход \text{2-го}) независимы. Тогда все исходы, т.\,е. пары (время прихода первого и~время
прихода второго)~--- равновероятны. Мы сейчас нарисуем зону, в~которой друзья встретились,
и~посмотрим, какая у~нее площадь (для~левой части рис.~\ref{f:h1}).

Если они пришли в~один и~тот же момент, то из~таких точек мы получим диагональ -- одинаковый момент
прихода. Ясно, что они встретятся (и~время ожидания будет равно~0).

А~если они немножко отклонились от~диагонали влево/вправо? Тогда тоже встретятся, потому что один из~них
пришел немножко раньше другого и~дождался второго. Надо понять, на~какое самое большое число минут им
можно отклониться друг от друга по времени прихода, чтобы встреча еще произошла? На~15~минут. На~одну четверть часа. Иначе будет как
в~известной песне\footnote{Договорились мы на~завтра: <<На том же месте, в~тот же час!>>}.

%%Рис. 2
%\xPICi{h2}{Договорились мы на~завтра: <<На том же месте, в~тот же час!>>}

Мы получили границы зоны встречи. Что происходит на~границе? Первый пришел, например, в~9~часов 35~минут,
а~второй в~9:50. Тогда  первый,
который пришел в 9:35, уже собирался уходить, и тут появился второй.

Теперь надо посчитать площадь <<встречи>> (то есть участка квадрата, описывающего пары моментов прихода, при которых
встреча произойдет) и~поделить ее на~общую площадь фазового пространства. Вычислим сначала площадь оставшейся части
для~случая квадрата (рис.~\ref{f:h3}).
$$
s=\left(\bfrac34\right)^{2} = \bfrac9{16}
$$
--- площадь оставшейся части,
$S = 1^{2} = 1$~--- площадь всего квадрата,
$$
S-s = 1-\bfrac9{16} = \bfrac7{16}
$$
--- площадь <<встречи>>.

%Рис. 3
\xPICi{h3}{Встреча возможна только внутри шестиугольника ($15'= 0{,}25$~часа).}

Число $\bfrac7{16}$ чуть-чуть меньше $\bfrac12$. То есть ждут всего 15~минут, а~вероятность встречи близка к~50\%.

\textit{Упражнение.} А~какой будет ответ, если фазовое пространство не~квадратное, а~прямоугольное (рис.~\ref{f:h1},
справа)?

Про теорию вероятностей можно говорить очень долго. Это отдельная, очень большая, интересная наука
и~для~школьной программы, и~для~людей, занимающихся другими науками. В~теории вероятностей есть свои
проблемы. Например, данные про большой город типа Москвы входят в~очень резкий контраст с~базовыми
предположениями теории вероятностей. Рассмотрим состояние пробок на~дорогах. Оно складывается из~миллиона случайных решений отдельных людей.
Каждый, у~кого есть машина, решает, поехать ли
на~машине или на~общественном транспорте, то есть примерно миллион человек одновременно решают
задачу, на~чем им ехать. И~типовое предположение теории вероятностей о~том, что выборы людей
независимы друг от~друга, предсказывает абсолютно одинаковые пробки при~одинаковых метеорологических
условиях. Если наша теория верна, если решения независимые, то должны быть идентичные дорожные
ситуации при~одинаковых условиях. Аварии учесть трудно. Но~одна, две мелких аварии не~сильно влияют
на~трафик. Математики очень мало знают про транспорт. Но~самое главное~--- есть стойкое ощущение,
что эта модель неверна. Люди друг с~другом каким-то образом связаны. Они реагируют на~фазы~Луны, пятна на Солнце или
на~что-то еще и~принимают одинаковые решения. (Например, если их просят назвать известного русского
поэта, все как один говорят: Пушкин.) Это~--- единственное объяснение, почему при~абсолютно
идентичных условиях бывают диаметрально противоположные по~структуре пробки. Сегодня город едет,
а~завтра~--- стоит.

Надо признать, что нам не~всё известно. Я, на~самом деле, считаю, что про социальные науки
(социологию, политологию, экономику) нам вообще почти ничего неизвестно. Математики врут, когда
говорят, что они разобрались в~том, как функционирует социум. Модели примитивные, никогда ничего
не~предсказывают. Иногда объясняют то, что было вчера.

С~географией дела обстоят лучше. Расселение меняется медленно. Редко бывает так: с~утра не~с~той
ноги встал и~с~досады оказался не~в~Москве, а~в~Иркутске\footnote{Хотя один подобный (похожий)
эпизод описан в <<поэме в прозе>> В.~Ерофеева <<Москва~--- Петушки>>.}. Эталоном науки должна быть
физика~--- наука о~неживой природе. Она разработана до~такой степени, которая никаким
инопланетянам, наверное, не~снилась. И~если с~физикой сравнивать науки о~социуме, то математический
блок социальных наук практически не~развит. Поэтому всяким <<гуру>>, которые появляются
на~<<Полит.ру>>, в~<<Ведомостях>> или прочих изданиях, вообще верить нельзя (в том числе и мне
самому!).
 Они делают прогнозы, а~через неделю уже всё по-другому.
Я~вижу, о~каких моделях они говорят, и~понимаю, что там обман в~каждом слове. А~в~математике,
в~лингвистике, в~других не~социальных науках нет места подвоху.

\centerline{* * *}

Вернемся к~комплексным числам. Я~хотел рассказать о том, как чудесным образом с~помощью комплексных
чисел решаются некоторые уравнения. На~прошлой лекции мы решили, что хотим иметь такое
невещественное число $i$, что $i^{2} = -1$. И~каждой точке плоскости сопоставили некоторое комплексное
число: $(x,y) \to x+yi$. Оказывается, эти числа подчиняются привычным математическим действиям: плюс, минус,
умножить, разделить. Математики довольно большую часть времени живут в~системах, которые называются
\textit{полями}. Поле~--- это такое обобщение обычных чисел. Это такие системы <<чисел>>, в~которых можно
совершать операции \textbf{плюс}, \textbf{минус}, \textbf{умножить} и~\textbf{разделить} по~нормальным обычным правилам. То есть вы
пишете какое-то алгебраическое выражение, раскрываете скобки, делите, сокращаете. Всё, что можно
сделать с~обычными действительными числами, можно сделать и с~элементами любого поля.
А~поля бывают очень разные и~иногда совершенно неожиданно выглядят\footnote{Как и обычные поля~--- сжатые, заросшие, заметенные снегом\ldots}.

Мы хотим, чтобы
множество комплексных чисел стало полем, то есть чтобы в~нём можно было делать всё, что мы
привыкли делать с~действительными числами, в~частности, умножать и делить. И~об этом мы сейчас поговорим.

Было доказано, что $x+yi = z+ti$ \textit{только в~том случае}, если имеют место равенства $x=z$ и~$y=t$.

То есть если это просто одна и та же точка на~плоскости. А~разные точки дают разные комплексные числа,
поэтому комплексные числа занимают как минимум всю плоскость. А~из~стремления к~минимализму мы
постараемся ограничиться \textbf{только} точками плоскости. Давайте учиться складывать, вычитать, умножать
и~делить точки плоскости.

Чему будет равна сумма $(x+yi) + (z+ti)$?

Мы должны получить какое-то комплексное число. Значит, у~нас будет часть с~$i$ и~часть без~$i$.
Если отложить часть <<без $i$>> по оси абсцисс, а часть <<с~$i$>>~--- по оси ординат, то у нас получится
какая-то точка на плоскости. Часто комплексное число отождествляют с~вектором (т.\,е. стрелочкой),
ведущим из начала координат в эту точку. Из~правила сложения векторов получается, что $(x+yi) +
(z+ti) = (x+z) + i(y+t)$.

Давайте посмотрим, почему так получается (рис.~\ref{f:h4}).

%$(z,t)$
%
%$(x,y)$
%
%$(x+z,y+t)$



%Рис. 4
\xPICi{h4}{Сложение комплексных чисел.}

Точки $(x,y)$ и~$(z,t)$ задают нам два вектора на~плоскости, выходящие из~начала координат. Если
сложить два вектора, получится вектор с~координатами $(x+z, y+t)$.

В~школе это называют правилом параллелограмма.

\textbf{Сумма двух точек плоскости строится так. Берем векторы, порождаемые нашими точками, и~складываем их
по~правилу параллелограмма.}

Вычитание от~сложения практически не~отличается:
$$
(x+yi)-(z+ti) = (x+yi) + (-z-ti) = (x-z) + i(y-t).
$$
%Вектор, порождаемый точкой $(z,t)$ повернется в~другую сторону, там, где достроен смежный параллелограмм (рис.~\ref{f:h5}).
Вектор, порождаемый точкой $(z,t)$, развернется в~другую сторону~--- туда, где достроен смежный параллелограмм (рис.~\ref{f:h5}).

%\newpage
%$(z,t)$
%
%$(x,y)$
%
%$(x-z,y-t)$
%\newpage

%Рис. 5
\xPICi{h5}{Вычитание комплексных чисел.}

Итак, операции \textbf{плюс} и~\textbf{минус} определены и~всегда осуществимы. Также видно, что у~каждого числа есть
противоположное к~нему: $(x+yi)$ и~$(-x-yi)$. С~точки зрения сложения и~вычитания система уже построена
и~ведет себя очевидным образом. Теперь переходим к~гораздо более интересной теме. А~именно:
\textbf{умножение и~деление комплексных чисел.}

\pagebreak

Я~хочу узнать, как должно выглядеть умножение
$$
(x+yi) (z+ti).
$$
Будем пользоваться распределительным законом, который математики называют \textit{дистрибутивным}. Проще
говоря, разрешается раскрывать скобки: $a(b+c) = ab+ac$ (как учили в~школе).

А~также $(a+b)(c+d)= ac+bc+ad+bd$.

Правило дистрибутивности вынуждает нас так умножать. Потому что так делается в~вещественных числах.
Давайте попробуем перемножить два комплексных числа
$$
(x+yi) (z+ti)=xz+xti+zyi+yti^{2}.
$$
Теперь давайте вспомним, что $i^{2}=-1$,
\begin{multline*}
(x+yi) (z+ti)=xz+xti+zyi+yti^{2}=
\\=
xz+xti+zyi-yt=(xz-yt)+(xt+yz)i.
\end{multline*}

Мы научились умножать. Произведением точек с~координатами $(x,y)$, $(z,t)$ служит точка на~плоскости
с~координатами $(xz-yt,\allowbreak xt+yz)$.

Но~этого для нас мало, потому что мы не~видим, где <<живет>> на~плоскости точка с~такими
координатами. Мы должны увидеть ее, понять, как ее построить. Как получить ее из~векторов,
порождаемых точками $(x,y)$, $(z,t)$. Какие у~этих векторов характеристики? У~них есть длины и~углы
поворота (\textit{отклонения}) от~оси~$x$. Пользуясь этими данными, мы должны получить новый вектор $(xz-yt,xt+yz)$.

Нам нужно провести некоторое исследование. Для~этого разработаем терминологию.

У~комплексного числа~--- точки на~плоскости~--- первая координата называется \textit{вещественной частью},
а~вторая~--- \textit{мнимой}. Мнимой ее называют потому, что, когда начинали с~комплексными числами общаться,
считали, что числа~$i$ не~существует. Существуют только вещественные числа. Остальные не~существуют,
они как бы у нас в~воображении, imaginary numbers.
 С~тех пор у~комплексных чисел есть действительная и~мнимая
части.

Рассмотрим еще такую конструкцию. Для~каждого вектора рисуется вектор, симметричный относительно
вещественной оси. Точка $(x;y)$ перейдет в~точку $(x,-y)$ (см. рис.~\ref{f:h6}).

%Рис. 6
\xPICi{h6}{Векторы, симметричные относительно оси абсцисс.}

Числу $x+yi$ естественным образом сопоставляется число $x-yi$, которое лежит по~другую сторону
от~вещественной оси.

Числа вида $(x+yi)$ и~$(x-yi)$ называются \textit{сопряженными}.
 Чему равно произведение этих чисел?
$$
(x+yi)(x-yi)= x^{2}+y^{2}.
$$
\textit{Что} такое $x^{2} + y^{2}$ в~геометрическом смысле? Это~--- длина вектора, обозначающего наше комплексное число, возведенная в~квадрат.
Квадрат длины комплексного числа, рассматриваемого как вектор, равен произведению его
самого и~ему сопряженного.

И~еще одна выкладка. Интересно, что получится, если я~перемножу векторы, сопряженные к~нашим исходным векторам:
$$
(x-yi) (z-ti) = (xz-yt)-(xt+yz)i.
$$

%%Рис. 7
%\xPICi{h7}{Внимание: над вектором написано <<корень $(x^{2} + y^{2})$>>, а~не~$x^{2} + y^{2}$.}


Вещественная часть не~изменилась, а~мнимая поменяла знак. Было $(xz-yt, xt+yz)$, стало $(xz-yt,
-xt-yz)$. Получается, что \textbf{если мы берем произведение двух сопряженных, то получается сопряженное
к~их произведению} (рис.~\ref{f:h8}).

%Рис. 8
\xPICi{h8}{Математики сказали бы так: умножение комплексных чисел <<уважает>>  операцию сопряжения,
и наоборот. Можно вначале сделать сопряжение каждого сомножителя, а потом перемножить их, а можно
вначале перемножить, а после сделать сопряжение перемноженных. Результат будет одинаковым.}

Хотелось бы уметь делить одну точку на~плоскости на~другую точку. Это тоже совсем не~сложно, если,
конечно, не~делить на~ноль. Но~на~ноль мы и~раньше не~могли делить. Так что ничего удивительного
в~том, что мы не~будем делить на~0, нет. Значит, так. Попробуем разделить на~число, которое
не~равно нулю. Используем основное свойство дроби: \textit{дробь не~изменится, если и~числитель,
и~знаменатель умножить на~одно и~то же число.} В~качестве такого числа мы возьмем число, сопряженное
к~$z+ti$:
\begin{multline*}
\bfrac{x+yi}{z+ti}=
\bfrac{(x+yi)(z-ti)}{(z+ti)(z-ti)}=
\\=
\bfrac{(xz+yt)+(yz-xt)i}{z^{2}+t^{2}}=
\bfrac{xz+yt}{z^{2}+t^{2}}+\bfrac{yz-xt}{z^{2}+t^{2}}i.
\end{multline*}
Итак, мы получили комплексное число в~стандартном виде: вещественная $\bfrac{xz+yt}{z^{2}+t^{2}}$
и~мнимая $\bfrac{yz-xt}{z^{2}+t^{2}}$ части.

Всё. Теперь мы умеем делить, умножать, складывать и~вычитать~--- всё как с~обычными действительными
числами. Однако мы пока не~видим, как геометрически это выглядит, а~это очень важно и~чрезвычайно
полезно.

Давайте все-таки это поймем. Для~этого перемножим $$(x+yi)(z+ti)(x-yi)(z-ti).$$
Если я~буду перемножать почленно, то получится
$$
(x+yi)(z+ti)(x-yi)(z-ti) = [(xz-yt)+(xt+yz)i][(xz-yt)-(xt+yz)i].
$$
Обратите внимание, получились сопряженные комплексные числа~--- значит, их произведение равно
\begin{multline*}
(x+yi)(z+ti)(x-yi)(z-ti) =
\\=
[(xz-yt)+(xt+yz)i][(xz-yt)-(xt+yz)i]) =
(xz-yt)^{2}+(xt+yz)^{2}.
\end{multline*}
А~если я~вспомню, что от~перемены мест множителей произведение не~меняется, и~переставлю скобки, то получу
$$
[(x+yi )(x-yi)][(z-ti) )(z+ti)]=
(x^{2}+y^{2})(z^{2}+t^{2}).
$$
Но~мы же умножали одно и~то же, значит, результаты совпадают:
$$
(x^{2}+y^{2})(z^{2}+t^{2})=
(xz-yt)^{2}+(xt+yz)^{2}.
$$
Это таинственное правило иногда изучается в~школе как одно из~правил сокращенного умножения.
Но~смысл его скрывается. Можно честно раскрыть все скобки и~получить верное равенство. Совершенно
честно, без~всяких комплексных чисел. Но~если вы сделаете это без~комплексных чисел, то природа
явления будет не~видна и~непонятна.
 А~с~помощью комплексных чисел мы говорим, что $(xz-yt)^{2}+(xt+yz)^{2}$~---
квадрат длины вектора, который является произведением исходных
векторов $(x,y)$ и~$(z,t)$. А~$(x^{2}+y^{2})$ и~$(z^{2}+t^{2})$~--- квадраты длин самих исходных векторов. Если я~извлеку
корень из~этих длин, то получится, что

\textbf{длина вектора произведения равна произведению длин исходных векторов}
$$
\sqrt{(xz-yt)^{2}+(xt+yz)^{2}} =
\sqrt{(x^{2}+y^{2})(z^{2}+t^{2})}.
$$


Мы узнали, что \textbf{при~перемножении комплексных чисел их длины перемножаются.} Осталось выяснить, куда
будет направлен вектор произведения. Вопрос, что же происходит с~углами поворота каждого
из~сомножителей?

Сейчас я~могу только сказать, что мое произведение лежит где-то на~окружности радиуса, равного
произведению длин наших векторов. Но~где именно? Сейчас мы рассмотрим преобразование плоскости.
Давайте нанесем на~наши оси координат единичную окружность. На~этой окружности <<живут>> точки 1,
$-1$, $i$ и~$-i$ (рис.~\ref{f:h9}).

%\newpage
%$i$
%
%$-i$
%
%$1$
%
%$-1$
%\newpage

%Рис. 9
\xPICi{h9}{Единичная окружность на~комплексной плоскости.}

Как записать координаты точки на~окружности? Какое комплексное число живет в~точке единичной
окружности, если вектор повернут на~угол~$\varphi$ (см. рис.~\ref{f:h10})?

%\newpage
%$\varphi$
%
%$(\cos\varphi,\sin\varphi)$
%
%\newpage

%Рис. 10
\xPICi{h10}{Нижний катет равен $\cos\varphi$, правый равен $\sin\varphi$.}


Точка данной окружности определяется углом, на~который повернулся вектор единичной длины.
Косинус~--- это координата по~оси~$x$, синус~--- по~оси~$y$. В~учебниках пишут, что косинус~--- это
отношение прилежащего катета к~гипотенузе. Но~здесь гипотенуза имеет длину 1. Поэтому косинус равен
просто горизонтальному катету. А~синус~--- это отношение другого катета к~гипотенузе. Гипотенуза
имеет длину 1, и синус~--- это просто
второй катет.

А~теперь я~совершу обещанное преобразование: умножу все точки плоскости на~комплексное число
$\cos\varphi+i\sin\varphi$.

Напомню, что при~умножении комплексных чисел длина получаемого вектора равна произведению длин перемножаемых
$$
\sqrt{(xz-yt)^{2}+(xt+yz)^{2}}=
\sqrt{x^{2}+y^{2}}\sqrt {z^{2}+t^{2}}.
$$


Подставим слева в~формулу $\cos\varphi$ и~$\sin\varphi$ вместо $z$ и~$t$
$$
\sqrt{(x^{2}+y^{2})(\cos^{2}\varphi+\sin^{2}\varphi)}=
\sqrt{(xz-yt)^{2}+(xt+yz)^{2}}.
$$

Но~$\cos^{2}\varphi+\sin^{2}\varphi=1$ (основное тригонометрическое тождество, следствие теоремы Пифагора). Получаем
$$
\sqrt{(x^{2}+y^{2})(1)}=
\sqrt{(xz-yt)^{2}+(xt+yz)^{2}}.
$$
Мы домножаем на~единицу, а~значит, длина вектора не~изменяется.

Получается, что при~умножении на~число $\cos\varphi+i\sin\varphi$ любое комплексное число остается на~той же окружности,
на~которой оно лежало.

Комплексное число <<жило>>, например, в~точке~$A$, на~расстоянии $\sqrt{x^{2}+y^{2}}$ от точки $(0,0)$; после преобразования
оно будет <<жить>> на~той же самой окружности в~какой-то другой точке~$B$, но~на~том же расстоянии
от~$(0,0)$ (см. рис.~\ref{f:h11}).

Похожим образом показывается, что \textbf{для~любых двух точек плоскости умножение на~$\cos\varphi+i\sin\varphi$ не~изменит
расстояния между ними.}

%\newpage
%$A$
%
%$B$
%
%$\varphi$
%\newpage


%Рис. 11
\xPICi{h11}{$B=A\cdot(\cos\varphi+i\sin\varphi)$}

Иными словами, умножение на~число $\cos\varphi+i\sin\varphi$, примененное ко всем точкам плоскости, является \textit{движением
плоскости}.

Давайте попробуем понять, что же это за~движение.

Для~простоты изложения по~ходу дела точки плоскости я~буду называть комплексными числами,
а~комплексные числа~--- точками плоскости. Это позволит стереть некоторый налет <<мнимости>>,
остающийся в~выражении \textit{комплексные числа}.

Пусть $q_{1}=x+yi$, $q_{2}= z+ti$~--- два комплексных числа, второе из которых не равно ни нулю,
ни единице, но при этом лежит на единичной окружности (то есть имеет модуль, или длину, равную
единице). Второе число, $q_2$, мы на время всего рассуждения зафиксируем, а первое число, $q_1$,
будем <<перебирать>>, подставляя всевозможные комплексные значения.

С~помощью формулы $q_{1}q_{2}$ мы сконструировали некоторое преобразование точек плоскости:
любая точка $q_1$ при этом преобразовании переходит в точку $q_{1}q_{2}$.
 Ключевое утверждение состоит
в том, что у этого преобразования будет только одна неподвижная точка: $q_{1}=0$ (то есть только
одна точка останется на месте).

Проведем доказательство этого утверждения. Допустим, какая-то точка $q_1$ осталась на месте.
Это означает, что $q_1 = q_1 q_2$. Перенесем оба выражения в левую часть, получим:
$$
q_1 (1 - q_2) = 0.
$$

Мы договорились, что $q_2 \neq 1$, а тогда $1 - q_2 \neq 0$, и на этот множитель можно
сократить обе части равенства. Следовательно, $q_1 = 0$, что и утверждалось. Таким
образом, наше преобразование плоскости является движением (что было установлено
выше) и оставляет на месте ровно одну точку, а именно точку $q_1 = 0$.


Один из~примеров движения плоскости ровно с~одной неподвижной точкой хорошо
известен: это~--- поворот на~некоторый угол относительно неподвижной точки. Но,
может быть, одними поворотами дело не~ограничивается? Этот вопрос исследовал
французский математик М.~Шаль. Оказалось, что ничего, кроме поворотов, в~этой ситуации быть не~может.
Принимая его исследования на веру,\footnote{Доказательство
данного утверждения не так уж и сложно, но уведет нас в сторону.} делаем вывод, что
изучаемое преобразование является поворотом.

Итак, это движение~--- поворот. Остается вопрос, на~какой угол мы повернули?
Для ответа на этот вопрос вспомним, что число $q_2$ лежит на окружности, то
есть равно $\cos\varphi+i\sin\varphi$ при некотором значении угла $\varphi$.

Я утверждаю, что наше движение является поворотом именно на угол~$\varphi$.
Потому что точка $q_1=1$ перешла в~точку $q_1 q_2 = \cos\varphi\brokenbin{+}i\sin\varphi$.
А~раз единица в~нее перешла, значит, мы повернули плоскость на~угол~$\varphi$.
Ведь комплексное число $q_1=1+0i$ имело в~начальный момент нулевой угол поворота.

Таким образом, любая точка переходит в~точку, которая получается поворотом
на~угол~$\varphi$ соответствовавшего исходной точке вектора.

В~частности, если я~беру некоторый вектор и~умножаю его на~вектор $\cos\varphi+i\sin\varphi$, то он переходит в~вектор,
повернутый на~угол~$\varphi$. Особый важный случай~--- это умножение на~вектор $\cos \pi/2 + i \sin \pi/2$, то есть
просто на~число~$i$. Умножение вектора на~$i$ приводит к~тому, что этот \textbf{вектор поворачивается на~$90^{\circ}$.}
Это особенно важно для~тех технических вузов, где изучают ТОЭ (теоретические основы
электротехники). Злые языки даже утверждают, что перед основным экзаменом по~ТОЭ там производится
предэкзамен: у~студента, заснувшего на~лекции, над ухом стреляют хлопушкой и~грозно спрашивают:
УМНОЖЕНИЕ на~$i$? Он должен сразу ответить: ПОВОРОТ НА 90 ГРАДУСОВ! (рис.~\ref{f:h12}).

%\newpage
%$A$
%
%$B=A\cdot i$
%\newpage

%Рис. 12
\xPICi{h12}{Умножение на $i$~--- это поворот на $90^{\circ}$.}

И~окончательно. \textbf{При~умножении комплексных чисел\linebreak углы складываются.} Это правило, которое мы вывели,
позволяет нам увидеть все арифметические операции над комплексными числами. А~именно, при~сложении
комплексных чисел складываем их как вектора~--- по~правилу параллелограмма. При~умножении~--- длины
векторов перемножаются, а~углы поворотов складываются. Слегка почесав в~затылке, можно даже сказать
так: \textbf{при~делении комплексных чисел их длины делятся, а~углы поворота вычитаются друг из друга}.



%%Рис. 13
%\xPICi{h13}{Наступление на~комплексные числа закончилось успешно.}

Сейчас будет бонус. Наконец-то мы запомним две зловредные формулы.

Давайте возьмем еще одну точку, лежащую на~единичной\linebreak окружности: $\cos\psi+i\sin\psi$.
Куда она перейдет при~умножении на~$\cos\varphi+i\sin\varphi$?

Она перейдет в~точку той же окружности, но~повернется на угол~$\varphi$. То есть суммарный угол для~произведения будет $(\psi+\varphi)$.

%%Рис. 14
%\xPICi{h14}{Комплексные числа~--- это большое чудо. Но~и~мелкие чудеса комплексных чисел
%заслуживают нашего внимания. Например, прямо с~неба падают формулы <<синус суммы>> и~<<косинус суммы>>.}

\textit{Получается,} что произведение
$$
(\cos\varphi+i\sin\varphi)(\cos\psi+i\sin\psi)
$$
равно $\cos(\varphi+\psi)+i\sin(\varphi+\psi)$.

Теперь раскроем скобки:
\begin{multline*}
(\cos\varphi+i\sin\varphi)(\cos\psi+i\sin\psi)=
\\=
\cos\varphi\cos\psi+\cos\varphi i\sin\psi+i\sin\varphi\cos\psi+i\sin\varphi i\sin\psi=
\\=
(\cos\varphi\cos\psi-\sin\varphi\sin\psi)+i(\cos\varphi\sin\psi+\sin\varphi\cos\psi).
\end{multline*}
С~другой стороны, это произведение равно $\cos(\varphi+\psi)+i\sin(\varphi+\psi)$. Получается, что
\begin{multline*}
\cos(\varphi+\psi)+i\sin(\varphi+\psi)=
\\=
(\cos\varphi\cos\psi-\sin\varphi\sin\psi)+i(\cos\varphi\sin\psi+\sin\varphi\cos\psi).
\end{multline*}

Но~если два комплексных числа равны друг другу, то вещественная часть равна вещественной, а мнимая~--- мнимой:
\begin{gather*}
\cos(\varphi+\psi)=
\cos\varphi\cos\psi-\sin\varphi\sin\psi,\\
\sin(\varphi+\psi)=
\cos\varphi\sin\psi+\sin\varphi\cos\psi.
\end{gather*}

Это и есть те <<зловредные>> формулы, которые доставляли вам головную боль в~школе, всем поголовно. Они очень легко
выводятся с~использованием комплексных чисел.

\textbf{Некоторые соображения о~преподавании математики\linebreak в~школе.}

У каждого человека есть некие безумные идеи, в которые он свято верит. Я свято
верю в то, что школьная математика должна быть устроена следующим образом.

Преподавание математики начинается с~\textbf{движений}, причем сразу же вводится понятие группы
движений~--- сперва прямой и окружности, затем плоскости. Давайте без обиняков это называть своими
именами~--- \textit{группа движений изучаемого объекта}. Потом следует полная характеризация этих
групп движений через то, сколько у~тех или иных движений имеется неподвижных точек.

Есть такая теорема <<о трех гвоздях>>.
%Если 3~точки плоскости остаются неподвижными при~движении,
%то движение является тождественным преобразованием, то есть оно вообще ничего не~меняет.
Если три~точки плоскости остаются неподвижными при~движении,
то движение является тождественным преобразованием, то есть оно вообще ничего
не~меняет.
Для прямой и для
окружности имеются очевидные аналоги этой теоремы, которые еще проще.

Завершим вкратце классификацию движений плоскости. Если у движения имеются две различные неподвижные
точки, то неподвижной окажется и вся прямая,
 их соединяющая, а само преобразование будет являться
\textbf{отражением относительно этой прямой}. Если у движения ровно одна неподвижная точка, то это
движение является \textbf{поворотом}. Если неподвижных точек нет, мы получаем два вида движения:
\textbf{параллельный перенос} и~\textbf{скользящая симметрия}. Больше никаких движений плоскости нет.

Это~--- теорема Шаля, которая должна входить во~все школьные программы. После того, как это прошли, нужно
приступать к~\textbf{комплексным числам}. Надо сразу сказать, что плоскость~--- это комплексные числа,
образующие \textbf{поле}. Все основные алгебраические понятия должны быть введены прямо в~детском
саду, чтобы потом в~школе уже было можно браться за~дело\footnote{Здесь Остапа понесло. Но в целом, если мы
заменим детский сад на младшую школу, то все алгебраические понятия и в самом деле можно ввести на
примере систем остатков от деления на некоторое фиксированное целое число!}.

После изучения комплексных чисел выводятся правила умножения и~сразу~--- переход к~тригонометрии.
\textit{Тригонометрия~--- это просто операции с~комплексными числами, лежащими на~окружности.}

\pagebreak

А~дальше можно переходить к~более интересным вещам, например, к~диофантовым уравнениям.

Обсудим, при чем тут комплексные числа (которых Диофант не~знал) и~диофантовы уравнения?
На~вещественной оси есть числа специальной природы, называемые \textit{целыми}. Они <<живут>> на~одинаковом
расстоянии в~обе стороны от~0 до~бесконечности. Это числа $0,\pm1,\pm2,\pm3,\pm4$  и~так далее.
 Между ними <<живут>>
всякие другие числа, которые нас пока сейчас интересовать не~будут (рис.~\ref{f:h15}). Диофантовы
уравнения~--- это уравнения, которые надо решать в~рациональных либо в~целых числах. Мы ограничимся
целыми решениями. На~прошлой лекции мы рассматривали следующее уравнение $$x^{2}+y^{2}=z^{2}.$$

%$-3$\quad
%$-2$\quad
%$-1$\quad
%$0$\quad
%$1$\quad
%$2$\quad
%$3$\quad

Мы его решили двумя способами: с~помощью анализа делимости в~обычных целых числах и~с~помощью
алгебраической геометрии. Есть еще и~\textbf{третий способ}.

%Рис. 15
\xPICi{h15}{Решетка целых чисел на~числовой оси
(такого рода решетки бывают и~на~плоскости).}

Подобно тому, как среди вещественных чисел можно выделить замечательное семейство целых, можно
выделить не~менее замечательные семейства и~среди комплексных чисел. Чем целые числа принципиально
отличаются от~вещественных? В~них (во~множестве целых чисел) нельзя делить. Иногда получается
разделить, а~иногда~--- нет. Анализ того, что на~что делится, приводит к~содержательной и красивой науке:
к простым числам, к основной теореме арифметики и, в~конечном счете, к~решению этого самого уравнения
$x^{2}+y^{2}=z^{2}$.

Теперь мы живем на~плоскости, и~хотелось бы сделать что-нибудь подобное во~множестве комплексных
чисел. Давайте по аналогии распространим целые числа на~плоскость. Как будут выглядеть целые числа
на~плоскости? Скажу по~секрету, что на~плоскости имеется огромное количество числовых систем, которые
обобщают и~продолжают целые числа.
 Можно построить числовые системы разными способами, и~они все
чрезвычайно важны для~многих диофантовых уравнений. Различные диофантовы уравнения требуют
различных числовых систем. Но~самое простое~--- это рассмотреть комплексные числа, у~которых просто
обе части (и~вещественная и~мнимая) являются целыми числами (рис.~\ref{f:h16}).

%\newpage
%$-2$\quad
%$-1$\quad
%$0$\quad
%$1$\quad
%$2$\quad
%
%$i$\quad
%$1+i$\quad
%$2+i$\quad
%
%$2i$\quad
%\newpage

%Рис. 16
\xPICi{h16}{Загадочные гауссовы числа: среди них есть <<простые>>, но~как раз
они-то и~оказались самыми загадочными.}

Узлы этой сетки и~есть <<целые числа>> на~плоскости. Первым их рассматривал Гаусс, мы назовем их
$Z[i]$. \textbf{$Z$~--- значит <<целое>>, $[i]$~--- конкретное комплексное число, присовокупленное к целым.}

Он к~целым числам прибавил число $i$ и~спросил себя: а~какие тогда должны быть числа еще взяты? Если
мы взяли $i$ и~взяли 1, то мы должны, конечно, взять их сумму~--- $1+i$. Потому что мы должны уметь
складывать, вычитать и~умножать (если хотим действовать по~правилам обычных целых чисел). Ясно, что
эти требования нас в конце концов приведут ко взятию произвольных целых кратных числа $i$, сложенных
с любыми целыми (обычными) числами.

\pagebreak

Числа вида $a+bi$, где $a$, $b$~--- целые числа, называют гауссовыми числами. Складывать и вычитать
их можно <<покомпонентно>>, то есть $(a+bi) \pm (c+di) = (a \pm c) + (b \pm d)i$. При этом <<на выходе>>
получаются снова Гауссовы числа, потому что сумма и разность целых чисел всегда являются целыми числами.

Но для полноценной работы с новыми числами нужно уметь их друг на друга умножать. Чудо состоит в том,
что при перемножении Гауссовых чисел по обычным правилам перемножения комплексных чисел <<на выходе>>
снова получаются Гауссовы, то есть целые комплексные числа. Читателю книги доставит удовольствие
самостоятельно перемножить два Гауссовых числа, чтобы увидеть, что целочисленность вещественной и
мниной частей результата умножения сохраняется.

Кроме того, новые числа удовлетворяют всем тем же принципам умножения, вычитания и~сложения, которые
верны для обычных целых чисел (потому что новые числа~--- это <<подмножество>> комплексных чисел, а
последние этим правилам подчиняются).

В то же время из-за того, что мы акцентируем внимание на~их <<цельности>>, то есть целочисленности,
у нас появляются нетривиальные моменты, связанные с~их делимостью друг на~друга (аналогично тому, как
в системе обычных целых чисел разрабатывается теория делимости, теория простых чисел и разложение на простые
множители).


В~частности, можно определить понятие \textit{простого гауссова числа.}

Так вот, оказывается, что всё, что мы знаем про целые числа~--- делимость, простота, основная теорема
арифметики~--- удивительным образом переносится на~$Z[i]$, то есть на систему Гауссовых чисел. Любое
Гауссово $a+bi$ с~целыми $a$ и~$b$ \textbf{единственным образом} раскладывается в~произведение простых
чисел, которые уже ни~на~что не~делятся.

Небольшое замечание: на~числа $1, i, -1$ и~$-i$ делятся все Гауссовы числа, так же, как в~целых числах
на~прямой все числа делятся на~1 и~$-1$.
 Например, $(a+bi):i=b-ai$. Это чуть-чуть усложняет ситуацию, потому
что однозначность разложения на~простые множители выполняется лишь с~точностью до~умножения и~деления
на~$1, i, -1$ и~$-i$. Потому что с точки зрения теории делимости $(a+bi)$ и~$(b-ai)$~--- это один
и~тот же простой множитель.

Для целых чисел на комплексной~плоскости вообще появляется много фокусов, которых не~было для целых
чисел на~прямой. Например, число~$2$ перестало быть простым. Ибо оно раскладывается на~множители
$2=(1+i)(1-i)$. Кстати, из~геометрии это тоже следует (рис.~\ref{f:h17}).

%\newpage
%$1+i$\quad
%$1-i$\quad
%$2$\quad
%\newpage


%Рис. 17
\xPICi{h17}{Вот чудеса-то: сумма чисел $(1+i)$ и~$(1-i)$ равна их произведению!
Но~обычное число~2 похитрее будет: $2+2=2\cdot2 =2^{2}$.}

По~правилу умножения мы должны взять произведение двух длин. Длина вектора $1+i$ равна длине вектора
$1-i$, и~обе равны $\sqrt2$, так как это гипотенуза прямоугольного треугольника с~единичными катетами.
Значит, у~произведения должна быть длина $\sqrt2\cdot\sqrt2=2$.

Посмотрим, что произойдет с~углами. При~умножении углы складываются. Но~они у~нас противоположные
по~знаку, значит, при~сложении получится~0. То есть при~умножении мы получим вектор длины~2,
направленный по~оси~$X$. Обратите внимание, что мы невзначай нашли одно из~решений уравнения
в~комплексных числах: $z+w=zw$ (подпись к~рис.~\ref{f:h17}).

Какие еще числа перестают быть простыми? Например, число~5. Теперь
$
5 = (2+i)(2-i) =2^{2} +1^{2}.
$
А число $3$ можно разложить на множители или нет? Есть ли тут какое-то общее правило?

Оказывается, есть. Более того, ответ на заданный вопрос теснейшим образом связан с вопросом про
<<обычные>> целые числа, а именно: какие простые числа можно представить в виде суммы двух полных
квадратов~--- то есть двух чисел, из которых можно нацело извлечь квадратный корень? Потрясающим
образом этот вопрос решается введением Гауссовых чисел и изучением их арифметики.

Окинем еще раз взглядом наши построения. Мы ввели комплексные числа. Потом в~них выделили
семейство <<целочисленных>> комплексных чисел и назвали их гауссовыми. Там развили делимость,
научились делить с~остатком, обнаружили <<Основную теорему арифметики>>. Зачем? Ответ таков:
некоторые вопросы из~арифметики обычных целых чисел можно решить только через гауссовы числа.

Какие простые числа представляются в~виде суммы двух квадратов? Эта задача чрезвычайно важная
в~теории \textit{кодирования}. (Здесь под словом <<кодирование>> понимается запись информации
в~таком виде, чтобы ее не~смогли прочесть посторонние лица. А~<<посторонние лица>> обычно очень
интересуются методами <<взлома>> использованного кода.) Человек, который что-то знает про
кодирование/декодирование, может взять и~разрушить систему Пентагона в~два щелчка мыши
(вот вам и~готов международный конфликт).

Вопросы математического кодирования~--- это вопросы примерно такого же типа, как
и~задача о~разложении простого числа в~сумму двух квадратов. И вот долгожданный
ответ на поставленный выше вопрос.



\smallskip

{\bf Теорема.} (Ферма~--- Эйлер~--- Гаусс. Гаусс здесь упомянут потому, что он ввел Гауссовы
числа и установил простым образом все три эквивалентности, приводимые в формулировке.)


\textit{<<Обычное>> простое число (не~комплексное) $p$ является суммой двух квадратов, то есть
$p=x^{2}+y^{2}$ ($x$ и~$y$~--- обычные целые числа), тогда и~только тогда, когда~$p$
перестает быть простым в~гауссовой системе чисел~$Z[i]$.
И~происходит это \textbf{тогда и~только тогда}, когда либо $p=2$, либо число <<$p$>> имеет остаток 1 при~делении на~4, то есть $p=4k+1$.}

\smallskip

У Гаусса несколько <<царских результатов>>. Он называл их разными именами. Например, есть некий
закон про поведение остатков при делении одних чисел на другие. Гаусс назвал его <<золотым
результатом>>, <<золотой результат Гаусса>>. Связь между представимостью простого числа $p$ в виде
суммы двух квадратов и его <<поведением>> в системе Гауссовых целых чисел~--- это королевская
теорема Гаусса. Как следствие, <<сокращая одну из эквивалентностей>> в теореме выше, получаем как
раз теорему Ферма~--- Эйлера: {\bf Простое число в обычных натуральных числах является суммой двух
квадратов тогда и только тогда, когда оно имеет остаток 1 при делении на~4.}
Это мгновенно
вычисляемая характеристика. Например, 97. При~делении на~4 дает остаток~1: $97=96+1=4\cdot24+1$.
Значит, по нашей теореме оно должно представляться в~виде суммы двух квадратов. Так и~есть:
$97=81+16=9^{2}+4^{2}$.

Возьмите число, в~котором 25~цифр. Проверьте, что оно имеет остаток 1
при~делении на~4, это очень просто.
 Проверить, что оно простое, немножко сложнее, но~тоже не~очень
долго. Так вот, если вы узнали, что оно простое, и~вычислили, что оно имеет остаток 1 при~делении
на~4, то вы можете спорить на~любую сумму с~любым неверующим Фомой, что есть два числа, суммой
квадратов которых исходное число является. Никакого полного доказательства этой теоремы, кроме как
через гауссовы числа, мне не~известно (существует, говорят, по~крайней мере 6~доказательств).

Давайте вернемся к~пифагоровым тройкам. Пифагоровы тройки очень красиво находятся с~помощью
гауссовых чисел. Предположим, есть тройка $x$, $y$, $z$ обычных целых чисел, которые являются сторонами
прямоугольного треугольника, то есть
$$
x^{2}+y^{2}=z^{2}.
$$
Опять рассмотрим прямоугольный треугольник, наименьший в~семействе. Иными словами, $x$, $y$, $z$ попарно
взаимно просты, у~них нет общих делителей. Тогда довольно просто показать, что
$(x+yi)$ и~$(x-yi)$~--- также взаимно просты (это следует из разной четности $x$ и $y$).

То есть у~гауссова числа и~сопряженного ему гауссова числа нет общих делителей.

Вспоминаем прошлую лекцию: $(x+yi)(x-yi)=z^{2}$.

Произведение равно квадрату некоторого числа. Значит, все (Гауссовы) простые множители числа
$z$ входят в~него в~четной степени. Это означает, что в~левой части уравнения стоит, с точностью
до обратимых множителей, произведение двух квадратов.


Этот прием применяется во~всех похожих структурах, не~только в~гауссовых числах. Если мы можем
доказать основную теорему арифметики, то будет верен и~этот замечательный результат: если произведение
двух взаимно простых чисел равно квадрату, то каждое из~этих чисел является квадратом с~точностью
до~умножения на~обратимые числа $1, i, -1$ и~$-i$ (для~гауссовых чисел) или до~умножения на любые
другие обратимые числа (если целые числа~--- не~гауссовы).


Заметая <<под ковер>> исследование дополнительных обратимых множителей, делаем вывод, что
$$
(x+yi) = (m+ni)^{2} = m^{2}+2mni-n^{2} = (m^{2}-n^{2})+2mni.
$$
Комплексные числа равны в том и только том случае, когда их вещественные и~мнимые части равны:
$$
x= m^{2}-n^{2},\quad
y=2mn.
$$
Отсюда уже нетрудно вывести и формулу для гипотенузы Пифагорова треугольника: $z=m^{2}+n^{2}$.

Вот мы и получили <<формулу индусов>>. Через гауссовы числа она выводится почти в~одну строчку.

Теперь~--- пара слов про великую теорему Ферма. Такие методы, как тот, который мы сейчас рассматривали,
развивавшиеся весь XIX~век, не~привели к~решению великой теоремы Ферма для~всех показателей.
Привело совершенно другое соображение. Соображение такое: если бы существовала тройка $a$, $b$, $c$
такая, что $a^{n}+b^{n} = c^{n}$, то существовала бы некоторая, как математики выражаются, \textit{эллиптическая кривая} с~набором свойств,
которые противоречат ее природе. Это~--- доказательство великой теоремы Ферма в~одной фразе.
Правда, к~этой <<одной фразе>> придется добавить фраз 20--30, чтобы хоть слегка пояснить, что это
за~зверь такой~--- \textbf{эллиптическая кривая}, и, главное, какое отношение она имеет к~великой теореме
Ферма.

Ну и~последний сюжет.

Диофант решал самые разные уравнения. Некоторые он сформулировал, но~был не~способен решить.
А~точнее, решения некоторых из~них не~найдены в~первых 6~томах. Мы ничего не~знаем про оставшиеся 7~томов,
и~я~не~удивлюсь, если в~них было всё, что потом открывали в~\text{XVII}, \text{XVIII}, \text{XIX} веках.
В~частности, Эйлер стал рассматривать одно из~тех уравнений, которые Диофант не~решил. \textit{Может ли быть
так, что квадрат некоторого натурального числа отличается от~куба другого натурального числа
на~единицу?} То есть требуется решить в~целых числах уравнение
$$
a^{2}=b^{3}\pm1.
$$
То, что квадрат одного числа просто равен кубу другого, очень легко представить себе, если
$a=c^{3}$ и $b=c^{2}$, при некотором целом $c$. В самом деле, тогда
$$
a^{2}=(c^{3})^{2}=c^{6}=(c^{2})^{3}=b^{3}.
$$
Возьмем, например, $c=3$. Тогда $a=27, b=9$: $27^{2} = 9^{3}=729$. Так что эта задача неинтересная. Правда,
число 729 напоминает мне один разговор.

Однажды два математика беседовали в~кафе. Один другому говорит: <<На свете нет ни одного числа,
которое не~было бы чем-то удивительным, просто ни одного>>. А~второй отвечает: <<Ну, как же? Ну,
я~возьму навскидку 1729. Что интересного в~числе 1729?>> А~второй посмотрел на~него и~сказал: <<Ты
сам не~догадываешься, насколько удивительное число ты назвал! Это первое из~натуральных чисел,
которое \textit{двумя разными способами} представляется в~виде суммы двух кубов>>.

Пальцем в~небо ткнул и попал
в~число 1729.
 И~вот что оказалось. Действительно, $1729 = 9^{3} + 10^{3}$, и~$1729=12^{3} +1^{3}$. Второй математик
был сражен этим аргументом.

Так вот, бывает ли, чтобы куб и~квадрат отличались на~единичку?

Допустим, ваш ребенок играет в~кубики. Он сложил из~них большой куб, а~вы украли у~него один кубик.
Тогда ребенок взял, развалил куб и~сложил большой огромный квадрат. Может ли такое быть? Эйлер
полностью решил эту задачу $(a^{2}=b^{3}\pm1)$.

Решим только одно уравнение из~двух, потому что другое очень сложное:
$a^{2}=b^{3}+1$~--- сложное,
$a^{2}=b^{3}-1$ простое.


В~обоих случаях можно выписать ответ в~явном виде.

У~второго уравнения решений нет, кроме тривиальных: $a=0$ и~$b=1$. Мы это сейчас докажем. А~у~первого,
кроме тривиальных ($a=1$ и~$b=0$), решением является пара $(2,3)$.
 Ведь $3^{2} = 2^{3}+1$. Других решений \textbf{нет}.
Эйлер и это доказал, но~весьма сложным путем.

Разберем простой вариант:
$$
a^{2}=b^{3}-1,\quad
a^{2}+1=b^{3},\quad
(a+i)(a-i)=b^{3}.
$$
Могут ли у~$(a+i)$ и~$(a-i)$ быть общие множители? Пусть $(a+i)$ и~$(a-i)$ делятся на~какое-то простое
гауссово число. Тогда их разность $$(a+i)-(a-i)=a+i-a+i=2i$$ тоже на~него делится.

Простых гауссовых чисел, которые делят число $2i$, всего одно: $(1+i)$. Есть еще $1-i$, но это <<то же самое простое число>>, ибо
$1-i=(-i)(1+i)$~--- то есть, одно получается из другого умножением на обратимое.

Значит, наши числа $(a+i)$ и~$(a-i)$, если они не взаимно просты, могут делиться только на~$(1+i)$. Но тогда их произведение делится
на~$(1+i)^{2}=2i$. Значит, $b$ делится на~2, а~$b^{3}$~--- на~8. Но~тогда $a^{2}$ будет иметь остаток 7 при~делении
на~8, так как $a^{2}+1=b^{3}$. А~значит, остаток 3 при~делении на~4. А, как мы выяснили на~предыдущей
лекции, таких квадратов не~существует. При~делении на~4 квадрат дает в~остатке либо 1, либо 0.
Поэтому такого быть не~может.

Значит, ни одного общего делителя у~чисел $(a+i)$ и~$(a-i)$ нет. Их произведение является поэтому кубом
некоторого гауссова числа. Согласно основной теореме арифметики, из~этого следует, что каждое
из~них само является кубом гауссова числа (снова с точностью до умножения на обратимый элемент $1,i,-1$ или $i$).
Но все они тоже кубы, так что сформулированное утверждение верно в точности: скажем, $a+i =(m+ni)^{3}$.

Вдумайтесь, что мы сделали. Мы взяли обычное уравнение в~целых числах. Зачем-то перешли в~гауссовы
числа и~внутри гауссовых чисел разложили левую часть на~множители. После чего, живя внутри
гауссовых чисел, мы сказали, что тогда $$a+i =(m+ni)^{3}.$$ При~этом $a$~--- целое \textbf{не~гауссово} число.
Гауссово число $(a+i)$ живет на~один шаг выше оси~$x$.

Это число должно быть равно кубу некоторого гауссова числа.

Теперь вспомним формулу куба суммы и~раскроем скобки:
\begin{multline*}
a+i =(m+ni)^{3}=
m^{3}+3m^{2}ni-3mn^{2}-n^{3}i=
\\=
(m^{3}-3mn^{2})+i(3m^{2}n-n^{3}).
\end{multline*}
Комплексные числа равны, значит равны их вещественная и~мнимая части:
$$
a=m^{3}-3mn^{2},\quad
1=3m^{2}n-n^{3}.
$$
Я~вернулся из~гауссовых чисел в~обычные целые числа. С~помощью гауссовых чисел я~сделал вывод,
который никогда в~жизни не~сделал бы без~них. Из~$a^{2}=b^{3}-1$ я~получил, что $$3m^{2}n-n^{3}=1.$$

Теперь уже всё просто:
$$
3m^{2}n-n^{3}=1,\quad
n(3m^{2}-n^{2})=1,
$$
$n$ и~$3m^{2}-n^{2}$~--- целые числа. Два числа дают в~произведении 1 тогда и~только тогда, когда
они одновременно равны 1 или~$-1$.
$$
n=\pm1,\quad
3m^{2}-n^{2}=\pm1.
$$
Вы заметили, <<единицу можно разложить на~множители \textit{единственным} способом: либо 1 умножить на~1,
либо $-1$ умножить на $-1$>>. Второй способ \textbf{неотличим от~первого}, так как второе решение можно сократить
на~<<обратимое число>>~($-1$). Так что второй случай кажется ненужным для~рассмотрения~--- вроде как
получается избыточная аргументация. Но, как будет видно ниже, второй случай отнюдь не~лишний.

Мой учитель Саша Шень рассказывал замечательную историю про то, как он стал математиком <<из-за избыточной
аргументации>>.
 Ему подали рыбу, филе (я~сам очень долго, лет до~30, думал, что филе~--- это название
рыбы). Так вот. Ему подали филе, и~он сказал: <<Мама, ну тут кости! Ты можешь вынуть кости?>>
А~мама применила следующий замечательный логический прием, поставив его на~дорогу математика. Она
сказала: <<Так! Саша, во-первых, это филе, и~костей в~нём быть не~может. А~во-вторых, где ты видел
рыбу без~костей?>> Саша настолько был потрясен такой <<железобетонной>> логикой, что после этого стал
математиком.

Итак, разберем наши два случая. Хотя они одинаковы с~точки зрения единственности разложения
на~множители, но~они не~одинаковы с~точки зрения наличия решений!

Первый случай: $n=1$, $3m^{2}-n^{2}=1$, следовательно, $3m^{2}=2$. Но~$m$~--- целое число. Значит, такого быть не~может.

Второй случай: $n=-1$, $3m^{2}-n^{2} =-1$, следовательно, $3m^{2}=0$. Получаем $m=0$.
$$
a+i=(m+ni)^{3} =(0-i)^{3}=(-i)^{3}=i.
$$
Так как $a+i = i$, то $a=0$. Но~$b^{3} =a^{2}+1$, значит, $b=1$.

Это~--- единственное решение исходного уравнения. Получается, что кроме тривиальных решений, других
решений уравнения $a^{2}=b^{3}-1$ \textbf{нет}.

Из~этой теории можно сделать следующий практический вывод. Если у~вас с~ребенком вышла такая
ситуация, что он сложил из кубиков большой куб, вы украли у него кубик, и~он сложит квадрат, значит, что-то
не~так. Значит, он кубик <<украл обратно>> (и~их было 729 скорее всего!). Вы можете сказать: <<Так, ты похитил у меня кубик!>>

--- Как, папа? Как ты это увидел? Ты, наверное, ясновидящий\ldots

--- Нет. Я~просто умею решать диофантовы уравнения, сынок.
